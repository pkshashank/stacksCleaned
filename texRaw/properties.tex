\IfFileExists{stacks-project.cls}{%
\documentclass{stacks-project}
}{%
\documentclass{amsart}
}

% For dealing with references we use the comment environment
\usepackage{verbatim}
\newenvironment{reference}{\comment}{\endcomment}
%\newenvironment{reference}{}{}
\newenvironment{slogan}{\comment}{\endcomment}
\newenvironment{history}{\comment}{\endcomment}

% For commutative diagrams we use Xy-pic
\usepackage[all]{xy}

% We use 2cell for 2-commutative diagrams.
\xyoption{2cell}
\UseAllTwocells

% We use multicol for the list of chapters between chapters
\usepackage{multicol}

% This is generally recommended for better output
\usepackage{lmodern}
\usepackage[T1]{fontenc}

% For cross-file-references
\usepackage{xr-hyper}

% Package for hypertext links:
\usepackage{hyperref}

% For any local file, say "hello.tex" you want to link to please
% use \externaldocument[hello-]{hello}
\externaldocument[introduction-]{introduction}
\externaldocument[conventions-]{conventions}
\externaldocument[sets-]{sets}
\externaldocument[categories-]{categories}
\externaldocument[topology-]{topology}
\externaldocument[sheaves-]{sheaves}
\externaldocument[sites-]{sites}
\externaldocument[stacks-]{stacks}
\externaldocument[fields-]{fields}
\externaldocument[algebra-]{algebra}
\externaldocument[brauer-]{brauer}
\externaldocument[homology-]{homology}
\externaldocument[derived-]{derived}
\externaldocument[simplicial-]{simplicial}
\externaldocument[more-algebra-]{more-algebra}
\externaldocument[smoothing-]{smoothing}
\externaldocument[modules-]{modules}
\externaldocument[sites-modules-]{sites-modules}
\externaldocument[injectives-]{injectives}
\externaldocument[cohomology-]{cohomology}
\externaldocument[sites-cohomology-]{sites-cohomology}
\externaldocument[dga-]{dga}
\externaldocument[dpa-]{dpa}
\externaldocument[sdga-]{sdga}
\externaldocument[hypercovering-]{hypercovering}
\externaldocument[schemes-]{schemes}
\externaldocument[constructions-]{constructions}
\externaldocument[properties-]{properties}
\externaldocument[morphisms-]{morphisms}
\externaldocument[coherent-]{coherent}
\externaldocument[divisors-]{divisors}
\externaldocument[limits-]{limits}
\externaldocument[varieties-]{varieties}
\externaldocument[topologies-]{topologies}
\externaldocument[descent-]{descent}
\externaldocument[perfect-]{perfect}
\externaldocument[more-morphisms-]{more-morphisms}
\externaldocument[flat-]{flat}
\externaldocument[groupoids-]{groupoids}
\externaldocument[more-groupoids-]{more-groupoids}
\externaldocument[etale-]{etale}
\externaldocument[chow-]{chow}
\externaldocument[intersection-]{intersection}
\externaldocument[pic-]{pic}
\externaldocument[weil-]{weil}
\externaldocument[adequate-]{adequate}
\externaldocument[dualizing-]{dualizing}
\externaldocument[duality-]{duality}
\externaldocument[discriminant-]{discriminant}
\externaldocument[derham-]{derham}
\externaldocument[local-cohomology-]{local-cohomology}
\externaldocument[algebraization-]{algebraization}
\externaldocument[curves-]{curves}
\externaldocument[resolve-]{resolve}
\externaldocument[models-]{models}
\externaldocument[functors-]{functors}
\externaldocument[equiv-]{equiv}
\externaldocument[pione-]{pione}
\externaldocument[etale-cohomology-]{etale-cohomology}
\externaldocument[proetale-]{proetale}
\externaldocument[relative-cycles-]{relative-cycles}
\externaldocument[more-etale-]{more-etale}
\externaldocument[trace-]{trace}
\externaldocument[crystalline-]{crystalline}
\externaldocument[spaces-]{spaces}
\externaldocument[spaces-properties-]{spaces-properties}
\externaldocument[spaces-morphisms-]{spaces-morphisms}
\externaldocument[decent-spaces-]{decent-spaces}
\externaldocument[spaces-cohomology-]{spaces-cohomology}
\externaldocument[spaces-limits-]{spaces-limits}
\externaldocument[spaces-divisors-]{spaces-divisors}
\externaldocument[spaces-over-fields-]{spaces-over-fields}
\externaldocument[spaces-topologies-]{spaces-topologies}
\externaldocument[spaces-descent-]{spaces-descent}
\externaldocument[spaces-perfect-]{spaces-perfect}
\externaldocument[spaces-more-morphisms-]{spaces-more-morphisms}
\externaldocument[spaces-flat-]{spaces-flat}
\externaldocument[spaces-groupoids-]{spaces-groupoids}
\externaldocument[spaces-more-groupoids-]{spaces-more-groupoids}
\externaldocument[bootstrap-]{bootstrap}
\externaldocument[spaces-pushouts-]{spaces-pushouts}
\externaldocument[spaces-chow-]{spaces-chow}
\externaldocument[groupoids-quotients-]{groupoids-quotients}
\externaldocument[spaces-more-cohomology-]{spaces-more-cohomology}
\externaldocument[spaces-simplicial-]{spaces-simplicial}
\externaldocument[spaces-duality-]{spaces-duality}
\externaldocument[formal-spaces-]{formal-spaces}
\externaldocument[restricted-]{restricted}
\externaldocument[spaces-resolve-]{spaces-resolve}
\externaldocument[formal-defos-]{formal-defos}
\externaldocument[defos-]{defos}
\externaldocument[cotangent-]{cotangent}
\externaldocument[examples-defos-]{examples-defos}
\externaldocument[algebraic-]{algebraic}
\externaldocument[examples-stacks-]{examples-stacks}
\externaldocument[stacks-sheaves-]{stacks-sheaves}
\externaldocument[criteria-]{criteria}
\externaldocument[artin-]{artin}
\externaldocument[quot-]{quot}
\externaldocument[stacks-properties-]{stacks-properties}
\externaldocument[stacks-morphisms-]{stacks-morphisms}
\externaldocument[stacks-limits-]{stacks-limits}
\externaldocument[stacks-cohomology-]{stacks-cohomology}
\externaldocument[stacks-perfect-]{stacks-perfect}
\externaldocument[stacks-introduction-]{stacks-introduction}
\externaldocument[stacks-more-morphisms-]{stacks-more-morphisms}
\externaldocument[stacks-geometry-]{stacks-geometry}
\externaldocument[moduli-]{moduli}
\externaldocument[moduli-curves-]{moduli-curves}
\externaldocument[examples-]{examples}
\externaldocument[exercises-]{exercises}
\externaldocument[guide-]{guide}
\externaldocument[desirables-]{desirables}
\externaldocument[coding-]{coding}
\externaldocument[obsolete-]{obsolete}
\externaldocument[fdl-]{fdl}
\externaldocument[index-]{index}

% Theorem environments.
%
\theoremstyle{plain}
\newtheorem{theorem}[subsection]{Theorem}
\newtheorem{proposition}[subsection]{Proposition}
\newtheorem{lemma}[subsection]{Lemma}

\theoremstyle{definition}
\newtheorem{definition}[subsection]{Definition}
\newtheorem{example}[subsection]{Example}
\newtheorem{exercise}[subsection]{Exercise}
\newtheorem{situation}[subsection]{Situation}

\theoremstyle{remark}
\newtheorem{remark}[subsection]{Remark}
\newtheorem{remarks}[subsection]{Remarks}

\numberwithin{equation}{subsection}

% Macros
%
\def\lim{\mathop{\mathrm{lim}}\nolimits}
\def\colim{\mathop{\mathrm{colim}}\nolimits}
\def\Spec{\mathop{\mathrm{Spec}}}
\def\Hom{\mathop{\mathrm{Hom}}\nolimits}
\def\Ext{\mathop{\mathrm{Ext}}\nolimits}
\def\SheafHom{\mathop{\mathcal{H}\!\mathit{om}}\nolimits}
\def\SheafExt{\mathop{\mathcal{E}\!\mathit{xt}}\nolimits}
\def\Sch{\mathit{Sch}}
\def\Mor{\mathop{\mathrm{Mor}}\nolimits}
\def\Ob{\mathop{\mathrm{Ob}}\nolimits}
\def\Sh{\mathop{\mathit{Sh}}\nolimits}
\def\NL{\mathop{N\!L}\nolimits}
\def\CH{\mathop{\mathrm{CH}}\nolimits}
\def\proetale{{pro\text{-}\acute{e}tale}}
\def\etale{{\acute{e}tale}}
\def\QCoh{\mathit{QCoh}}
\def\Ker{\mathop{\mathrm{Ker}}}
\def\Im{\mathop{\mathrm{Im}}}
\def\Coker{\mathop{\mathrm{Coker}}}
\def\Coim{\mathop{\mathrm{Coim}}}

% Boxtimes
%
\DeclareMathSymbol{\boxtimes}{\mathbin}{AMSa}{"02}

%
% Macros for moduli stacks/spaces
%
\def\QCohstack{\mathcal{QC}\!\mathit{oh}}
\def\Cohstack{\mathcal{C}\!\mathit{oh}}
\def\Spacesstack{\mathcal{S}\!\mathit{paces}}
\def\Quotfunctor{\mathrm{Quot}}
\def\Hilbfunctor{\mathrm{Hilb}}
\def\Curvesstack{\mathcal{C}\!\mathit{urves}}
\def\Polarizedstack{\mathcal{P}\!\mathit{olarized}}
\def\Complexesstack{\mathcal{C}\!\mathit{omplexes}}
% \Pic is the operator that assigns to X its picard group, usage \Pic(X)
% \Picardstack_{X/B} denotes the Picard stack of X over B
% \Picardfunctor_{X/B} denotes the Picard functor of X over B
\def\Pic{\mathop{\mathrm{Pic}}\nolimits}
\def\Picardstack{\mathcal{P}\!\mathit{ic}}
\def\Picardfunctor{\mathrm{Pic}}
\def\Deformationcategory{\mathcal{D}\!\mathit{ef}}


% OK, start here.
%
\begin{document}

\title{Properties of Schemes}


\maketitle

\phantomsection
\label{section-phantom}

\tableofcontents

\section{Introduction}
\label{section-introduction}

\noindent
In this chapter we introduce some absolute properties of schemes.
A foundational reference is \cite{EGA}.




\section{Constructible sets}
\label{section-constructible}

\noindent
Constructible and locally constructible sets are introduced in
Topology, Section \ref{topology-section-constructible}.
We may characterize locally constructible subsets of schemes as
follows.

\begin{lemma}
\label{lemma-locally-constructible}
Let $X$ be a scheme.
A subset $E$ of $X$ is locally constructible in $X$ if and only if
$E \cap U$ is constructible in $U$ for every affine open $U$ of $X$.
\end{lemma}

\begin{proof}
Assume $E$ is locally constructible. Then there exists an open covering
$X = \bigcup U_i$ such that $E \cap U_i$ is constructible in $U_i$
for each $i$. Let $V \subset X$ be any affine open. We can find a finite
open affine covering $V = V_1 \cup \ldots \cup V_m$ such that for each $j$
we have $V_j \subset U_i$ for some $i = i(j)$. By
Topology, Lemma \ref{topology-lemma-open-immersion-constructible-inverse-image}
we see that each $E \cap V_j$ is constructible in $V_j$. Since the inclusions
$V_j \to V$ are quasi-compact (see
Schemes, Lemma \ref{schemes-lemma-quasi-compact-affine})
we conclude that $E \cap V$ is constructible in $V$ by
Topology, Lemma \ref{topology-lemma-collate-constructible}.
The converse implication is immediate.
\end{proof}

\begin{lemma}
\label{lemma-generic-point-in-constructible}
Let $X$ be a scheme and let $E \subset X$ be a locally constructible subset.
Let $\xi \in X$ be a generic point of an irreducible component of $X$.
\begin{enumerate}
\item If $\xi \in E$, then an open neighbourhood of
$\xi$ is contained in $E$.
\item If $\xi \not \in E$, then an open neighbourhood
of $\xi$ is disjoint from $E$.
\end{enumerate}
\end{lemma}

\begin{proof}
As the complement of a locally constructible subset is locally
constructible it suffices to show (2). We may assume $X$ is
affine and hence $E$ constructible (Lemma \ref{lemma-locally-constructible}).
In this case $X$ is a spectral space
(Algebra, Lemma \ref{algebra-lemma-spec-spectral}).
Then $\xi \not \in E$ implies $\xi \not \in \overline{E}$ by
Topology, Lemma \ref{topology-lemma-constructible-stable-specialization-closed}
and the fact that there are no points of $X$ different from $\xi$
which specialize to $\xi$.
\end{proof}

\begin{lemma}
\label{lemma-quasi-separated-quasi-compact-open-retrocompact}
Let $X$ be a quasi-separated scheme. The intersection of any two
quasi-compact opens of $X$ is a quasi-compact open of $X$.
Every quasi-compact open of $X$ is retrocompact in $X$.
\end{lemma}

\begin{proof}
If $U$ and $V$ are quasi-compact open then
$U \cap V = \Delta^{-1}(U \times V)$, where $\Delta : X \to X \times X$
is the diagonal. As $X$ is quasi-separated we see that $\Delta$ is
quasi-compact. Hence we see that $U \cap V$ is quasi-compact as
$U \times V$ is quasi-compact (details omitted; use
Schemes, Lemma \ref{schemes-lemma-affine-covering-fibre-product}
to see $U \times V$ is a finite union of affines).
The other assertions follow from the first and
Topology, Lemma \ref{topology-lemma-topology-quasi-separated-scheme}.
\end{proof}

\begin{lemma}
\label{lemma-quasi-compact-quasi-separated-spectral}
Let $X$ be a quasi-compact and quasi-separated scheme.
Then the underlying topological space of $X$ is a spectral space.
\end{lemma}

\begin{proof}
By Topology, Definition \ref{topology-definition-spectral-space}
we have to check that $X$ is sober, quasi-compact, has a basis
of quasi-compact opens, and the intersection of any two
quasi-compact opens is quasi-compact. This follows from
Schemes, Lemma \ref{schemes-lemma-scheme-sober} and
\ref{schemes-lemma-basis-affine-opens}
and
Lemma \ref{lemma-quasi-separated-quasi-compact-open-retrocompact}
above.
\end{proof}

\begin{lemma}
\label{lemma-constructible-quasi-compact-quasi-separated}
Let $X$ be a quasi-compact and quasi-separated scheme.
Any locally constructible subset of $X$ is constructible.
\end{lemma}

\begin{proof}
As $X$ is quasi-compact we can choose a finite affine open covering
$X = V_1 \cup \ldots \cup V_m$. As $X$ is quasi-separated each $V_i$ is
retrocompact in $X$ by
Lemma \ref{lemma-quasi-separated-quasi-compact-open-retrocompact}.
Hence by
Topology, Lemma \ref{topology-lemma-collate-constructible}
we see that $E \subset X$ is constructible in $X$ if and only if
$E \cap V_j$ is constructible in $V_j$. Thus we win by
Lemma \ref{lemma-locally-constructible}.
\end{proof}

\begin{lemma}
\label{lemma-retrocompact}
Let $X$ be a scheme. A subset $E$ of $X$ is retrocompact in $X$ if and only if
$E \cap U$ is quasi-compact for every affine open $U$ of $X$.
\end{lemma}

\begin{proof}
Immediate from the fact that every quasi-compact open of $X$ is a finite
union of affine opens.
\end{proof}

\begin{lemma}
\label{lemma-stratification-locally-finite-constructible}
A partition $X = \coprod_{i \in I} X_i$ of a scheme $X$ with
retrocompact parts is locally finite if and only if the parts
are locally constructible.
\end{lemma}

\begin{proof}
See Topology, Definitions
\ref{topology-definition-quasi-compact},
\ref{topology-definition-paritition}, and
\ref{topology-definition-locally-finite}
for the definitions of retrocompact, partition, and locally finite.

\medskip\noindent
If the partition is locally finite and $U \subset X$ is an
affine open, then we see that $U = \coprod_{i \in I} U \cap X_i$
is a finite partition (more precisely, all but a finite number
of its parts are empty). Hence $U \cap X_i$ is quasi-compact
and its complement is retrocompact in $U$ as a finite union
of retrocompact parts. Thus $U \cap X_i$ is constructible
by Topology, Lemma \ref{topology-lemma-locally-closed-constructible-image}.
It follows that $X_i$ is locally constructible by
Lemma \ref{lemma-locally-constructible}.

\medskip\noindent
Assume the parts are locally constructible. Then for any affine
open $U \subset X$ we obtain a covering $U = \coprod X_i \cap U$
by constructible subsets. Since the constructible topology is
quasi-compact, see
Topology, Lemma \ref{topology-lemma-constructible-hausdorff-quasi-compact},
this covering has a finite refinement, i.e.,
the partition is locally finite.
\end{proof}




\section{Integral, irreducible, and reduced schemes}
\label{section-integral}

\begin{definition}
\label{definition-integral}
Let $X$ be a scheme. We say $X$ is {\it integral} if it is nonempty and
for every nonempty affine open $\Spec(R) = U \subset X$ the ring $R$
is an integral domain.
\end{definition}

\begin{lemma}
\label{lemma-characterize-reduced}
Let $X$ be a scheme.
The following are equivalent.
\begin{enumerate}
\item The scheme $X$ is reduced, see
Schemes, Definition \ref{schemes-definition-reduced}.
\item There exists an affine open covering $X = \bigcup U_i$
such that each $\Gamma(U_i, \mathcal{O}_X)$ is reduced.
\item For every affine open $U \subset X$ the ring
$\mathcal{O}_X(U)$ is reduced.
\item For every open $U \subset X$ the ring $\mathcal{O}_X(U)$ is reduced.
\end{enumerate}
\end{lemma}

\begin{proof}
See Schemes, Lemmas \ref{schemes-lemma-reduced} and
\ref{schemes-lemma-affine-reduced}.
\end{proof}

\begin{lemma}
\label{lemma-characterize-irreducible}
Let $X$ be a scheme.
The following are equivalent.
\begin{enumerate}
\item The scheme $X$ is irreducible.
\item There exists an affine open covering $X = \bigcup_{i \in I} U_i$
such that $I$ is not empty, $U_i$ is irreducible for all $i \in I$, and
$U_i \cap U_j \not = \emptyset$ for all $i, j \in I$.
\item The scheme $X$ is nonempty and every nonempty affine open
$U \subset X$ is irreducible.
\end{enumerate}
\end{lemma}

\begin{proof}
Assume (1). By Schemes, Lemma \ref{schemes-lemma-scheme-sober}
we see that $X$ has a unique generic point $\eta$. Then
$X = \overline{\{\eta\}}$. Hence $\eta$ is an element of
every nonempty affine open $U \subset X$. This implies
that $\eta \in U$ is dense hence $U$ is irreducible.
It also implies any two nonempty affines meet.
Thus (1) implies both (2) and (3).

\medskip\noindent
Assume (2). Suppose $X = Z_1 \cup Z_2$ is a union of two closed subsets.
For every $i$ we see that either $U_i \subset Z_1$ or $U_i \subset Z_2$.
Pick some $i \in I$ and assume $U_i \subset Z_1$ (possibly after renumbering
$Z_1$, $Z_2$). For any $j \in I$ the open subset $U_i \cap U_j$ is dense in
$U_j$ and contained in the closed subset $Z_1 \cap U_j$. We conclude that
also $U_j \subset Z_1$. Thus $X = Z_1$ as desired.

\medskip\noindent
Assume (3). Choose an affine open covering $X = \bigcup_{i \in I} U_i$.
We may assume that each $U_i$ is nonempty.
Since $X$ is nonempty we see that $I$ is not empty.
By assumption each $U_i$ is irreducible.
Suppose $U_i \cap U_j = \emptyset$ for some pair $i, j \in I$.
Then the open $U_i \amalg  U_j = U_i \cup U_j$ is affine, see
Schemes, Lemma \ref{schemes-lemma-disjoint-union-affines}.
Hence it is irreducible by assumption which is absurd. We conclude that (3)
implies (2). The lemma is proved.
\end{proof}

\begin{lemma}
\label{lemma-characterize-integral}
A scheme $X$ is integral if and only if it is reduced and irreducible.
\end{lemma}

\begin{proof}
If $X$ is irreducible, then every affine open $\Spec(R) = U \subset X$
is irreducible. If $X$ is reduced, then $R$ is reduced, by
Lemma \ref{lemma-characterize-reduced} above. Hence $R$ is reduced
and $(0)$ is a prime ideal, i.e., $R$ is an integral domain.

\medskip\noindent
If $X$ is integral, then for every nonempty affine open
$\Spec(R) = U \subset X$ the ring $R$ is reduced
and hence $X$ is reduced by Lemma \ref{lemma-characterize-reduced}.
Moreover, every nonempty affine open is irreducible.
Hence $X$ is irreducible, see Lemma \ref{lemma-characterize-irreducible}.
\end{proof}

\noindent
In Examples, Section
\ref{examples-section-connected-locally-integral-not-integral}
we construct a connected affine scheme all of whose local rings are domains,
but which is not integral.











\section{Types of schemes defined by properties of rings}
\label{section-properties-rings}

\noindent
In this section we study what properties of rings
allow one to define local properties of schemes.

\begin{definition}
\label{definition-property-local}
Let $P$ be a property of rings.
We say that $P$ is {\it local} if the following hold:
\begin{enumerate}
\item For any ring $R$, and any $f \in R$ we have
$P(R) \Rightarrow P(R_f)$.
\item For any ring $R$, and $f_i \in R$ such that
$(f_1, \ldots, f_n) = R$ then
$\forall i, P(R_{f_i}) \Rightarrow P(R)$.
\end{enumerate}
\end{definition}

\begin{definition}
\label{definition-locally-P}
Let $P$ be a property of rings. Let $X$ be a scheme.
We say $X$ is {\it locally $P$} if for any $x \in X$
there exists an affine open neighbourhood $U$ of $x$
in $X$ such that $\mathcal{O}_X(U)$ has property $P$.
\end{definition}

\noindent
This is only a good notion if the property is local.
Even if $P$ is a local property we will not
automatically use this definition to say that a scheme is
``locally $P$'' unless we also explicitly state the definition
elsewhere.

\begin{lemma}
\label{lemma-locally-P}
Let $X$ be a scheme. Let $P$ be a local property of rings.
The following are equivalent:
\begin{enumerate}
\item The scheme $X$ is locally $P$.
\item For every affine open $U \subset X$ the property
$P(\mathcal{O}_X(U))$ holds.
\item There exists an affine open covering $X = \bigcup U_i$ such that
each $\mathcal{O}_X(U_i)$ satisfies $P$.
\item There exists an open covering $X = \bigcup X_j$
such that each open subscheme $X_j$ is locally $P$.
\end{enumerate}
Moreover, if $X$ is locally $P$ then every open subscheme
is locally $P$.
\end{lemma}

\begin{proof}
Of course (1) $\Leftrightarrow$ (3) and (2) $\Rightarrow$ (1).
If (3) $\Rightarrow$ (2), then the final statement of the lemma
holds and it follows easily that (4) is also equivalent to (1).
Thus we show (3) $\Rightarrow$ (2).

\medskip\noindent
Let $X = \bigcup U_i$ be an affine open covering, say
$U_i = \Spec(R_i)$. Assume $P(R_i)$.
Let $\Spec(R) = U \subset X$ be an arbitrary affine open.
By Schemes, Lemma \ref{schemes-lemma-good-subcover}
there exists a standard covering of $U = \Spec(R)$ by
standard opens $D(f_j)$ such that each ring $R_{f_j}$ is a
principal localization of one of the rings $R_i$. By
Definition \ref{definition-property-local} (1) we get $P(R_{f_j})$.
Whereupon $P(R)$ by Definition \ref{definition-property-local} (2).
\end{proof}

\noindent
Here is a sample application.

\begin{lemma}
\label{lemma-reduced-is-locally-reduced}
Let $X$ be a scheme. Then $X$ is reduced if and only if $X$ is
``locally reduced'' in the sense of Definition \ref{definition-locally-P}.
\end{lemma}

\begin{proof}
This is clear from Lemma \ref{lemma-characterize-reduced}.
\end{proof}

\begin{lemma}
\label{lemma-properties-local}
The following properties of a ring $R$ are local.
\begin{enumerate}
\item (Cohen-Macaulay.)
The ring $R$ is Noetherian and CM, see
Algebra, Definition \ref{algebra-definition-ring-CM}.
\item (Regular.)
The ring $R$ is Noetherian and regular, see
Algebra, Definition \ref{algebra-definition-regular}.
\item (Absolutely Noetherian.)
The ring $R$ is of finite type over $Z$.
\item Add more here as needed.\footnote{But we only list those properties
here which we have not already dealt with separately somewhere else.}
\end{enumerate}
\end{lemma}

\begin{proof}
Omitted.
\end{proof}















\section{Noetherian schemes}
\label{section-noetherian}

\noindent
Recall that a ring $R$ is {\it Noetherian} if it satisfies the ascending
chain condition of ideals. Equivalently every ideal of $R$ is finitely
generated.

\begin{definition}
\label{definition-noetherian}
Let $X$ be a scheme.
\begin{enumerate}
\item We say $X$ is {\it locally Noetherian} if every
$x \in X$ has an affine open neighbourhood
$\Spec(R) = U \subset X$ such that the ring $R$ is Noetherian.
\item We say $X$ is {\it Noetherian} if $X$ is locally Noetherian
and quasi-compact.
\end{enumerate}
\end{definition}

\noindent
Here is the standard result characterizing locally Noetherian schemes.

\begin{lemma}
\label{lemma-locally-Noetherian}
Let $X$ be a scheme. The following are equivalent:
\begin{enumerate}
\item The scheme $X$ is locally Noetherian.
\item For every affine open $U \subset X$ the ring $\mathcal{O}_X(U)$
is Noetherian.
\item There exists an affine open covering $X = \bigcup U_i$ such that
each $\mathcal{O}_X(U_i)$ is Noetherian.
\item There exists an open covering $X = \bigcup X_j$
such that each open subscheme $X_j$ is locally Noetherian.
\end{enumerate}
Moreover, if $X$ is locally Noetherian then every open subscheme
is locally Noetherian.
\end{lemma}

\begin{proof}
To show this it suffices to show that being Noetherian is a local
property of rings, see Lemma \ref{lemma-locally-P}.
Any localization of a Noetherian ring is Noetherian, see
Algebra, Lemma \ref{algebra-lemma-Noetherian-permanence}.
By Algebra, Lemma \ref{algebra-lemma-cover} we see the second
property to Definition \ref{definition-property-local}.
\end{proof}

\begin{lemma}
\label{lemma-immersion-into-noetherian}
Any immersion $Z \to X$ with $X$ locally Noetherian is quasi-compact.
\end{lemma}

\begin{proof}
A closed immersion is clearly quasi-compact.
A composition of quasi-compact morphisms is quasi-compact,
see Topology, Lemma \ref{topology-lemma-composition-quasi-compact}.
Hence it suffices to show that an open immersion into
a locally Noetherian scheme is quasi-compact.
Using Schemes, Lemma \ref{schemes-lemma-quasi-compact-affine}
we reduce to the case where $X$ is affine.
Any open subset of the spectrum of a Noetherian ring
is quasi-compact (for example
combine Algebra, Lemma \ref{algebra-lemma-Noetherian-topology} and
Topology, Lemmas \ref{topology-lemma-Noetherian} and
\ref{topology-lemma-Noetherian-quasi-compact}).
\end{proof}

\begin{lemma}
\label{lemma-locally-Noetherian-quasi-separated}
A locally Noetherian scheme is quasi-separated.
\end{lemma}

\begin{proof}
By Schemes, Lemma \ref{schemes-lemma-characterize-quasi-separated}
we have to show that the intersection $U \cap V$ of two
affine opens of $X$ is quasi-compact. This follows from
Lemma \ref{lemma-immersion-into-noetherian} above on
considering the open immersion $U \cap V \to U$ for example.
(But really it is just because any open of the spectrum of a
Noetherian ring is quasi-compact.)
\end{proof}

\begin{lemma}
\label{lemma-Noetherian-topology}
A (locally) Noetherian scheme has a (locally)
Noetherian underlying topological space,
see Topology, Definition \ref{topology-definition-noetherian}.
\end{lemma}

\begin{proof}
This is because a Noetherian scheme is a finite union of spectra
of Noetherian rings and
Algebra, Lemma \ref{algebra-lemma-Noetherian-topology} and
Topology, Lemma \ref{topology-lemma-finite-union-Noetherian}.
\end{proof}

\begin{lemma}
\label{lemma-locally-closed-in-Noetherian}
Any locally closed subscheme of a (locally) Noetherian
scheme is (locally) Noetherian.
\end{lemma}

\begin{proof}
Omitted. Hint: Any quotient, and any localization of a Noetherian
ring is Noetherian. For the Noetherian case use again
that any subset of a Noetherian space is a Noetherian space
(with induced topology).
\end{proof}

\begin{lemma}
\label{lemma-Noetherian-irreducible-components}
A Noetherian scheme has a finite number of irreducible components.
\end{lemma}

\begin{proof}
The underlying topological space of a Noetherian scheme is Noetherian
(Lemma \ref{lemma-Noetherian-topology})
and we conclude because a Noetherian topological space
has only finitely many irreducible components
(Topology, Lemma \ref{topology-lemma-Noetherian}).
\end{proof}

\begin{lemma}
\label{lemma-morphism-Noetherian-schemes-quasi-compact}
Any morphism of schemes $f : X \to Y$ with $X$ Noetherian
is quasi-compact.
\end{lemma}

\begin{proof}
Use Lemma \ref{lemma-Noetherian-topology}
and use that any subset of a Noetherian topological
space is quasi-compact (see Topology,
Lemmas \ref{topology-lemma-Noetherian} and
\ref{topology-lemma-Noetherian-quasi-compact}).
\end{proof}

\noindent
Here is a fun lemma.
It says that every locally Noetherian scheme has plenty of
closed points (at least one in every closed subset).

\begin{lemma}
\label{lemma-locally-Noetherian-closed-point}
Any nonempty locally Noetherian scheme has a closed point.
Any nonempty closed subset of a locally Noetherian scheme has a closed point.
Equivalently, any point of a locally Noetherian scheme specializes
to a closed point.
\end{lemma}

\begin{proof}
The second assertion follows from the first (using
Schemes, Lemma \ref{schemes-lemma-reduced-closed-subscheme}
and Lemma \ref{lemma-locally-closed-in-Noetherian}).
Consider any nonempty affine open $U \subset X$.
Let $x \in U$ be a closed point. If $x$ is a closed point
of $X$ then we are done. If not, let $X_0 \subset X$ be the
reduced induced closed subscheme structure on $\overline{\{x\}}$.
Then $U_0 = U \cap X_0$ is an affine open of $X_0$ by
Schemes, Lemma \ref{schemes-lemma-closed-subspace-scheme} and
$U_0 = \{x\}$. Let $y \in X_0$, $y \not = x$ be a specialization of $x$.
Consider the local ring $R = \mathcal{O}_{X_0, y}$.
This is a Noetherian local ring as $X_0$ is Noetherian
by Lemma \ref{lemma-locally-closed-in-Noetherian}. Denote $V \subset \Spec(R)$
the inverse image of $U_0$ in $\Spec(R)$ by the canonical morphism
$\Spec(R) \to X_0$ (see Schemes, Section \ref{schemes-section-points}.)
By construction $V$ is a singleton with unique point corresponding to $x$ (use
Schemes, Lemma \ref{schemes-lemma-specialize-points}).
By
Algebra, Lemma \ref{algebra-lemma-Noetherian-local-domain-dim-2-infinite-opens}
we see that $\dim(R) = 1$.
In other words, we see that $y$ is an immediate specialization
of $x$ (see Topology, Definition \ref{topology-definition-dimension-function}).
In other words, any
point $y \not = x$ such that $x \leadsto y$ is an immediate
specialization of $x$. Clearly each of these points is a
closed point as desired.
\end{proof}

\begin{lemma}
\label{lemma-locally-Noetherian-specialization-dvr}
Let $X$ be a locally Noetherian scheme.
Let $x' \leadsto x$ be a specialization of points of $X$.
Then
\begin{enumerate}
\item there exists a discrete valuation ring $R$ and a morphism
$f : \Spec(R) \to X$ such that the generic point $\eta$ of
$\Spec(R)$ maps to $x'$ and the special point maps to $x$, and
\item given a finitely generated field extension $K/\kappa(x')$
we may arrange it so that the extension $\kappa(\eta)/\kappa(x')$
induced by $f$ is isomorphic to the given one.
\end{enumerate}
\end{lemma}

\begin{proof}
Let $x' \leadsto x$ be a specialization in $X$, and let
$K/\kappa(x')$ be a finitely generated extension of fields. By
Schemes, Lemma \ref{schemes-lemma-specialize-points}
and the discussion following
Schemes, Lemma \ref{schemes-lemma-characterize-points}
this leads to ring maps $\mathcal{O}_{X, x} \to \kappa(x') \to K$.
Let $R \subset K$ be any discrete valuation ring whose field of fractions is
$K$ and which dominates the image of $\mathcal{O}_{X, x} \to K$, see
Algebra, Lemma \ref{algebra-lemma-exists-dvr}.
The ring map $\mathcal{O}_{X, x} \to R$ induces the morphism
$f : \Spec(R) \to X$, see
Schemes, Lemma \ref{schemes-lemma-morphism-from-spec-local-ring}.
This morphism has all the desired properties by construction.
\end{proof}

\begin{lemma}
\label{lemma-thin-infinite-sequence}
Let $S$ be a Noetherian scheme. Let $T \subset S$ be an infinite subset.
Then there exists an infinite subset $T' \subset T$
such that there are no nontrivial specializations among the points $T'$.
\end{lemma}

\begin{proof}
Let $T_0 \subset T$ be the set of $t \in T$ which do not specialize
to another point of $T$. If $T_0$ is infinite, then $T' = T_0$ works.
Hence we may and do assume $T_0$ is finite.
Inductively, for $i > 0$, consider the set $T_i \subset T$
of $t \in T$ such that
\begin{enumerate}
\item $t \not \in T_{i - 1} \cup T_{i - 2} \cup \ldots \cup T_0$,
\item there exist a nontrivial specialization $t \leadsto t'$ with
$t' \in T_{i - 1}$, and
\item for any nontrivial specialization
$t \leadsto t'$ with $t' \in T$ we have
$t' \in T_{i - 1} \cup T_{i - 2} \cup \ldots \cup T_0$.
\end{enumerate}
Again, if $T_i$ is infinite, then $T' = T_i$ works.
Let $d$ be the maximum of the dimensions of the local rings
$\mathcal{O}_{S, t}$ for $t \in T_0$; then $d$ is an integer
because $T_0$ is finite and the dimensions of the local rings
are finite by Algebra, Proposition \ref{algebra-proposition-dimension}.
Then $T_i = \emptyset$ for $i > d$.
Namely, if $t \in T_i$ then we can find a sequence
of nontrivial specializations
$t = t_i \leadsto t_{i - 1} \leadsto \ldots \leadsto t_0$
with $t_0 \in T_0$. As
the points $t = t_i, t_{i - 1}, \ldots, t_0$ are in
$\Spec(\mathcal{O}_{S, t_0})$
(Schemes, Lemma \ref{schemes-lemma-specialize-points}),
we see that $i \leq d$.
Thus $\bigcup T_i = T_d \cup \ldots \cup T_0$ is a finite subset of $T$.

\medskip\noindent
Suppose $t \in T$ is not in $\bigcup T_i$. Then there must be a specialization
$t \leadsto t'$ with $t' \in T$ and $t' \not \in \bigcup T_i$. (Namely, if
every specialization of $t$ is in the finite set $T_d \cup \ldots \cup T_0$,
then there is a maximum $i$ such that there is some specialization
$t \leadsto t'$ with $t' \in T_i$ and then $t \in T_{i + 1}$ by construction.)
Hence we get an infinite sequence
$$
t \leadsto t' \leadsto t'' \leadsto \ldots
$$
of nontrivial specializations between points of $T \setminus \bigcup T_i$.
This is impossible because the underlying topological space of $S$
is Noetherian by Lemma \ref{lemma-locally-Noetherian-quasi-separated}.
\end{proof}

\begin{lemma}
\label{lemma-maximal-points}
Let $S$ be a Noetherian scheme. Let $T \subset S$ be a subset. Let
$T_0 \subset T$ be the set of $t \in T$ such that there is no nontrivial
specialization $t' \leadsto t$ with $t' \in T'$. Then (a) there are
no specializations among the points of $T_0$, (b) every point of
$T$ is a specialization of a point of $T_0$, and (c) the closures
of $T$ and $T_0$ are the same.
\end{lemma}

\begin{proof}
Recall that $\dim(\mathcal{O}_{S, s}) < \infty$ for any $s \in S$, see
Algebra, Proposition \ref{algebra-proposition-dimension}. Let $t \in T$.
If $t' \leadsto t$, then by dimension theory
$\dim(\mathcal{O}_{S, t'}) \leq \dim(\mathcal{O}_{S, t})$
with equality if and only if $t' = t$. Thus if we pick $t' \leadsto t$
with $\dim(\mathcal{O}_{T, t'})$ minimal, then $t' \in T_0$.
In other words, 
every $t \in T$ is the specialization of an element of $T_0$.
\end{proof}

\begin{lemma}
\label{lemma-countable-dense-subset}
Let $S$ be a Noetherian scheme. Let $T \subset S$ be an infinite dense subset.
Then there exist a countable subset $E \subset T$ which is dense in $S$.
\end{lemma}

\begin{proof}
Let $T'$ be the set of points $s \in S$ such that $\overline{\{s\}} \cap T$
contains a countable subset whose closure is $\overline{\{s\}}$.
Since a finite set is countable we have $T \subset T'$.
For $s \in T'$ choose such a countable subset
$E_s \subset \overline{\{s\}} \cap T$.
Let $E' = \{s_1, s_2, s_3, \ldots\} \subset T'$
be a countable subset. Then the closure of $E'$ in $S$ is the
closure of the countable subset $\bigcup_n E_{s_n}$ of $T$.
It follows that if $Z$
is an irreducible component of the closure of $E'$, then the generic
point of $Z$ is in $T'$.

\medskip\noindent
Denote $T'_0 \subset T'$ the subset of $t \in T'$ such that
there is no nontrivial specialization $t' \leadsto t$ with $t' \in T'$
as in Lemma \ref{lemma-maximal-points} whose results we will use
without further mention. If $T'_0$ is infinite, then we choose a
countable subset $E' \subset T'_0$. By the argument in the first
paragraph, the generic points of the irreducible components of the
closure of $E'$ are in $T'$. However, since one of these points specializes to
infinitely many distinct elements of $E' \subset T'_0$
this is a contradiction. Thus $T'_0$ is finite, say
$T'_0 = \{s_1, \ldots, s_m\}$. Then it follows that $S$, which is
the closure of $T$, is contained in the closure of
$\{s_1, \ldots, s_m\}$, which in turn is contained in the closure
of the countable subset $E_{s_1} \cup \ldots \cup E_{s_m} \subset T$
as desired.
\end{proof}








\section{Jacobson schemes}
\label{section-jacobson}

\noindent
Recall that a space is said to be {\it Jacobson} if the closed points are
dense in every closed subset, see
Topology, Section \ref{topology-section-space-jacobson}.

\begin{definition}
\label{definition-jacobson}
A scheme $S$ is said to be {\it Jacobson} if its underlying topological
space is Jacobson.
\end{definition}

\noindent
Recall that a ring $R$ is Jacobson if every radical ideal of $R$
is the intersection of maximal ideals, see
Algebra, Definition \ref{algebra-definition-ring-jacobson}.

\begin{lemma}
\label{lemma-affine-jacobson}
An affine scheme $\Spec(R)$ is Jacobson if and only if
the ring $R$ is Jacobson.
\end{lemma}

\begin{proof}
This is Algebra, Lemma \ref{algebra-lemma-jacobson}.
\end{proof}

\noindent
Here is the standard result characterizing Jacobson schemes.
Intuitively it claims that Jacobson $\Leftrightarrow$ locally Jacobson.

\begin{lemma}
\label{lemma-locally-jacobson}
Let $X$ be a scheme. The following are equivalent:
\begin{enumerate}
\item The scheme $X$ is Jacobson.
\item The scheme $X$ is ``locally Jacobson'' in the sense of
Definition \ref{definition-locally-P}.
\item For every affine open $U \subset X$ the ring $\mathcal{O}_X(U)$
is Jacobson.
\item There exists an affine open covering $X = \bigcup U_i$ such that
each $\mathcal{O}_X(U_i)$ is Jacobson.
\item There exists an open covering $X = \bigcup X_j$
such that each open subscheme $X_j$ is Jacobson.
\end{enumerate}
Moreover, if $X$ is Jacobson then every open subscheme
is Jacobson.
\end{lemma}

\begin{proof}
The final assertion of the lemma holds by
Topology, Lemma \ref{topology-lemma-jacobson-inherited}.
The equivalence of (5) and (1) is
Topology, Lemma \ref{topology-lemma-jacobson-local}.
Hence, using Lemma \ref{lemma-affine-jacobson},
we see that (1) $\Leftrightarrow$ (2).
To finish proving the lemma it suffices to show that ``Jacobson''
is a local property of rings, see Lemma \ref{lemma-locally-P}.
Any localization of a Jacobson ring at an element is Jacobson, see
Algebra, Lemma \ref{algebra-lemma-Jacobson-invert-element}.
Suppose $R$ is a ring, $f_1, \ldots, f_n \in R$ generate the unit
ideal and each $R_{f_i}$ is Jacobson. Then we see that
$\Spec(R) = \bigcup D(f_i)$ is a union of open subsets
which are all Jacobson, and hence $\Spec(R)$ is Jacobson
by Topology, Lemma \ref{topology-lemma-jacobson-local} again.
This proves the second property of Definition \ref{definition-property-local}.
\end{proof}

\noindent
Many schemes used commonly in algebraic geometry are Jacobson, see
Morphisms, Lemma \ref{morphisms-lemma-ubiquity-Jacobson-schemes}.
We mention here the following interesting case.

\begin{lemma}
\label{lemma-complement-closed-point-Jacobson}
Examples of Noetherian Jacobson schemes.
\begin{enumerate}
\item If $(R, \mathfrak m)$ is a Noetherian local ring, then
the punctured spectrum $\Spec(R) \setminus \{\mathfrak m\}$
is a Jacobson scheme.
\item If $R$ is a Noetherian ring with Jacobson radical $\text{rad}(R)$
then $\Spec(R) \setminus V(\text{rad}(R))$ is a Jacobson scheme.
\item If $(R, I)$ is a Zariski pair (More on Algebra, Definition
\ref{more-algebra-definition-zariski-pair})
with $R$ Noetherian, then $\Spec(R) \setminus V(I)$ is a
Jacobson scheme.
\end{enumerate}
\end{lemma}

\begin{proof}
Proof of (3). Observe that $\Spec(R) - V(I)$ has a covering by the affine
opens $\Spec(R_f)$ for $f \in I$. The rings $R_f$ are Jacobson by
More on Algebra, Lemma
\ref{more-algebra-lemma-noetherian-zariski-jacobson-complement}.
Hence $\Spec(R) \setminus V(I)$ is Jacobson by
Lemma \ref{lemma-locally-jacobson}.
Parts (1) and (2) are special cases of (3).

\medskip\noindent
Direct proof of case (1).
Since $\Spec(R)$ is a Noetherian scheme,
$S$ is a Noetherian scheme (Lemma \ref{lemma-locally-closed-in-Noetherian}).
Hence $S$ is a sober, Noetherian topological space (use
Schemes, Lemma \ref{schemes-lemma-scheme-sober}).
Assume $S$ is not Jacobson to
get a contradiction. By
Topology, Lemma \ref{topology-lemma-non-jacobson-Noetherian-characterize}
there exists some non-closed point $\xi \in S$
such that $\{\xi\}$ is locally closed. This corresponds
to a prime $\mathfrak p \subset R$ such that (1) there exists
a prime $\mathfrak q$, $\mathfrak p \subset \mathfrak q \subset \mathfrak m$
with both inclusions strict, and (2) $\{\mathfrak p\}$ is open in
$\Spec(R/\mathfrak p)$. This is impossible by Algebra,
Lemma \ref{algebra-lemma-Noetherian-local-domain-dim-2-infinite-opens}.
\end{proof}







\section{Normal schemes}
\label{section-normal}

\noindent
Recall that a ring $R$ is said to be normal if all its local rings
are normal domains,
see Algebra, Definition \ref{algebra-definition-ring-normal}.
A normal domain is a domain which is integrally closed in its field
of fractions, see
Algebra, Definition \ref{algebra-definition-domain-normal}.
Thus it makes sense to define a normal scheme as follows.

\begin{definition}
\label{definition-normal}
A scheme $X$ is {\it normal} if and only if for all $x \in X$ the local ring
$\mathcal{O}_{X, x}$ is a normal domain.
\end{definition}

\noindent
This seems to be the definition used in EGA, see \cite[0, 4.1.4]{EGA}.
Suppose $X = \Spec(A)$, and $A$ is reduced. Then saying that $X$ is
normal is not equivalent to saying that $A$ is integrally closed in its
total ring of fractions. However, if $A$ is Noetherian then this is the case
(see Algebra, Lemma \ref{algebra-lemma-characterize-reduced-ring-normal}).

\begin{lemma}
\label{lemma-locally-normal}
Let $X$ be a scheme. The following are equivalent:
\begin{enumerate}
\item The scheme $X$ is normal.
\item For every affine open $U \subset X$ the ring $\mathcal{O}_X(U)$
is normal.
\item There exists an affine open covering $X = \bigcup U_i$ such that
each $\mathcal{O}_X(U_i)$ is normal.
\item There exists an open covering $X = \bigcup X_j$
such that each open subscheme $X_j$ is normal.
\end{enumerate}
Moreover, if $X$ is normal then every open subscheme
is normal.
\end{lemma}

\begin{proof}
This is clear from the definitions.
\end{proof}

\begin{lemma}
\label{lemma-normal-reduced}
A normal scheme is reduced.
\end{lemma}

\begin{proof}
Immediate from the definitions.
\end{proof}

\begin{lemma}
\label{lemma-integral-normal}
Let $X$ be an integral scheme.
Then $X$ is normal if and only if for every nonempty affine open
$U \subset X$ the ring $\mathcal{O}_X(U)$ is a normal domain.
\end{lemma}

\begin{proof}
This follows from
Algebra, Lemma \ref{algebra-lemma-normality-is-local}.
\end{proof}

\begin{lemma}
\label{lemma-normal-locally-finite-nr-irreducibles}
Let $X$ be a scheme such that any quasi-compact open has a finite number
of irreducible components. The following are equivalent:
\begin{enumerate}
\item $X$ is normal, and
\item $X$ is a disjoint union of normal integral schemes.
\end{enumerate}
\end{lemma}

\begin{proof}
It is immediate from the definitions that (2) implies (1).
Let $X$ be a normal scheme such that every quasi-compact open
has a finite number of irreducible components.
If $X$ is affine then $X$ satisfies (2) by
Algebra, Lemma \ref{algebra-lemma-characterize-reduced-ring-normal}.
For a general $X$, let $X = \bigcup X_i$ be
an affine open covering. Note that also each $X_i$ has
but a finite number of irreducible components, and the lemma holds
for each $X_i$. Let $T \subset X$ be an irreducible component.
By the affine case each intersection $T \cap X_i$ is open in $X_i$
and an integral normal scheme.
Hence $T \subset X$ is open, and an integral normal scheme.
This proves that $X$ is the disjoint union of its irreducible components,
which are integral normal schemes.
\end{proof}

\begin{lemma}
\label{lemma-normal-Noetherian}
Let $X$ be a Noetherian scheme.
The following are equivalent:
\begin{enumerate}
\item $X$ is normal, and
\item $X$ is a finite disjoint union of normal integral schemes.
\end{enumerate}
\end{lemma}

\begin{proof}
This is a special case of
Lemma \ref{lemma-normal-locally-finite-nr-irreducibles} because a Noetherian
scheme has a Noetherian underlying topological space
(Lemma \ref{lemma-Noetherian-topology}
and
Topology, Lemma \ref{topology-lemma-Noetherian}).
\end{proof}

\begin{lemma}
\label{lemma-normal-locally-Noetherian}
Let $X$ be a locally Noetherian scheme.
The following are equivalent:
\begin{enumerate}
\item $X$ is normal, and
\item $X$ is a disjoint union of integral normal schemes.
\end{enumerate}
\end{lemma}

\begin{proof}
Omitted. Hint: This is purely topological from
Lemma \ref{lemma-normal-Noetherian}.
\end{proof}

\begin{remark}
\label{remark-normal-connected-irreducible}
Let $X$ be a normal scheme. If $X$ is locally Noetherian then we see that
$X$ is integral if and only if $X$ is connected, see
Lemma \ref{lemma-normal-locally-Noetherian}.
But there exists a connected affine scheme $X$ such that
$\mathcal{O}_{X, x}$ is a domain for all $x \in X$, but $X$ is not
irreducible, see Examples, Section
\ref{examples-section-connected-locally-integral-not-integral}.
This example is even a normal scheme (proof omitted), so beware!
\end{remark}

\begin{lemma}
\label{lemma-normal-integral-sections}
\begin{slogan}
The ring of functions on a normal scheme is normal.
\end{slogan}
Let $X$ be an integral normal scheme.
Then $\Gamma(X, \mathcal{O}_X)$ is a normal domain.
\end{lemma}

\begin{proof}
Set $R = \Gamma(X, \mathcal{O}_X)$.
It is clear that $R$ is a domain.
Suppose $f = a/b$ is an element of its fraction field
which is integral over $R$. Say we have
$f^d + \sum_{i = 0, \ldots, d - 1} a_i f^i = 0$ with
$a_i \in R$. Let $U \subset X$ be a nonempty affine open.
Since $b \in R$ is not zero and since $X$ is integral we see
that also $b|_U \in \mathcal{O}_X(U)$ is not zero.
Hence $a/b$ is an element of the fraction field of
$\mathcal{O}_X(U)$ which is integral over $\mathcal{O}_X(U)$
(because we can use the same polynomial
$f^d + \sum_{i = 0, \ldots, d - 1} a_i|_U f^i = 0$ on $U$).
Since $\mathcal{O}_X(U)$ is a normal domain
(Lemma \ref{lemma-integral-normal}), we see that
$f_U = (a|_U)/(b|_U) \in \mathcal{O}_X(U)$. It is clear
that $f_U|_V = f_V$ whenever $V \subset U \subset X$ are
nonempty affine open. Hence the local sections $f_U$ glue to an element
$g \in R = \Gamma(X, \mathcal{O}_X)$. Then $bg$ and $a$
restrict to the same element of $\mathcal{O}_X(U)$ for
all $U$ as above, hence $bg = a$, in other words, $g$ maps to $f$
in the fraction field of $R$.
\end{proof}










\section{Cohen-Macaulay schemes}
\label{section-Cohen-Macaulay}

\noindent
Recall, see Algebra, Definition \ref{algebra-definition-local-ring-CM},
that a local Noetherian ring $(R, \mathfrak m)$ is
said to be Cohen-Macaulay if $\text{depth}_{\mathfrak m}(R) = \dim(R)$.
Recall that a Noetherian ring $R$ is said to be Cohen-Macaulay if
every local ring $R_{\mathfrak p}$ of $R$ is Cohen-Macaulay,
see Algebra, Definition \ref{algebra-definition-ring-CM}.

\begin{definition}
\label{definition-Cohen-Macaulay}
Let $X$ be a scheme. We say $X$ is {\it Cohen-Macaulay} if
for every $x \in X$ there exists an affine open neighbourhood
$U \subset X$ of $x$ such that the ring $\mathcal{O}_X(U)$ is
Noetherian and Cohen-Macaulay.
\end{definition}

\begin{lemma}
\label{lemma-characterize-Cohen-Macaulay}
Let $X$ be a scheme. The following are equivalent:
\begin{enumerate}
\item $X$ is Cohen-Macaulay,
\item $X$ is locally Noetherian and all of its local rings are Cohen-Macaulay,
and
\item $X$ is locally Noetherian and for any closed point $x \in X$
the local ring $\mathcal{O}_{X, x}$ is Cohen-Macaulay.
\end{enumerate}
\end{lemma}

\begin{proof}
Algebra, Lemma \ref{algebra-lemma-localize-CM} says that the localization of
a Cohen-Macaulay local ring is Cohen-Macaulay. The lemma follows
by combining this with Lemma \ref{lemma-locally-Noetherian},
with the existence of closed
points on locally Noetherian schemes
(Lemma \ref{lemma-locally-Noetherian-closed-point}), and
the definitions.
\end{proof}

\begin{lemma}
\label{lemma-locally-Cohen-Macaulay}
Let $X$ be a scheme. The following are equivalent:
\begin{enumerate}
\item The scheme $X$ is Cohen-Macaulay.
\item For every affine open $U \subset X$ the ring $\mathcal{O}_X(U)$
is Noetherian and Cohen-Macaulay.
\item There exists an affine open covering $X = \bigcup U_i$ such that
each $\mathcal{O}_X(U_i)$ is Noetherian and Cohen-Macaulay.
\item There exists an open covering $X = \bigcup X_j$
such that each open subscheme $X_j$ is Cohen-Macaulay.
\end{enumerate}
Moreover, if $X$ is Cohen-Macaulay then every open subscheme
is Cohen-Macaulay.
\end{lemma}

\begin{proof}
Combine Lemmas \ref{lemma-locally-Noetherian}
and \ref{lemma-characterize-Cohen-Macaulay}.
\end{proof}

\noindent
More information on Cohen-Macaulay schemes and depth can be found in
Cohomology of Schemes, Section \ref{coherent-section-depth}.








\section{Regular schemes}
\label{section-regular}

\noindent
Recall, see Algebra, Definition \ref{algebra-definition-regular-local},
that a local Noetherian ring $(R, \mathfrak m)$ is
said to be {\it regular} if $\mathfrak m$ can be generated
by $\dim(R)$ elements.
Recall that a Noetherian ring $R$ is said to be {\it regular} if
every local ring $R_{\mathfrak p}$ of $R$ is regular,
see Algebra, Definition \ref{algebra-definition-regular}.

\begin{definition}
\label{definition-regular}
Let $X$ be a scheme. We say $X$ is {\it regular}, or {\it nonsingular} if
for every $x \in X$ there exists an affine open neighbourhood
$U \subset X$ of $x$ such that the ring $\mathcal{O}_X(U)$ is
Noetherian and regular.
\end{definition}

\begin{lemma}
\label{lemma-characterize-regular}
Let $X$ be a scheme. The following are equivalent:
\begin{enumerate}
\item $X$ is regular,
\item $X$ is locally Noetherian and all of its local rings are regular,
and
\item $X$ is locally Noetherian and for any closed point $x \in X$
the local ring $\mathcal{O}_{X, x}$ is regular.
\end{enumerate}
\end{lemma}

\begin{proof}
By the discussion in Algebra preceding Algebra, Definition
\ref{algebra-definition-regular} we know that the localization of
a regular local ring is regular. The lemma follows
by combining this with Lemma \ref{lemma-locally-Noetherian},
with the existence of closed
points on locally Noetherian schemes
(Lemma \ref{lemma-locally-Noetherian-closed-point}), and
the definitions.
\end{proof}

\begin{lemma}
\label{lemma-locally-regular}
Let $X$ be a scheme. The following are equivalent:
\begin{enumerate}
\item The scheme $X$ is regular.
\item For every affine open $U \subset X$ the ring $\mathcal{O}_X(U)$
is Noetherian and regular.
\item There exists an affine open covering $X = \bigcup U_i$ such that
each $\mathcal{O}_X(U_i)$ is Noetherian and regular.
\item There exists an open covering $X = \bigcup X_j$
such that each open subscheme $X_j$ is regular.
\end{enumerate}
Moreover, if $X$ is regular then every open subscheme is regular.
\end{lemma}

\begin{proof}
Combine Lemmas \ref{lemma-locally-Noetherian}
and \ref{lemma-characterize-regular}.
\end{proof}

\begin{lemma}
\label{lemma-regular-normal}
A regular scheme is normal.
\end{lemma}

\begin{proof}
See
Algebra, Lemma \ref{algebra-lemma-regular-normal}.
\end{proof}




\section{Dimension}
\label{section-dimension}

\noindent
The dimension of a scheme is just the dimension of its underlying
topological space.

\begin{definition}
\label{definition-dimension}
Let $X$ be a scheme.
\begin{enumerate}
\item The {\it dimension} of $X$ is just the dimension of $X$
as a topological spaces, see
Topology, Definition \ref{topology-definition-Krull}.
\item For $x \in X$ we denote $\dim_x(X)$ the dimension of the underlying
topological space of $X$ at $x$ as in
Topology, Definition \ref{topology-definition-Krull}.
We say $\dim_x(X)$ is the {\it dimension of $X$ at $x$}.
\end{enumerate}
\end{definition}

\noindent
As a scheme has a sober underlying topological space
(Schemes, Lemma \ref{schemes-lemma-scheme-sober})
we may compute the dimension of $X$ as the supremum of the lengths $n$
of chains
$$
T_0 \subset T_1 \subset \ldots \subset T_n
$$
of irreducible closed subsets of $X$, or as the supremum of the lengths $n$
of chains of specializations
$$
\xi_n \leadsto \xi_{n - 1} \leadsto \ldots \leadsto \xi_0
$$
of points of $X$.

\begin{lemma}
\label{lemma-dimension}
Let $X$ be a scheme. The following are equal
\begin{enumerate}
\item The dimension of $X$.
\item The supremum of the dimensions of the local rings of $X$.
\item The supremum of $\dim_x(X)$ for $x \in X$.
\end{enumerate}
\end{lemma}

\begin{proof}
Note that given a chain of specializations
$$
\xi_n \leadsto \xi_{n - 1} \leadsto \ldots \leadsto \xi_0
$$
of points of $X$ all of the points $\xi_i$ correspond to prime ideals
of the local ring of $X$ at $\xi_0$ by
Schemes, Lemma \ref{schemes-lemma-specialize-points}.
Hence we see that the dimension of $X$ is the supremum of the dimensions
of its local rings. In particular $\dim_x(X) \geq \dim(\mathcal{O}_{X, x})$
as $\dim_x(X)$ is the minimum of the dimensions of open neighbourhoods of
$x$. Thus $\sup_{x \in X} \dim_x(X) \geq \dim(X)$. On the other hand,
it is clear that $\sup_{x \in X} \dim_x(X) \leq \dim(X)$
as $\dim(U) \leq \dim(X)$ for any open subset of $X$.
\end{proof}

\begin{lemma}
\label{lemma-codimension-local-ring}
Let $X$ be a scheme. Let $Y \subset X$ be an irreducible closed
subset. Let $\xi \in Y$ be the generic point. Then
$$
\text{codim}(Y, X) = \dim(\mathcal{O}_{X, \xi})
$$
where the codimension is as defined in
Topology, Definition \ref{topology-definition-codimension}.
\end{lemma}

\begin{proof}
By Topology, Lemma \ref{topology-lemma-codimension-at-generic-point}
we may replace $X$ by an affine open neighbourhood of $\xi$. In this
case the result follows easily from
Algebra, Lemma \ref{algebra-lemma-irreducible-components-containing-x}.
\end{proof}

\begin{lemma}
\label{lemma-generic-point}
Let $X$ be a scheme. Let $x \in X$. Then $x$ is a generic point of
an irreducible component of $X$ if and only if $\dim(\mathcal{O}_{X, x}) = 0$.
\end{lemma}

\begin{proof}
This follows from Lemma \ref{lemma-codimension-local-ring} for example.
\end{proof}

\begin{lemma}
\label{lemma-locally-Noetherian-dimension-0}
A locally Noetherian scheme of dimension $0$ is a disjoint
union of spectra of Artinian local rings.
\end{lemma}

\begin{proof}
A Noetherian ring of dimension $0$ is a finite product of Artinian local
rings, see
Algebra, Proposition \ref{algebra-proposition-dimension-zero-ring}.
Hence an affine open of a locally Noetherian scheme $X$ of dimension
$0$ has discrete underlying topological space. This implies that
the topology on $X$ is discrete. The lemma follows easily from these
remarks.
\end{proof}

\begin{lemma}
\label{lemma-dimension-zero}
\begin{reference}
Email from Ofer Gabber dated June 4, 2016
\end{reference}
Let $X$ be a scheme of dimension zero. The following are equivalent
\begin{enumerate}
\item $X$ is quasi-separated,
\item $X$ is separated,
\item $X$ is Hausdorff,
\item every affine open is closed.
\end{enumerate}
In this case the connected components of $X$ are points and every
quasi-compact open of $X$ is affine. In particular, if $X$
is quasi-compact, then $X$ is affine.
\end{lemma}

\begin{proof}
As the dimension of $X$ is zero, we see that for any affine open
$U \subset X$ the space $U$ is profinite and satisfies
a bunch of other properties which we will use freely below, see
Algebra, Lemma \ref{algebra-lemma-ring-with-only-minimal-primes}.
We choose an affine open covering $X = \bigcup U_i$.

\medskip\noindent
If (4) holds, then $U_i \cap U_j$ is a closed subset of
$U_i$, hence quasi-compact, hence $X$ is quasi-separated,
by Schemes, Lemma \ref{schemes-lemma-characterize-quasi-separated},
hence (1) holds.

\medskip\noindent
If (1) holds, then $U_i \cap U_j$ is a quasi-compact open
of $U_i$ hence closed in $U_i$. Then $U_i \cap U_j \to U_i$
is an open immersion whose image is closed, hence it is a
closed immersion. In particular $U_i \cap U_j$ is affine
and $\mathcal{O}(U_i) \to \mathcal{O}_X(U_i \cap U_j)$ is surjective.
Thus $X$ is separated
by Schemes, Lemma \ref{schemes-lemma-characterize-separated},
hence (2) holds.

\medskip\noindent
Assume (2) and let $x, y \in X$. Say $x \in U_i$. If $y \in U_i$
too, then we can find disjoint open neighbourhoods of $x$ and $y$
because $U_i$ is Hausdorff. Say $y \not \in U_i$ and $y \in U_j$.
Then $y \not \in U_i \cap U_j$ which is an affine open of $U_j$
and hence closed in $U_j$. Thus we can find an open neighbourhood
of $y$ not meeting $U_i$ and we conclude that $X$ is Hausdorff,
hence (3) holds.

\medskip\noindent
Assume (3). Let $U \subset X$ be affine open.
Then $U$ is closed in $X$ by Topology, Lemma
\ref{topology-lemma-quasi-compact-in-Hausdorff}.
This proves (4) holds.

\medskip\noindent
Assume $X$ satisfies the equivalent conditions (1) -- (4).
We prove the final statements of the lemma. Say $x, y \in X$
with $x \not = y$. Since $y$ does not specialize to $x$ we can
choose $U \subset X$ affine open with $x \in U$ and $y \not \in U$.
Then we see that $X = U \amalg (X \setminus U)$ is a decomposistion
into open and closed subsets which shows that $x$ and $y$ do not belong
to the same connected component of $X$. Next, assume $U \subset X$
is a quasi-compact open. Write $U = U_1 \cup \ldots \cup U_n$ as a union
of affine opens. We will prove by induction on $n$ that $U$ is affine.
This immediately reduces us to the case $n = 2$. In this case we have
$U = (U_1 \setminus U_2) \amalg (U_1 \cap U_2) \amalg (U_2 \setminus U_1)$
and the arguments above show that each of the pieces is affine.
\end{proof}

\begin{lemma}
\label{lemma-isolated-dim-0-locally-noetherian}
Let $x$ be a point of a locally Noetherian scheme $X$.
Then $\dim_x(X) = 0$ if and only if $x$ is an isolated point of $X$.
\end{lemma}

\begin{proof}
If $x$ is an isolated point, then $\{x\}$ is an open subset of $X$
(Topology, Definition \ref{topology-definition-isolated-point})
and hence $\dim_x(X) = \dim(\{x\}) = 0$ by definition. Conversely, if
$\dim_x(X) = 0$, then there exists an open neighbourhood $U \subset X$
of $x$ such that $\dim(U) = 0$
(Topology, Definition \ref{topology-definition-Krull}).
By Lemma \ref{lemma-locally-Noetherian-dimension-0}
we see that the topology on $U$ is discrete and hence
$x$ is an isolated point.
\end{proof}





\section{Catenary schemes}
\label{section-catenary}

\noindent
Recall that a topological space $X$ is called {\it catenary} if
for every pair of irreducible closed subsets $T \subset T'$
there exist a maximal chain of irreducible closed subsets
$$
T = T_0 \subset T_1 \subset \ldots \subset T_e = T'
$$
and every such chain has the same length. See
Topology, Definition \ref{topology-definition-catenary}.

\begin{definition}
\label{definition-catenary}
Let $S$ be a scheme. We say $S$ is {\it catenary} if the
underlying topological space of $S$ is catenary.
\end{definition}

\noindent
Recall that a ring $A$ is called {\it catenary} if
for any pair of prime ideals $\mathfrak p \subset \mathfrak q$
there exists a maximal chain of primes
$$
\mathfrak p =
\mathfrak p_0 \subset \ldots \subset \mathfrak p_e
= \mathfrak q
$$
and all of these have the same length. See
Algebra, Definition \ref{algebra-definition-catenary}.

\begin{lemma}
\label{lemma-catenary-local}
Let $S$ be a scheme. The following are equivalent
\begin{enumerate}
\item $S$ is catenary,
\item there exists an open covering of $S$ all of whose members are
catenary schemes,
\item for every affine open $\Spec(R) = U \subset S$ the ring
$R$ is catenary, and
\item there exists an affine open covering $S = \bigcup U_i$ such
that each $U_i$ is the spectrum of a catenary ring.
\end{enumerate}
Moreover, in this case any locally closed subscheme of $S$ is catenary
as well.
\end{lemma}

\begin{proof}
Combine Topology, Lemma \ref{topology-lemma-catenary}, and
Algebra, Lemma \ref{algebra-lemma-catenary}.
\end{proof}

\begin{lemma}
\label{lemma-catenary-dimension-function}
Let $S$ be a locally Noetherian scheme.
The following are equivalent:
\begin{enumerate}
\item $S$ is catenary, and
\item locally in the Zariski topology there exists a dimension function
on $S$ (see Topology, Definition \ref{topology-definition-dimension-function}).
\end{enumerate}
\end{lemma}

\begin{proof}
This follows from
Topology, Lemmas
\ref{topology-lemma-catenary},
\ref{topology-lemma-dimension-function-catenary}, and
\ref{topology-lemma-locally-dimension-function},
Schemes, Lemma \ref{schemes-lemma-scheme-sober}
and finally Lemma \ref{lemma-Noetherian-topology}.
\end{proof}

\noindent
It turns out that a scheme is catenary if and only if its local
rings are catenary.

\begin{lemma}
\label{lemma-catenary-local-rings-catenary}
Let $X$ be a scheme. The following are equivalent
\begin{enumerate}
\item $X$ is catenary, and
\item for any $x \in X$ the local ring $\mathcal{O}_{X, x}$ is
catenary.
\end{enumerate}
\end{lemma}

\begin{proof}
Assume $X$ is catenary. Let $x \in X$. By Lemma \ref{lemma-catenary-local}
we may replace $X$ by an affine open neighbourhood of $x$, and
then $\Gamma(X, \mathcal{O}_X)$ is a catenary ring. By
Algebra, Lemma \ref{algebra-lemma-localization-catenary} any
localization of a catenary ring is
catenary. Whence $\mathcal{O}_{X, x}$ is catenary.

\medskip\noindent
Conversely assume all local rings of $X$ are catenary.
Let $Y \subset Y'$ be an inclusion of irreducible closed
subsets of $X$. Let $\xi \in Y$ be the generic point.
Let $\mathfrak p \subset \mathcal{O}_{X, \xi}$ be the prime
corresponding to the generic point of $Y'$, see
Schemes, Lemma \ref{schemes-lemma-specialize-points}. By that same
lemma the irreducible closed subsets of $X$ in between $Y$ and $Y'$
correspond to primes $\mathfrak q \subset \mathcal{O}_{X, \xi}$
with $\mathfrak p \subset \mathfrak q \subset \mathfrak m_{\xi}$.
Hence we see all maximal chains of these are finite and have the
same length as $\mathcal{O}_{X, \xi}$ is a catenary ring.
\end{proof}






\section{Serre's conditions}
\label{section-Rk}

\noindent
Here are two technical notions that are often useful.
See also Cohomology of Schemes, Section \ref{coherent-section-depth}.

\begin{definition}
\label{definition-Rk}
Let $X$ be a locally Noetherian scheme. Let $k \geq 0$.
\begin{enumerate}
\item We say $X$ is {\it regular in codimension $k$},
or we say $X$ has property {\it $(R_k)$} if for every $x \in X$
we have
$$
\dim(\mathcal{O}_{X, x}) \leq k
\Rightarrow
\mathcal{O}_{X, x}\text{ is regular}
$$
\item We say $X$ has property {\it $(S_k)$} if for every $x \in X$ we have
$\text{depth}(\mathcal{O}_{X, x}) \geq \min(k, \dim(\mathcal{O}_{X, x}))$.
\end{enumerate}
\end{definition}

\noindent
The phrase ``regular in codimension $k$'' makes sense since we have seen
in Section \ref{section-catenary} that if $Y \subset X$ is irreducible
closed with generic point $x$, then
$\dim(\mathcal{O}_{X, x}) = \text{codim}(Y, X)$. For example condition
$(R_0)$ means that for every generic point $\eta \in X$ of an irreducible
component of $X$ the local ring $\mathcal{O}_{X, \eta}$ is a field.
But for general Noetherian schemes it can happen that the regular locus
of $X$ is badly behaved, so care has to be taken.

\begin{lemma}
\label{lemma-scheme-regular-iff-all-Rk}
Let $X$ be a locally Noetherian scheme.
Then $X$ is regular if and only if $X$ has $(R_k)$ for all $k \geq 0$.
\end{lemma}

\begin{proof}
Follows from Lemma \ref{lemma-characterize-regular} and the definitions.
\end{proof}

\begin{lemma}
\label{lemma-scheme-CM-iff-all-Sk}
Let $X$ be a locally Noetherian scheme.
Then $X$ is Cohen-Macaulay if and only if $X$ has $(S_k)$ for all $k \geq 0$.
\end{lemma}

\begin{proof}
By Lemma \ref{lemma-characterize-Cohen-Macaulay}
we reduce to looking at local rings.
Hence the lemma is true because a Noetherian local ring is Cohen-Macaulay
if and only if it has depth equal to its dimension.
\end{proof}

\begin{lemma}
\label{lemma-criterion-reduced}
Let $X$ be a locally Noetherian scheme.
Then $X$ is reduced if and only if $X$ has properties $(S_1)$ and $(R_0)$.
\end{lemma}

\begin{proof}
This is Algebra, Lemma \ref{algebra-lemma-criterion-reduced}.
\end{proof}

\begin{lemma}
\label{lemma-criterion-normal}
Let $X$ be a locally Noetherian scheme.
Then $X$ is normal if and only if $X$ has properties $(S_2)$ and $(R_1)$.
\end{lemma}

\begin{proof}
This is Algebra, Lemma \ref{algebra-lemma-criterion-normal}.
\end{proof}

\begin{lemma}
\label{lemma-normal-dimension-1-regular}
Let $X$ be a locally Noetherian scheme which is normal and
has dimension $\leq 1$. Then $X$ is regular.
\end{lemma}

\begin{proof}
This follows from Lemma \ref{lemma-criterion-normal} and the definitions.
\end{proof}

\begin{lemma}
\label{lemma-normal-dimension-2-Cohen-Macaulay}
Let $X$ be a locally Noetherian scheme which is normal and
has dimension $\leq 2$. Then $X$ is Cohen-Macaulay.
\end{lemma}

\begin{proof}
This follows from Lemma \ref{lemma-criterion-normal} and the definitions.
\end{proof}







\section{Japanese and Nagata schemes}
\label{section-nagata}

\noindent
The notions considered in this section are not prominently defined in EGA.
A ``universally Japanese scheme'' is mentioned and defined in
\cite[IV Corollary 5.11.4]{EGA}. A ``Japanese scheme'' is mentioned in
\cite[IV Remark 10.4.14 (ii)]{EGA} but no definition is given.
A Nagata scheme (as given below) occurs in a few
places in the literature (see for example \cite[Definition 8.2.30]{Liu} and
\cite[Page 142]{Greco}).

\medskip\noindent
We briefly recall that a domain $R$ is called {\it Japanese} if the integral
closure of $R$ in any finite extension of its fraction field is finite over
$R$. A ring $R$ is called {\it universally Japanese} if for any finite type
ring map $R \to S$ with $S$ a domain $S$ is Japanese. A ring $R$ is called
{\it Nagata} if it is Noetherian and $R/\mathfrak p$ is Japanese for every
prime $\mathfrak p$ of $R$.

\begin{definition}
\label{definition-nagata}
Let $X$ be a scheme.
\begin{enumerate}
\item Assume $X$ integral. We say $X$ is {\it Japanese}
if for every $x \in X$ there exists an
affine open neighbourhood $x \in U \subset X$ such that the ring
$\mathcal{O}_X(U)$ is Japanese (see
Algebra, Definition \ref{algebra-definition-N}).
\item We say $X$ is {\it universally Japanese} if for every $x \in X$
there exists an affine open neighbourhood $x \in U \subset X$ such that
the ring $\mathcal{O}_X(U)$ is universally Japanese (see
Algebra, Definition \ref{algebra-definition-nagata}).
\item We say $X$ is {\it Nagata} if for every $x \in X$ there exists an
affine open neighbourhood $x \in U \subset X$ such that the ring
$\mathcal{O}_X(U)$ is Nagata (see
Algebra, Definition \ref{algebra-definition-nagata}).
\end{enumerate}
\end{definition}

\noindent
Being Nagata is the same thing as being locally Noetherian
and universally Japanese, see
Lemma \ref{lemma-nagata-universally-Japanese}.

\begin{remark}
\label{remark-non-integral-Japanese}
In \cite{Hoobler-finite} a (locally Noetherian) scheme $X$ is called
Japanese if for every $x \in X$ and every associated prime $\mathfrak p$
of $\mathcal{O}_{X, x}$ the ring $\mathcal{O}_{X, x}/\mathfrak p$ is
Japanese. We do not use this definition since there exists a one
dimensional Noetherian domain with excellent (in particular
Japanese) local rings whose normalization is not finite. See
\cite[Example 1]{Hochster-loci} or \cite{Heinzer-Levy} or
\cite[Expos\'e XIX]{Traveaux}.
On the other hand, we could circumvent this problem by calling a scheme
$X$ Japanese if for every affine open $\Spec(A) \subset X$ the ring
$A/\mathfrak p$ is Japanese for every associated prime $\mathfrak p$ of $A$.
\end{remark}

\begin{lemma}
\label{lemma-nagata-locally-Noetherian}
A Nagata scheme is locally Noetherian.
\end{lemma}

\begin{proof}
This is true because a Nagata ring is Noetherian by definition.
\end{proof}

\begin{lemma}
\label{lemma-locally-Japanese}
Let $X$ be an integral scheme. The following are equivalent:
\begin{enumerate}
\item The scheme $X$ is Japanese.
\item For every affine open $U \subset X$ the domain $\mathcal{O}_X(U)$
is Japanese.
\item There exists an affine open covering $X = \bigcup U_i$
such that each $\mathcal{O}_X(U_i)$ is Japanese.
\item There exists an open covering $X = \bigcup X_j$
such that each open subscheme $X_j$ is Japanese.
\end{enumerate}
Moreover, if $X$ is Japanese then every open subscheme
is Japanese.
\end{lemma}

\begin{proof}
This follows from Lemma \ref{lemma-locally-P} and
Algebra, Lemmas \ref{algebra-lemma-localize-N} and
\ref{algebra-lemma-Japanese-local}.
\end{proof}

\begin{lemma}
\label{lemma-locally-universally-Japanese}
Let $X$ be a scheme. The following are equivalent:
\begin{enumerate}
\item The scheme $X$ is universally Japanese.
\item For every affine open $U \subset X$ the ring $\mathcal{O}_X(U)$
is universally Japanese.
\item There exists an affine open covering $X = \bigcup U_i$
such that each $\mathcal{O}_X(U_i)$ is universally Japanese.
\item There exists an open covering $X = \bigcup X_j$
such that each open subscheme $X_j$ is universally Japanese.
\end{enumerate}
Moreover, if $X$ is universally Japanese then every open subscheme
is universally Japanese.
\end{lemma}

\begin{proof}
This follows from Lemma \ref{lemma-locally-P} and
Algebra, Lemmas \ref{algebra-lemma-universally-japanese} and
\ref{algebra-lemma-nagata-local}.
\end{proof}

\begin{lemma}
\label{lemma-locally-nagata}
Let $X$ be a scheme. The following are equivalent:
\begin{enumerate}
\item The scheme $X$ is Nagata.
\item For every affine open $U \subset X$ the ring $\mathcal{O}_X(U)$
is Nagata.
\item There exists an affine open covering $X = \bigcup U_i$
such that each $\mathcal{O}_X(U_i)$ is Nagata.
\item There exists an open covering $X = \bigcup X_j$
such that each open subscheme $X_j$ is Nagata.
\end{enumerate}
Moreover, if $X$ is Nagata then every open subscheme is Nagata.
\end{lemma}

\begin{proof}
This follows from Lemma \ref{lemma-locally-P} and
Algebra, Lemmas \ref{algebra-lemma-nagata-localize} and
\ref{algebra-lemma-nagata-local}.
\end{proof}

\begin{lemma}
\label{lemma-characterize-nagata}
Let $X$ be a locally Noetherian scheme.
Then $X$ is Nagata if and only if every integral closed subscheme
$Z \subset X$ is Japanese.
\end{lemma}

\begin{proof}
Assume $X$ is Nagata. Let $Z \subset X$ be an integral closed subscheme.
Let $z \in Z$.
Let $\Spec(A) = U \subset X$ be an affine open containing $z$
such that $A$ is Nagata. Then
$Z \cap U \cong \Spec(A/\mathfrak p)$ for some prime $\mathfrak p$,
see Schemes, Lemma \ref{schemes-lemma-closed-subspace-scheme} (and
Definition \ref{definition-integral}). By
Algebra, Definition \ref{algebra-definition-nagata} we see
that $A/\mathfrak p$ is Japanese. Hence $Z$ is Japanese by definition.

\medskip\noindent
Assume every integral closed subscheme of $X$ is Japanese.
Let $\Spec(A) = U \subset X$ be any affine open.
As $X$ is locally Noetherian we see that $A$ is Noetherian
(Lemma \ref{lemma-locally-Noetherian}). Let $\mathfrak p \subset A$
be a prime ideal. We have to show that $A/\mathfrak p$ is Japanese.
Let $T \subset U$ be the closed subset $V(\mathfrak p) \subset \Spec(A)$.
Let $\overline{T} \subset X$ be the closure. Then $\overline{T}$ is
irreducible as the closure of an irreducible subset. Hence the reduced
closed subscheme defined by $\overline{T}$ is an integral closed
subscheme (called $\overline{T}$ again), see
Schemes, Lemma \ref{schemes-lemma-reduced-closed-subscheme}.
In other words, $\Spec(A/\mathfrak p)$ is an affine
open of an integral closed subscheme of $X$. This subscheme is Japanese
by assumption and by Lemma \ref{lemma-locally-Japanese} we see that
$A/\mathfrak p$ is Japanese.
\end{proof}

\begin{lemma}
\label{lemma-nagata-universally-Japanese}
Let $X$ be a scheme.
The following are equivalent:
\begin{enumerate}
\item $X$ is Nagata, and
\item $X$ is locally Noetherian and universally Japanese.
\end{enumerate}
\end{lemma}

\begin{proof}
This is
Algebra, Proposition \ref{algebra-proposition-nagata-universally-japanese}.
\end{proof}

\noindent
This discussion will be continued in
Morphisms, Section \ref{morphisms-section-nagata}.





\section{The singular locus}
\label{section-singular-locus}

\noindent
Here is the definition.

\begin{definition}
\label{definition-singular-locus}
Let $X$ be a locally Noetherian scheme. The {\it regular locus}
$\text{Reg}(X)$ of $X$ is the set of $x \in X$ such that $\mathcal{O}_{X, x}$
is a regular local ring. The {\it singular locus} $\text{Sing}(X)$ is the
complement $X \setminus \text{Reg}(X)$, i.e., the set of points $x \in X$
such that $\mathcal{O}_{X, x}$ is not a regular local ring.
\end{definition}

\noindent
The regular locus of a locally Noetherian scheme is stable under
generalizations, see the discussion preceding
Algebra, Definition \ref{algebra-definition-regular}.
However, for general locally Noetherian schemes the regular locus
need not be open. In
More on Algebra, Section \ref{more-algebra-section-singular-locus}
the reader can find some criteria for when this is the case.
We will discuss this further in
Morphisms, Section \ref{morphisms-section-singular-locus}.






\section{Local irreducibility}
\label{section-unibranch}

\noindent
Recall that in More on Algebra, Section \ref{more-algebra-section-unibranch}
we introduced the notion of a (geometrically) unibranch local ring.

\begin{definition}
\label{definition-unibranch}
\begin{reference}
\cite[Chapter IV (6.15.1)]{EGA4}
\end{reference}
Let $X$ be a scheme. Let $x \in X$. We say $X$ is {\it unibranch at $x$}
if the local ring $\mathcal{O}_{X, x}$ is unibranch. We say $X$ is
{\it geometrically unibranch at $x$}
if the local ring $\mathcal{O}_{X, x}$ is geometrically unibranch.
We say $X$ is {\it unibranch} if $X$ is unibranch at all of its points.
We say $X$ is {\it geometrically unibranch} if $X$ is
geometrically unibranch at all of its points.
\end{definition}

\noindent
To be sure, it can happen that a local ring $A$ is geometrically unibranch
(in the sense of
More on Algebra, Definition \ref{more-algebra-definition-unibranch})
but the scheme $\Spec(A)$ is not geometrically unibranch in the sense
of Definition \ref{definition-unibranch}. For example this happens
if $A$ is the local ring at the vertex of the cone over an irreducible
plane curve which has ordinary double point singularity (a node).

\begin{lemma}
\label{lemma-normal-geometrically-unibranch}
A normal scheme is geometrically unibranch.
\end{lemma}

\begin{proof}
This follows from the definitions. Namely, a scheme
is normal if the local rings are normal domains. It is immediate
from the More on Algebra, Definition \ref{more-algebra-definition-unibranch}
that a local normal domain is geometrically unibranch.
\end{proof}

\begin{lemma}
\label{lemma-geometrically-unibranch}
\begin{reference}
Compare with \cite[Proposition 2.3]{Etale-coverings}
\end{reference}
Let $X$ be a Noetherian scheme. The following are equivalent
\begin{enumerate}
\item $X$ is geometrically unibranch (Definition \ref{definition-unibranch}),
\item for every point $x \in X$ which is not the generic point of
an irreducible component of $X$, the punctured spectrum of the
strict henselization $\mathcal{O}_{X, x}^{sh}$ is connected.
\end{enumerate}
\end{lemma}

\begin{proof}
More on Algebra, Lemma \ref{more-algebra-lemma-geometrically-unibranch}
shows that (1) implies that the punctured spectra in (2) are
irreducible and in particular connected.

\medskip\noindent
Assume (2). Let $x \in X$. We have to show that $\mathcal{O}_{X, x}$
is geometrically unibranch. By induction on $\dim(\mathcal{O}_{X, x})$
we may assume that the result holds for every nontrivial generalization of $x$.
We may replace $X$ by $\Spec(\mathcal{O}_{X, x})$. In other words,
we may assume that $X = \Spec(A)$ with $A$ local and that
$A_\mathfrak p$ is geometrically unibranch for each nonmaximal
prime $\mathfrak p \subset A$.

\medskip\noindent
Let $A^{sh}$ be the strict henselization of $A$. If
$\mathfrak q \subset A^{sh}$ is a prime lying over $\mathfrak p \subset A$,
then $A_\mathfrak p \to A^{sh}_\mathfrak q$ is a
filtered colimit of \'etale algebras. Hence the strict henselizations of
$A_\mathfrak p$ and $A^{sh}_\mathfrak q$ are isomorphic.
Thus by More on Algebra, Lemma \ref{more-algebra-lemma-geometrically-unibranch}
we conclude that $A^{sh}_\mathfrak q$
has a unique minimal prime ideal for every nonmaximal prime $\mathfrak q$ of
$A^{sh}$.

\medskip\noindent
Let $\mathfrak q_1, \ldots, \mathfrak q_r$ be the minimal primes
of $A^{sh}$. We have to show that $r = 1$. By the above
we see that $V(\mathfrak q_1) \cap V(\mathfrak q_j) = \{\mathfrak m^{sh}\}$
for $j = 2, \ldots, r$. Hence $V(\mathfrak q_1) \setminus \{\mathfrak m^{sh}\}$
is an open and closed subset of the punctured spectrum of $A^{sh}$
which is a contradiction with the assumption that this punctured spectrum
is connected unless $r = 1$.
\end{proof}

\begin{definition}
\label{definition-number-of-branches}
Let $X$ be a scheme. Let $x \in X$. The {\it number of branches of $X$
at $x$} is the number of branches of the local ring $\mathcal{O}_{X, x}$
as defined in
More on Algebra, Definition \ref{more-algebra-definition-number-of-branches}.
The {\it number of geometric branches of $X$ at $x$} is the number of
geometric branches of the local ring $\mathcal{O}_{X, x}$ as defined in
More on Algebra, Definition \ref{more-algebra-definition-number-of-branches}.
\end{definition}

\noindent
Often we want to compare this with the branches of the complete local
ring, but the comparison is not straightforward in general; some information
on this topic can be found in More on Algebra, Section
\ref{more-algebra-section-branches-completion}.

\begin{lemma}
\label{lemma-number-of-branches-irreducible-components}
Let $X$ be a scheme and $x \in X$. Let $X_i$, $i \in I$ be the
irreducible components of $X$ passing through $x$.
Then the number of (geometric) branches of $X$ at $x$
is the sum over $i \in I$ of the number of (geometric)
branches of $X_i$ at $x$.
\end{lemma}

\begin{proof}
We view the $X_i$ as integral closed subschemes of $X$, see
Schemes, Definition \ref{schemes-definition-reduced-induced-scheme} and
Lemma \ref{lemma-characterize-integral}.
Observe that the number of (geometric) branches of $X_i$ at $x$
is at least $1$ for all $i$ (essentially by definition).
Recall that the $X_i$ correspond $1$-to-$1$ with the minimal
prime ideals $\mathfrak p_i \subset \mathcal{O}_{X, x}$, see
Algebra, Lemma \ref{algebra-lemma-irreducible-components-containing-x}.
Thus, if $I$ is infinite, then $\mathcal{O}_{X, x}$ has infinitely
many minimal primes, whence both $\mathcal{O}_{X, x}^h$
and $\mathcal{O}_{X, x}^{sh}$ have infinitely many minimal
primes (combine Algebra, Lemmas
\ref{algebra-lemma-injective-minimal-primes-in-image} and
\ref{algebra-lemma-minimal-prime-image-minimal-prime} and
the injectivity of the maps
$\mathcal{O}_{X, x} \to  \mathcal{O}_{X, x}^h \to \mathcal{O}_{X, x}^{sh}$).
In this case the number of (geometric) branches of $X$ at $x$
is defined to be $\infty$ which is also true for the sum.
Thus we may assume $I$ is finite.
Let $A'$ be the integral closure of $\mathcal{O}_{X, x}$
in the total ring of fractions $Q$ of $(\mathcal{O}_{X, x})_{red}$.
Let $A'_i$ be the  integral closure of $\mathcal{O}_{X, x}/\mathfrak p_i$
in the total ring of fractions $Q_i$ of $\mathcal{O}_{X, x}/\mathfrak p_i$.
By Algebra, Lemma \ref{algebra-lemma-total-ring-fractions-no-embedded-points}
we have $Q = \prod_{i \in I} Q_i$. Thus $A' = \prod A'_i$.
Then the equality of the lemma follows from
More on Algebra, Lemma \ref{more-algebra-lemma-number-of-branches-1}
which expresses the number of (geometric) branches in terms
of the maximal ideals of $A'$.
\end{proof}

\begin{lemma}
\label{lemma-number-of-branches-1}
Let $X$ be a scheme. Let $x \in X$.
\begin{enumerate}
\item The number of branches of $X$ at $x$ is $1$ if and only if
$X$ is unibranch at $x$.
\item The number of geometric branches of $X$ at $x$ is $1$ if and only if
$X$ is geometrically unibranch at $x$.
\end{enumerate}
\end{lemma}

\begin{proof}
This lemma follows immediately from the definitions and the corresponding
result for rings, see More on Algebra, Lemma
\ref{more-algebra-lemma-number-of-branches-1}.
\end{proof}






\section{Characterizing modules of finite type and finite presentation}
\label{section-characterizing-finite-type-presentation}

\noindent
Let $X$ be a scheme.
Let $\mathcal{F}$ be a quasi-coherent $\mathcal{O}_X$-module.
The following lemma implies that $\mathcal{F}$ is of finite type
(see Modules, Definition \ref{modules-definition-finite-type})
if and only if $\mathcal{F}$ is
on each open affine $\Spec(A) = U \subset X$
of the form $\widetilde M$ for some finite $A$-module $M$.
Similarly, $\mathcal{F}$ is of finite presentation
(see Modules, Definition \ref{modules-definition-finite-presentation})
if and only if $\mathcal{F}$ is
on each open affine $\Spec(A) = U \subset X$
of the form $\widetilde M$ for some finitely presented $A$-module $M$.

\begin{lemma}
\label{lemma-finite-type-module}
Let $X = \Spec(R)$ be an affine scheme.
The quasi-coherent sheaf of $\mathcal{O}_X$-modules
$\widetilde M$ is a finite type $\mathcal{O}_X$-module
if and only if $M$ is a finite $R$-module.
\end{lemma}

\begin{proof}
Assume $\widetilde M$ is a finite type $\mathcal{O}_X$-module.
This means there exists an open covering of $X$ such that
$\widetilde M$ restricted to the members of this covering is
globally generated by finitely many sections.
Thus there also exists a standard open covering
$X = \bigcup_{i = 1, \ldots, n} D(f_i)$ such that $\widetilde M|_{D(f_i)}$
is generated by finitely many sections. Thus $M_{f_i}$ is finitely
generated for each $i$. Hence we conclude by
Algebra, Lemma \ref{algebra-lemma-cover}.
\end{proof}

\begin{lemma}
\label{lemma-finite-presentation-module}
Let $X = \Spec(R)$ be an affine scheme. The quasi-coherent sheaf
of $\mathcal{O}_X$-modules $\widetilde M$ is an $\mathcal{O}_X$-module of
finite presentation if and only if $M$ is an $R$-module of finite presentation.
\end{lemma}

\begin{proof}
Assume $\widetilde M$ is an $\mathcal{O}_X$-module of finite presentation.
By Lemma \ref{lemma-finite-type-module} we see that $M$ is a finite $R$-module.
Choose a surjection $R^n \to M$ with kernel $K$. By
Schemes, Lemma \ref{schemes-lemma-spec-sheaves}
there is a short exact sequence
$$
0 \to \widetilde{K} \to
\bigoplus \mathcal{O}_X^{\oplus n} \to
\widetilde{M} \to 0
$$
By
Modules, Lemma
\ref{modules-lemma-kernel-surjection-finite-free-onto-finite-presentation}
we see that $\widetilde{K}$ is a finite type $\mathcal{O}_X$-module.
Hence by Lemma \ref{lemma-finite-type-module}
again we see that $K$ is a finite $R$-module.
Hence $M$ is an $R$-module of finite presentation.
\end{proof}







\section{Sections over principal opens}
\label{section-principal-opens}

\noindent
Here is a typical result of this kind. We will use a more naive but
more direct method of proof in later lemmas.

\begin{lemma}
\label{lemma-invert-f-sections}
\begin{slogan}
Sections of quasi-coherent sheaves have only meromorphic singularities
at infinity.
\end{slogan}
Let $X$ be a scheme. Let $f \in \Gamma(X, \mathcal{O}_X)$.
Denote $X_f \subset X$ the open where $f$ is invertible, see
Schemes, Lemma \ref{schemes-lemma-f-open}.
If $X$ is quasi-compact and quasi-separated, the canonical map
$$
\Gamma(X, \mathcal{O}_X)_f \longrightarrow \Gamma(X_f, \mathcal{O}_X)
$$
is an isomorphism. Moreover, if $\mathcal{F}$ is a quasi-coherent
sheaf of $\mathcal{O}_X$-modules the map
$$
\Gamma(X, \mathcal{F})_f \longrightarrow \Gamma(X_f, \mathcal{F})
$$
is an isomorphism.
\end{lemma}

\begin{proof}
Write $R = \Gamma(X, \mathcal{O}_X)$. Consider the canonical morphism
$$
\varphi : X \longrightarrow \Spec(R)
$$
of schemes, see
Schemes, Lemma
\ref{schemes-lemma-morphism-into-affine}.
Then the inverse image of the standard open $D(f)$ on the
right hand side is $X_f$ on the left hand side.
Moreover, since $X$ is assumed quasi-compact and quasi-separated
the morphism $\varphi$ is quasi-compact and quasi-separated,
see Schemes, Lemma \ref{schemes-lemma-quasi-compact-affine} and
\ref{schemes-lemma-compose-after-separated}. Hence by
Schemes, Lemma \ref{schemes-lemma-push-forward-quasi-coherent}
we see that $\varphi_*\mathcal{F}$ is quasi-coherent.
Hence we see that $\varphi_*\mathcal{F} = \widetilde M$
with $M = \Gamma(X, \mathcal{F})$ as an $R$-module.
Thus we see that
$$
\Gamma(X_f, \mathcal{F}) =
\Gamma(D(f), \varphi_*\mathcal{F}) =
\Gamma(D(f), \widetilde M) = M_f
$$
which is exactly the content of the lemma. The first displayed isomorphism
of the lemma follows by taking $\mathcal{F} = \mathcal{O}_X$.
\end{proof}

\noindent
Recall that given a scheme $X$, an invertible sheaf $\mathcal{L}$
on $X$, and a sheaf of $\mathcal{O}_X$-modules $\mathcal{F}$
we get a graded ring
$\Gamma_*(X, \mathcal{L}) =
\bigoplus\nolimits_{n \geq 0} \Gamma(X, \mathcal{L}^{\otimes n})$
and a graded $\Gamma_*(X, \mathcal{L})$-module
$\Gamma_*(X, \mathcal{L}, \mathcal{F}) =
\bigoplus\nolimits_{n \in \mathbf{Z}}
\Gamma(X, \mathcal{F} \otimes_{\mathcal{O}_X} \mathcal{L}^{\otimes n})$
see Modules, Definition \ref{modules-definition-gamma-star}.
If we have moreover a section $s \in \Gamma(X, \mathcal{L})$, then
we obtain a map
\begin{equation}
\label{equation-module-invert-s}
\Gamma_*(X, \mathcal{L}, \mathcal{F})_{(s)}
\longrightarrow
\Gamma(X_s, \mathcal{F}|_{X_s})
\end{equation}
which sends $t/s^n$ where
$t \in \Gamma(X, \mathcal{F} \otimes_{\mathcal{O}_X} \mathcal{L}^{\otimes n})$
to $t|_{X_s} \otimes s|_{X_s}^{-n}$. This makes sense
because $X_s \subset X$ is by definition the open over which
$s$ has an inverse, see Modules, Lemma \ref{modules-lemma-s-open}.

\begin{lemma}
\label{lemma-invert-s-sections}
Let $X$ be a scheme. Let $\mathcal{L}$ be an invertible sheaf on $X$.
Let $s \in \Gamma(X, \mathcal{L})$. Let $\mathcal{F}$ be a quasi-coherent
$\mathcal{O}_X$-module.
\begin{enumerate}
\item If $X$ is quasi-compact, then (\ref{equation-module-invert-s})
is injective, and
\item if $X$ is quasi-compact and quasi-separated, then
(\ref{equation-module-invert-s}) is an isomorphism.
\end{enumerate}
In particular, the canonical map
$$
\Gamma_*(X, \mathcal{L})_{(s)}
\longrightarrow
\Gamma(X_s, \mathcal{O}_X),\quad
a/s^n \longmapsto a \otimes s^{-n}
$$
is an isomorphism if $X$ is quasi-compact and quasi-separated.
\end{lemma}

\begin{proof}
Assume $X$ is quasi-compact. Choose a finite affine open covering
$X = U_1 \cup \ldots \cup U_m$ with $U_j$ affine and
$\mathcal{L}|_{U_j} \cong \mathcal{O}_{U_j}$. Via this isomorphism,
the image $s|_{U_j}$ corresponds to some
$f_j \in \Gamma(U_j, \mathcal{O}_{U_j})$. Then
$X_s \cap U_j = D(f_j)$.

\medskip\noindent
Proof of (1). Let $t/s^n$ be an element in the kernel of
(\ref{equation-module-invert-s}). Then $t|_{X_s} = 0$.
Hence $(t|_{U_j})|_{D(f_j)} = 0$. By
Lemma \ref{lemma-invert-f-sections} we conclude that
$f_j^{e_j} t|_{U_j} = 0$ for some
$e_j \geq 0$. Let $e = \max(e_j)$. Then we see that $t \otimes s^e$
restricts to zero on $U_j$ for all $j$, hence is zero. Since $t/s^n$
is equal to $t \otimes s^e/s^{n + e}$ in
$\Gamma_*(X, \mathcal{L}, \mathcal{F})_{(s)}$ we conclude that $t/s^n = 0$
as desired.

\medskip\noindent
Proof of (2). Assume $X$ is quasi-compact and quasi-separated.
Then $U_j \cap U_{j'}$ is quasi-compact for all pairs $j, j'$, see
Schemes, Lemma \ref{schemes-lemma-characterize-quasi-separated}.
By part (1) we know (\ref{equation-module-invert-s}) is injective.
Let $t' \in \Gamma(X_s, \mathcal{F}|_{X_s})$. For every $j$, there exist an
integer $e_j \geq 0$ and $t'_j \in \Gamma(U_j, \mathcal{F}|_{U_j})$ such that
$t'|_{D(f_j)}$ corresponds to $t'_j/f_j^{e_j}$
via the isomorphism of Lemma \ref{lemma-invert-f-sections}.
Set $e = \max(e_j)$ and
$$
t_j = f_j^{e - e_j} t'_j \otimes q_j^e \in
\Gamma(U_j,
(\mathcal{F} \otimes_{\mathcal{O}_X} \mathcal{L}^{\otimes e})|_{U_j})
$$
where $q_j \in \Gamma(U_j, \mathcal{L}|_{U_j})$ is the trivializing
section coming from the isomorphism
$\mathcal{L}|_{U_j} \cong \mathcal{O}_{U_j}$. In particular we have
$s|_{U_j} = f_j q_j$. Using this a calculation shows that
$t_j|_{U_j \cap U_{j'}}$ and $t_{j'}|_{U_j \cap U_{j'}}$
map to the same section of $\mathcal{F}$ over $U_j \cap U_{j'} \cap X_s$.
By quasi-compactness of $U_j \cap U_{j'}$ and part (1) there exists an
integer $e' \geq 0$ such that
$$
t_j|_{U_j \cap U_{j'}} \otimes s^{e'}|_{U_j \cap U_{j'}} =
t_{j'}|_{U_j \cap U_{j'}} \otimes s^{e'}|_{U_j \cap U_{j'}}
$$
as sections of $\mathcal{F} \otimes \mathcal{L}^{\otimes e + e'}$ over
$U_j \cap U_{j'}$. We may choose the same $e'$ to work for all pairs
$j, j'$. Then the sheaf conditions implies there is a section
$t \in \Gamma(X, \mathcal{F} \otimes \mathcal{L}^{\otimes e + e'})$
whose restriction to $U_j$ is $t_j \otimes s^{e'}|_{U_j}$.
A simple computation shows that $t/s^{e + e'}$ maps to $t'$
as desired.
\end{proof}

\noindent
Let $X$ be a scheme. Let $\mathcal{L}$ be an invertible $\mathcal{O}_X$-module.
Let $\mathcal{F}$ and $\mathcal{G}$ be quasi-coherent $\mathcal{O}_X$-modules.
Consider the graded $\Gamma_*(X, \mathcal{L})$-module
$$
M = \bigoplus\nolimits_{n \in \mathbf{Z}} \Hom_{\mathcal{O}_X}(\mathcal{F},
\mathcal{G} \otimes_{\mathcal{O}_X} \mathcal{L}^{\otimes n})
$$
Next, let $s \in \Gamma(X, \mathcal{L})$ be a section. Then there is a
canonical map
\begin{equation}
\label{equation-hom-invert-s}
M_{(s)} \longrightarrow
\Hom_{\mathcal{O}_{X_s}}(\mathcal{F}|_{X_s}, \mathcal{G}|_{X_s})
\end{equation}
which sends $\alpha/s^n$ to the map $\alpha|_{X_s} \otimes s|_{X_s}^{-n}$.
The following lemma, combined with
Lemma \ref{lemma-extend-finite-presentation},
says roughly that, if $X$ is quasi-compact and quasi-separated,
the category of finitely presented $\mathcal{O}_{X_s}$-modules
is the category of finitely presented $\mathcal{O}_X$-modules
with the multiplicative system of maps
$s^n: \mathcal{F} \to
\mathcal{F} \otimes_{\mathcal{O}_X} \mathcal{L}^{\otimes n}$ inverted.

\begin{lemma}
\label{lemma-section-maps-backwards}
Let $X$ be a scheme. Let $\mathcal{L}$ be an invertible $\mathcal{O}_X$-module.
Let $s \in \Gamma(X, \mathcal{L})$ be a section.
Let $\mathcal{F}$, $\mathcal{G}$ be quasi-coherent $\mathcal{O}_X$-modules.
\begin{enumerate}
\item If $X$ is quasi-compact and $\mathcal{F}$ is of finite type,
then (\ref{equation-hom-invert-s}) is injective, and
\item if $X$ is quasi-compact and quasi-separated and $\mathcal{F}$
is of finite presentation, then
(\ref{equation-hom-invert-s})
is bijective.
\end{enumerate}
\end{lemma}

\begin{proof}
We first prove the lemma in case $X = \Spec(A)$ is affine
and $\mathcal{L} = \mathcal{O}_X$. In this case $s$ corresponds
to an element $f \in A$. Say
$\mathcal{F} = \widetilde{M}$ and $\mathcal{G} = \widetilde{N}$
for some $A$-modules $M$ and $N$. Then the lemma translates
(via Lemmas \ref{lemma-finite-type-module} and
\ref{lemma-finite-presentation-module}) into
the following algebra statements
\begin{enumerate}
\item If $M$ is a finite $A$-module and $\varphi : M \to N$ is
an $A$-module map such that the induced map $M_f \to N_f$ is zero,
then $f^n\varphi = 0$ for some $n$.
\item If $M$ is a finitely presented $A$-module, then
$\Hom_A(M, N)_f = \Hom_{A_f}(M_f, N_f)$.
\end{enumerate}
The second statement is
Algebra, Lemma \ref{algebra-lemma-hom-from-finitely-presented} and we omit
the proof of the first statement.

\medskip\noindent
Next, we prove (1) for general $X$.
Assume $X$ is quasi-compact and hoose a finite affine open covering
$X = U_1 \cup \ldots \cup U_m$ with $U_j$ affine and
$\mathcal{L}|_{U_j} \cong \mathcal{O}_{U_j}$. Via this isomorphism,
the image $s|_{U_j}$ corresponds to some
$f_j \in \Gamma(U_j, \mathcal{O}_{U_j})$. Then
$X_s \cap U_j = D(f_j)$.
Let $\alpha/s^n$ be an element in the kernel of
(\ref{equation-hom-invert-s}). Then $\alpha|_{X_s} = 0$.
Hence $(\alpha|_{U_j})|_{D(f_j)} = 0$. By the affine case treated above
we conclude that $f_j^{e_j} \alpha|_{U_j} = 0$ for some
$e_j \geq 0$. Let $e = \max(e_j)$. Then we see that $\alpha \otimes s^e$
restricts to zero on $U_j$ for all $j$, hence is zero. Since $\alpha/s^n$
is equal to $\alpha \otimes s^e/s^{n + e}$ in $M_{(s)}$ we conclude that
$\alpha/s^n = 0$ as desired.

\medskip\noindent
Proof of (2). Since $\mathcal{F}$ is of finite presentation, the
sheaf $\SheafHom_{\mathcal{O}_X}(\mathcal{F}, \mathcal{G})$ is
quasi-coherent, see Schemes, Section \ref{schemes-section-quasi-coherent}.
Moreover, it is clear that
$$
\SheafHom_{\mathcal{O}_X}(\mathcal{F},
\mathcal{G} \otimes_{\mathcal{O}_X} \mathcal{L}^{\otimes n}) =
\SheafHom_{\mathcal{O}_X}(\mathcal{F}, \mathcal{G})
\otimes_{\mathcal{O}_X} \mathcal{L}^{\otimes n}
$$
for all $n$. Hence in this case the statement follows from
Lemma \ref{lemma-invert-s-sections} applied to
$\SheafHom_{\mathcal{O}_X}(\mathcal{F}, \mathcal{G})$.
\end{proof}





\section{Quasi-affine schemes}
\label{section-quasi-affine}

\begin{definition}
\label{definition-quasi-affine}
A scheme $X$ is called {\it quasi-affine} if it is quasi-compact
and isomorphic to an open subscheme of an affine scheme.
\end{definition}

\begin{lemma}
\label{lemma-quasi-coherent-quasi-affine}
Let $A$ be a ring and let $U \subset \Spec(A)$ be a quasi-compact
open subscheme. For $\mathcal{F}$ quasi-coherent on $U$ the canonical map
$$
\widetilde{H^0(U, \mathcal{F})}|_U \to \mathcal{F}
$$
is an isomorphism.
\end{lemma}

\begin{proof}
Denote $j : U \to \Spec(A)$ the inclusion morphism. Then
$H^0(U, \mathcal{F}) = H^0(\Spec(A), j_*\mathcal{F})$ and
$j_*\mathcal{F}$ is quasi-coherent by
Schemes, Lemma \ref{schemes-lemma-push-forward-quasi-coherent}.
Hence $j_*\mathcal{F} = \widetilde{H^0(U, \mathcal{F})}$ by
Schemes, Lemma \ref{schemes-lemma-equivalence-quasi-coherent}.
Restricting back to $U$ we get the lemma.
\end{proof}

\begin{lemma}
\label{lemma-invert-f-affine}
Let $X$ be a scheme. Let $f \in \Gamma(X, \mathcal{O}_X)$.
Assume $X$ is quasi-compact and quasi-separated and assume that
$X_f$ is affine. Then the canonical morphism
$$
j : X \longrightarrow \Spec(\Gamma(X, \mathcal{O}_X))
$$
from Schemes, Lemma \ref{schemes-lemma-morphism-into-affine}
induces an isomorphism of $X_f = j^{-1}(D(f))$ onto the standard affine
open $D(f) \subset \Spec(\Gamma(X, \mathcal{O}_X))$.
\end{lemma}

\begin{proof}
This is clear as $j$ induces an isomorphism of rings
$\Gamma(X, \mathcal{O}_X)_f \to \mathcal{O}_X(X_f)$ by
Lemma \ref{lemma-invert-f-sections} above.
\end{proof}

\begin{lemma}
\label{lemma-quasi-affine}
Let $X$ be a scheme. Then $X$ is quasi-affine if and only if
the canonical morphism
$$
X \longrightarrow \Spec(\Gamma(X, \mathcal{O}_X))
$$
from Schemes, Lemma \ref{schemes-lemma-morphism-into-affine} is
a quasi-compact open immersion.
\end{lemma}

\begin{proof}
If the displayed morphism is a quasi-compact open immersion then
$X$ is isomorphic to a quasi-compact open subscheme of
$\Spec(\Gamma(X, \mathcal{O}_X))$ and clearly $X$ is quasi-affine.

\medskip\noindent
Assume $X$ is quasi-affine, say $X \subset \Spec(R)$ is
quasi-compact open. This in particular implies that $X$ is
separated, see
Schemes, Lemma \ref{schemes-lemma-subscheme-of-separated-scheme}.
Let $A = \Gamma(X, \mathcal{O}_X)$.
Consider the ring map $R \to A$ coming from
$R = \Gamma(\Spec(R), \mathcal{O}_{\Spec(R)})$
and the restriction mapping of the sheaf $\mathcal{O}_{\Spec(R)}$.
By Schemes, Lemma \ref{schemes-lemma-morphism-into-affine}
we obtain a factorization:
$$
X \longrightarrow
\Spec(A) \longrightarrow
\Spec(R)
$$
of the inclusion morphism. Let $x \in X$. Choose $r \in R$ such that
$x \in D(r)$ and $D(r) \subset X$. Denote $f \in A$ the image of $r$
in $A$. The open $X_f$ of Lemma \ref{lemma-invert-f-sections}
above is equal to $D(r) \subset X$ and hence $A_f \cong R_r$ by the
conclusion of that lemma.
Hence $D(r) \to \Spec(A)$ is an isomorphism onto the
standard affine open $D(f)$ of $\Spec(A)$. Since $X$
can be covered by such affine opens $D(f)$ we win.
\end{proof}

\begin{lemma}
\label{lemma-cartesian-diagram-quasi-affine}
Let $U \to V$ be an open immersion of quasi-affine schemes. Then
$$
\xymatrix{
U \ar[d] \ar[rr]_-j & & \Spec(\Gamma(U, \mathcal{O}_U)) \ar[d] \\
U \ar[r] & V \ar[r]^-{j'} & \Spec(\Gamma(V, \mathcal{O}_V))
}
$$
is cartesian.
\end{lemma}

\begin{proof}
The diagram is commutative by Schemes, Lemma
\ref{schemes-lemma-morphism-into-affine}.
Write $A = \Gamma(U, \mathcal{O}_U)$ and $B = \Gamma(V, \mathcal{O}_V)$. Let
$g \in B$ be such that $V_g$ is affine and contained in $U$. This
means that if $f$ is the image of $g$ in $A$, then $U_f = V_g$. By Lemma
\ref{lemma-invert-f-affine} we see that $j'$ induces an isomorphism of
$V_g$ with the standard open $D(g)$ of $\Spec(B)$. Thus
$V_g \times_{\Spec(B)} \Spec(A) \to \Spec(A)$ is an
isomorphism onto $D(f) \subset \Spec(A)$. By Lemma \ref{lemma-invert-f-affine}
again $j$ maps $U_f$ isomorphically to $D(f)$. Thus we see that
$U_f = U_f \times_{\Spec(B)} \Spec(A)$. Since by
Lemma \ref{lemma-quasi-affine} we can cover $U$ by $V_g = U_f$ as above,
we see that $U \to U \times_{\Spec(B)} \Spec(A)$ is an isomorphism.
\end{proof}

\begin{lemma}
\label{lemma-quasi-affine-presentation}
Let $X$ be a quasi-affine scheme. There exists an integer $n \geq 0$,
an affine scheme $T$, and a morphism $T \to X$ such that for every
morphism $X' \to X$ with $X'$ affine the fibre product $X' \times_X T$
is isomorphic to $\mathbf{A}^n_{X'}$ over $X'$.
\end{lemma}

\begin{proof}
By definition, there exists a ring $A$ such that $X$ is isomorphic to a
quasi-compact open subscheme $U \subset \Spec(A)$. Recall that the standard
opens $D(f) \subset \Spec(A)$ form a basis for the topology, see
Algebra, Section \ref{algebra-section-spectrum-ring}. Since $U$ is
quasi-compact we can choose $f_1, \ldots, f_n \in A$ such that
$U = D(f_1) \cup \ldots \cup D(f_n)$. Thus we may assume
$X = \Spec(A) \setminus V(I)$ where $I = (f_1, \ldots, f_n)$. We set
$$
T = \Spec(A[t, x_1, \ldots, x_n]/(f_1 x_1 + \ldots + f_n x_n - 1))
$$
The structure morphism $T \to \Spec(A)$ factors through the open $X$
to give the morphism $T \to X$. If $X' = \Spec(A')$ and the morphism
$X' \to X$ corresponds to the ring map $A \to A'$, then the images
$f'_1, \ldots, f'_n \in A'$ of $f_1, \ldots, f_n$
generate the unit ideal in $A'$.
Say $1 = f'_1 a'_1 + \ldots + f'_n a'_n$.
The base change $X' \times_X T$ is the spectrum of
$A'[t, x_1, \ldots, x_n]/(f'_1 x_1 + \ldots + f'_n x_n - 1)$.
We claim the $A'$-algebra homomorphism
$$
\varphi :
A'[y_1, \ldots, y_n]
\longrightarrow
A'[t, x_1, \ldots, x_n, x_{n + 1}]/(f'_1 x_1 + \ldots + f'_n x_n - 1)
$$
sending $y_i$ to $a'_i t + x_i$ is an isomorphism. The claim finishes
the proof of the lemma. The inverse of $\varphi$ is given by the $A'$-algebra
homomorphism
$$
\psi :
A'[t, x_1, \ldots, x_n, x_{n + 1}]/(f'_1 x_1 + \ldots + f'_n x_n - 1)
\longrightarrow
A'[y_1, \ldots, y_n]
$$
sending $t$ to $-1 + f'_1 y_1 + \ldots + f'_n y_n$ and $x_i$ to
$y_i + a'_i - a'_i(f'_1 y_1 + \ldots + f'_n y_n)$ for $i = 1, \ldots, n$.
This makes sense because $\sum f'_ix_i$ is mapped to
$$
\begin{matrix}
\sum f'_i(y_i + a'_i - a'_i(\sum f'_j y_j)) =
(\sum f'_iy_i) + 1 - (\sum f'_j y_j) = 1
\end{matrix}
$$
To see the maps are mutually inverse one computes as follows:
$$
\begin{matrix}
\varphi(\psi(t) = \varphi(-1 +  \sum f'_i y_i) =
-1 + \sum f'_i (a'_i t + x_i) = t \\
\varphi(\psi(x_i)) = \varphi(y_i + a'_i - a'_i(\sum f'_j y_j)) =
a'_i t + x_i + a'_i - a'_i(\sum f'_ja'_jt + f'_jx_j) = x_i \\
\psi(\varphi(y_i)) = \psi(a'_i t + x_i) =
a'_i(-1 + \sum f'_j y_j) + y_i + a'_i - a'_i(\sum f'_j y_j) = y_i
\end{matrix}
$$
This finishes the proof.
\end{proof}




\section{Flat modules}
\label{section-flat}

\noindent
On any ringed space $(X, \mathcal{O}_X)$
we know what it means for an $\mathcal{O}_X$-module
to be flat (at a point), see
Modules, Definition \ref{modules-definition-flat}
(Definition \ref{modules-definition-flat-at-point}).
For quasi-coherent sheaves on an affine scheme this matches the notion
defined in the algebra chapter.

\begin{lemma}
\label{lemma-flat-module}
\begin{slogan}
Flatness is the same for modules and sheaves.
\end{slogan}
Let $X = \Spec(R)$ be an affine scheme.
Let $\mathcal{F} = \widetilde{M}$ for some $R$-module $M$.
The quasi-coherent sheaf $\mathcal{F}$ is a flat
$\mathcal{O}_X$-module if and only if $M$ is a flat $R$-module.
\end{lemma}

\begin{proof}
Flatness of $\mathcal{F}$ may be checked on the stalks, see
Modules, Lemma \ref{modules-lemma-flat-stalks-flat}.
The same is true in the case of modules over a ring, see
Algebra, Lemma \ref{algebra-lemma-flat-localization}.
And since $\mathcal{F}_x = M_{\mathfrak p}$ if $x$ corresponds
to $\mathfrak p$ the lemma is true.
\end{proof}





\section{Locally free modules}
\label{section-finite-locally-free}

\noindent
On any ringed space we know what it means for an $\mathcal{O}_X$-module
to be (finite) locally free. On an affine scheme this matches the notion
defined in the algebra chapter.

\begin{lemma}
\label{lemma-locally-free-module}
Let $X = \Spec(R)$ be an affine scheme.
Let $\mathcal{F} = \widetilde{M}$ for some $R$-module $M$.
The quasi-coherent sheaf $\mathcal{F}$ is a (finite) locally free
$\mathcal{O}_X$-module of if and only if $M$ is a (finite)
locally free $R$-module.
\end{lemma}

\begin{proof}
Follows from the definitions, see
Modules, Definition \ref{modules-definition-locally-free}
and
Algebra, Definition \ref{algebra-definition-locally-free}.
\end{proof}

\noindent
We can characterize finite locally free modules in many different ways.

\begin{lemma}
\label{lemma-finite-locally-free}
Let $X$ be a scheme.
Let $\mathcal{F}$ be a quasi-coherent $\mathcal{O}_X$-module.
The following are equivalent:
\begin{enumerate}
\item $\mathcal{F}$ is a flat $\mathcal{O}_X$-module of finite presentation,
\item $\mathcal{F}$ is $\mathcal{O}_X$-module of finite presentation and
for all $x \in X$ the stalk $\mathcal{F}_x$ is a free
$\mathcal{O}_{X, x}$-module,
\item $\mathcal{F}$ is a locally free, finite type $\mathcal{O}_X$-module,
\item $\mathcal{F}$ is a finite locally free $\mathcal{O}_X$-module, and
\item $\mathcal{F}$ is an $\mathcal{O}_X$-module of finite type,
for every $x \in X$ the stalk $\mathcal{F}_x$ is a free
$\mathcal{O}_{X, x}$-module, and the function
$$
\rho_\mathcal{F} : X \to \mathbf{Z}, \quad
x \longmapsto
\dim_{\kappa(x)} \mathcal{F}_x \otimes_{\mathcal{O}_{X, x}} \kappa(x)
$$
is locally constant in the Zariski topology on $X$.
\end{enumerate}
\end{lemma}

\begin{proof}
This lemma immediately reduces to the affine case.
In this case the lemma is a reformulation of
Algebra, Lemma \ref{algebra-lemma-finite-projective}.
The translation uses
Lemmas \ref{lemma-finite-type-module},
\ref{lemma-finite-presentation-module},
\ref{lemma-flat-module}, and
\ref{lemma-locally-free-module}.
\end{proof}

\begin{lemma}
\label{lemma-finite-locally-free-reduced}
Let $X$ be a reduced scheme. Let $\mathcal{F}$ be a quasi-coherent
$\mathcal{O}_X$-module. Then the equivalent conditions of
Lemma \ref{lemma-finite-locally-free} are also equivalent to
\begin{enumerate}
\item[(6)] $\mathcal{F}$ is an $\mathcal{O}_X$-module of finite type and
the function
$$
\rho_\mathcal{F} : X \to \mathbf{Z}, \quad
x \longmapsto
\dim_{\kappa(x)} \mathcal{F}_x \otimes_{\mathcal{O}_{X, x}} \kappa(x)
$$
is locally constant in the Zariski topology on $X$.
\end{enumerate}
\end{lemma}

\begin{proof}
This lemma immediately reduces to the affine case.
In this case the lemma is a reformulation of
Algebra, Lemma \ref{algebra-lemma-finite-projective-reduced}.
\end{proof}




\section{Locally projective modules}
\label{section-locally-projective}

\noindent
A consequence of the work done in the algebra chapter is that it
makes sense to define a locally projective module as follows.

\begin{definition}
\label{definition-locally-projective}
Let $X$ be a scheme. Let $\mathcal{F}$ be a quasi-coherent
$\mathcal{O}_X$-module. We say $\mathcal{F}$ is {\it locally projective}
if for every affine open $U \subset X$ the $\mathcal{O}_X(U)$-module
$\mathcal{F}(U)$ is projective.
\end{definition}

\begin{lemma}
\label{lemma-locally-projective}
Let $X$ be a scheme.
Let $\mathcal{F}$ be a quasi-coherent $\mathcal{O}_X$-module.
The following are equivalent
\begin{enumerate}
\item $\mathcal{F}$ is locally projective, and
\item there exists an affine open covering $X = \bigcup U_i$
such that the $\mathcal{O}_X(U_i)$-module
$\mathcal{F}(U_i)$ is projective for every $i$.
\end{enumerate}
In particular, if $X = \Spec(A)$ and $\mathcal{F} = \widetilde{M}$
then $\mathcal{F}$ is locally projective if and only if $M$ is a projective
$A$-module.
\end{lemma}

\begin{proof}
First, note that if $M$ is a projective $A$-module and $A \to B$ is a
ring map, then $M \otimes_A B$ is a projective $B$-module, see
Algebra, Lemma \ref{algebra-lemma-ascend-properties-modules}.
Hence if $U$ is an affine open such that $\mathcal{F}(U)$ is a projective
$\mathcal{O}_X(U)$-module, then the standard open $D(f)$ is an
affine open such that $\mathcal{F}(D(f))$ is a projective
$\mathcal{O}_X(D(f))$-module for all $f \in \mathcal{O}_X(U)$.
Assume (2) holds. Let $U \subset X$ be an arbitrary affine open.
We can find an open covering $U = \bigcup_{j = 1, \ldots, m} D(f_j)$
by finitely many standard opens $D(f_j)$ such that for each
$j$ the open $D(f_j)$ is a standard open of some $U_i$, see
Schemes, Lemma \ref{schemes-lemma-standard-open-two-affines}.
Hence, if we set $A = \mathcal{O}_X(U)$ and if $M$ is an $A$-module
such that $\mathcal{F}|_U$ corresponds to $M$, then we see that
$M_{f_j}$ is a projective $A_{f_j}$-module. It follows that
$A \to B = \prod A_{f_j}$ is a faithfully flat ring map
such that $M \otimes_A B$ is a projective $B$-module.
Hence $M$ is projective by
Algebra, Theorem \ref{algebra-theorem-ffdescent-projectivity}.
\end{proof}

\begin{lemma}
\label{lemma-locally-projective-pullback}
Let $f : X \to Y$ be a morphism of schemes.
Let $\mathcal{G}$ be a quasi-coherent $\mathcal{O}_Y$-module.
If $\mathcal{G}$ is locally projective on $Y$, then $f^*\mathcal{G}$
is locally projective on $X$.
\end{lemma}

\begin{proof}
See
Algebra, Lemma \ref{algebra-lemma-ascend-properties-modules}.
\end{proof}






\section{Extending quasi-coherent sheaves}
\label{section-extending-quasi-coherent-sheaves}

\noindent
It is sometimes useful to be able to show that a given quasi-coherent
sheaf on an open subscheme extends to the whole scheme.

\begin{lemma}
\label{lemma-extend-trivial}
Let $j : U \to X$ be a quasi-compact open immersion of schemes.
\begin{enumerate}
\item Any quasi-coherent sheaf on $U$ extends to a quasi-coherent
sheaf on $X$.
\item Let $\mathcal{F}$ be a quasi-coherent sheaf on $X$.
Let $\mathcal{G} \subset \mathcal{F}|_U$ be a quasi-coherent
subsheaf. There exists a quasi-coherent subsheaf $\mathcal{H}$ of
$\mathcal{F}$ such that $\mathcal{H}|_U = \mathcal{G}$
as subsheaves of $\mathcal{F}|_U$.
\item Let $\mathcal{F}$ be a quasi-coherent sheaf on $X$.
Let $\mathcal{G}$ be a quasi-coherent sheaf on $U$.
Let $\varphi : \mathcal{G} \to \mathcal{F}|_U$ be a morphism
of $\mathcal{O}_U$-modules. There exists a quasi-coherent sheaf $\mathcal{H}$
of $\mathcal{O}_X$-modules and a map $\psi : \mathcal{H} \to \mathcal{F}$
such that $\mathcal{H}|_U = \mathcal{G}$ and that
$\psi|_U = \varphi$.
\end{enumerate}
\end{lemma}

\begin{proof}
An immersion is separated
(see Schemes, Lemma \ref{schemes-lemma-immersions-monomorphisms})
and $j$ is quasi-compact by assumption.
Hence for any quasi-coherent sheaf $\mathcal{G}$ on $U$ the sheaf
$j_*\mathcal{G}$ is an extension to $X$. See
Schemes, Lemma \ref{schemes-lemma-push-forward-quasi-coherent} and
Sheaves, Section \ref{sheaves-section-open-immersions}.

\medskip\noindent
Assume $\mathcal{F}$, $\mathcal{G}$ are as in (2).
Then $j_*\mathcal{G}$ is a quasi-coherent sheaf on $X$ (see above).
It is a subsheaf of $j_*j^*\mathcal{F}$.
Hence the kernel
$$
\mathcal{H} =
\Ker(\mathcal{F} \oplus j_* \mathcal{G}
\longrightarrow j_*j^*\mathcal{F})
$$
is quasi-coherent as well, see
Schemes, Section \ref{schemes-section-quasi-coherent}.
It is formal to check that $\mathcal{H} \subset \mathcal{F}$ and that
$\mathcal{H}|_U = \mathcal{G}$ (using the material in
Sheaves, Section \ref{sheaves-section-open-immersions} again).

\medskip\noindent
Part (3) is proved in the same manner as (2). Just take
$\mathcal{H} = \Ker(\mathcal{F} \oplus j_* \mathcal{G}
\to j_*j^*\mathcal{F})$ with its obvious map to $\mathcal{F}$
and its obvious identification with $\mathcal{G}$ over $U$.
\end{proof}

\begin{lemma}
\label{lemma-extend}
Let $X$ be a quasi-compact and quasi-separated scheme.
Let $U \subset X$ be a quasi-compact open.
Let $\mathcal{F}$ be a quasi-coherent $\mathcal{O}_X$-module.
Let $\mathcal{G} \subset \mathcal{F}|_U$ be a quasi-coherent
$\mathcal{O}_U$-submodule which is of finite type. Then
there exists a quasi-coherent submodule $\mathcal{G}' \subset \mathcal{F}$
which is of finite type such that $\mathcal{G}'|_U = \mathcal{G}$.
\end{lemma}

\begin{proof}
Let $n$ be the minimal number of affine opens $U_i \subset X$,
$i = 1, \ldots , n$ such that $X = U \cup \bigcup U_i$.
(Here we use that $X$ is quasi-compact.) Suppose
we can prove the lemma for the case $n = 1$. Then we can successively
extend $\mathcal{G}$
to a $\mathcal{G}_1$ over $U \cup U_1$
to a $\mathcal{G}_2$ over $U \cup U_1 \cup U_2$
to a $\mathcal{G}_3$ over $U \cup U_1 \cup U_2 \cup U_3$,
and so on.
Thus we reduce to the case $n = 1$.

\medskip\noindent
Thus we may assume that $X = U \cup V$ with $V$ affine.
Since $X$ is quasi-separated and $U$, $V$ are quasi-compact open,
we see that $U \cap V$ is a quasi-compact open. It suffices to prove the
lemma for the system $(V, U \cap V, \mathcal{F}|_V, \mathcal{G}|_{U \cap V})$
since we can glue the resulting sheaf $\mathcal{G}'$ over $V$
to the given sheaf $\mathcal{G}$ over $U$ along the common value
over $U \cap V$.
Thus we reduce to the case where $X$ is affine.

\medskip\noindent
Assume $X = \Spec(R)$. Write $\mathcal{F} = \widetilde M$
for some $R$-module $M$. By Lemma \ref{lemma-extend-trivial} above we may
find a quasi-coherent subsheaf $\mathcal{H} \subset \mathcal{F}$
which restricts to $\mathcal{G}$ over $U$.
Write $\mathcal{H} = \widetilde N$ for some $R$-module $N$.
For every $u \in U$ there exists an $f \in R$ such that
$u \in D(f) \subset U$ and such that $N_f$ is finitely generated,
see Lemma \ref{lemma-finite-type-module}.
Since $U$ is quasi-compact we can cover it by finitely
many $D(f_i)$ such that $N_{f_i}$ is generated by
finitely many elements, say $x_{i, 1}/f_i^N, \ldots, x_{i, r_i}/f_i^N$.
Let $N' \subset N$ be the submodule generated by the elements
$x_{i, j}$. Then the subsheaf
$\mathcal{G}' = \widetilde{N'} \subset \mathcal{H} \subset \mathcal{F}$
works.
\end{proof}

\begin{lemma}
\label{lemma-quasi-coherent-colimit-finite-type}
Let $X$ be a quasi-compact and quasi-separated scheme.
Any quasi-coherent sheaf of $\mathcal{O}_X$-modules
is the directed colimit of its quasi-coherent
$\mathcal{O}_X$-submodules which are of finite type.
\end{lemma}

\begin{proof}
The colimit is directed because if $\mathcal{G}_1$, $\mathcal{G}_2$
are quasi-coherent subsheaves of finite type, then the image of
$\mathcal{G}_1 \oplus \mathcal{G}_2 \to \mathcal{F}$ is
a quasi-coherent submodule of finite type.
Let $U \subset X$ be any affine open, and let
$s \in \Gamma(U, \mathcal{F})$ be any section.
Let $\mathcal{G} \subset \mathcal{F}|_U$ be the
subsheaf generated by $s$. Then clearly $\mathcal{G}$
is quasi-coherent and has finite type as an $\mathcal{O}_U$-module.
By Lemma \ref{lemma-extend} we see that $\mathcal{G}$ is the restriction
of a quasi-coherent subsheaf $\mathcal{G}' \subset \mathcal{F}$
which has finite type. Since $X$ has a basis for the topology consisting
of affine opens we conclude that every local section of
$\mathcal{F}$ is locally contained in a quasi-coherent submodule
of finite type. Thus we win.
\end{proof}

\begin{lemma}
\label{lemma-extend-finite-presentation}
Let $X$ be a quasi-compact and quasi-separated scheme.
Let $\mathcal{F}$ be a quasi-coherent $\mathcal{O}_X$-module.
Let $U \subset X$ be a quasi-compact open.
Let $\mathcal{G}$ be an $\mathcal{O}_U$-module which is of finite presentation.
Let $\varphi : \mathcal{G} \to \mathcal{F}|_U$ be a morphism of
$\mathcal{O}_U$-modules.
Then there exists an $\mathcal{O}_X$-module
$\mathcal{G}'$ of finite presentation, and a morphism
of $\mathcal{O}_X$-modules $\varphi' : \mathcal{G}' \to \mathcal{F}$
such that $\mathcal{G}'|_U = \mathcal{G}$ and such that
$\varphi'|_U = \varphi$.
\end{lemma}

\begin{proof}
The beginning of the proof is a repeat of the beginning of the
proof of Lemma \ref{lemma-extend}. We write it out carefully anyway.

\medskip\noindent
Let $n$ be the minimal number of affine opens $U_i \subset X$,
$i = 1, \ldots , n$ such that $X = U \cup \bigcup U_i$.
(Here we use that $X$ is quasi-compact.) Suppose
we can prove the lemma for the case $n = 1$. Then we can successively
extend the pair $(\mathcal{G}, \varphi)$
to a pair $(\mathcal{G}_1, \varphi_1)$ over $U \cup U_1$
to a pair $(\mathcal{G}_2, \varphi_2)$ over $U \cup U_1 \cup U_2$
to a pair $(\mathcal{G}_3, \varphi_3)$ over $U \cup U_1 \cup U_2 \cup U_3$,
and so on.
Thus we reduce to the case $n = 1$.

\medskip\noindent
Thus we may assume that $X = U \cup V$ with $V$ affine.
Since $X$ is quasi-separated and $U$ quasi-compact,
we see that $U \cap V \subset V$ is quasi-compact.
Suppose we prove the lemma for the system
$(V, U \cap V, \mathcal{F}|_V, \mathcal{G}|_{U \cap V}, \varphi|_{U \cap V})$
thereby producing $(\mathcal{G}', \varphi')$ over $V$.
Then we can glue $\mathcal{G}'$ over $V$ to the given sheaf $\mathcal{G}$
over $U$ along the common value over $U \cap V$, and similarly we can glue
the map $\varphi'$ to the map $\varphi$ along the common value over
$U \cap V$. Thus we reduce to the case where $X$ is affine.

\medskip\noindent
Assume $X = \Spec(R)$.
By Lemma \ref{lemma-extend-trivial} above we may
find a quasi-coherent sheaf $\mathcal{H}$ with
a map $\psi : \mathcal{H} \to \mathcal{F}$ over $X$
which restricts to $\mathcal{G}$ and $\varphi$ over $U$.
By Lemma \ref{lemma-extend} we can find a finite type
quasi-coherent $\mathcal{O}_X$-submodule
$\mathcal{H}' \subset \mathcal{H}$
such that $\mathcal{H}'|_U = \mathcal{G}$. Thus after
replacing $\mathcal{H}$ by $\mathcal{H}'$
and $\psi$ by the restriction of $\psi$ to $\mathcal{H}'$
we may assume that $\mathcal{H}$ is of finite type.
By Lemma \ref{lemma-finite-presentation-module}
we conclude that $\mathcal{H} = \widetilde{N}$ with
$N$ a finitely generated $R$-module. Hence there exists a surjection
as in the following short exact sequence of
quasi-coherent $\mathcal{O}_X$-modules
$$
0 \to \mathcal{K} \to \mathcal{O}_X^{\oplus n} \to \mathcal{H} \to 0
$$
where $\mathcal{K}$ is defined as the kernel.
Since $\mathcal{G}$ is of finite presentation and
$\mathcal{H}|_U = \mathcal{G}$ by
Modules, Lemma
\ref{modules-lemma-kernel-surjection-finite-free-onto-finite-presentation}
the restriction $\mathcal{K}|_U$ is
an $\mathcal{O}_U$-module of finite type. Hence by Lemma \ref{lemma-extend}
again we see that there exists a finite type quasi-coherent
$\mathcal{O}_X$-submodule $\mathcal{K}' \subset \mathcal{K}$ such
that $\mathcal{K}'|_U = \mathcal{K}|_U$. The solution to the problem
posed in the lemma is to set
$$
\mathcal{G}' = \mathcal{O}_X^{\oplus n}/\mathcal{K}'
$$
which is clearly of finite presentation and restricts to give $\mathcal{G}$
on $U$ with $\varphi'$ equal to the composition
$$
\mathcal{G}' = \mathcal{O}_X^{\oplus n}/\mathcal{K}'
\to \mathcal{O}_X^{\oplus n}/\mathcal{K} = \mathcal{H} \xrightarrow{\psi}
\mathcal{F}.
$$
This finishes the proof of the lemma.
\end{proof}

\begin{lemma}
\label{lemma-lift-finite-presentation}
Let $X$ be a quasi-compact and quasi-separated scheme. Let $U \subset X$
be a quasi-compact open. Let $\mathcal{G}$ be an $\mathcal{O}_U$-module.
\begin{enumerate}
\item If $\mathcal{G}$ is quasi-coherent and of finite type, then
there exists a quasi-coherent $\mathcal{O}_X$-module $\mathcal{G}'$
of finite type such that $\mathcal{G}'|_U = \mathcal{G}$.
\item If $\mathcal{G}$ is of finite presentation, then
there exists an $\mathcal{O}_X$-module $\mathcal{G}'$
of finite presentation such that $\mathcal{G}'|_U = \mathcal{G}$.
\end{enumerate}
\end{lemma}

\begin{proof}
Part (2) is the special case of Lemma \ref{lemma-extend-finite-presentation}
where $\mathcal{F} = 0$. For part (1) we first write
$\mathcal{G} = \mathcal{F}|_U$ for some quasi-coherent $\mathcal{O}_X$-module
by Lemma \ref{lemma-extend-trivial}
and then we apply Lemma \ref{lemma-extend} with $\mathcal{G} = \mathcal{F}|_U$.
\end{proof}

\noindent
The following lemma says that every quasi-coherent sheaf on a quasi-compact
and quasi-separated scheme is a filtered colimit of $\mathcal{O}$-modules
of finite presentation. Actually, we reformulate this in (perhaps more
familiar) terms of directed colimits over directed sets in the next lemma.

\begin{lemma}
\label{lemma-directed-colimit-diagram-finite-presentation}
\begin{slogan}
Quasi-coherent modules on quasi-compact and quasi-separated schemes
are filtered colimits of finitely presented modules.
\end{slogan}
Let $X$ be a scheme. Assume $X$ is quasi-compact and quasi-separated.
Let $\mathcal{F}$ be a quasi-coherent $\mathcal{O}_X$-module.
There exist
\begin{enumerate}
\item a filtered index category $\mathcal{I}$ (see
Categories, Definition \ref{categories-definition-directed}),
\item a diagram $\mathcal{I} \to \textit{Mod}(\mathcal{O}_X)$ (see
Categories, Section \ref{categories-section-limits}),
$i \mapsto \mathcal{F}_i$,
\item morphisms of $\mathcal{O}_X$-modules
$\varphi_i : \mathcal{F}_i \to \mathcal{F}$
\end{enumerate}
such that each $\mathcal{F}_i$ is of finite presentation
and such that the morphisms $\varphi_i$ induce an isomorphism
$$
\colim_i \mathcal{F}_i
=
\mathcal{F}.
$$
\end{lemma}

\begin{proof}
Choose a set $I$ and for each $i \in I$ an $\mathcal{O}_X$-module
of finite presentation and a homomorphism of $\mathcal{O}_X$-modules
$\varphi_i : \mathcal{F}_i \to \mathcal{F}$ with the following
property: For any $\psi : \mathcal{G} \to \mathcal{F}$ with $\mathcal{G}$
of finite presentation there is an $i \in I$ such that there exists
an isomorphism $\alpha : \mathcal{F}_i \to \mathcal{G}$ with
$\varphi_i = \psi \circ \alpha$. It is clear from
Modules, Lemma \ref{modules-lemma-set-isomorphism-classes-finite-type-modules}
that such a set exists (see also its proof).
We denote $\mathcal{I}$ the category
with $\Ob(\mathcal{I}) = I$ and given $i, i' \in I$
we set
$$
\Mor_\mathcal{I}(i, i') =
\{\alpha : \mathcal{F}_i \to \mathcal{F}_{i'} \mid
\alpha \circ \varphi_{i'} = \varphi_i
\}.
$$
We claim that $\mathcal{I}$ is a filtered category and that
$\mathcal{F} = \colim_i \mathcal{F}_i$.

\medskip\noindent
Let $i, i' \in I$. Then we can consider the morphism
$$
\mathcal{F}_i \oplus \mathcal{F}_{i'} \longrightarrow \mathcal{F}
$$
which is the direct sum of $\varphi_i$ and $\varphi_{i'}$.
Since a direct sum of finitely presented $\mathcal{O}_X$-modules
is finitely presented we see that there exists some $i'' \in I$
such that $\varphi_{i''} : \mathcal{F}_{i''} \to \mathcal{F}$
is isomorphic to the displayed arrow towards $\mathcal{F}$ above.
Since there are commutative diagrams
$$
\xymatrix{
\mathcal{F}_i \ar[r] \ar[d] & \mathcal{F} \ar@{=}[d] \\
\mathcal{F}_i \oplus \mathcal{F}_{i'} \ar[r] & \mathcal{F}
}
\quad
\text{and}
\quad
\xymatrix{
\mathcal{F}_{i'} \ar[r] \ar[d] & \mathcal{F} \ar@{=}[d] \\
\mathcal{F}_i \oplus \mathcal{F}_{i'} \ar[r] & \mathcal{F}
}
$$
we see that there are morphisms $i \to i''$ and $i' \to i''$
in $\mathcal{I}$. Next, suppose that we have $i, i' \in I$ and
morphisms $\alpha, \beta : i \to i'$ (corresponding to $\mathcal{O}_X$-module
maps $\alpha, \beta : \mathcal{F}_i \to \mathcal{F}_{i'}$).
In this case consider the coequalizer
$$
\mathcal{G} =
\Coker(
\mathcal{F}_i \xrightarrow{\alpha - \beta} \mathcal{F}_{i'}
)
$$
Note that $\mathcal{G}$ is an $\mathcal{O}_X$-module of finite presentation.
Since by definition of morphisms in the category $\mathcal{I}$
we have $\varphi_{i'} \circ \alpha = \varphi_{i'} \circ \beta$
we see that we get an induced map $\psi : \mathcal{G} \to \mathcal{F}$.
Hence again the pair $(\mathcal{G}, \psi)$ is isomorphic to
the pair $(\mathcal{F}_{i''}, \varphi_{i''})$ for some $i''$.
Hence we see that there exists a morphism $i' \to i''$ in
$\mathcal{I}$ which equalizes $\alpha$ and $\beta$. Thus we have
shown that the category $\mathcal{I}$ is filtered.

\medskip\noindent
We still have to show that the colimit of the diagram is $\mathcal{F}$.
By definition of the colimit, and by our definition of the category
$\mathcal{I}$ there is a canonical map
$$
\varphi :
\colim_i \mathcal{F}_i
\longrightarrow
\mathcal{F}.
$$
Pick $x \in X$. Let us show that $\varphi_x$ is an isomorphism.
Recall that
$$
(\colim_i \mathcal{F}_i)_x
=
\colim_i \mathcal{F}_{i, x},
$$
see
Sheaves, Section \ref{sheaves-section-limits-sheaves}.
First we show that the map $\varphi_x$ is injective.
Suppose that $s \in \mathcal{F}_{i, x}$ is an element
such that $s$ maps to zero in $\mathcal{F}_x$. Then there exists
a quasi-compact open $U$ such that $s$ comes from $s \in \mathcal{F}_i(U)$
and such that $\varphi_i(s) = 0$ in $\mathcal{F}(U)$.
By Lemma \ref{lemma-extend}
we can find a finite type quasi-coherent subsheaf
$\mathcal{K} \subset \Ker(\varphi_i)$ which restricts to
the quasi-coherent $\mathcal{O}_U$-submodule of $\mathcal{F}_i$
generated by $s$:
$\mathcal{K}|_U = \mathcal{O}_U\cdot s \subset \mathcal{F}_i|_U$.
Clearly, $\mathcal{F}_i/\mathcal{K}$ is of finite presentation and
the map $\varphi_i$ factors through the quotient map
$\mathcal{F}_i \to \mathcal{F}_i/\mathcal{K}$. Hence we can find
an $i' \in I$ and a morphism $\alpha : \mathcal{F}_i \to \mathcal{F}_{i'}$
in $\mathcal{I}$ which can be identified with the quotient map
$\mathcal{F}_i \to \mathcal{F}_i/\mathcal{K}$. Then it follows
that the section $s$ maps to zero in $\mathcal{F}_{i'}(U)$ and
in particular in
$(\colim_i \mathcal{F}_i)_x =
\colim_i \mathcal{F}_{i, x}$.
The injectivity follows.
Finally, we show that the map $\varphi_x$ is surjective.
Pick $s \in \mathcal{F}_x$. Choose a quasi-compact open neighbourhood
$U \subset X$ of $x$ such that $s$ corresponds to a section
$s \in \mathcal{F}(U)$. Consider the map
$s : \mathcal{O}_U \to \mathcal{F}$ (multiplication by $s$).
By Lemma \ref{lemma-extend-finite-presentation}
there exists an $\mathcal{O}_X$-module $\mathcal{G}$
of finite presentation and an $\mathcal{O}_X$-module map
$\mathcal{G} \to \mathcal{F}$ such that $\mathcal{G}|_U \to \mathcal{F}|_U$
is identified with
$s : \mathcal{O}_U \to \mathcal{F}$.
Again by definition of $\mathcal{I}$ there exists an $i \in I$
such that $\mathcal{G} \to \mathcal{F}$ is isomorphic to
$\varphi_i : \mathcal{F}_i \to \mathcal{F}$. Clearly there exists
a section $s' \in \mathcal{F}_i(U)$ mapping to $s \in \mathcal{F}(U)$.
This proves surjectivity and the proof of the lemma is complete.
\end{proof}

\begin{lemma}
\label{lemma-directed-colimit-finite-presentation}
Let $X$ be a scheme. Assume $X$ is quasi-compact and quasi-separated.
Let $\mathcal{F}$ be a quasi-coherent $\mathcal{O}_X$-module.
There exist
\begin{enumerate}
\item a directed set $I$ (see
Categories, Definition \ref{categories-definition-directed-set}),
\item a system $(\mathcal{F}_i, \varphi_{ii'})$
over $I$ in $\textit{Mod}(\mathcal{O}_X)$ (see
Categories, Definition \ref{categories-definition-system-over-poset})
\item morphisms of $\mathcal{O}_X$-modules
$\varphi_i : \mathcal{F}_i \to \mathcal{F}$
\end{enumerate}
such that each $\mathcal{F}_i$ is of finite presentation
and such that the morphisms $\varphi_i$ induce an isomorphism
$$
\colim_i \mathcal{F}_i
=
\mathcal{F}.
$$
\end{lemma}

\begin{proof}
This is a direct consequence of
Lemma \ref{lemma-directed-colimit-diagram-finite-presentation} and
Categories, Lemma \ref{categories-lemma-directed-category-system}
(combined with the fact that
colimits exist in the category of sheaves of $\mathcal{O}_X$-modules, see
Sheaves, Section \ref{sheaves-section-limits-sheaves}).
\end{proof}

\begin{lemma}
\label{lemma-finite-directed-colimit-surjective-maps}
Let $X$ be a scheme. Assume $X$ is quasi-compact and quasi-separated.
Let $\mathcal{F}$ be a finite type quasi-coherent $\mathcal{O}_X$-module.
Then we can write $\mathcal{F} = \colim \mathcal{F}_i$ with $\mathcal{F}_i$
of finite presentation and all transition maps
$\mathcal{F}_i \to \mathcal{F}_{i'}$ surjective.
\end{lemma}

\begin{proof}
Write $\mathcal{F} = \colim \mathcal{G}_i$ as a filtered colimit of
finitely presented $\mathcal{O}_X$-modules
(Lemma \ref{lemma-directed-colimit-finite-presentation}).
We claim that $\mathcal{G}_i \to \mathcal{F}$ is surjective for some $i$.
Namely, choose a finite affine open covering $X = U_1 \cup \ldots \cup U_m$.
Choose sections $s_{jl} \in \mathcal{F}(U_j)$ generating
$\mathcal{F}|_{U_j}$, see Lemma \ref{lemma-finite-type-module}.
By Sheaves, Lemma \ref{sheaves-lemma-directed-colimits-sections}
we see that $s_{jl}$ is in the image of $\mathcal{G}_i \to \mathcal{F}$
for $i$ large enough. Hence $\mathcal{G}_i \to \mathcal{F}$ is surjective
for $i$ large enough. Choose such an $i$ and let
$\mathcal{K} \subset \mathcal{G}_i$ be the kernel of the map
$\mathcal{G}_i \to \mathcal{F}$. Write $\mathcal{K} = \colim \mathcal{K}_a$
as the filtered colimit of its finite type quasi-coherent submodules
(Lemma \ref{lemma-quasi-coherent-colimit-finite-type}). Then
$\mathcal{F} = \colim \mathcal{G}_i/\mathcal{K}_a$ is a solution
to the problem posed by the lemma.
\end{proof}

\begin{lemma}
\label{lemma-application-directed-colimit}
Let $X$ be a quasi-compact and quasi-separated scheme.
Let $\mathcal{F}$ be a finite type quasi-coherent $\mathcal{O}_X$-module.
Let $U \subset X$ be a quasi-compact open such that $\mathcal{F}|_U$
is of finite presentation. Then there exists a map of $\mathcal{O}_X$-modules
$\varphi : \mathcal{G} \to \mathcal{F}$ with
(a) $\mathcal{G}$ of finite presentation,
(b) $\varphi$ is surjective, and
(c) $\varphi|_U$ is an isomorphism.
\end{lemma}

\begin{proof}
Write $\mathcal{F} = \colim \mathcal{F}_i$ as a directed colimit
with each $\mathcal{F}_i$ of finite presentation,
see Lemma \ref{lemma-directed-colimit-finite-presentation}.
Choose a finite affine open covering $X = \bigcup V_j$ and choose
finitely many sections $s_{jl} \in \mathcal{F}(V_j)$ generating
$\mathcal{F}|_{V_j}$, see Lemma \ref{lemma-finite-type-module}.
By Sheaves, Lemma \ref{sheaves-lemma-directed-colimits-sections}
we see that $s_{jl}$ is in the image of $\mathcal{F}_i \to \mathcal{F}$
for $i$ large enough. Hence $\mathcal{F}_i \to \mathcal{F}$ is surjective
for $i$ large enough. Choose such an $i$ and let
$\mathcal{K} \subset \mathcal{F}_i$ be the kernel of the map
$\mathcal{F}_i \to \mathcal{F}$. Since $\mathcal{F}_U$ is of finite
presentation, we see that $\mathcal{K}|_U$ is of finite type, see
Modules, Lemma
\ref{modules-lemma-kernel-surjection-finite-free-onto-finite-presentation}.
Hence we can find a finite type quasi-coherent submodule
$\mathcal{K}' \subset \mathcal{K}$ with $\mathcal{K}'|_U = \mathcal{K}|_U$,
see Lemma \ref{lemma-extend}. Then
$\mathcal{G} = \mathcal{F}_i/\mathcal{K}'$
with the given map $\mathcal{G} \to \mathcal{F}$ is a solution.
\end{proof}

\noindent
Let $X$ be a scheme. In the following lemma we use the notion
of a {\it quasi-coherent $\mathcal{O}_X$-algebra $\mathcal{A}$
of finite presentation}. This means that for every affine open
$\Spec(R) \subset X$ we have $\mathcal{A} = \widetilde{A}$
where $A$ is a (commutative) $R$-algebra which is of finite presentation
as an $R$-algebra.

\begin{lemma}
\label{lemma-algebra-directed-colimit-finite-presentation}
Let $X$ be a scheme. Assume $X$ is quasi-compact and quasi-separated.
Let $\mathcal{A}$ be a quasi-coherent $\mathcal{O}_X$-algebra.
There exist
\begin{enumerate}
\item a directed set $I$ (see
Categories, Definition \ref{categories-definition-directed-set}),
\item a system $(\mathcal{A}_i, \varphi_{ii'})$
over $I$ in the category of $\mathcal{O}_X$-algebras,
\item morphisms of $\mathcal{O}_X$-algebras
$\varphi_i : \mathcal{A}_i \to \mathcal{A}$
\end{enumerate}
such that each $\mathcal{A}_i$ is a quasi-coherent $\mathcal{O}_X$-algebra
of finite presentation and such that the morphisms $\varphi_i$
induce an isomorphism
$$
\colim_i \mathcal{A}_i
=
\mathcal{A}.
$$
\end{lemma}

\begin{proof}
First we write $\mathcal{A} = \colim_i \mathcal{F}_i$ as a directed
colimit of finitely presented quasi-coherent sheaves as in
Lemma \ref{lemma-directed-colimit-finite-presentation}.
For each $i$ let $\mathcal{B}_i = \text{Sym}(\mathcal{F}_i)$ be the
symmetric algebra on $\mathcal{F}_i$ over $\mathcal{O}_X$. Write
$\mathcal{I}_i = \Ker(\mathcal{B}_i \to \mathcal{A})$. Write
$\mathcal{I}_i = \colim_j \mathcal{F}_{i, j}$ where
$\mathcal{F}_{i, j}$ is a finite type quasi-coherent submodule of
$\mathcal{I}_i$, see
Lemma \ref{lemma-quasi-coherent-colimit-finite-type}.
Set $\mathcal{I}_{i, j} \subset \mathcal{I}_i$
equal to the $\mathcal{B}_i$-ideal generated by $\mathcal{F}_{i, j}$.
Set $\mathcal{A}_{i, j} = \mathcal{B}_i/\mathcal{I}_{i, j}$.
Then $\mathcal{A}_{i, j}$ is a quasi-coherent finitely presented
$\mathcal{O}_X$-algebra. Define $(i, j) \leq (i', j')$ if
$i \leq i'$ and the map $\mathcal{B}_i \to \mathcal{B}_{i'}$
maps the ideal $\mathcal{I}_{i, j}$ into the ideal $\mathcal{I}_{i', j'}$.
Then it is clear that $\mathcal{A} = \colim_{i, j} \mathcal{A}_{i, j}$.
\end{proof}

\noindent
Let $X$ be a scheme. In the following lemma we use the notion
of a {\it quasi-coherent $\mathcal{O}_X$-algebra $\mathcal{A}$
of finite type}. This means that for every affine open
$\Spec(R) \subset X$ we have $\mathcal{A} = \widetilde{A}$
where $A$ is a (commutative) $R$-algebra which is of finite type
as an $R$-algebra.

\begin{lemma}
\label{lemma-algebra-directed-colimit-finite-type}
Let $X$ be a scheme. Assume $X$ is quasi-compact and quasi-separated.
Let $\mathcal{A}$ be a quasi-coherent $\mathcal{O}_X$-algebra.
Then $\mathcal{A}$ is the directed colimit of its finite type
quasi-coherent $\mathcal{O}_X$-subalgebras.
\end{lemma}

\begin{proof}
If $\mathcal{A}_1, \mathcal{A}_2 \subset \mathcal{A}$ are
quasi-coherent $\mathcal{O}_X$-subalgebras of finite type, then the image of
$\mathcal{A}_1 \otimes_{\mathcal{O}_X} \mathcal{A}_2 \to \mathcal{A}$
is also a quasi-coherent $\mathcal{O}_X$-subalgebra of finite type
(some details omitted) which contains both $\mathcal{A}_1$ and $\mathcal{A}_2$.
In this way we see that the system is directed.
To show that $\mathcal{A}$ is the colimit of this system, write
$\mathcal{A} = \colim_i \mathcal{A}_i$ as a directed
colimit of finitely presented quasi-coherent $\mathcal{O}_X$-algebras as in
Lemma \ref{lemma-algebra-directed-colimit-finite-presentation}.
Then the images $\mathcal{A}'_i = \Im(\mathcal{A}_i \to \mathcal{A})$ are
quasi-coherent subalgebras of $\mathcal{A}$ of finite type. Since
$\mathcal{A}$ is the colimit of these the result follows.
\end{proof}

\noindent
Let $X$ be a scheme. In the following lemma we use the notion
of a {\it finite (resp.\ integral) quasi-coherent
$\mathcal{O}_X$-algebra $\mathcal{A}$}. This means that for every
affine open $\Spec(R) \subset X$ we have
$\mathcal{A} = \widetilde{A}$ where $A$ is a (commutative) $R$-algebra
which is finite (resp.\ integral) as an $R$-algebra.

\begin{lemma}
\label{lemma-finite-algebra-directed-colimit-finite-finitely-presented}
Let $X$ be a scheme. Assume $X$ is quasi-compact and quasi-separated.
Let $\mathcal{A}$ be a finite quasi-coherent $\mathcal{O}_X$-algebra.
Then $\mathcal{A} = \colim \mathcal{A}_i$ is a directed colimit of finite
and finitely presented quasi-coherent $\mathcal{O}_X$-algebras
such that all transition maps $\mathcal{A}_{i'} \to \mathcal{A}_i$
are surjective.
\end{lemma}

\begin{proof}
By Lemma \ref{lemma-finite-directed-colimit-surjective-maps}
there exists a finitely presented $\mathcal{O}_X$-module
$\mathcal{F}$ and a surjection $\mathcal{F} \to \mathcal{A}$.
Using the algebra structure we obtain a surjection
$$
\text{Sym}^*_{\mathcal{O}_X}(\mathcal{F}) \longrightarrow \mathcal{A}
$$
Denote $\mathcal{J}$ the kernel. Write $\mathcal{J} = \colim \mathcal{E}_i$
as a filtered colimit of finite type $\mathcal{O}_X$-submodules
$\mathcal{E}_i$ (Lemma \ref{lemma-quasi-coherent-colimit-finite-type}). Set
$$
\mathcal{A}_i = \text{Sym}^*_{\mathcal{O}_X}(\mathcal{F})/(\mathcal{E}_i)
$$
where $(\mathcal{E}_i)$ indicates the ideal sheaf generated by
the image of $\mathcal{E}_i \to \text{Sym}^*_{\mathcal{O}_X}(\mathcal{F})$.
Then each $\mathcal{A}_i$ is a finitely presented $\mathcal{O}_X$-algebra,
the transition maps are surjections,
and $\mathcal{A} = \colim \mathcal{A}_i$. To finish the proof we still
have to show that $\mathcal{A}_i$ is a finite $\mathcal{O}_X$-algebra
for $i$ sufficiently large. To do this we choose an affine open
covering $X = U_1 \cup \ldots \cup U_m$. Take generators
$f_{j, 1}, \ldots, f_{j, N_j} \in \Gamma(U_i, \mathcal{F})$.
As $\mathcal{A}(U_j)$ is a finite $\mathcal{O}_X(U_j)$-algebra we
see that for each $k$ there exists a monic polynomial
$P_{j, k} \in \mathcal{O}(U_j)[T]$ such that $P_{j, k}(f_{j, k})$
is zero in $\mathcal{A}(U_j)$. Since
$\mathcal{A} = \colim \mathcal{A}_i$ by construction, we
have $P_{j, k}(f_{j, k}) = 0$ in $\mathcal{A}_i(U_j)$
for all sufficiently large $i$. For such $i$ the algebras
$\mathcal{A}_i$ are finite.
\end{proof}

\begin{lemma}
\label{lemma-integral-algebra-directed-colimit-finite}
Let $X$ be a scheme. Assume $X$ is quasi-compact and quasi-separated.
Let $\mathcal{A}$ be an integral quasi-coherent $\mathcal{O}_X$-algebra.
Then
\begin{enumerate}
\item $\mathcal{A}$ is the directed colimit of its finite
quasi-coherent $\mathcal{O}_X$-subalgebras, and
\item $\mathcal{A}$ is a direct colimit of finite and finitely
presented quasi-coherent $\mathcal{O}_X$-algebras.
\end{enumerate}
\end{lemma}

\begin{proof}
By Lemma \ref{lemma-algebra-directed-colimit-finite-type} we have
$\mathcal{A} = \colim \mathcal{A}_i$ where
$\mathcal{A}_i \subset \mathcal{A}$ runs through the
quasi-coherent $\mathcal{O}_X$-algebras of finite type.
Any finite type quasi-coherent $\mathcal{O}_X$-subalgebra
of $\mathcal{A}$ is finite (apply Algebra, Lemma
\ref{algebra-lemma-characterize-finite-in-terms-of-integral}
to $\mathcal{A}_i(U) \subset \mathcal{A}(U)$ for affine opens $U$
in $X$). This proves (1).

\medskip\noindent
To prove (2), write $\mathcal{A} = \colim \mathcal{F}_i$
as a colimit of finitely presented $\mathcal{O}_X$-modules using
Lemma \ref{lemma-directed-colimit-finite-presentation}.
For each $i$, let $\mathcal{J}_i$ be the kernel of the map
$$
\text{Sym}^*_{\mathcal{O}_X}(\mathcal{F}_i) \longrightarrow \mathcal{A}
$$
For $i' \geq i$ there is an induced map $\mathcal{J}_i \to \mathcal{J}_{i'}$
and we have $\mathcal{A} =
\colim \text{Sym}^*_{\mathcal{O}_X}(\mathcal{F}_i)/\mathcal{J}_i$.
Moreover, the quasi-coherent $\mathcal{O}_X$-algebras
$\text{Sym}^*_{\mathcal{O}_X}(\mathcal{F}_i)/\mathcal{J}_i$
are finite (see above). Write $\mathcal{J}_i = \colim \mathcal{E}_{ik}$
as a colimit of finitely presented $\mathcal{O}_X$-modules.
Given $i' \geq i$ and $k$ there exists a $k'$ such that we
have a map $\mathcal{E}_{ik} \to \mathcal{E}_{i'k'}$
making
$$
\xymatrix{
\mathcal{J}_i \ar[r] & \mathcal{J}_{i'} \\
\mathcal{E}_{ik} \ar[u] \ar[r] & \mathcal{E}_{i'k'} \ar[u]
}
$$
commute. This follows from
Modules, Lemma \ref{modules-lemma-finite-presentation-quasi-compact-colimit}.
This induces a map
$$
\mathcal{A}_{ik} =
\text{Sym}^*_{\mathcal{O}_X}(\mathcal{F}_i)/(\mathcal{E}_{ik})
\longrightarrow
\text{Sym}^*_{\mathcal{O}_X}(\mathcal{F}_{i'})/(\mathcal{E}_{i'k'}) =
\mathcal{A}_{i'k'}
$$
where $(\mathcal{E}_{ik})$ denotes the ideal generated by $\mathcal{E}_{ik}$.
The quasi-coherent $\mathcal{O}_X$-algebras $\mathcal{A}_{ki}$
are of finite presentation and finite for $k$ large enough
(see proof of
Lemma \ref{lemma-finite-algebra-directed-colimit-finite-finitely-presented}).
Finally, we have
$$
\colim \mathcal{A}_{ik} = \colim \mathcal{A}_i = \mathcal{A}
$$
Namely, the first equality was shown in the proof of
Lemma \ref{lemma-finite-algebra-directed-colimit-finite-finitely-presented}
and the second equality because $\mathcal{A}$ is the colimit of
the modules $\mathcal{F}_i$.
\end{proof}





\section{Gabber's result}
\label{section-gabber}

\noindent
In this section we prove a result of Gabber which guarantees that on every
scheme there exists a cardinal $\kappa$ such that every quasi-coherent
module $\mathcal{F}$ is the union of its quasi-coherent
$\kappa$-generated subsheaves. It follows that the category of quasi-coherent
sheaves on a scheme is a Grothendieck abelian category having
limits and enough injectives\footnote{Nicely explained in a
\href{https://amathew.wordpress.com/2011/07/30/quasi-coherent-sheaves-presentable-categories-and-a-result-of-gabber/}{blog post}
by Akhil Mathew.}.

\begin{definition}
\label{definition-kappa-generated}
Let $(X, \mathcal{O}_X)$ be a ringed space. Let $\kappa$ be an infinite
cardinal. We say a sheaf of $\mathcal{O}_X$-modules $\mathcal{F}$ is
{\it $\kappa$-generated} if there exists an open covering
$X = \bigcup U_i$ such that $\mathcal{F}|_{U_i}$ is generated by
a subset $R_i \subset \mathcal{F}(U_i)$ whose cardinality is
at most $\kappa$.
\end{definition}

\noindent
Note that a direct sum of at most $\kappa$ $\kappa$-generated modules is
again $\kappa$-generated because $\kappa \otimes \kappa = \kappa$, see
Sets, Section \ref{sets-section-cardinals}.
In particular this holds for the direct sum of two $\kappa$-generated modules.
Moreover, a quotient of a $\kappa$-generated sheaf is $\kappa$-generated.
(But the same needn't be true for submodules.)

\begin{lemma}
\label{lemma-set-of-iso-classes}
Let $(X, \mathcal{O}_X)$ be a ringed space. Let $\kappa$ be a cardinal.
There exists a set $T$ and a family $(\mathcal{F}_t)_{t \in T}$ of
$\kappa$-generated $\mathcal{O}_X$-modules such that every $\kappa$-generated
$\mathcal{O}_X$-module is isomorphic to one of the $\mathcal{F}_t$.
\end{lemma}

\begin{proof}
There is a set of coverings of $X$ (provided we disallow repeats).
Suppose $X = \bigcup U_i$ is a covering and suppose $\mathcal{F}_i$
is an $\mathcal{O}_{U_i}$-module. Then there is a set of isomorphism
classes of $\mathcal{O}_X$-modules $\mathcal{F}$ with the property
that $\mathcal{F}|_{U_i} \cong \mathcal{F}_i$ since there is a set of
glueing maps. This reduces us to proving there is a set of (isomorphism
classes of) quotients
$\oplus_{k \in \kappa} \mathcal{O}_X \to \mathcal{F}$
for any ringed space $X$. This is clear.
\end{proof}

\noindent
Here is the result the title of this section refers to.

\begin{lemma}
\label{lemma-colimit-kappa}
Let $X$ be a scheme. There exists a cardinal $\kappa$ such that
every quasi-coherent module $\mathcal{F}$ is the directed colimit
of its quasi-coherent $\kappa$-generated submodules.
\end{lemma}

\begin{proof}
Choose an affine open covering $X = \bigcup_{i \in I} U_i$. For each pair
$i, j$ choose an affine open covering
$U_i \cap U_j = \bigcup_{k \in I_{ij}} U_{ijk}$.
Write $U_i = \Spec(A_i)$ and $U_{ijk} = \Spec(A_{ijk})$.
Let $\kappa$ be any infinite cardinal $\geq$ than the cardinality
of any of the sets $I$, $I_{ij}$.

\medskip\noindent
Let $\mathcal{F}$ be a quasi-coherent sheaf. Set $M_i = \mathcal{F}(U_i)$
and $M_{ijk} = \mathcal{F}(U_{ijk})$. Note that
$$
M_i \otimes_{A_i} A_{ijk} = M_{ijk} = M_j \otimes_{A_j} A_{ijk}.
$$
see
Schemes, Lemma \ref{schemes-lemma-widetilde-pullback}.
Using the axiom of choice we choose a map
$$
(i, j, k, m) \mapsto S(i, j, k, m)
$$
which associates to every $i, j \in I$, $k \in I_{ij}$ and $m \in M_i$
a finite subset $S(i, j, k, m) \subset M_j$ such that we have
$$
m \otimes 1 = \sum\nolimits_{m' \in S(i, j, k, m)} m' \otimes a_{m'}
$$
in $M_{ijk}$ for some $a_{m'} \in A_{ijk}$. Moreover, let's agree
that $S(i, i, k, m) = \{m\}$ for all $i, j = i, k, m$ as above.
Fix such a map.

\medskip\noindent
Given a family $\mathcal{S} = (S_i)_{i \in I}$ of subsets
$S_i \subset M_i$ of cardinality at most $\kappa$ we set
$\mathcal{S}' = (S'_i)$ where
$$
S'_j = \bigcup\nolimits_{(i, k, m)\text{ such that }m \in S_i}
S(i, j, k, m)
$$
Note that $S_i \subset S'_i$. Note that $S'_i$ has cardinality at most
$\kappa$ because it is a union over a set of cardinality at most $\kappa$
of finite sets. Set $\mathcal{S}^{(0)} = \mathcal{S}$,
$\mathcal{S}^{(1)} = \mathcal{S}'$ and by induction
$\mathcal{S}^{(n + 1)} = (\mathcal{S}^{(n)})'$. Then set
$\mathcal{S}^{(\infty)} = \bigcup_{n \geq 0} \mathcal{S}^{(n)}$.
Writing $\mathcal{S}^{(\infty)} = (S^{(\infty)}_i)$ we see that
for any element $m \in S^{(\infty)}_i$ the image of $m$ in
$M_{ijk}$ can be written as a finite sum $\sum m' \otimes a_{m'}$
with $m' \in S_j^{(\infty)}$. In this way we see that setting
$$
N_i = A_i\text{-submodule of }M_i\text{ generated by }S^{(\infty)}_i
$$
we have
$$
N_i \otimes_{A_i} A_{ijk} = N_j \otimes_{A_j} A_{ijk}.
$$
as submodules of $M_{ijk}$. Thus there exists a quasi-coherent subsheaf
$\mathcal{G} \subset \mathcal{F}$ with $\mathcal{G}(U_i) = N_i$.
Moreover, by construction the sheaf $\mathcal{G}$ is $\kappa$-generated.

\medskip\noindent
Let $\{\mathcal{G}_t\}_{t \in T}$ be the set of $\kappa$-generated
quasi-coherent subsheaves. If $t, t' \in T$ then
$\mathcal{G}_t + \mathcal{G}_{t'}$ is also a $\kappa$-generated
quasi-coherent subsheaf as it is the image of the map
$\mathcal{G}_t \oplus \mathcal{G}_{t'} \to \mathcal{F}$.
Hence the system (ordered by inclusion) is directed.
The arguments above show that every section of $\mathcal{F}$ over $U_i$
is in one of the $\mathcal{G}_t$ (because we can start with $\mathcal{S}$
such that the given section is an element of $S_i$). Hence
$\colim_t \mathcal{G}_t \to \mathcal{F}$ is both injective and surjective
as desired.
\end{proof}

\begin{proposition}
\label{proposition-coherator}
Let $X$ be a scheme.
\begin{enumerate}
\item The category $\QCoh(\mathcal{O}_X)$ is a Grothendieck
abelian category. Consequently, $\QCoh(\mathcal{O}_X)$
has enough injectives and all limits.
\item The inclusion functor
$\QCoh(\mathcal{O}_X) \to \textit{Mod}(\mathcal{O}_X)$
has a right adjoint\footnote{This functor is sometimes called
the {\it coherator}.}
$$
Q : \textit{Mod}(\mathcal{O}_X) \longrightarrow \QCoh(\mathcal{O}_X)
$$
such that for every quasi-coherent sheaf $\mathcal{F}$ the adjunction mapping
$Q(\mathcal{F}) \to \mathcal{F}$ is an isomorphism.
\end{enumerate}
\end{proposition}

\begin{proof}
Part (1) means $\QCoh(\mathcal{O}_X)$ (a) has all colimits,
(b) filtered colimits are exact, and (c) has a generator, see
Injectives, Section \ref{injectives-section-grothendieck-conditions}.
By Schemes, Section \ref{schemes-section-quasi-coherent}
colimits in $\QCoh(\mathcal{O}_X)$ exist and agree
with colimits in $\textit{Mod}(\mathcal{O}_X)$. By
Modules, Lemma \ref{modules-lemma-limits-colimits}
filtered colimits are exact. Hence (a) and (b) hold.
To construct a generator $U$, pick a cardinal $\kappa$ as in
Lemma \ref{lemma-colimit-kappa}. Pick a collection
$(\mathcal{F}_t)_{t \in T}$ of $\kappa$-generated quasi-coherent sheaves as in
Lemma \ref{lemma-set-of-iso-classes}. Set
$U = \bigoplus_{t \in T} \mathcal{F}_t$. Since every object of
$\QCoh(\mathcal{O}_X)$ is a filtered colimit of $\kappa$-generated
quasi-coherent modules, i.e., of objects isomorphic to $\mathcal{F}_t$,
it is clear that $U$ is a generator.
The assertions on limits and injectives hold in any
Grothendieck abelian category, see
Injectives, Theorem
\ref{injectives-theorem-injective-embedding-grothendieck} and
Lemma \ref{injectives-lemma-grothendieck-products}.

\medskip\noindent
Proof of (2). To construct $Q$ we use the following general procedure.
Given an object $\mathcal{F}$ of $\textit{Mod}(\mathcal{O}_X)$
we consider the functor
$$
\QCoh(\mathcal{O}_X)^{opp} \longrightarrow \textit{Sets},\quad
\mathcal{G} \longmapsto \Hom_X(\mathcal{G}, \mathcal{F})
$$
This functor transforms colimits into limits,
hence is representable, see
Injectives, Lemma \ref{injectives-lemma-grothendieck-brown}.
Thus there exists a quasi-coherent sheaf $Q(\mathcal{F})$
and a functorial isomorphism
$\Hom_X(\mathcal{G}, \mathcal{F}) = \Hom_X(\mathcal{G}, Q(\mathcal{F}))$
for $\mathcal{G}$ in $\QCoh(\mathcal{O}_X)$. By the Yoneda lemma
(Categories, Lemma \ref{categories-lemma-yoneda})
the construction $\mathcal{F} \leadsto Q(\mathcal{F})$ is
functorial in $\mathcal{F}$. By construction $Q$ is a right
adjoint to the inclusion functor.
The fact that $Q(\mathcal{F}) \to \mathcal{F}$ is an isomorphism
when $\mathcal{F}$ is quasi-coherent is a formal consequence of the fact
that the inclusion functor
$\QCoh(\mathcal{O}_X) \to \textit{Mod}(\mathcal{O}_X)$
is fully faithful.
\end{proof}







\section{Sections with support in a closed subset}
\label{section-sections-with-support-in-closed}

\noindent
Given any topological space $X$, a closed subset $Z \subset X$, and an
abelian sheaf $\mathcal{F}$ you can take the subsheaf of sections whose
support is contained in $Z$. If $X$ is a scheme, $Z$ a closed
subscheme, and $\mathcal{F}$ a quasi-coherent module there is a variant
where you take sections which are scheme theoretically supported
on $Z$. However, in the scheme setting you have to be careful because
the resulting $\mathcal{O}_X$-module may not be quasi-coherent.

\begin{lemma}
\label{lemma-quasi-coherent-finite-type-ideals}
Let $X$ be a quasi-compact and quasi-separated scheme.
Let $U \subset X$ be an open subscheme. The following are equivalent:
\begin{enumerate}
\item $U$ is retrocompact in $X$,
\item $U$ is quasi-compact,
\item $U$ is a finite union of affine opens, and
\item there exists a finite type quasi-coherent sheaf of ideals
$\mathcal{I} \subset \mathcal{O}_X$ such that $X \setminus U = V(\mathcal{I})$
(set theoretically).
\end{enumerate}
\end{lemma}

\begin{proof}
The equivalence of (1), (2), and (3) follows from
Lemma \ref{lemma-quasi-separated-quasi-compact-open-retrocompact}.
Assume (1), (2), (3). Let $T = X \setminus U$. By
Schemes, Lemma \ref{schemes-lemma-reduced-closed-subscheme} there exists
a unique quasi-coherent sheaf of ideals $\mathcal{J}$ cutting
out the reduced induced closed subscheme structure on $T$.
Note that $\mathcal{J}|_U = \mathcal{O}_U$ which is an
$\mathcal{O}_U$-modules of finite type.
By Lemma \ref{lemma-extend} there exists a quasi-coherent subsheaf
$\mathcal{I} \subset \mathcal{J}$ which is of finite type
and has the property that $\mathcal{I}|_U = \mathcal{J}|_U$.
Then $X \setminus U = V(\mathcal{I})$ and we obtain (4). Conversely,
if $\mathcal{I}$ is as in (4) and $W = \Spec(R) \subset X$ is an affine
open, then $\mathcal{I}|_W = \widetilde{I}$ for some finitely generated
ideal $I \subset R$, see Lemma \ref{lemma-finite-type-module}.
It follows that $U \cap W = \Spec(R) \setminus V(I)$ is quasi-compact,
see Algebra, Lemma \ref{algebra-lemma-qc-open}. Hence $U \subset X$
is retrocompact by Lemma \ref{lemma-retrocompact}.
\end{proof}

\begin{lemma}
\label{lemma-sections-annihilated-by-ideal}
Let $X$ be a scheme.
Let $\mathcal{I} \subset \mathcal{O}_X$ be a quasi-coherent sheaf of ideals.
Let $\mathcal{F}$ be a quasi-coherent $\mathcal{O}_X$-module.
Consider the sheaf of $\mathcal{O}_X$-modules $\mathcal{F}'$
which associates to every open $U \subset X$
$$
\mathcal{F}'(U)
=
\{s \in \mathcal{F}(U) \mid
\mathcal{I}s = 0\}
$$
Assume $\mathcal{I}$ is of finite type. Then
\begin{enumerate}
\item $\mathcal{F}'$ is a quasi-coherent sheaf of $\mathcal{O}_X$-modules,
\item on any affine open $U \subset X$ we have
$\mathcal{F}'(U) = \{s \in \mathcal{F}(U) \mid \mathcal{I}(U)s = 0\}$, and
\item $\mathcal{F}'_x = \{s \in \mathcal{F}_x \mid \mathcal{I}_x s = 0\}$.
\end{enumerate}
\end{lemma}

\begin{proof}
It is clear that the rule defining $\mathcal{F}'$ gives a subsheaf
of $\mathcal{F}$ (the sheaf condition is easy to verify). Hence we
may work locally on $X$ to verify the other statements. In other words
we may assume that $X = \Spec(A)$, $\mathcal{F} = \widetilde{M}$
and $\mathcal{I} = \widetilde{I}$. It is clear that in this case
$\mathcal{F}'(U) = \{x \in M \mid Ix = 0\} =: M'$ because $\widetilde{I}$
is generated by its global sections $I$ which proves (2).
To show $\mathcal{F}'$ is quasi-coherent it suffices to show that
for every $f \in A$ we have
$\{x \in M_f \mid I_f x = 0\} = (M')_f$.
Write $I = (g_1, \ldots, g_t)$, which is possible because $\mathcal{I}$
is of finite type, see Lemma \ref{lemma-finite-type-module}.
If $x = y/f^n$ and $I_fx = 0$, then that means that for every $i$
there exists an $m \geq 0$ such that $f^mg_ix = 0$.
We may choose one $m$ which works for all $i$ (and this is where we
use that $I$ is finitely generated). Then we see that $f^mx \in M'$
and $x/f^n = f^mx/f^{n + m}$ in $(M')_f$ as desired.
The proof of (3) is similar and omitted.
\end{proof}

\begin{definition}
\label{definition-subsheaf-sections-annihilated-by-ideal}
Let $X$ be a scheme.
Let $\mathcal{I} \subset \mathcal{O}_X$ be a quasi-coherent sheaf of ideals
of finite type.
Let $\mathcal{F}$ be a quasi-coherent $\mathcal{O}_X$-module.
The subsheaf $\mathcal{F}' \subset \mathcal{F}$ defined in
Lemma \ref{lemma-sections-annihilated-by-ideal} above is called
the {\it subsheaf of sections annihilated by $\mathcal{I}$}.
\end{definition}

\begin{lemma}
\label{lemma-push-sections-annihilated-by-ideal}
Let $f : X \to Y$ be a quasi-compact and quasi-separated morphism
of schemes. Let $\mathcal{I} \subset \mathcal{O}_Y$ be a quasi-coherent
sheaf of ideals of finite type. Let $\mathcal{F}$ be a quasi-coherent
$\mathcal{O}_X$-module. Let $\mathcal{F}' \subset \mathcal{F}$
be the subsheaf of sections annihilated by $f^{-1}\mathcal{I}\mathcal{O}_X$.
Then $f_*\mathcal{F}' \subset f_*\mathcal{F}$ is the subsheaf
of sections annihilated by $\mathcal{I}$.
\end{lemma}

\begin{proof}
Omitted. (Hint: The assumption that $f$ is quasi-compact and
quasi-separated implies that $f_*\mathcal{F}$ is quasi-coherent
so that Lemma \ref{lemma-sections-annihilated-by-ideal} applies
to $\mathcal{I}$ and $f_*\mathcal{F}$.)
\end{proof}

\noindent
For an abelian sheaf on a topological space we have discussed the subsheaf
of sections with support in a closed subset in
Modules, Remark \ref{modules-remark-sections-support-in-closed}.
For quasi-coherent modules this submodule isn't always a quasi-coherent
module, but if the closed subset has a retrocompact complement, then
it is.

\begin{lemma}
\label{lemma-sections-supported-on-closed-subset}
Let $X$ be a scheme. Let $Z \subset X$ be a closed subset.
Let $\mathcal{F}$ be a quasi-coherent $\mathcal{O}_X$-module.
Consider the sheaf of $\mathcal{O}_X$-modules $\mathcal{F}'$
which associates to every open $U \subset X$
$$
\mathcal{F}'(U)
=
\{s \in \mathcal{F}(U) \mid
\text{the support of }s\text{ is contained in }Z \cap U\}
$$
If $X \setminus Z$ is a retrocompact open of $X$, then
\begin{enumerate}
\item for an affine open $U \subset X$ there exist a finitely generated
ideal $I \subset \mathcal{O}_X(U)$ such that $Z \cap U = V(I)$,
\item for $U$ and $I$ as in (1) we have
$\mathcal{F}'(U) = \{x \in \mathcal{F}(U) \mid
I^nx = 0 \text{ for some } n\}$,
\item $\mathcal{F}'$ is a quasi-coherent sheaf of $\mathcal{O}_X$-modules.
\end{enumerate}
\end{lemma}

\begin{proof}
Part (1) is Algebra, Lemma \ref{algebra-lemma-qc-open}.
Let $U = \Spec(A)$ and $I$ be as in (1).
Then $\mathcal{F}|_U$ is the quasi-coherent sheaf associated to
some $A$-module $M$. We have
$$
\mathcal{F}'(U) = \{x \in M \mid x = 0\text{ in }M_\mathfrak p
\text{ for all }\mathfrak p \not \in Z\}.
$$
by Modules, Definition \ref{modules-definition-support}. Thus
$x \in \mathcal{F}'(U)$ if and only if $V(\text{Ann}(x)) \subset V(I)$, see
Algebra, Lemma \ref{algebra-lemma-support-element}. Since $I$ is
finitely generated this is equivalent to $I^n x = 0$ for some $n$.
This proves (2).

\medskip\noindent
Proof of (3). Observe that given $U \subset X$ open there is an exact sequence
$$
0 \to \mathcal{F}'(U) \to \mathcal{F}(U) \to \mathcal{F}(U \setminus Z)
$$
If we denote $j : X \setminus Z \to X$ the inclusion morphism, then
we observe that $\mathcal{F}(U \setminus Z)$ is the sections of the
module $j_*(\mathcal{F}|_{X \setminus Z})$ over $U$. Thus we have
an exact sequence
$$
0 \to \mathcal{F}' \to \mathcal{F} \to j_*(\mathcal{F}|_{X \setminus Z})
$$
The restriction $\mathcal{F}|_{X \setminus Z}$ is quasi-coherent.
Hence $j_*(\mathcal{F}|_{X \setminus Z})$ is quasi-coherent
by Schemes, Lemma \ref{schemes-lemma-push-forward-quasi-coherent}
and our assumption that $j$ is quasi-compact (any open immersion
is separated). Hence $\mathcal{F}'$ is quasi-coherent
as a kernel of a map of quasi-coherent modules, see
Schemes, Section \ref{schemes-section-quasi-coherent}.
\end{proof}

\begin{definition}
\label{definition-subsheaf-sections-supported-on-closed}
Let $X$ be a scheme.
Let $T \subset X$ be a closed subset whose complement
is retrocompact in $X$.
Let $\mathcal{F}$ be a quasi-coherent $\mathcal{O}_X$-module.
The quasi-coherent subsheaf $\mathcal{F}' \subset \mathcal{F}$ defined in
Lemma \ref{lemma-sections-supported-on-closed-subset} is called
the {\it subsheaf of sections supported on $T$}.
\end{definition}

\begin{lemma}
\label{lemma-push-sections-supported-on-closed-subset}
Let $f : X \to Y$ be a quasi-compact and quasi-separated morphism
of schemes. Let $Z \subset Y$ be a closed subset such that
$Y \setminus Z$ is retrocompact in $Y$. Let $\mathcal{F}$ be a quasi-coherent
$\mathcal{O}_X$-module. Let $\mathcal{F}' \subset \mathcal{F}$
be the subsheaf of sections supported in $f^{-1}Z$.
Then $f_*\mathcal{F}' \subset f_*\mathcal{F}$ is the subsheaf
of sections supported in $Z$.
\end{lemma}

\begin{proof}
Omitted. (Hint: First show that $X \setminus f^{-1}Z$ is retrocompact
in $X$ as $Y \setminus Z$ is retrocompact in $Y$. Hence
Lemma \ref{lemma-sections-supported-on-closed-subset}
applies to $f^{-1}Z$ and $\mathcal{F}$. As $f$ is quasi-compact and
quasi-separated we see that $f_*\mathcal{F}$ is quasi-coherent.
Hence Lemma \ref{lemma-sections-supported-on-closed-subset}
applies to $Z$ and $f_*\mathcal{F}$. Finally, match the sheaves
directly.)
\end{proof}








\section{Sections of quasi-coherent sheaves}
\label{section-sections-quasi-coherent}

\noindent
Here is a computation of sections of a quasi-coherent sheaf on a quasi-compact
open of an affine spectrum.

\begin{lemma}
\label{lemma-sections-over-quasi-compact-open-in-affine}
Let $A$ be a ring.
Let $I \subset A$ be a finitely generated ideal.
Let $M$ be an $A$-module.
Then there is a canonical map
$$
\colim_n \Hom_A(I^n, M)
\longrightarrow
\Gamma(\Spec(A) \setminus V(I), \widetilde{M}).
$$
This map is always injective.
If for all $x \in M$ we have $Ix = 0 \Rightarrow x = 0$
then this map is an isomorphism. In general, set
$M_n = \{x \in M \mid I^nx = 0\}$, then there is an
isomorphism
$$
\colim_n \Hom_A(I^n, M/M_n)
\longrightarrow
\Gamma(\Spec(A) \setminus V(I), \widetilde{M}).
$$
\end{lemma}

\begin{proof}
Since $I^{n + 1} \subset I^n$ and $M_n \subset M_{n + 1}$ we can
use composition via these maps to get canonical maps of $A$-modules
$$
\Hom_A(I^n, M)
\longrightarrow
\Hom_A(I^{n + 1}, M)
$$
and
$$
\Hom_A(I^n, M/M_n)
\longrightarrow
\Hom_A(I^{n + 1}, M/M_{n + 1})
$$
which we will use as the transition maps in the systems. Given an
$A$-module map $\varphi : I^n \to M$, then we get a map of
sheaves $\widetilde{\varphi} : \widetilde{I^n} \to \widetilde{M}$
which we can restrict to the open $\Spec(A) \setminus V(I)$.
Since $\widetilde{I^n}$ restricted to this open gives the structure
sheaf we get an element of
$\Gamma(\Spec(A) \setminus V(I), \widetilde{M})$.
We omit the verification that this is compatible with the transition maps
in the system $\Hom_A(I^n, M)$. This gives the first arrow.
To get the second arrow we note that
$\widetilde{M}$ and $\widetilde{M/M_n}$ agree over the open
$\Spec(A) \setminus V(I)$ since the sheaf $\widetilde{M_n}$
is clearly supported on $V(I)$. Hence we can use the same mechanism
as before.

\medskip\noindent
Next, we work out how to define this arrow in terms of algebra.
Say $I = (f_1, \ldots, f_t)$. Then
$\Spec(A) \setminus V(I) = \bigcup_{i = 1, \ldots, t} D(f_i)$.
Hence
$$
0 \to
\Gamma(\Spec(A) \setminus V(I), \widetilde{M}) \to
\bigoplus\nolimits_i M_{f_i} \to
\bigoplus\nolimits_{i, j} M_{f_if_j}
$$
is exact. Suppose that $\varphi : I^n \to M$ is an $A$-module map.
Consider the vector of elements $\varphi(f_i^n)/f_i^n \in M_{f_i}$.
It is easy to see that this vector maps to zero in the
second direct sum of the exact sequence above. Whence an element
of $\Gamma(\Spec(A) \setminus V(I), \widetilde{M})$.
We omit the verification that this description agrees with the one
given above.

\medskip\noindent
Let us show that the first arrow is injective using this description.
Namely, if $\varphi$ maps to zero, then for each $i$ the element
$\varphi(f_i^n)/f_i^n$ is zero in $M_{f_i}$. In other words we
see that for each $i$ we have $f_i^m\varphi(f_i^n) = 0$ for some $m \geq 0$.
We may choose a single $m$ which works for all $i$. Then we see that
$\varphi(f_i^{n + m}) = 0$ for all $i$. It is easy to see that
this means that $\varphi|_{I^{t(n + m - 1) + 1}} = 0$ in other
words that $\varphi$ maps to zero in the $t(n + m - 1) + 1$st
term of the colimit. Hence injectivity follows.

\medskip\noindent
Note that each $M_n = 0$ in case we have
$Ix = 0 \Rightarrow x = 0$ for $x \in M$. Thus
to finish the proof of the lemma it suffices to show that
the second arrow is an isomorphism.

\medskip\noindent
Let us attempt to construct an inverse of the second map of the lemma.
Let $s \in \Gamma(\Spec(A) \setminus V(I), \widetilde{M})$.
This corresponds to a vector $x_i/f_i^n$ with $x_i \in M$ of the
first direct sum of the exact sequence above.
Hence for each $i, j$ there exists $m \geq 0$
such that $f_i^m f_j^m (f_j^n x_i - f_i^n x_j) = 0$ in $M$.
We may choose a single $m$ which works for all pairs $i, j$.
After replacing $x_i$ by $f_i^mx_i$ and $n$ by $n + m$ we see
that we get $f_j^nx_i = f_i^nx_j$ in $M$ for all $i, j$.
Let us introduce
$$
K_n = \{x \in M \mid f_1^nx = \ldots = f_t^nx = 0\}
$$
We claim there is an $A$-module map
$$
\varphi :
I^{t(n - 1) + 1}
\longrightarrow
M/K_n
$$
which maps the monomial
$f_1^{e_1} \ldots f_t^{e_t}$ with $\sum e_i = t(n - 1) + 1$
to the class modulo $K_n$ of the expression
$f_1^{e_1} \ldots f_i^{e_i - n} \ldots f_t^{e_t}x_i$
where $i$ is chosen such that $e_i \geq n$ (note that there
is at least one such $i$).
To see that this is indeed the case suppose that
$$
\sum\nolimits_{E = (e_1, \ldots, e_t), |E| = t(n - 1) + 1}
a_E f_1^{e_1} \ldots f_t^{e_t} = 0
$$
is a relation between the monomials with coefficients $a_E$ in $A$.
Then we would map this to
$$
z =
\sum\nolimits_{E = (e_1, \ldots, e_t), |E| = t(n - 1) + 1}
a_E f_1^{e_1} \ldots f_{i(E)}^{e_{i(E)} - n} \ldots f_t^{e_t}x_{i(E)}
$$
where for each multiindex $E$ we have chosen a particular $i(E)$
such that $e_{i(E)} \geq n$.
Note that if we multiply this by $f_j^n$ for any $j$, then
we get zero, since by the relations $f_j^nx_i = f_i^nx_j$ above we get
\begin{align*}
f_j^nz & = \sum\nolimits_{E = (e_1, \ldots, e_t), |E| = t(n - 1) + 1}
a_E f_1^{e_1} \ldots f_j^{e_j + n}
\ldots f_{i(E)}^{e_{i(E)} - n} \ldots f_t^{e_t}x_{i(E)} \\
& =
\sum\nolimits_{E = (e_1, \ldots, e_t), |E| = t(n - 1) + 1}
a_E f_1^{e_1} \ldots f_t^{e_t}x_j
= 0.
\end{align*}
Hence $z \in K_n$ and we see that every relation gets mapped to zero
in $M/K_n$. This proves the claim.

\medskip\noindent
Note that $K_n \subset M_{t(n - 1) + 1}$. Hence the
map $\varphi$ in particular gives rise to an $A$-module map
$I^{t(n - 1) + 1} \to M/M_{t(n - 1) + 1}$.
This proves the second arrow of the lemma is surjective.
We omit the proof of injectivity.
\end{proof}

\begin{example}
\label{example-does-not-work-in-general}
We will give two examples showing that the first displayed map of
Lemma \ref{lemma-sections-over-quasi-compact-open-in-affine}
is not an isomorphism.

\medskip\noindent
Let $k$ be a field. Consider the ring
$$
A = k[x, y, z_1, z_2, \ldots]/(x^nz_n).
$$
Set $I = (x)$ and let $M = A$. Then the element $y/x$ defines a section of
the structure sheaf of $\Spec(A)$ over $D(x) = \Spec(A) \setminus V(I)$.
We claim that $y/x$ is not in the image of the canonical map
$\colim \Hom_A(I^n, A) \to A_x = \mathcal{O}(D(x))$. Namely, if so it
would come from a homomorphism $\varphi : I^n \to A$ for some $n$.
Set $a = \varphi(x^n)$. Then we would have $x^m(xa - x^ny) = 0$ for some
$m > 0$. This would mean that $x^{m + 1}a = x^{m + n}y$. This would mean that
$\varphi(x^{n + m + 1}) = x^{m + n}y$. This leads to a contradiction
because it would imply that
$$
0 = \varphi(0) = \varphi(z_{n + m + 1} x^{n + m + 1}) =
x^{m + n}y z_{n + m + 1}
$$
which is not true in the ring $A$.

\medskip\noindent
Let $k$ be a field. Consider the ring
$$
A = k[f, g, x, y, \{a_n, b_n\}_{n \geq 1}]/
(fy - gx, \{a_nf^n + b_ng^n\}_{n \geq 1}).
$$
Set $I = (f, g)$ and let $M = A$. Then $x/f \in A_f$ and $y/g \in A_g$
map to the same element of $A_{fg}$. Hence these define a section $s$
of the structure sheaf of $\Spec(A)$ over
$D(f) \cup D(g) = \Spec(A) \setminus V(I)$. However, there is no
$n \geq 0$ such that $s$ comes from an $A$-module map
$\varphi : I^n \to A$ as in the source of the first displayed arrow
of Lemma \ref{lemma-sections-over-quasi-compact-open-in-affine}.
Namely, given such a module map set
$x_n = \varphi(f^n)$ and $y_n = \varphi(g^n)$.
Then $f^mx_n = f^{n + m - 1}x$ and
$g^my_n = g^{n + m - 1}y$ for some $m \geq 0$ (see proof
of the lemma). But then we would have
$0 = \varphi(0) =
\varphi(a_{n + m}f^{n + m} + b_{n + m}g^{n + m}) =
a_{n + m}f^{n + m - 1}x + b_{n + m}g^{n + m - 1}y$ which is not the case
in the ring $A$.
\end{example}

\noindent
We will improve on the following lemma in the Noetherian case, see
Cohomology of Schemes, Lemma \ref{coherent-lemma-homs-over-open}.

\begin{lemma}
\label{lemma-sections-over-quasi-compact-open}
Let $X$ be a quasi-compact scheme.
Let $\mathcal{I} \subset \mathcal{O}_X$ be a
quasi-coherent sheaf of ideals of finite type.
Let $Z \subset X$ be the closed subscheme
defined by $\mathcal{I}$ and set $U = X \setminus Z$.
Let $\mathcal{F}$ be a quasi-coherent $\mathcal{O}_X$-module.
The canonical map
$$
\colim_n \Hom_{\mathcal{O}_X}(\mathcal{I}^n,
\mathcal{F})
\longrightarrow
\Gamma(U, \mathcal{F})
$$
is injective. Assume further that $X$ is quasi-separated.
Let $\mathcal{F}_n \subset \mathcal{F}$
be subsheaf of sections annihilated by $\mathcal{I}^n$.
The canonical map
$$
\colim_n \Hom_{\mathcal{O}_X}(\mathcal{I}^n,
\mathcal{F}/\mathcal{F}_n)
\longrightarrow
\Gamma(U, \mathcal{F})
$$
is an isomorphism.
\end{lemma}

\begin{proof}
Let $\Spec(A) = W \subset X$ be an affine open. Write
$\mathcal{F}|_W = \widetilde{M}$ for some $A$-module $M$
and $\mathcal{I}|_W = \widetilde{I}$ for some finite type
ideal $I \subset A$. Restricting the first displayed map
of the lemma to $W$ we obtain the first displayed map of
Lemma \ref{lemma-sections-over-quasi-compact-open-in-affine}.
Since we can cover $X$ by a finite number of affine opens this
proves the first displayed map of the lemma is injective.

\medskip\noindent
We have $\mathcal{F}_n|_W = \widetilde{M_n}$ where
$M_n \subset M$ is defined as in
Lemma \ref{lemma-sections-over-quasi-compact-open-in-affine}
(details omitted). The lemma guarantees that we have a bijection
$$
\colim_n  \Hom_{\mathcal{O}_W}(
\mathcal{I}^n|_W, (\mathcal{F}/\mathcal{F}_n)|_W)
\longrightarrow
\Gamma(U \cap W, \mathcal{F})
$$
for any such affine open $W$.

\medskip\noindent
To see the second displayed arrow of the lemma is bijective,
we choose a finite affine open covering $X = \bigcup_{j = 1, \ldots, m} W_j$.
The injectivity follows immediately from the above and the finiteness of
the covering. If $X$ is quasi-separated, then for each pair
$j, j'$ we choose a finite affine open covering
$$
W_j \cap W_{j'} = \bigcup\nolimits_{k = 1, \ldots, m_{jj'}} W_{jj'k}.
$$
Let $s \in \Gamma(U, \mathcal{F})$. As seen above for each $j$ there exists
an $n_j$ and a map
$\varphi_j : \mathcal{I}^{n_j}|_{W_j} \to
(\mathcal{F}/\mathcal{F}_{n_j})|_{W_j}$
which corresponds to $s|_{U \cap W_j}$.
By the same token for each triple $(j, j', k)$ there exists an integer
$n_{jj'k}$ such that the restriction of $\varphi_j$ and $\varphi_{j'}$
as maps $\mathcal{I}^{n_{jj'k}} \to \mathcal{F}/\mathcal{F}_{n_{jj'k}}$
agree over $W_{jj'k}$. Let $n = \max\{n_j, n_{jj'k}\}$ and we see that
the $\varphi_j$ glue as maps
$\mathcal{I}^n \to \mathcal{F}/\mathcal{F}_n$ over $X$.
This proves surjectivity of the map.
\end{proof}













\section{Ample invertible sheaves}
\label{section-ample}

\noindent
Recall from Modules, Lemma \ref{modules-lemma-s-open}
that given an invertible sheaf $\mathcal{L}$ on a locally ringed
space $X$, and given a global section $s$ of $\mathcal{L}$
the set $X_s = \{x \in X \mid s \not \in \mathfrak m_x\mathcal{L}_x\}$
is open. A general remark is that
$X_s \cap X_{s'} = X_{ss'}$, where $ss'$ denote
the section $s \otimes s' \in \Gamma(X, \mathcal{L} \otimes \mathcal{L}')$.

\begin{definition}
\label{definition-ample}
\begin{reference}
\cite[II Definition 4.5.3]{EGA}
\end{reference}
Let $X$ be a scheme.
Let $\mathcal{L}$ be an invertible $\mathcal{O}_X$-module.
We say $\mathcal{L}$ is {\it ample} if
\begin{enumerate}
\item $X$ is quasi-compact, and
\item for every $x \in X$ there exists an $n \geq 1$
and $s \in \Gamma(X, \mathcal{L}^{\otimes n})$ such
that $x \in X_s$ and $X_s$ is affine.
\end{enumerate}
\end{definition}

\begin{lemma}
\label{lemma-ample-power-ample}
\begin{reference}
\cite[II Proposition 4.5.6(i)]{EGA}
\end{reference}
Let $X$ be a scheme. Let $\mathcal{L}$ be an invertible $\mathcal{O}_X$-module.
Let $n \geq 1$. Then $\mathcal{L}$ is ample if and only if
$\mathcal{L}^{\otimes n}$ is ample.
\end{lemma}

\begin{proof}
This follows from the fact that $X_{s^n} = X_s$.
\end{proof}

\begin{lemma}
\label{lemma-ample-on-closed}
Let $X$ be a scheme.
Let $\mathcal{L}$ be an ample invertible $\mathcal{O}_X$-module.
For any closed subscheme $Z \subset X$ the restriction of
$\mathcal{L}$ to $Z$ is ample.
\end{lemma}

\begin{proof}
This is clear since a closed subset of a quasi-compact space is quasi-compact
and a closed subscheme of an affine scheme is affine (see
Schemes, Lemma \ref{schemes-lemma-closed-immersion-affine-case}).
\end{proof}

\begin{lemma}
\label{lemma-affine-cap-s-open}
Let $X$ be a scheme. Let $\mathcal{L}$ be an invertible $\mathcal{O}_X$-module.
Let $s \in \Gamma(X, \mathcal{L})$. For any affine $U \subset X$
the intersection $U \cap X_s$ is affine.
\end{lemma}

\begin{proof}
This translates into the following algebra problem.
Let $R$ be a ring. Let $N$ be an invertible $R$-module
(i.e., locally free of rank 1). Let $s \in N$ be an element.
Then $V = \{\mathfrak p \mid s \not \in \mathfrak p N\}$ is
an affine open subset of $\Spec(R)$.

\medskip\noindent
Let $A = \bigoplus_{n \geq 0} A_n$ be the symmetric algebra of $N$
(which is commutative) and view $s$ as an element of $A_1$.
Set $B = A/(s - 1)A$. This is an $R$-algebra whose construction
commutes with any base change $R \to R'$. Thus $B' = B \otimes_R R'$
is the zero ring if $s$ maps to zero in $N' = N \otimes_R R'$. It
follows that if $x \in \Spec(R) \setminus V$, then $B \otimes_R \kappa(x) = 0$.
We conclude that $\Spec(B) \to \Spec(R)$ factors through $V$ as the
fibres over $x \not \in V$ are empty. On the other hand, if
$\Spec(R') \subset V$ is an affine open, then $s$ maps to a basis
element of $N'$ and we see that $B' = R'[s]/(s - 1) \cong R'$.
It follows that $\Spec(B) \to V$ is an isomorphism and $V$ is
indeed affine.
\end{proof}

\begin{lemma}
\label{lemma-ample-tensor-globally-generated}
\begin{reference}
\cite[II Proposition 4.5.6(ii)]{EGA}
\end{reference}
Let $X$ be a scheme. Let $\mathcal{L}$ and $\mathcal{M}$
be invertible $\mathcal{O}_X$-modules. If
\begin{enumerate}
\item $\mathcal{L}$ is ample, and
\item the open sets $X_t$ where $t \in \Gamma(X, \mathcal{M}^{\otimes m})$
for $m > 0$ cover $X$,
\end{enumerate}
then $\mathcal{L} \otimes \mathcal{M}$ is ample.
\end{lemma}

\begin{proof}
We check the conditions of Definition \ref{definition-ample}.
As $\mathcal{L}$ is ample we see that $X$ is quasi-compact.
Let $x \in X$. Choose $n \geq 1$, $m \geq 1$,
$s \in \Gamma(X, \mathcal{L}^{\otimes n})$, and
$t \in \Gamma(X, \mathcal{M}^{\otimes m})$
such that $x \in X_s$, $x \in X_t$ and $X_s$ is affine.
Then $s^mt^n \in \Gamma(X, (\mathcal{L} \otimes \mathcal{M})^{\otimes nm})$,
$x \in X_{s^mt^n}$, and $X_{s^mt^n}$ is affine by
Lemma \ref{lemma-affine-cap-s-open}.
\end{proof}

\begin{lemma}
\label{lemma-affine-s-opens}
Let $X$ be a scheme. Let $\mathcal{L}$ be an invertible $\mathcal{O}_X$-module.
Assume the open sets $X_s$, where $s \in \Gamma(X, \mathcal{L}^{\otimes n})$
and $n \geq 1$, form a basis for the topology on $X$.
Then among those opens, the open sets $X_s$ which are affine
form a basis for the topology on $X$.
\end{lemma}

\begin{proof}
Let $x \in X$. Choose an affine open neighbourhood
$\Spec(R) = U \subset X$ of $x$.
By assumption, there exists
a $n \geq 1$ and a $s \in \Gamma(X, \mathcal{L}^{\otimes n})$
such that $X_s \subset U$. By Lemma \ref{lemma-affine-cap-s-open} above
the intersection $X_s = U \cap X_s$ is affine.
Since $U$ can be chosen arbitrarily small we win.
\end{proof}

\begin{lemma}
\label{lemma-affine-s-opens-cover-quasi-separated}
Let $X$ be a scheme and $\mathcal{L}$ be an invertible $\mathcal{O}_X$-module.
Assume for every point $x$ of $X$ there exists $n \geq 1$ and
$s \in \Gamma(X, \mathcal{L}^{\otimes n})$ such that
$x \in X_s$ and $X_s$ is affine. Then $X$ is separated.
\end{lemma}

\begin{proof}
We show first that $X$ is quasi-separated.  By assumption we
can find a covering of $X$ by affine opens of the form
$X_s$.  By Lemma \ref{lemma-affine-cap-s-open}, the
intersection of any two such sets is affine, so Schemes, Lemma
\ref{schemes-lemma-characterize-quasi-separated} implies
that $X$ is quasi-separated.

\medskip\noindent
To show that $X$ is separated, we can use the valuative
criterion, Schemes, Lemma
\ref{schemes-lemma-valuative-criterion-separatedness}.
Thus, let $A$ be a valuation ring with fraction field $K$
and consider two morphisms $f, g : \Spec(A) \to X$ such that
the two compositions $\Spec(K) \to \Spec(A) \to X$ agree.
As $A$ is local, there exists $p, q \ge 1$, $s \in \Gamma(X,
\mathcal{L}^{\otimes p})$, and $t \in \Gamma(X,
\mathcal{L}^{\otimes q})$ such that $X_s$ and $X_t$ are
affine, $f(\Spec A) \subseteq X_s$, and $g(\Spec A)
\subseteq X_t$.  We now replace $s$ by $s^q$, $t$ by $t^p$,
and $\mathcal{L}$ by $\mathcal{L}^{\otimes pq}$.  This is
harmless as $X_s = X_{s^q}$ and $X_t = X_{t^p}$, and now $s$
and $t$ are both sections of the same sheaf $\mathcal{L}$.

\medskip\noindent
The quasi-coherent module $f^*\mathcal{L}$ corresponds to an $A$-module $M$ and
$g^*\mathcal{L}$ corresponds to an $A$-module $N$ by our
classification of quasi-coherent modules over affine schemes
(Schemes, Lemma \ref{schemes-lemma-quasi-coherent-affine}).
The $A$-modules $M$ and $N$ are locally free of rank
$1$ (Lemma \ref{lemma-locally-free-module}) and as $A$ is
local they are free (Algebra, Lemma
\ref{algebra-lemma-K0-local}).  Therefore we may identify
$M$ and $N$ with $A$-submodules of $M \otimes_A K$ and $N
\otimes_A K$.  The equality $f|_{\Spec(K)} = g|_{\Spec(K)}$
determines an isomorphism $\phi \colon M \otimes_A K \to N
\otimes_A K$.

\medskip\noindent
Let $x \in M$ and $y \in N$ be the elements corresponding to
the pullback of $s$ along $f$ and $g$, respectively.  These
satisfy $\phi(x \otimes 1) = y \otimes 1$.  The image of $f$
is contained in $X_s$, so $x \not\in \mathfrak{m}_A M$, that
is, $x$ generates $M$.  Hence $\phi$ determines an
isomorphism of $M$ with the submodule of $N$ generated by
$y$.  Arguing symmetrically using $t$, $\phi^{-1}$
determines an isomorphism of $N$ with a submodule of $M$.
Consequently $\phi$ restricts to an isomorphism of $M$ and
$N$.  Since $x$ generates $M$, its image $y$ generates
$N$, implying $y \not\in \mathfrak{m}_A N$.  Therefore
$g(\Spec(A)) \subseteq X_s$.  Because $X_s$ is affine, it is
separated by Schemes, Lemma \ref{schemes-lemma-affine-separated},
and we conclude $f = g$.
\end{proof}

\begin{lemma}
\label{lemma-ample-separated}
Let $X$ be a scheme. If there exists an ample invertible sheaf on $X$
then $X$ is separated.
\end{lemma}

\begin{proof}
Follows immediately from
Lemma \ref{lemma-affine-s-opens-cover-quasi-separated} and
Definition \ref{definition-ample}.
\end{proof}

\begin{lemma}
\label{lemma-map-into-proj}
Let $X$ be a scheme.
Let $\mathcal{L}$ be an invertible $\mathcal{O}_X$-module.
Set $S = \Gamma_*(X, \mathcal{L})$ as a graded ring.
If every point of $X$ is contained in one of the
open subschemes $X_s$, for some $s \in S_{+}$ homogeneous, then
there is a canonical morphism of schemes
$$
f : X \longrightarrow Y = \text{Proj}(S),
$$
to the homogeneous spectrum of $S$ (see
Constructions, Section \ref{constructions-section-proj}).
This morphism has the following properties
\begin{enumerate}
\item $f^{-1}(D_{+}(s)) = X_s$ for any $s \in S_{+}$ homogeneous,
\item there are $\mathcal{O}_X$-module maps
$f^*\mathcal{O}_Y(n) \to \mathcal{L}^{\otimes n}$
compatible with multiplication maps, see
Constructions, Equation (\ref{constructions-equation-multiply}),
\item the composition
$S_n \to \Gamma(Y, \mathcal{O}_Y(n)) \to \Gamma(X, \mathcal{L}^{\otimes n})$
is the identity map, and
\item for every $x \in X$ there is an integer $d \geq 1$
and an open neighbourhood $U \subset X$ of $x$
such that $f^*\mathcal{O}_Y(dn)|_U \to \mathcal{L}^{\otimes dn}|_U$
is an isomorphism for all $n \in \mathbf{Z}$.
\end{enumerate}
\end{lemma}

\begin{proof}
Denote $\psi : S \to \Gamma_*(X, \mathcal{L})$ the identity map.
We are going to use the triple
$(U(\psi), r_{\mathcal{L}, \psi}, \theta)$ of
Constructions, Lemma \ref{constructions-lemma-invertible-map-into-proj}.
By assumption the open subscheme $U(\psi)$ of equals $X$. Hence
$r_{\mathcal{L}, \psi} : U(\psi) \to Y$ is defined on all of $X$.
We set $f = r_{\mathcal{L}, \psi}$.
The maps in part (2) are the components of $\theta$.
Part (3) follows from condition (2) in the lemma cited above.
Part (1) follows from (3) combined with condition (1) in the lemma
cited above. Part (4) follows from the last statement in
Constructions, Lemma \ref{constructions-lemma-invertible-map-into-proj}
since the map $\alpha$ mentioned there is an isomorphism.
\end{proof}

\begin{lemma}
\label{lemma-map-into-proj-quasi-compact}
Let $X$ be a scheme. Let $\mathcal{L}$ be an invertible $\mathcal{O}_X$-module.
Set $S = \Gamma_*(X, \mathcal{L})$.
Assume (a) every point of $X$ is contained in one of the
open subschemes $X_s$, for some $s \in S_{+}$ homogeneous,
and (b) $X$ is quasi-compact. Then the canonical morphism of schemes
$f : X \longrightarrow \text{Proj}(S)$ of Lemma \ref{lemma-map-into-proj}
above is quasi-compact with dense image.
\end{lemma}

\begin{proof}
To prove $f$ is quasi-compact it suffices to show that $f^{-1}(D_{+}(s))$
is quasi-compact for any $s \in S_{+}$ homogeneous. Write
$X = \bigcup_{i = 1, \ldots, n} X_i$ as a finite union of
affine opens. By Lemma \ref{lemma-affine-cap-s-open} each intersection
$X_s \cap X_i$ is affine. Hence $X_s = \bigcup_{i = 1, \ldots, n} X_s \cap X_i$
is quasi-compact. Assume that the image of $f$ is not dense to get
a contradiction. Then, since the opens $D_+(s)$ with $s \in S_+$ homogeneous
form a basis for the topology on $\text{Proj}(S)$, we can find such
an $s$ with $D_+(s) \not = \emptyset$ and $f(X) \cap D_+(s) = \emptyset$.
By Lemma \ref{lemma-map-into-proj}
this means $X_s = \emptyset$. By Lemma \ref{lemma-invert-s-sections}
this means that a power $s^n$ is the zero section of
$\mathcal{L}^{\otimes n\deg(s)}$.
This in turn means that $D_+(s) = \emptyset$ which is the
desired contradiction.
\end{proof}

\begin{lemma}
\label{lemma-ample-immersion-into-proj}
Let $X$ be a scheme. Let $\mathcal{L}$ be an invertible $\mathcal{O}_X$-module.
Set $S = \Gamma_*(X, \mathcal{L})$.
Assume $\mathcal{L}$ is ample. Then the canonical morphism of schemes
$f : X \longrightarrow \text{Proj}(S)$ of Lemma \ref{lemma-map-into-proj}
is an open immersion with dense image.
\end{lemma}

\begin{proof}
By Lemma \ref{lemma-affine-s-opens-cover-quasi-separated} we see
that $X$ is quasi-separated. Choose finitely many
$s_1, \ldots, s_n \in S_{+}$ homogeneous
such that $X_{s_i}$ are affine, and $X = \bigcup X_{s_i}$.
Say $s_i$ has degree $d_i$. The inverse image of
$D_{+}(s_i)$ under $f$ is $X_{s_i}$, see Lemma \ref{lemma-map-into-proj}.
By Lemma \ref{lemma-invert-s-sections} the ring map
$$
(S^{(d_i)})_{(s_i)} = \Gamma(D_{+}(s_i), \mathcal{O}_{\text{Proj}(S)})
\longrightarrow
\Gamma(X_{s_i}, \mathcal{O}_X)
$$
is an isomorphism. Hence $f$ induces an isomorphism
$X_{s_i} \to D_{+}(s_i)$. Thus $f$ is an isomorphism of $X$ onto the open
subscheme $\bigcup_{i = 1, \ldots, n} D_{+}(s_i)$ of $\text{Proj}(S)$.
The image is dense by Lemma \ref{lemma-map-into-proj-quasi-compact}.
\end{proof}

\begin{lemma}
\label{lemma-open-in-proj-ample}
Let $X$ be a scheme.
Let $S$ be a graded ring. Assume $X$ is quasi-compact,
and assume there exists an open immersion
$$
j : X \longrightarrow Y = \text{Proj}(S).
$$
Then $j^*\mathcal{O}_Y(d)$ is an invertible ample sheaf
for some $d > 0$.
\end{lemma}

\begin{proof}
This is Constructions, Lemma \ref{constructions-lemma-ample-on-proj}.
\end{proof}

\begin{proposition}
\label{proposition-characterize-ample}
Let $X$ be a quasi-compact scheme.
Let $\mathcal{L}$ be an invertible sheaf on $X$.
Set $S = \Gamma_*(X, \mathcal{L})$.
The following are equivalent:
\begin{enumerate}
\item
\label{item-ample}
$\mathcal{L}$ is ample,
\item
\label{item-immersion}
the open sets $X_s$, with $s \in S_{+}$ homogeneous,
cover $X$ and the associated morphism $X \to \text{Proj}(S)$
is an open immersion,
\item
\label{item-s-basis}
the open sets $X_s$, with $s \in S_{+}$ homogeneous,
form a basis for the topology of $X$,
\item
\label{item-s-affine-basis}
the open sets $X_s$, with $s \in S_{+}$ homogeneous,
which are affine form a basis for the topology of $X$,
\item
\label{item-qc-gg}
for every quasi-coherent sheaf $\mathcal{F}$ on $X$
the sum of the images of the canonical maps
$$
\Gamma(X, \mathcal{F} \otimes_{\mathcal{O}_X} \mathcal{L}^{\otimes n})
\otimes_{\mathbf{Z}} \mathcal{L}^{\otimes -n}
\longrightarrow
\mathcal{F}
$$
with $n \geq 1$ equals $\mathcal{F}$,
\item
\label{item-qc-i-gg}
same property as (\ref{item-qc-gg}) with $\mathcal{F}$
ranging over all quasi-coherent sheaves of ideals,
\item
\label{item-c-gg}
$X$ is quasi-separated and
for every quasi-coherent sheaf $\mathcal{F}$ of finite type on $X$
there exists an integer $n_0$ such that
$\mathcal{F} \otimes_{\mathcal{O}_X} \mathcal{L}^{\otimes n}$
is globally generated for all $n \geq n_0$,
\item
\label{item-c-q}
$X$ is quasi-separated and
for every quasi-coherent sheaf $\mathcal{F}$ of finite type on $X$
there exist integers $n > 0$, $k \geq 0$ such that
$\mathcal{F}$ is a quotient of a direct sum of $k$ copies of
$\mathcal{L}^{\otimes - n}$, and
\item
\label{item-c-i-q}
same as in (\ref{item-c-q}) with $\mathcal{F}$ ranging over all
sheaves of ideals of finite type on $X$.
\end{enumerate}
\end{proposition}

\begin{proof}
Lemma \ref{lemma-ample-immersion-into-proj} is
(\ref{item-ample}) $\Rightarrow$ (\ref{item-immersion}).
Lemmas \ref{lemma-ample-power-ample} and \ref{lemma-open-in-proj-ample}
provide the implication
(\ref{item-ample}) $\Leftarrow$ (\ref{item-immersion}).
The implications (\ref{item-immersion}) $\Rightarrow$
(\ref{item-s-affine-basis}) $\Rightarrow$ (\ref{item-s-basis})
are clear from Constructions, Section \ref{constructions-section-proj}.
Lemma \ref{lemma-affine-s-opens} is
(\ref{item-s-basis}) $\Rightarrow$ (\ref{item-ample}).
Thus we see that the first 4 conditions are all equivalent.

\medskip\noindent
Assume the equivalent conditions (1) -- (4).
Note that in particular $X$ is separated (as an open
subscheme of the separated scheme $\text{Proj}(S)$).
Let $\mathcal{F}$ be a quasi-coherent sheaf on $X$.
Choose $s \in S_{+}$ homogeneous such that $X_s$ is affine.
We claim that any section $m \in \Gamma(X_s, \mathcal{F})$
is in the image of one of the maps displayed in
(\ref{item-qc-gg}) above. This will imply (\ref{item-qc-gg})
since these affines $X_s$ cover $X$.
Namely, by Lemma \ref{lemma-invert-s-sections} we may write
$m$ as the image of $m' \otimes s^{-n}$ for some
$n \geq 1$, some
$m' \in \Gamma(X, \mathcal{F} \otimes \mathcal{L}^{\otimes n})$.
This proves the claim.

\medskip\noindent
Clearly (\ref{item-qc-gg}) $\Rightarrow$ (\ref{item-qc-i-gg}).
Let us assume (\ref{item-qc-i-gg}) and prove $\mathcal{L}$ is
ample. Pick $x \in X$. Let $U \subset X$ be an affine open
which contains $x$. Set $Z = X \setminus U$. We may think of
$Z$ as a reduced closed subscheme, see
Schemes, Section \ref{schemes-section-reduced}.
Let $\mathcal{I} \subset \mathcal{O}_X$ be the quasi-coherent
sheaf of ideals corresponding to the closed subscheme $Z$.
By assumption (\ref{item-qc-i-gg}), there exists an $n \geq 1$ and a section
$s \in \Gamma(X, \mathcal{I} \otimes \mathcal{L}^{\otimes n})$
such that $s$ does not vanish at $x$ (more precisely such that
$s \not \in \mathfrak m_x \mathcal{I}_x \otimes \mathcal{L}_x^{\otimes n}$).
We may think of $s$ as a section of $\mathcal{L}^{\otimes n}$.
Since it clearly vanishes along $Z$ we see that
$X_s \subset U$. Hence $X_s$ is affine, see
Lemma \ref{lemma-affine-cap-s-open}.
This proves that $\mathcal{L}$ is ample.
At this point we have proved that (1) -- (6) are equivalent.

\medskip\noindent
Assume the equivalent conditions (1) -- (6). In the following
we will use the fact that the tensor product of two sheaves of
modules which are globally generated is globally generated without
further mention (see
Modules, Lemma \ref{modules-lemma-tensor-product-globally-generated}).
By (1) we can find elements $s_i \in S_{d_i}$ with $d_i \geq 1$
such that $X = \bigcup_{i = 1, \ldots, n} X_{s_i}$.
Set $d = d_1\ldots d_n$. It follows that $\mathcal{L}^{\otimes d}$
is globally generated by
$$
s_1^{d/d_1}, \ldots, s_n^{d/d_n}.
$$
This means that if $\mathcal{L}^{\otimes j}$ is globally generated
then so is $\mathcal{L}^{\otimes j + dn}$ for all $n \geq 0$.
Fix a $j \in \{0, \ldots, d - 1\}$. For any point $x \in X$ there
exists an $n \geq 1$ and a global section $s$ of $\mathcal{L}^{j + dn}$
which does not vanish at $x$, as follows from (\ref{item-qc-gg}) applied
to $\mathcal{F} = \mathcal{L}^{\otimes j}$ and ample invertible
sheaf $\mathcal{L}^{\otimes d}$. Since $X$ is quasi-compact there
we may find a finite list of integers $n_i$ and global sections
$s_i$ of $\mathcal{L}^{\otimes j + dn_i}$ which do not vanish at any point
of $X$. Since $\mathcal{L}^{\otimes d}$ is globally generated this means that
$\mathcal{L}^{\otimes j + dn}$ is globally generated where $n = \max\{n_i\}$.
Since we proved this for every congruence class mod $d$ we
conclude that there exists an $n_0 = n_0(\mathcal{L})$ such that
$\mathcal{L}^{\otimes n}$ is globally generated for all $n \geq n_0$.
At this point we see that if $\mathcal{F}$ is globally generated then
so is $\mathcal{F} \otimes \mathcal{L}^{\otimes n}$ for all
$n \geq n_0$.

\medskip\noindent
We continue to assume the equivalent conditions (1) -- (6).
Let $\mathcal{F}$ be a quasi-coherent
sheaf of $\mathcal{O}_X$-modules of finite type.
Denote $\mathcal{F}_n \subset \mathcal{F}$ the image of the canonical
map of (\ref{item-qc-gg}). By construction
$\mathcal{F}_n \otimes \mathcal{L}^{\otimes n}$ is
globally generated. By (\ref{item-qc-gg}) we see
$\mathcal{F}$ is the sum of the subsheaves $\mathcal{F}_n$,
$n \geq 1$. By
Modules, Lemma \ref{modules-lemma-finite-type-quasi-compact-colimit}
we see that $\mathcal{F} = \sum_{n = 1, \ldots, N} \mathcal{F}_n$
for some $N \geq 1$. It follows that
$\mathcal{F} \otimes \mathcal{L}^{\otimes n}$ is globally
generated whenever $n \geq N + n_0(\mathcal{L})$ with $n_0(\mathcal{L})$
as above. We conclude that (1) -- (6) implies (\ref{item-c-gg}).

\medskip\noindent
Assume (\ref{item-c-gg}). Let $\mathcal{F}$ be a quasi-coherent
sheaf of $\mathcal{O}_X$-modules of finite type.
By (\ref{item-c-gg}) there exists an integer $n \geq 1$ such that
the canonical map
$$
\Gamma(X, \mathcal{F} \otimes_{\mathcal{O}_X} \mathcal{L}^{\otimes n})
\otimes_{\mathbf{Z}} \mathcal{L}^{\otimes -n}
\longrightarrow
\mathcal{F}
$$
is surjective. Let $I$ be the set of finite subsets of
$\Gamma(X, \mathcal{F} \otimes_{\mathcal{O}_X} \mathcal{L}^{\otimes n})$
partially ordered by inclusion. Then $I$ is a directed partially ordered set.
For $i = \{s_1, \ldots, s_{r(i)}\}$ let $\mathcal{F}_i \subset \mathcal{F}$
be the image of the map
$$
\bigoplus\nolimits_{j = 1, \ldots, r(i)} \mathcal{L}^{\otimes -n}
\longrightarrow
\mathcal{F}
$$
which is multiplication by $s_j$ on the $j$th factor. The surjectivity above
implies that $\mathcal{F} = \colim_{i \in I} \mathcal{F}_i$.
Hence Modules, Lemma \ref{modules-lemma-finite-type-quasi-compact-colimit}
applies and we conclude that
$\mathcal{F} = \mathcal{F}_i$ for some $i$.
Hence we have proved (\ref{item-c-q}). In other words,
(\ref{item-c-gg}) $\Rightarrow$ (\ref{item-c-q}).

\medskip\noindent
The implication (\ref{item-c-q}) $\Rightarrow$ (\ref{item-c-i-q}) is trivial.

\medskip\noindent
Finally, assume (\ref{item-c-i-q}).
Let $\mathcal{I} \subset \mathcal{O}_X$ be a quasi-coherent sheaf
of ideals. By Lemma \ref{lemma-quasi-coherent-colimit-finite-type}
(this is where we use the condition that $X$ be quasi-separated)
we see that $\mathcal{I} = \colim_\alpha I_\alpha$ with
each $I_\alpha$ quasi-coherent of finite type. Since by assumption each of
the $I_\alpha$ is a quotient of negative tensor powers of
$\mathcal{L}$ we conclude the same for $\mathcal{I}$ (but of course
without the finiteness or boundedness of the powers). Hence
we conclude that (\ref{item-c-i-q}) implies (\ref{item-qc-i-gg}).
This ends the proof of the proposition.
\end{proof}

\begin{lemma}
\label{lemma-ample-on-locally-closed}
Let $X$ be a scheme. Let $\mathcal{L}$ be an ample invertible
$\mathcal{O}_X$-module. Let $i : X' \to X$ be a morphism of schemes.
Assume at least one of the following conditions holds
\begin{enumerate}
\item $i$ is a quasi-compact immersion,
\item $X'$ is quasi-compact and $i$ is an immersion,
\item $i$ is quasi-compact and induces a homeomorphism
between $X'$ and $i(X')$,
\item $X'$ is quasi-compact and $i$ induces a homeomorphism
between $X'$ and $i(X')$.
\end{enumerate}
Then $i^*\mathcal{L}$ is ample on $X'$.
\end{lemma}

\begin{proof}
Observe that in cases (1) and (3) the scheme $X'$ is quasi-compact
as $X$ is quasi-compact by Definition \ref{definition-ample}.
Thus it suffices to prove (2) and (4). Since (2) is a special case
of (4) it suffices to prove (4).

\medskip\noindent
Assume condition (4) holds. For $s \in \Gamma(X, \mathcal{L}^{\otimes d})$
denote $s' = i^*s$ the pullback of $s$ to $X'$. Note that
$s'$ is a section of $(i^*\mathcal{L})^{\otimes d}$. By
Proposition \ref{proposition-characterize-ample}
the opens $X_s$, for $s \in \Gamma(X, \mathcal{L}^{\otimes d})$,
form a basis for the topology on $X$. Since $X'_{s'} = i^{-1}(X_s)$ by
Modules, Remark \ref{modules-remark-pullback-s-open}
and since $X' \to i(X')$ is a homeomorphism, we conclude
the opens $X'_{s'}$ form a basis for the topology of $X'$. Hence
$i^*\mathcal{L}$ is ample by
Proposition \ref{proposition-characterize-ample}.
\end{proof}

\begin{lemma}
\label{lemma-ample-on-product}
Let $S$ be a quasi-separated scheme. Let $X$, $Y$ be schemes over $S$.
Let $\mathcal{L}$ be an ample invertible $\mathcal{O}_X$-module
and let $\mathcal{N}$ be an ample invertible $\mathcal{O}_Y$-module.
Then $\mathcal{M} = \text{pr}_1^*\mathcal{L}
\otimes_{\mathcal{O}_{X \times_S Y}} \text{pr}_2^*\mathcal{N}$
is an ample invertible sheaf on $X \times_S Y$.
\end{lemma}

\begin{proof}
The morphism $i : X \times_S Y \to X \times Y$ is a quasi-compact
immersion, see Schemes, Lemma \ref{schemes-lemma-fibre-product-after-map}.
On the other hand, $\mathcal{M}$ is the pullback by
$i$ of the corresponding invertible module on $X \times Y$.
By Lemma \ref{lemma-ample-on-locally-closed} it suffices to prove the
lemma for $X \times Y$. We check (1) and (2) of
Definition \ref{definition-ample} for $\mathcal{M}$ on $X \times Y$.

\medskip\noindent
Since $X$ and $Y$ are quasi-compact, so is $X \times Y$.
Let $z \in X \times Y$ be a point. Let $x \in X$ and $y \in Y$
be the projections. Choose $n > 0$ and
$s \in \Gamma(X, \mathcal{L}^{\otimes n})$
such that $X_s$ is an affine open neighbourhood of $x$.
Choose $m > 0$ and
$t \in \Gamma(Y, \mathcal{N}^{\otimes m})$
such that $Y_t$ is an affine open neighbourhood of $y$.
Then $r = \text{pr}_1^*s \otimes \text{pr}_2^*t$ is a section
of $\mathcal{M}$ with $(X \times Y)_r = X_s \times Y_t$.
This is an affine open neighbourhood of $z$ and the proof is complete.
\end{proof}












\section{Affine and quasi-affine schemes}
\label{section-affine-quasi-affine}

\begin{lemma}
\label{lemma-quasi-affine-O-ample}
Let $X$ be a scheme.
Then $X$ is quasi-affine if and only if $\mathcal{O}_X$ is ample.
\end{lemma}

\begin{proof}
Suppose that $X$ is quasi-affine. Set $A = \Gamma(X, \mathcal{O}_X)$.
Consider the open immersion
$$
j : X \longrightarrow \Spec(A)
$$
from Lemma \ref{lemma-quasi-affine}. Note that
$\Spec(A) = \text{Proj}(A[T])$, see
Constructions, Example \ref{constructions-example-trivial-proj}.
Hence we can apply Lemma \ref{lemma-open-in-proj-ample}
to deduce that $\mathcal{O}_X$ is ample.

\medskip\noindent
Suppose that $\mathcal{O}_X$ is ample.
Note that $\Gamma_*(X, \mathcal{O}_X) \cong A[T]$
as graded rings. Hence the result follows from Lemmas
\ref{lemma-ample-immersion-into-proj} and \ref{lemma-quasi-affine}
taking into account that
$\Spec(A) = \text{Proj}(A[T])$ for any ring $A$
as seen above.
\end{proof}

\begin{lemma}
\label{lemma-quasi-affine-locally-closed}
Let $X$ be a quasi-affine scheme. For any quasi-compact immersion
$i : X' \to X$ the scheme $X'$ is quasi-affine.
\end{lemma}

\begin{proof}
This can be proved directly without making use of the material on
ample invertible sheaves; we urge the reader to do this on a napkin.
Since $X$ is quasi-affine, we have that $\mathcal{O}_X$ is ample by
Lemma \ref{lemma-quasi-affine-O-ample}.
Then $\mathcal{O}_{X'}$ is ample by
Lemma \ref{lemma-ample-on-locally-closed}. Then $X'$ is quasi-affine by
Lemma \ref{lemma-quasi-affine-O-ample}.
\end{proof}

\begin{lemma}
\label{lemma-characterize-affine}
Let $X$ be a scheme. Suppose that there exist finitely many elements
$f_1, \ldots, f_n \in \Gamma(X, \mathcal{O}_X)$ such that
\begin{enumerate}
\item each $X_{f_i}$ is an affine open of $X$, and
\item the ideal generated by $f_1, \ldots, f_n$ in
$\Gamma(X, \mathcal{O}_X)$ is equal to the unit ideal.
\end{enumerate}
Then $X$ is affine.
\end{lemma}

\begin{proof}
Assume we have $f_1, \ldots, f_n$ as in the lemma.
We may write $1 = \sum g_i f_i$ for some $g_j \in \Gamma(X, \mathcal{O}_X)$
and hence it is clear that $X = \bigcup X_{f_i}$. (The $f_i$'s cannot
all vanish at a point.) Since each $X_{f_i}$
is quasi-compact (being affine) it follows that $X$ is quasi-compact.
Hence we see that $X$ is quasi-affine by
Lemma \ref{lemma-quasi-affine-O-ample} above.
Consider the open immersion
$$
j : X \to \Spec(\Gamma(X, \mathcal{O}_X)),
$$
see Lemma \ref{lemma-quasi-affine}. The inverse image of the standard open
$D(f_i)$ on the right hand side is equal to $X_{f_i}$ on the
left hand side and the morphism $j$ induces an isomorphism
$X_{f_i} \cong D(f_i)$, see
Lemma \ref{lemma-invert-f-affine}. Since the $f_i$ generate the unit ideal
we see that $\Spec(\Gamma(X, \mathcal{O}_X))
= \bigcup_{i = 1, \ldots, n} D(f_i)$. Thus $j$ is an isomorphism.
\end{proof}







\section{Quasi-coherent sheaves and ample invertible sheaves}
\label{section-ample-quasi-coherent}

\noindent
Theme of this section: in the presence of an ample invertible sheaf
every quasi-coherent sheaf comes from a graded module.

\begin{situation}
\label{situation-ample}
Let $X$ be a scheme.
Let $\mathcal{L}$ be an ample invertible sheaf on $X$.
Set $S = \Gamma_*(X, \mathcal{L})$ as a graded ring.
Set $Y = \text{Proj}(S)$.
Let $f : X \to Y$ be the canonical morphism of Lemma \ref{lemma-map-into-proj}.
It comes equipped with a $\mathbf{Z}$-graded $\mathcal{O}_X$-algebra map
$\bigoplus f^*\mathcal{O}_Y(n) \to \bigoplus \mathcal{L}^{\otimes n}$.
\end{situation}

\noindent
The following lemma is really a special case of the next lemma
but it seems like a good idea to point out its validity first.

\begin{lemma}
\label{lemma-ample-gcd-is-one}
In Situation \ref{situation-ample}.
The canonical morphism $f : X \to Y$
maps $X$ into the open subscheme $W = W_1 \subset Y$
where $\mathcal{O}_Y(1)$ is invertible and where
all multiplication maps
$\mathcal{O}_Y(n) \otimes_{\mathcal{O}_Y} \mathcal{O}_Y(m) \to
\mathcal{O}_Y(n + m)$
are isomorphisms (see
Constructions, Lemma \ref{constructions-lemma-where-invertible}).
Moreover, the maps $f^*\mathcal{O}_Y(n) \to \mathcal{L}^{\otimes n}$
are all isomorphisms.
\end{lemma}

\begin{proof}
By Proposition \ref{proposition-characterize-ample} there exists an integer
$n_0$ such that $\mathcal{L}^{\otimes n}$ is globally generated for all
$n \geq n_0$. Let $x \in X$ be a point. By the above we can find
$a \in S_{n_0}$ and $b \in S_{n_0 + 1}$ such that
$a$ and $b$ do not vanish at $x$. Hence
$f(x) \in D_{+}(a) \cap D_{+}(b) = D_{+}(ab)$. By
Constructions, Lemma \ref{constructions-lemma-where-invertible}
we see that $f(x) \in W_1$ as desired. By
Constructions, Lemma \ref{constructions-lemma-invertible-map-into-proj}
which was used in the construction of the map $f$
the maps
$f^*\mathcal{O}_Y(n_0) \to \mathcal{L}^{\otimes n_0}$ and
$f^*\mathcal{O}_Y(n_0 + 1) \to \mathcal{L}^{\otimes n_0 + 1}$
are isomorphisms in a neighbourhood of $x$. By compatibility with
the algebra structure and the fact that $f$ maps into $W$
we conclude all the maps
$f^*\mathcal{O}_Y(n) \to \mathcal{L}^{\otimes n}$ are isomorphisms
in a neighbourhood of $x$. Hence we win.
\end{proof}

\noindent
Recall from Modules, Definition \ref{modules-definition-gamma-star}
that given a locally ringed space $X$, an invertible sheaf $\mathcal{L}$,
and a $\mathcal{O}_X$-module $\mathcal{F}$ we have the graded
$\Gamma_*(X, \mathcal{L})$-module
$$
\Gamma_*(X, \mathcal{L}, \mathcal{F}) =
\bigoplus\nolimits_{n \in \mathbf{Z}}
\Gamma(X, \mathcal{F} \otimes_{\mathcal{O}_X} \mathcal{L}^{\otimes n}).
$$
The following lemma says that, in Situation \ref{situation-ample},
we can recover a quasi-coherent $\mathcal{O}_X$-module $\mathcal{F}$
from this graded module. Take a look also at
Constructions, Lemma \ref{constructions-lemma-quasi-coherent-projective-space}
where we prove this lemma in the special case $X = \mathbf{P}^n_R$.

\begin{lemma}
\label{lemma-ample-quasi-coherent}
In Situation \ref{situation-ample}.
Let $\mathcal{F}$ be a quasi-coherent sheaf on $X$.
Set $M = \Gamma_*(X, \mathcal{L}, \mathcal{F})$ as a graded $S$-module.
There are isomorphisms
$$
f^*\widetilde{M} \longrightarrow \mathcal{F}
$$
functorial in $\mathcal{F}$ such that
$M_0 \to \Gamma(\text{Proj}(S), \widetilde{M}) \to \Gamma(X, \mathcal{F})$
is the identity map.
\end{lemma}

\begin{proof}
Let $s \in S_{+}$ be homogeneous such that $X_s$ is affine open in $X$.
Recall that $\widetilde{M}|_{D_{+}(s)}$ corresponds to the
$S_{(s)}$-module $M_{(s)}$, see
Constructions, Lemma \ref{constructions-lemma-proj-sheaves}.
Recall that $f^{-1}(D_{+}(s)) = X_s$.
As $X$ carries an ample invertible sheaf it is quasi-compact and
quasi-separated, see Section \ref{section-ample}.
By Lemma \ref{lemma-invert-s-sections} there is a canonical isomorphism
$M_{(s)} = \Gamma_*(X, \mathcal{L}, \mathcal{F})_{(s)} \to
\Gamma(X_s, \mathcal{F})$.
Since $\mathcal{F}$ is quasi-coherent this leads to
a canonical isomorphism
$$
f^*\widetilde{M}|_{X_s} \to \mathcal{F}|_{X_s}
$$
Since $\mathcal{L}$ is ample on $X$ we know that $X$ is covered
by the affine opens of the form $X_s$. Hence it suffices to prove
that the displayed maps glue on overlaps. Proof of this is
omitted.
\end{proof}

\begin{remark}
\label{remark-neurotic}
With assumptions and notation of Lemma \ref{lemma-ample-quasi-coherent}.
Denote the displayed map of the lemma by $\theta_\mathcal{F}$.
Note that the isomorphism $f^*\mathcal{O}_Y(n) \to \mathcal{L}^{\otimes n}$
of Lemma \ref{lemma-ample-gcd-is-one} is just
$\theta_{\mathcal{L}^{\otimes n}}$.
Consider the multiplication maps
$$
\widetilde{M} \otimes_{\mathcal{O}_Y} \mathcal{O}_Y(n)
\longrightarrow
\widetilde{M(n)}
$$
see
Constructions, Equation (\ref{constructions-equation-multiply-more-generally}).
Pull this back to $X$ and consider
$$
\xymatrix{
f^*\widetilde{M} \otimes_{\mathcal{O}_X} f^*\mathcal{O}_Y(n)
\ar[r]
\ar[d]_{\theta_\mathcal{F} \otimes \theta_{\mathcal{L}^{\otimes n}}}
&
f^*\widetilde{M(n)}
\ar[d]^{\theta_{\mathcal{F} \otimes \mathcal{L}^{\otimes n}}}
\\
\mathcal{F} \otimes \mathcal{L}^{\otimes n} \ar[r]^{\text{id}} &
\mathcal{F} \otimes \mathcal{L}^{\otimes n}
}
$$
Here we have used the obvious identification
$M(n) = \Gamma_*(X, \mathcal{L}, \mathcal{F} \otimes \mathcal{L}^{\otimes n})$.
This diagram commutes. Proof omitted.
\end{remark}

\noindent
It should be possible to deduce the following lemma from
Lemma \ref{lemma-ample-quasi-coherent} (or conversely) but it seems
simpler to just repeat the proof.

\begin{lemma}
\label{lemma-proj-quasi-coherent}
Let $S$ be a graded ring such that $X = \text{Proj}(S)$ is quasi-compact.
Let $\mathcal{F}$ be a quasi-coherent $\mathcal{O}_X$-module. Set
$M = \bigoplus_{n \in \mathbf{Z}} \Gamma(X, \mathcal{F}(n))$ as
a graded $S$-module, see
Constructions, Section \ref{constructions-section-invertible-on-proj}.
The map
$$
\widetilde{M} \longrightarrow \mathcal{F}
$$
of Constructions, Lemma
\ref{constructions-lemma-comparison-proj-quasi-coherent}
is an isomorphism.
If $X$ is covered by standard opens $D_+(f)$ where $f$ has degree $1$,
then the induced maps
$M_n \to \Gamma(X, \mathcal{F}(n))$ are the identity maps.
\end{lemma}

\begin{proof}
Since $X$ is quasi-compact we can find homogeneous elements
$f_1, \ldots, f_n \in S$ of positive degrees such that
$X = D_+(f_1) \cup \ldots \cup D_+(f_n)$. Let $d$ be the
least common multiple of the degrees of $f_1, \ldots, f_n$.
After replacing $f_i$ by a power we may assume that each
$f_i$ has degree $d$. Then we see that $\mathcal{L} = \mathcal{O}_X(d)$ is
invertible, the multiplication maps
$\mathcal{O}_X(ad) \otimes \mathcal{O}_X(bd) \to \mathcal{O}_X((a + b)d)$
are isomorphisms, and each $f_i$ determines a global section $s_i$
of $\mathcal{L}$ such that $X_{s_i} = D_+(f_i)$, see
Constructions, Lemmas \ref{constructions-lemma-where-invertible} and
\ref{constructions-lemma-principal-open}.
Thus $\Gamma(X, \mathcal{F}(ad)) =
\Gamma(X, \mathcal{F} \otimes \mathcal{L}^{\otimes a})$.
Recall that $\widetilde{M}|_{D_{+}(f_i)}$ corresponds to the
$S_{(f_i)}$-module $M_{(f_i)}$, see
Constructions, Lemma \ref{constructions-lemma-proj-sheaves}.
Since the degree of $f_i$ is $d$, the isomorphism class of
$M_{(f_i)}$ depends only on the homogeneous summands of $M$ of
degree divisible by $d$. More precisely, the isomorphism class of
$M_{(f_i)}$ depends only on the graded $\Gamma_*(X, \mathcal{L})$-module
$\Gamma_*(X, \mathcal{L}, \mathcal{F})$
and the image $s_i$ of $f_i$ in $\Gamma_*(X, \mathcal{L})$.
The scheme $X$ is quasi-compact by assumption and
separated by Constructions, Lemma \ref{constructions-lemma-proj-separated}.
By Lemma \ref{lemma-invert-s-sections} there is a canonical isomorphism
$$
M_{(f_i)} = \Gamma_*(X, \mathcal{L}, \mathcal{F})_{(s_i)} \to
\Gamma(X_{s_i}, \mathcal{F}).
$$
The construction of the map in Constructions, Lemma
\ref{constructions-lemma-comparison-proj-quasi-coherent}
then shows that it is an isomorphism over $D_+(f_i)$
hence an isomorphism as $X$ is covered by these opens.
We omit the proof of the final statement.
\end{proof}




\section{Finding suitable affine opens}
\label{section-finding-affine-opens}

\noindent
In this section we collect some results on the existence of
affine opens in more and less general situations.

\begin{lemma}
\label{lemma-maximal-points-affine}
Let $X$ be a quasi-separated scheme.
Let $Z_1, \ldots, Z_n$ be pairwise distinct irreducible components of $X$,
see Topology, Section \ref{topology-section-irreducible-components}.
Let $\eta_i \in Z_i$ be their generic points, see
Schemes, Lemma \ref{schemes-lemma-scheme-sober}.
There exist affine open neighbourhoods $\eta_i \in U_i$
such that $U_i \cap U_j = \emptyset$ for all $i \not = j$.
In particular, $U = U_1 \cup \ldots \cup U_n$ is an affine
open containing all of the points $\eta_1, \ldots, \eta_n$.
\end{lemma}

\begin{proof}
Let $V_i$ be any affine open containing $\eta_i$
and disjoint from the closed set $Z_1 \cup \ldots \hat Z_i \ldots \cup Z_n$.
Since $X$ is quasi-separated for each $i$ the union
$W_i = \bigcup_{j, j \not = i} V_i \cap V_j$ is a quasi-compact
open of $V_i$ not containing $\eta_i$.
We can find open neighbourhoods $U_i \subset V_i$
containing $\eta_i$ and disjoint from $W_i$ by
Algebra, Lemma \ref{algebra-lemma-standard-open-containing-maximal-point}.
Finally, $U$ is affine since it is the spectrum of
the ring $R_1 \times \ldots \times R_n$ where $R_i = \mathcal{O}_X(U_i)$,
see Schemes, Lemma \ref{schemes-lemma-disjoint-union-affines}.
\end{proof}

\begin{remark}
\label{remark-maximal-points-affine}
Lemma \ref{lemma-maximal-points-affine} above is false if $X$
is not quasi-separated. Here is an example. Take
$R = \mathbf{Q}[x, y_1, y_2, \ldots]/((x-i)y_i)$.
Consider the minimal prime ideal $\mathfrak p = (y_1, y_2, \ldots)$
of $R$. Glue two copies of $\Spec(R)$ along the
(not quasi-compact) open $\Spec(R) \setminus V(\mathfrak p)$
to get a scheme $X$ (glueing as in
Schemes, Example \ref{schemes-example-affine-space-zero-doubled}).
Then the two maximal points of $X$ corresponding to $\mathfrak p$
are not contained in a common affine open. The reason is
that any open of $\Spec(R)$ containing $\mathfrak p$
contains infinitely many of the ``lines'' $x = i$, $y_j = 0$,
$j \not = i$ with parameter $y_i$. Details omitted.
\end{remark}

\noindent
Notwithstanding the example above, for ``most'' finite sets of irreducible
closed subsets one can apply Lemma \ref{lemma-maximal-points-affine} above,
at least if $X$ is quasi-compact. This is true because $X$ contains a dense
open which is separated.

\begin{lemma}
\label{lemma-quasi-compact-dense-open-separated}
Let $X$ be a quasi-compact scheme.
There exists a dense open $V \subset X$ which is separated.
\end{lemma}

\begin{proof}
Say $X = \bigcup_{i = 1, \ldots, n} U_i$ is a union of $n$ affine open
subschemes. We will prove the lemma by induction on $n$. It is trivial for
$n = 1$. Let $V' \subset \bigcup_{i = 1, \ldots, n - 1} U_i$ be a separated
dense open subscheme, which exists by induction hypothesis. Consider
$$
V = V' \amalg (U_n \setminus \overline{V'}).
$$
It is clear that $V$ is separated and a dense open subscheme of $X$.
\end{proof}

\noindent
It turns out that, even if $X$ is quasi-separated as well as quasi-compact,
there does not exist a separated, quasi-compact dense open, see
Examples, Lemma
\ref{examples-lemma-no-dense-separated-quasi-compact-open-in-qcqs}.
Here is a slight refinement of Lemma \ref{lemma-maximal-points-affine} above.

\begin{lemma}
\label{lemma-point-and-maximal-points-affine}
Let $X$ be a quasi-separated scheme. Let $Z_1, \ldots, Z_n$ be pairwise
distinct irreducible components of $X$. Let $\eta_i \in Z_i$ be their
generic points. Let $x \in X$ be arbitrary.
There exists an affine open $U \subset X$ containing
$x$ and all the $\eta_i$.
\end{lemma}

\begin{proof}
Suppose that $x \in Z_1 \cap \ldots \cap Z_r$ and
$x \not \in Z_{r + 1}, \ldots, Z_n$. Then we may choose
an affine open $W \subset X$ such that
$x \in W$ and $W \cap Z_i = \emptyset$ for
$i = r + 1, \ldots, n$. Note that clearly $\eta_i \in W$
for $i = 1, \ldots, r$. By Lemma \ref{lemma-maximal-points-affine}
we may choose affine opens $U_i \subset X$ which are pairwise
disjoint such that $\eta_i \in U_i$ for $i = r + 1, \ldots, n$.
Since $X$ is quasi-separated the opens $W \cap U_i$
are quasi-compact and do not contain $\eta_i$ for
$i = r + 1, \ldots, n$. Hence by
Algebra, Lemma \ref{algebra-lemma-standard-open-containing-maximal-point}
we may shrink $U_i$ such that $W \cap U_i = \emptyset$
for $i = r + 1, \ldots, n$. Then the union
$U = W \cup \bigcup_{i = r + 1, \ldots, n} U_i$ is disjoint and hence
(by Schemes, Lemma \ref{schemes-lemma-disjoint-union-affines})
a suitable affine open.
\end{proof}

\begin{lemma}
\label{lemma-ample-finite-set-in-affine}
Let $X$ be a scheme. Assume either
\begin{enumerate}
\item The scheme $X$ is quasi-affine.
\item The scheme $X$ is isomorphic to a locally closed subscheme
of an affine scheme.
\item There exists an ample invertible sheaf on $X$.
\item The scheme $X$ is isomorphic to a locally closed subscheme
of $\text{Proj}(S)$ for some graded ring $S$.
\end{enumerate}
Then for any finite subset $E \subset X$ there exists an
affine open $U \subset X$ with $E \subset U$.
\end{lemma}

\begin{proof}
By Properties, Definition \ref{definition-quasi-affine}
a quasi-affine scheme is a quasi-compact open subscheme
of an affine scheme. Any affine scheme $\Spec(R)$ is isomorphic to
$\text{Proj}(R[X])$ where $R[X]$ is graded by setting $\deg(X) = 1$.
By Proposition \ref{proposition-characterize-ample}
if $X$ has an ample invertible sheaf then $X$ is isomorphic to an open
subscheme of $\text{Proj}(S)$ for some graded ring $S$.
Hence, it suffices to prove the lemma in case (4).
(We urge the reader to prove case (2) directly for themselves.)

\medskip\noindent
Thus assume $X \subset \text{Proj}(S)$ is a locally closed subscheme
where $S$ is some graded ring. Let $T = \overline{X} \setminus X$.
Recall that the standard opens $D_{+}(f)$ form a basis of the
topology on $\text{Proj}(S)$. Since $E$ is finite we may choose finitely many
homogeneous elements $f_i \in S_{+}$ such that
$$
E \subset
D_{+}(f_1) \cup \ldots \cup D_{+}(f_n) \subset
\text{Proj}(S) \setminus T
$$
Suppose that $E = \{\mathfrak p_1, \ldots, \mathfrak p_m\}$
as a subset of $\text{Proj}(S)$.
Consider the ideal $I = (f_1, \ldots, f_n) \subset S$.
Since $I \not \subset \mathfrak p_j$ for all $j = 1, \ldots, m$
we see from Algebra, Lemma \ref{algebra-lemma-graded-silly} that
there exists a homogeneous element $f \in I$, $f \not \in \mathfrak p_j$
for all $j = 1, \ldots, m$. Then $E \subset D_{+}(f) \subset
D_{+}(f_1) \cup \ldots \cup D_{+}(f_n)$. Since $D_{+}(f)$ does not
meet $T$ we see that $X \cap D_{+}(f)$ is a closed subscheme of the
affine scheme $D_{+}(f)$, hence is an affine open of $X$ as desired.
\end{proof}

\begin{lemma}
\label{lemma-ample-finite-set-in-principal-affine}
Let $X$ be a scheme. Let $\mathcal{L}$ be an ample invertible sheaf on $X$.
Let
$$
E \subset W \subset X
$$
with $E$ finite and $W$ open in $X$. Then there exists an $n > 0$
and a section $s \in \Gamma(X, \mathcal{L}^{\otimes n})$ such that
$X_s$ is affine and $E \subset X_s \subset W$.
\end{lemma}

\begin{proof}
The reader can modify the proof of Lemma \ref{lemma-ample-finite-set-in-affine}
to prove this lemma; we will instead deduce the lemma from it.
By Lemma \ref{lemma-ample-finite-set-in-affine} we can choose an affine
open $U \subset W$ such that $E \subset U$.
Consider the graded ring $S = \Gamma_*(X, \mathcal{L}) =
\bigoplus_{n \geq 0} \Gamma(X, \mathcal{L}^{\otimes n})$.
For each $x \in E$ let $\mathfrak p_x \subset S$ be the graded ideal
of sections vanishing at $x$. It is clear that $\mathfrak p_x$ is
a prime ideal and since some power of $\mathcal{L}$ is globally
generated, it is clear that $S_{+} \not \subset \mathfrak p_x$.
Let $I \subset S$ be the graded ideal of sections vanishing on all
points of $X \setminus U$. Since the sets $X_s$ form a basis
for the topology we see that $I \not \subset \mathfrak p_x$ for
all $x \in E$.
By (graded) prime avoidance (Algebra, Lemma \ref{algebra-lemma-graded-silly})
we can find $s \in I$ homogeneous
with $s \not \in \mathfrak p_x$ for all $x \in E$.
Then $E \subset X_s \subset U$ and $X_s$ is affine by
Lemma \ref{lemma-affine-cap-s-open}.
\end{proof}

\begin{lemma}
\label{lemma-quasi-affine-invertible-nonvanishing-section}
Let $X$ be a quasi-affine scheme. Let $\mathcal{L}$ be an invertible
$\mathcal{O}_X$-module. Let $E \subset W \subset X$ with $E$ finite
and $W$ open. Then there exists an $s \in \Gamma(X, \mathcal{L})$
such that $X_s$ is affine and $E \subset X_s \subset W$.
\end{lemma}

\begin{proof}
The proof of this lemma has a lot in common with the proof of
Algebra, Lemma \ref{algebra-lemma-silly}.
Say $E = \{x_1, \ldots, x_n\}$. If $E = W = \emptyset$, then $s = 0$
works. If $W \not = \emptyset$, then we may assume $E \not = \emptyset$
by adding a point if necessary. Thus we may assume $n \geq 1$.
We will prove the lemma by induction on $n$.

\medskip\noindent
Base case: $n = 1$. After replacing $W$ by an affine open neighbourhood
of $x_1$ in $W$, we may assume $W$ is affine. Combining
Lemmas \ref{lemma-quasi-affine-O-ample} and
Proposition \ref{proposition-characterize-ample}
we see that every quasi-coherent
$\mathcal{O}_X$-module is globally generated.
Hence there exists
a global section $s$ of $\mathcal{L}$ which does not vanish at $x_1$.
On the other hand, let $Z \subset X$ be the
reduced induced closed subscheme on $X \setminus W$.
Applying global generation to the quasi-coherent ideal sheaf
$\mathcal{I}$ of $Z$ we find a global section $f$ of $\mathcal{I}$
which does not vanish at $x_1$. Then $s' = fs$ is a global section
of $\mathcal{L}$ which does not vanish at $x_1$ such that
$X_{s'} \subset W$. Then $X_{s'}$ is affine by
Lemma \ref{lemma-affine-cap-s-open}.

\medskip\noindent
Induction step for $n > 1$. If there is a specialization
$x_i \leadsto x_j$ for $i \not = j$, then it suffices
to prove the lemma for $\{x_1, \ldots, x_n\} \setminus \{x_i\}$
and we are done by induction. Thus we may assume there are no
specializations among the $x_i$.
By either Lemma \ref{lemma-ample-finite-set-in-affine} or
Lemma \ref{lemma-ample-finite-set-in-principal-affine}
we may assume $W$ is affine.
By induction we can find a global section
$s$ of $\mathcal{L}$ such that $X_s \subset W$ is affine and contains
$x_1, \ldots, x_{n - 1}$. If $x_n \in X_s$ then we are done.
Assume $s$ is zero at $x_n$. By the case $n = 1$ we can find
a global section $s'$ of $\mathcal{L}$ with
$\{x_n\} \subset X_{s'} \subset
W \setminus \overline{\{x_1, \ldots, x_{n - 1}\}}$.
Here we use that $x_n$ is not a specialization of $x_1, \ldots, x_{n - 1}$.
Then $s + s'$
is a global section of $\mathcal{L}$ which is nonvanishing
at $x_1, \ldots, x_n$ with $X_{s + s'} \subset W$ and
we conclude as before.
\end{proof}

\begin{lemma}
\label{lemma-ring-affine-open-injective-local-ring}
Let $X$ be a scheme and $x \in X$ a point. There exists an affine open
neighbourhood $U \subset X$ of $x$ such that the canonical map
$\mathcal{O}_X(U) \to \mathcal{O}_{X, x}$ is injective in each of
the following cases:
\begin{enumerate}
\item $X$ is integral,
\item $X$ is locally Noetherian,
\item $X$ is reduced and has a finite number of irreducible components.
\end{enumerate}
\end{lemma}

\begin{proof}
After translation into algebra, this follows from
Algebra, Lemma \ref{algebra-lemma-subring-of-local-ring}.
\end{proof}







\begin{multicols}{2}[\section{Other chapters}]
\noindent
Preliminaries
\begin{enumerate}
\item \hyperref[introduction-section-phantom]{Introduction}
\item \hyperref[conventions-section-phantom]{Conventions}
\item \hyperref[sets-section-phantom]{Set Theory}
\item \hyperref[categories-section-phantom]{Categories}
\item \hyperref[topology-section-phantom]{Topology}
\item \hyperref[sheaves-section-phantom]{Sheaves on Spaces}
\item \hyperref[sites-section-phantom]{Sites and Sheaves}
\item \hyperref[stacks-section-phantom]{Stacks}
\item \hyperref[fields-section-phantom]{Fields}
\item \hyperref[algebra-section-phantom]{Commutative Algebra}
\item \hyperref[brauer-section-phantom]{Brauer Groups}
\item \hyperref[homology-section-phantom]{Homological Algebra}
\item \hyperref[derived-section-phantom]{Derived Categories}
\item \hyperref[simplicial-section-phantom]{Simplicial Methods}
\item \hyperref[more-algebra-section-phantom]{More on Algebra}
\item \hyperref[smoothing-section-phantom]{Smoothing Ring Maps}
\item \hyperref[modules-section-phantom]{Sheaves of Modules}
\item \hyperref[sites-modules-section-phantom]{Modules on Sites}
\item \hyperref[injectives-section-phantom]{Injectives}
\item \hyperref[cohomology-section-phantom]{Cohomology of Sheaves}
\item \hyperref[sites-cohomology-section-phantom]{Cohomology on Sites}
\item \hyperref[dga-section-phantom]{Differential Graded Algebra}
\item \hyperref[dpa-section-phantom]{Divided Power Algebra}
\item \hyperref[sdga-section-phantom]{Differential Graded Sheaves}
\item \hyperref[hypercovering-section-phantom]{Hypercoverings}
\end{enumerate}
Schemes
\begin{enumerate}
\setcounter{enumi}{25}
\item \hyperref[schemes-section-phantom]{Schemes}
\item \hyperref[constructions-section-phantom]{Constructions of Schemes}
\item \hyperref[properties-section-phantom]{Properties of Schemes}
\item \hyperref[morphisms-section-phantom]{Morphisms of Schemes}
\item \hyperref[coherent-section-phantom]{Cohomology of Schemes}
\item \hyperref[divisors-section-phantom]{Divisors}
\item \hyperref[limits-section-phantom]{Limits of Schemes}
\item \hyperref[varieties-section-phantom]{Varieties}
\item \hyperref[topologies-section-phantom]{Topologies on Schemes}
\item \hyperref[descent-section-phantom]{Descent}
\item \hyperref[perfect-section-phantom]{Derived Categories of Schemes}
\item \hyperref[more-morphisms-section-phantom]{More on Morphisms}
\item \hyperref[flat-section-phantom]{More on Flatness}
\item \hyperref[groupoids-section-phantom]{Groupoid Schemes}
\item \hyperref[more-groupoids-section-phantom]{More on Groupoid Schemes}
\item \hyperref[etale-section-phantom]{\'Etale Morphisms of Schemes}
\end{enumerate}
Topics in Scheme Theory
\begin{enumerate}
\setcounter{enumi}{41}
\item \hyperref[chow-section-phantom]{Chow Homology}
\item \hyperref[intersection-section-phantom]{Intersection Theory}
\item \hyperref[pic-section-phantom]{Picard Schemes of Curves}
\item \hyperref[weil-section-phantom]{Weil Cohomology Theories}
\item \hyperref[adequate-section-phantom]{Adequate Modules}
\item \hyperref[dualizing-section-phantom]{Dualizing Complexes}
\item \hyperref[duality-section-phantom]{Duality for Schemes}
\item \hyperref[discriminant-section-phantom]{Discriminants and Differents}
\item \hyperref[derham-section-phantom]{de Rham Cohomology}
\item \hyperref[local-cohomology-section-phantom]{Local Cohomology}
\item \hyperref[algebraization-section-phantom]{Algebraic and Formal Geometry}
\item \hyperref[curves-section-phantom]{Algebraic Curves}
\item \hyperref[resolve-section-phantom]{Resolution of Surfaces}
\item \hyperref[models-section-phantom]{Semistable Reduction}
\item \hyperref[functors-section-phantom]{Functors and Morphisms}
\item \hyperref[equiv-section-phantom]{Derived Categories of Varieties}
\item \hyperref[pione-section-phantom]{Fundamental Groups of Schemes}
\item \hyperref[etale-cohomology-section-phantom]{\'Etale Cohomology}
\item \hyperref[crystalline-section-phantom]{Crystalline Cohomology}
\item \hyperref[proetale-section-phantom]{Pro-\'etale Cohomology}
\item \hyperref[relative-cycles-section-phantom]{Relative Cycles}
\item \hyperref[more-etale-section-phantom]{More \'Etale Cohomology}
\item \hyperref[trace-section-phantom]{The Trace Formula}
\end{enumerate}
Algebraic Spaces
\begin{enumerate}
\setcounter{enumi}{64}
\item \hyperref[spaces-section-phantom]{Algebraic Spaces}
\item \hyperref[spaces-properties-section-phantom]{Properties of Algebraic Spaces}
\item \hyperref[spaces-morphisms-section-phantom]{Morphisms of Algebraic Spaces}
\item \hyperref[decent-spaces-section-phantom]{Decent Algebraic Spaces}
\item \hyperref[spaces-cohomology-section-phantom]{Cohomology of Algebraic Spaces}
\item \hyperref[spaces-limits-section-phantom]{Limits of Algebraic Spaces}
\item \hyperref[spaces-divisors-section-phantom]{Divisors on Algebraic Spaces}
\item \hyperref[spaces-over-fields-section-phantom]{Algebraic Spaces over Fields}
\item \hyperref[spaces-topologies-section-phantom]{Topologies on Algebraic Spaces}
\item \hyperref[spaces-descent-section-phantom]{Descent and Algebraic Spaces}
\item \hyperref[spaces-perfect-section-phantom]{Derived Categories of Spaces}
\item \hyperref[spaces-more-morphisms-section-phantom]{More on Morphisms of Spaces}
\item \hyperref[spaces-flat-section-phantom]{Flatness on Algebraic Spaces}
\item \hyperref[spaces-groupoids-section-phantom]{Groupoids in Algebraic Spaces}
\item \hyperref[spaces-more-groupoids-section-phantom]{More on Groupoids in Spaces}
\item \hyperref[bootstrap-section-phantom]{Bootstrap}
\item \hyperref[spaces-pushouts-section-phantom]{Pushouts of Algebraic Spaces}
\end{enumerate}
Topics in Geometry
\begin{enumerate}
\setcounter{enumi}{81}
\item \hyperref[spaces-chow-section-phantom]{Chow Groups of Spaces}
\item \hyperref[groupoids-quotients-section-phantom]{Quotients of Groupoids}
\item \hyperref[spaces-more-cohomology-section-phantom]{More on Cohomology of Spaces}
\item \hyperref[spaces-simplicial-section-phantom]{Simplicial Spaces}
\item \hyperref[spaces-duality-section-phantom]{Duality for Spaces}
\item \hyperref[formal-spaces-section-phantom]{Formal Algebraic Spaces}
\item \hyperref[restricted-section-phantom]{Algebraization of Formal Spaces}
\item \hyperref[spaces-resolve-section-phantom]{Resolution of Surfaces Revisited}
\end{enumerate}
Deformation Theory
\begin{enumerate}
\setcounter{enumi}{89}
\item \hyperref[formal-defos-section-phantom]{Formal Deformation Theory}
\item \hyperref[defos-section-phantom]{Deformation Theory}
\item \hyperref[cotangent-section-phantom]{The Cotangent Complex}
\item \hyperref[examples-defos-section-phantom]{Deformation Problems}
\end{enumerate}
Algebraic Stacks
\begin{enumerate}
\setcounter{enumi}{93}
\item \hyperref[algebraic-section-phantom]{Algebraic Stacks}
\item \hyperref[examples-stacks-section-phantom]{Examples of Stacks}
\item \hyperref[stacks-sheaves-section-phantom]{Sheaves on Algebraic Stacks}
\item \hyperref[criteria-section-phantom]{Criteria for Representability}
\item \hyperref[artin-section-phantom]{Artin's Axioms}
\item \hyperref[quot-section-phantom]{Quot and Hilbert Spaces}
\item \hyperref[stacks-properties-section-phantom]{Properties of Algebraic Stacks}
\item \hyperref[stacks-morphisms-section-phantom]{Morphisms of Algebraic Stacks}
\item \hyperref[stacks-limits-section-phantom]{Limits of Algebraic Stacks}
\item \hyperref[stacks-cohomology-section-phantom]{Cohomology of Algebraic Stacks}
\item \hyperref[stacks-perfect-section-phantom]{Derived Categories of Stacks}
\item \hyperref[stacks-introduction-section-phantom]{Introducing Algebraic Stacks}
\item \hyperref[stacks-more-morphisms-section-phantom]{More on Morphisms of Stacks}
\item \hyperref[stacks-geometry-section-phantom]{The Geometry of Stacks}
\end{enumerate}
Topics in Moduli Theory
\begin{enumerate}
\setcounter{enumi}{107}
\item \hyperref[moduli-section-phantom]{Moduli Stacks}
\item \hyperref[moduli-curves-section-phantom]{Moduli of Curves}
\end{enumerate}
Miscellany
\begin{enumerate}
\setcounter{enumi}{109}
\item \hyperref[examples-section-phantom]{Examples}
\item \hyperref[exercises-section-phantom]{Exercises}
\item \hyperref[guide-section-phantom]{Guide to Literature}
\item \hyperref[desirables-section-phantom]{Desirables}
\item \hyperref[coding-section-phantom]{Coding Style}
\item \hyperref[obsolete-section-phantom]{Obsolete}
\item \hyperref[fdl-section-phantom]{GNU Free Documentation License}
\item \hyperref[index-section-phantom]{Auto Generated Index}
\end{enumerate}
\end{multicols}


\bibliography{my}
\bibliographystyle{amsalpha}

\end{document}
