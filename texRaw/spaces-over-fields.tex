\IfFileExists{stacks-project.cls}{%
\documentclass{stacks-project}
}{%
\documentclass{amsart}
}

% For dealing with references we use the comment environment
\usepackage{verbatim}
\newenvironment{reference}{\comment}{\endcomment}
%\newenvironment{reference}{}{}
\newenvironment{slogan}{\comment}{\endcomment}
\newenvironment{history}{\comment}{\endcomment}

% For commutative diagrams we use Xy-pic
\usepackage[all]{xy}

% We use 2cell for 2-commutative diagrams.
\xyoption{2cell}
\UseAllTwocells

% We use multicol for the list of chapters between chapters
\usepackage{multicol}

% This is generally recommended for better output
\usepackage{lmodern}
\usepackage[T1]{fontenc}

% For cross-file-references
\usepackage{xr-hyper}

% Package for hypertext links:
\usepackage{hyperref}

% For any local file, say "hello.tex" you want to link to please
% use \externaldocument[hello-]{hello}
\externaldocument[introduction-]{introduction}
\externaldocument[conventions-]{conventions}
\externaldocument[sets-]{sets}
\externaldocument[categories-]{categories}
\externaldocument[topology-]{topology}
\externaldocument[sheaves-]{sheaves}
\externaldocument[sites-]{sites}
\externaldocument[stacks-]{stacks}
\externaldocument[fields-]{fields}
\externaldocument[algebra-]{algebra}
\externaldocument[brauer-]{brauer}
\externaldocument[homology-]{homology}
\externaldocument[derived-]{derived}
\externaldocument[simplicial-]{simplicial}
\externaldocument[more-algebra-]{more-algebra}
\externaldocument[smoothing-]{smoothing}
\externaldocument[modules-]{modules}
\externaldocument[sites-modules-]{sites-modules}
\externaldocument[injectives-]{injectives}
\externaldocument[cohomology-]{cohomology}
\externaldocument[sites-cohomology-]{sites-cohomology}
\externaldocument[dga-]{dga}
\externaldocument[dpa-]{dpa}
\externaldocument[sdga-]{sdga}
\externaldocument[hypercovering-]{hypercovering}
\externaldocument[schemes-]{schemes}
\externaldocument[constructions-]{constructions}
\externaldocument[properties-]{properties}
\externaldocument[morphisms-]{morphisms}
\externaldocument[coherent-]{coherent}
\externaldocument[divisors-]{divisors}
\externaldocument[limits-]{limits}
\externaldocument[varieties-]{varieties}
\externaldocument[topologies-]{topologies}
\externaldocument[descent-]{descent}
\externaldocument[perfect-]{perfect}
\externaldocument[more-morphisms-]{more-morphisms}
\externaldocument[flat-]{flat}
\externaldocument[groupoids-]{groupoids}
\externaldocument[more-groupoids-]{more-groupoids}
\externaldocument[etale-]{etale}
\externaldocument[chow-]{chow}
\externaldocument[intersection-]{intersection}
\externaldocument[pic-]{pic}
\externaldocument[weil-]{weil}
\externaldocument[adequate-]{adequate}
\externaldocument[dualizing-]{dualizing}
\externaldocument[duality-]{duality}
\externaldocument[discriminant-]{discriminant}
\externaldocument[derham-]{derham}
\externaldocument[local-cohomology-]{local-cohomology}
\externaldocument[algebraization-]{algebraization}
\externaldocument[curves-]{curves}
\externaldocument[resolve-]{resolve}
\externaldocument[models-]{models}
\externaldocument[functors-]{functors}
\externaldocument[equiv-]{equiv}
\externaldocument[pione-]{pione}
\externaldocument[etale-cohomology-]{etale-cohomology}
\externaldocument[proetale-]{proetale}
\externaldocument[relative-cycles-]{relative-cycles}
\externaldocument[more-etale-]{more-etale}
\externaldocument[trace-]{trace}
\externaldocument[crystalline-]{crystalline}
\externaldocument[spaces-]{spaces}
\externaldocument[spaces-properties-]{spaces-properties}
\externaldocument[spaces-morphisms-]{spaces-morphisms}
\externaldocument[decent-spaces-]{decent-spaces}
\externaldocument[spaces-cohomology-]{spaces-cohomology}
\externaldocument[spaces-limits-]{spaces-limits}
\externaldocument[spaces-divisors-]{spaces-divisors}
\externaldocument[spaces-over-fields-]{spaces-over-fields}
\externaldocument[spaces-topologies-]{spaces-topologies}
\externaldocument[spaces-descent-]{spaces-descent}
\externaldocument[spaces-perfect-]{spaces-perfect}
\externaldocument[spaces-more-morphisms-]{spaces-more-morphisms}
\externaldocument[spaces-flat-]{spaces-flat}
\externaldocument[spaces-groupoids-]{spaces-groupoids}
\externaldocument[spaces-more-groupoids-]{spaces-more-groupoids}
\externaldocument[bootstrap-]{bootstrap}
\externaldocument[spaces-pushouts-]{spaces-pushouts}
\externaldocument[spaces-chow-]{spaces-chow}
\externaldocument[groupoids-quotients-]{groupoids-quotients}
\externaldocument[spaces-more-cohomology-]{spaces-more-cohomology}
\externaldocument[spaces-simplicial-]{spaces-simplicial}
\externaldocument[spaces-duality-]{spaces-duality}
\externaldocument[formal-spaces-]{formal-spaces}
\externaldocument[restricted-]{restricted}
\externaldocument[spaces-resolve-]{spaces-resolve}
\externaldocument[formal-defos-]{formal-defos}
\externaldocument[defos-]{defos}
\externaldocument[cotangent-]{cotangent}
\externaldocument[examples-defos-]{examples-defos}
\externaldocument[algebraic-]{algebraic}
\externaldocument[examples-stacks-]{examples-stacks}
\externaldocument[stacks-sheaves-]{stacks-sheaves}
\externaldocument[criteria-]{criteria}
\externaldocument[artin-]{artin}
\externaldocument[quot-]{quot}
\externaldocument[stacks-properties-]{stacks-properties}
\externaldocument[stacks-morphisms-]{stacks-morphisms}
\externaldocument[stacks-limits-]{stacks-limits}
\externaldocument[stacks-cohomology-]{stacks-cohomology}
\externaldocument[stacks-perfect-]{stacks-perfect}
\externaldocument[stacks-introduction-]{stacks-introduction}
\externaldocument[stacks-more-morphisms-]{stacks-more-morphisms}
\externaldocument[stacks-geometry-]{stacks-geometry}
\externaldocument[moduli-]{moduli}
\externaldocument[moduli-curves-]{moduli-curves}
\externaldocument[examples-]{examples}
\externaldocument[exercises-]{exercises}
\externaldocument[guide-]{guide}
\externaldocument[desirables-]{desirables}
\externaldocument[coding-]{coding}
\externaldocument[obsolete-]{obsolete}
\externaldocument[fdl-]{fdl}
\externaldocument[index-]{index}

% Theorem environments.
%
\theoremstyle{plain}
\newtheorem{theorem}[subsection]{Theorem}
\newtheorem{proposition}[subsection]{Proposition}
\newtheorem{lemma}[subsection]{Lemma}

\theoremstyle{definition}
\newtheorem{definition}[subsection]{Definition}
\newtheorem{example}[subsection]{Example}
\newtheorem{exercise}[subsection]{Exercise}
\newtheorem{situation}[subsection]{Situation}

\theoremstyle{remark}
\newtheorem{remark}[subsection]{Remark}
\newtheorem{remarks}[subsection]{Remarks}

\numberwithin{equation}{subsection}

% Macros
%
\def\lim{\mathop{\mathrm{lim}}\nolimits}
\def\colim{\mathop{\mathrm{colim}}\nolimits}
\def\Spec{\mathop{\mathrm{Spec}}}
\def\Hom{\mathop{\mathrm{Hom}}\nolimits}
\def\Ext{\mathop{\mathrm{Ext}}\nolimits}
\def\SheafHom{\mathop{\mathcal{H}\!\mathit{om}}\nolimits}
\def\SheafExt{\mathop{\mathcal{E}\!\mathit{xt}}\nolimits}
\def\Sch{\mathit{Sch}}
\def\Mor{\mathop{\mathrm{Mor}}\nolimits}
\def\Ob{\mathop{\mathrm{Ob}}\nolimits}
\def\Sh{\mathop{\mathit{Sh}}\nolimits}
\def\NL{\mathop{N\!L}\nolimits}
\def\CH{\mathop{\mathrm{CH}}\nolimits}
\def\proetale{{pro\text{-}\acute{e}tale}}
\def\etale{{\acute{e}tale}}
\def\QCoh{\mathit{QCoh}}
\def\Ker{\mathop{\mathrm{Ker}}}
\def\Im{\mathop{\mathrm{Im}}}
\def\Coker{\mathop{\mathrm{Coker}}}
\def\Coim{\mathop{\mathrm{Coim}}}

% Boxtimes
%
\DeclareMathSymbol{\boxtimes}{\mathbin}{AMSa}{"02}

%
% Macros for moduli stacks/spaces
%
\def\QCohstack{\mathcal{QC}\!\mathit{oh}}
\def\Cohstack{\mathcal{C}\!\mathit{oh}}
\def\Spacesstack{\mathcal{S}\!\mathit{paces}}
\def\Quotfunctor{\mathrm{Quot}}
\def\Hilbfunctor{\mathrm{Hilb}}
\def\Curvesstack{\mathcal{C}\!\mathit{urves}}
\def\Polarizedstack{\mathcal{P}\!\mathit{olarized}}
\def\Complexesstack{\mathcal{C}\!\mathit{omplexes}}
% \Pic is the operator that assigns to X its picard group, usage \Pic(X)
% \Picardstack_{X/B} denotes the Picard stack of X over B
% \Picardfunctor_{X/B} denotes the Picard functor of X over B
\def\Pic{\mathop{\mathrm{Pic}}\nolimits}
\def\Picardstack{\mathcal{P}\!\mathit{ic}}
\def\Picardfunctor{\mathrm{Pic}}
\def\Deformationcategory{\mathcal{D}\!\mathit{ef}}


% OK, start here.
%
\begin{document}

\title{Algebraic Spaces over Fields}


\maketitle

\phantomsection
\label{section-phantom}

\tableofcontents

\section{Introduction}
\label{section-introduction}

\noindent
This chapter is the analogue of the chapter on varieties in the setting
of algebraic spaces. A reference for algebraic spaces is
\cite{Kn}.


\section{Conventions}
\label{section-conventions}

\noindent
The standing assumption is that all schemes are contained in
a big fppf site $\Sch_{fppf}$. And all rings $A$ considered
have the property that $\Spec(A)$ is (isomorphic) to an
object of this big site.

\medskip\noindent
Let $S$ be a scheme and let $X$ be an algebraic space over $S$.
In this chapter and the following we will write $X \times_S X$
for the product of $X$ with itself (in the category of algebraic
spaces over $S$), instead of $X \times X$.



\section{Generically finite morphisms}
\label{section-generically-finite}

\noindent
This section continues the discussion in
Decent Spaces, Section \ref{decent-spaces-section-generically-finite}
and the analogue for morphisms of algebraic spaces of
Varieties, Section \ref{varieties-section-generically-finite}.

\begin{lemma}
\label{lemma-quasi-finite-in-codim-1}
Let $S$ be a scheme. Let $f : X \to Y$ be a morphism of algebraic spaces
over $S$. Assume $f$ is locally of finite type and $Y$ is locally Noetherian.
Let $y \in |Y|$ be a point of codimension $\leq 1$ on $Y$.
Let $X^0 \subset |X|$ be the set of points of codimension $0$ on $X$.
Assume in addition one of the following conditions is satisfied
\begin{enumerate}
\item for every $x \in X^0$ the transcendence degree of $x/f(x)$ is $0$,
\item for every $x \in X^0$ with $f(x) \leadsto y$
the transcendence degree of $x/f(x)$ is $0$,
\item $f$ is quasi-finite at every $x \in X^0$,
\item $f$ is quasi-finite at a dense set of points of $|X|$,
\item add more here.
\end{enumerate}
Then $f$ is quasi-finite at every point of $X$ lying over $y$.
\end{lemma}

\begin{proof}
We want to reduce the proof to the case of schemes. To do this we
choose a commutative diagram
$$
\xymatrix{
U \ar[r] \ar[d]_g & X \ar[d]^f \\
V \ar[r] & Y
}
$$
where $U$, $V$ are schemes and where the horizontal arrows are \'etale
and surjective. Pick $v \in V$ mapping to $y$. Observe that
$V$ is locally Noetherian and that $\dim(\mathcal{O}_{V, v}) \leq 1$
(see Properties of Spaces, Definitions
\ref{spaces-properties-definition-dimension-local-ring} and
Remark \ref{spaces-properties-remark-list-properties-local-etale-topology}).
The fibre $U_v$ of $U \to V$ over $v$ surjects onto
$f^{-1}(\{y\}) \subset |X|$. The inverse image of $X^0$ in $U$
is exactly the set of
generic points of irreducible components of $U$ (Properties of Spaces, Lemma
\ref{spaces-properties-lemma-codimension-0-points}).
If $\eta \in U$ is such a point with image $x \in X^0$, then
the transcendence degree of $x / f(x)$ is the transcendence
degree of $\kappa(\eta)$ over $\kappa(g(\eta))$
(Morphisms of Spaces, Definition
\ref{spaces-morphisms-definition-dimension-fibre}).
Observe that $U \to V$ is quasi-finite at $u \in U$ if and only if
$f$ is quasi-finite at the image of $u$ in $X$.

\medskip\noindent
Case (1). Here case (1) of
Varieties, Lemma \ref{varieties-lemma-quasi-finite-in-codim-1} applies
and we conclude that $U \to V$ is quasi-finite at all points of $U_v$.
Hence $f$ is quasi-finite at every point lying over $y$.

\medskip\noindent
Case (2). Let $u \in U$ be a generic point of an irreducible component
whose image in $V$ specializes to $v$. Then the image $x \in X^0$ of
$u$ has the property that $f(x) \leadsto y$. Hence we see that
case (2) of
Varieties, Lemma \ref{varieties-lemma-quasi-finite-in-codim-1} applies
and we conclude as before.

\medskip\noindent
Case (3) follows from case (3) of
Varieties, Lemma \ref{varieties-lemma-quasi-finite-in-codim-1}.

\medskip\noindent
In case (4), since $|U| \to |X|$ is open, we see that
the set of points where $U \to V$ is quasi-finite is dense as well.
Hence case (4) of
Varieties, Lemma \ref{varieties-lemma-quasi-finite-in-codim-1} applies.
\end{proof}

\begin{lemma}
\label{lemma-finite-in-codim-1}
Let $S$ be a scheme. Let $f : X \to Y$ be a morphism of algebraic spaces
over $S$. Assume $f$ is proper and $Y$ is locally Noetherian.
Let $y \in Y$ be a point of codimension $\leq 1$ in $Y$.
Let $X^0 \subset |X|$ be the set of points of codimension $0$ on $X$.
Assume in addition one of the
following conditions is satisfied
\begin{enumerate}
\item for every $x \in X^0$ the transcendence degree of $x/f(x)$ is $0$,
\item for every $x \in X^0$ with $f(x) \leadsto y$ the transcendence degree
of $x/f(x)$ is $0$,
\item $f$ is quasi-finite at every $x \in X^0$,
\item $f$ is quasi-finite at a dense set of points of $|X|$,
\item add more here.
\end{enumerate}
Then there exists an open subspace $Y' \subset Y$ containing $y$ such that
$Y' \times_Y X \to Y'$ is finite.
\end{lemma}

\begin{proof}
By Lemma \ref{lemma-quasi-finite-in-codim-1} the morphism $f$ is
quasi-finite at every point lying over $y$. Let $\overline{y} : \Spec(k) \to Y$
be a geometric point lying over $y$. Then $|X_{\overline{y}}|$ is a
discrete space (Decent Spaces, Lemma
\ref{decent-spaces-lemma-conditions-on-fibre-and-qf}).
Since $X_{\overline{y}}$ is quasi-compact as $f$ is proper we conclude
that $|X_{\overline{y}}|$ is finite.
Thus we can apply Cohomology of Spaces, Lemma
\ref{spaces-cohomology-lemma-proper-finite-fibre-finite-in-neighbourhood}
to conclude.
\end{proof}

\begin{lemma}
\label{lemma-modification-normal-iso-over-codimension-1}
Let $S$ be a scheme. Let $X$ be a Noetherian algebraic space over $S$.
Let $f : Y \to X$ be a birational proper morphism of algebraic spaces
with $Y$ reduced.
Let $U \subset X$ be the maximal open over which $f$ is an isomorphism.
Then $U$ contains
\begin{enumerate}
\item every point of codimension $0$ in $X$,
\item every $x \in |X|$ of codimension $1$ on $X$ such that the local ring of
$X$ at $x$ is normal (Properties of Spaces, Remark
\ref{spaces-properties-remark-list-properties-local-ring-local-etale-topology}),
and
\item every $x \in |X|$ such that the fibre of $|Y| \to |X|$ over $x$ is
finite and such that the local ring of $X$ at $x$ is normal.
\end{enumerate}
\end{lemma}

\begin{proof}
Part (1) follows from Decent Spaces, Lemma
\ref{decent-spaces-lemma-birational-isomorphism-over-dense-open}
(and the fact that the Noetherian algebraic spaces $X$ and $Y$
are quasi-separated and hence decent).
Part (2) follows from part (3) and Lemma \ref{lemma-finite-in-codim-1}
(and the fact that finite morphisms have finite fibres).
Let $x \in |X|$ be as in (3). By
Cohomology of Spaces, Lemma
\ref{spaces-cohomology-lemma-proper-finite-fibre-finite-in-neighbourhood}
(which applies by Decent Spaces, Lemma
\ref{decent-spaces-lemma-conditions-on-fibre-and-qf})
we may assume $f$ is finite. Choose an affine scheme $X'$ and
an \'etale morphism $X' \to X$ and a point $x' \in X$ mapping to $x$.
It suffices to show there exists an open neighbourhood $U'$ of $x' \in X'$
such that $Y \times_X X' \to X'$ is an isomorphism over $U'$
(namely, then $U$ contains the image of $U'$ in $X$, see Spaces, Lemma
\ref{spaces-lemma-descent-representable-transformations-property}).
Then $Y \times_X X' \to X$ is a finite birational
(Decent Spaces, Lemma \ref{decent-spaces-lemma-birational-etale-localization})
morphism. Since a finite morphism is affine we reduce to
the case of a finite birational morphism of Noetherian affine schemes
$Y \to X$ and $x \in X$ such that $\mathcal{O}_{X, x}$ is a
normal domain. This is treated in Varieties, Lemma
\ref{varieties-lemma-modification-normal-iso-over-codimension-1}.
\end{proof}






\section{Integral algebraic spaces}
\label{section-integral-spaces}

\noindent
We have not yet defined the notion of an integral algebraic space. The
problem is that being integral is not an \'etale local property of schemes.
We could use the property, that $X$ is reduced and $|X|$ is irreducible,
given in Properties, Lemma \ref{properties-lemma-characterize-integral}
to define integral algebraic spaces. In this case the algebraic
space described in Spaces, Example \ref{spaces-example-infinite-product}
would be integral which does not seem right.
To avoid this type of pathology we will in addition assume that $X$ is a
decent algebraic space, although perhaps a weaker alternative exists.

\begin{definition}
\label{definition-integral-algebraic-space}
Let $S$ be a scheme. We say an algebraic space $X$ over $S$ is
{\it integral} if it is reduced, decent, and $|X|$ is irreducible.
\end{definition}

\noindent
In this case the irreducible topological space $|X|$ is sober
(Decent Spaces, Proposition \ref{decent-spaces-proposition-reasonable-sober}).
Hence it has a unique generic point $x$. In fact, in
Decent Spaces, Lemma
\ref{decent-spaces-lemma-finitely-many-irreducible-components}
we characterized decent algebraic spaces with finitely many
irreducible components. Applying that lemma we see that an
algebraic space $X$ is integral if it is
reduced, has an irreducible dense open subscheme $X'$
with generic point $x'$ and the morphism $x' \to X$ is quasi-compact.

\begin{lemma}
\label{lemma-integral-algebraic-space-rational-functions}
Let $S$ be a scheme. Let $X$ be an integral algebraic space over $S$.
Let $\eta \in |X|$ be the generic point of $X$.
There are canonical identifications
$$
R(X) = \mathcal{O}_{X, \eta}^h = \kappa(\eta)
$$
where $R(X)$ is the ring of rational functions defined in
Morphisms of Spaces, Definition
\ref{spaces-morphisms-definition-ring-of-rational-functions},
$\kappa(\eta)$ is the residue field defined in
Decent Spaces, Definition \ref{decent-spaces-definition-residue-field},
and $\mathcal{O}_{X, \eta}^h$ is the henselian local ring defined in
Decent Spaces, Definition
\ref{decent-spaces-definition-elemenary-etale-neighbourhood}.
In particular, these rings are fields.
\end{lemma}

\begin{proof}
Since $X$ is a scheme in an open neighbourhood of $\eta$ (see discussion
above), this follows immediately from the corresponding result for
schemes, see Morphisms, Lemma
\ref{morphisms-lemma-integral-scheme-rational-functions}.
We also use: the henselianization of a field is itself
and that our definitions of these objects
for algebraic spaces are compatible with those for schemes.
Details omitted.
\end{proof}

\noindent
This leads to the following definition.

\begin{definition}
\label{definition-function-field}
Let $S$ be a scheme. Let $X$ be an integral algebraic space over $S$.
The {\it function field}, or the {\it field of rational functions}
of $X$ is the field $R(X)$ of
Lemma \ref{lemma-integral-algebraic-space-rational-functions}.
\end{definition}

\noindent
We may occasionally indicate this field $k(X)$ instead of $R(X)$.

\begin{lemma}
\label{lemma-integral-sections}
Let $S$ be a scheme. Let $X$ be an integral algebraic space over $S$.
Then $\Gamma(X, \mathcal{O}_X)$ is a domain.
\end{lemma}

\begin{proof}
Set $R = \Gamma(X, \mathcal{O}_X)$. If $f, g \in R$ are nonzero and
$fg = 0$ then $X = V(f) \cup V(g)$ where $V(f)$ denotes the closed subspace
of $X$ cut out by $f$. Since $X$ is irreducible, we see that either
$V(f) = X$ or $V(g) = X$. Then either $f = 0$ or $g = 0$ by
Properties of Spaces, Lemma \ref{spaces-properties-lemma-reduced-space}.
\end{proof}

\noindent
Here is a lemma about normal integral algebraic spaces.

\begin{lemma}
\label{lemma-normal-integral-cover-by-affines}
Let $S$ be a scheme. Let $X$ be a normal integral algebraic space over $S$.
For every $x \in |X|$ there exists a normal integral affine scheme $U$
and an \'etale morphism $U \to X$ such that $x$ is in the image.
\end{lemma}

\begin{proof}
Choose an affine scheme $U$ and an \'etale morphism $U \to X$ such that
$x$ is in the image. Let $u_i$, $i \in I$ be the generic points of irreducible
components of $U$. Then each $u_i$ maps to the generic point of $X$
(Decent Spaces, Lemma \ref{decent-spaces-lemma-decent-generic-points}). By 
our definition of a decent space
(Decent Spaces, Definition \ref{decent-spaces-definition-very-reasonable}),
we see that $I$ is finite. Hence $U = \Spec(A)$ where $A$ is a normal ring
with finitely many minimal primes.
Thus $A = \prod_{i \in I} A_i$ is a product of normal domains by
Algebra, Lemma \ref{algebra-lemma-characterize-reduced-ring-normal}.
Then $U = \coprod U_i$ with $U_i = \Spec(A_i)$ and $x$ is in the image of
$U_i \to X$ for some $i$. This proves the lemma.
\end{proof}

\begin{lemma}
\label{lemma-normal-integral-sections}
Let $S$ be a scheme. Let $X$ be a normal integral algebraic space over $S$.
Then $\Gamma(X, \mathcal{O}_X)$ is a normal domain.
\end{lemma}

\begin{proof}
Set $R = \Gamma(X, \mathcal{O}_X)$. Then $R$ is a domain by
Lemma \ref{lemma-integral-sections}.
Let $f = a/b$ be an element of the fraction field of $R$
which is integral over $R$.
For any $U \to X$ \'etale with $U$ a scheme there is at most one
$f_U \in \Gamma(U, \mathcal{O}_U)$ with $b|_U f_U = a|_U$.
Namely, $U$ is reduced and the generic points of $U$ map to
the generic point of $X$ which implies that $b|_U$ is a
nonzerodivisor.
For every $x \in |X|$ we choose $U \to X$ as in
Lemma \ref{lemma-normal-integral-cover-by-affines}.
Then there is a unique $f_U \in \Gamma(U, \mathcal{O}_U)$
with $b|_U f_U = a|_U$ because
$\Gamma(U, \mathcal{O}_U)$ is a normal domain by
Properties, Lemma \ref{properties-lemma-normal-integral-sections}.
By the uniqueness mentioned above these $f_U$
glue and define a global section $f$ of the structure
sheaf, i.e., of $R$.
\end{proof}

\begin{lemma}
\label{lemma-decent-irreducible-closed}
Let $S$ be a scheme. Let $X$ be a decent algebraic space over $S$.
There are canonical bijections between the following sets:
\begin{enumerate}
\item the set of points of $X$, i.e., $|X|$,
\item the set of irreducible closed subsets of $|X|$,
\item the set of integral closed subspaces of $X$.
\end{enumerate}
The bijection from (1) to (2) sends $x$ to $\overline{\{x\}}$.
The bijection from (3) to (2) sends $Z$ to $|Z|$.
\end{lemma}

\begin{proof}
Our map defines a bijection between (1) and (2) as $|X|$ is
sober by 
Decent Spaces, Proposition \ref{decent-spaces-proposition-reasonable-sober}.
Given $T \subset |X|$ closed and irreducible, there is a
unique reduced closed subspace $Z \subset X$ such that
$|Z| = T$, namely, $Z$ is the reduced induced subspace structure
on $T$, see Properties of Spaces, Definition
\ref{spaces-properties-definition-reduced-induced-space}.
This is an integral algebraic space because it is decent,
reduced, and irreducible.
\end{proof}






\section{Morphisms between integral algebraic spaces}
\label{section-morphisms-between-integral-spaces}

\noindent
The following lemma characterizes dominant morphisms of finite degree
between integral algebraic spaces.

\begin{lemma}
\label{lemma-finite-degree}
Let $S$ be a scheme. Let $X$, $Y$ be integral algebraic spaces over $S$
Let $x \in |X|$ and $y \in |Y|$ be the generic points. Let $f : X \to Y$
be locally of finite type. Assume $f$ is dominant
(Morphisms of Spaces, Definition \ref{spaces-morphisms-definition-dominant}).
The following are equivalent:
\begin{enumerate}
\item the transcendence degree of $x/y$ is $0$,
\item the extension $\kappa(x)/\kappa(y)$ (see proof) is finite,
\item there exist nonempty affine opens $U \subset X$ and $V \subset Y$
such that $f(U) \subset V$ and $f|_U : U \to V$ is finite,
\item $f$ is quasi-finite at $x$, and
\item $x$ is the only point of $|X|$ mapping to $y$.
\end{enumerate}
If $f$ is separated or if $f$ is quasi-compact, then these are
also equivalent to
\begin{enumerate}
\item[(6)] there exists a nonempty affine open $V \subset Y$ such
that $f^{-1}(V) \to V$ is finite.
\end{enumerate}
\end{lemma}

\begin{proof}
By elementary topology, we see that $f(x) = y$ as $f$ is dominant.
Let $Y' \subset Y$ be the schematic locus of $Y$ and let
$X' \subset f^{-1}(Y')$ be the schematic locus of $f^{-1}(Y')$.
By the discussion above, using
Decent Spaces, Proposition \ref{decent-spaces-proposition-reasonable-sober} and
Theorem \ref{decent-spaces-theorem-decent-open-dense-scheme},
we see that $x \in |X'|$ and $y \in |Y'|$.
Then $f|_{X'} : X' \to Y'$ is a morphism of integral schemes
which is locally of finite type. Thus we see that (1), (2), (3)
are equivalent by Morphisms, Lemma \ref{morphisms-lemma-finite-degree}.

\medskip\noindent
Condition (4) implies condition (1) by
Morphisms of Spaces, Lemma \ref{spaces-morphisms-lemma-compare-tr-deg}
applied to $X \to Y \to Y$.
On the other hand, condition (3) implies condition (4) as
a finite morphism is quasi-finite and as $x \in U$ because $x$
is the generic point. Thus (1) -- (4) are equivalent.

\medskip\noindent
Assume the equivalent conditions (1) -- (4). Suppose that
$x' \mapsto y$. Then $x \leadsto x'$ is a specialization in the
fibre of $|X| \to |Y|$ over $y$. If $x' \not = x$, then $f$ is not
quasi-finite at $x$ by Decent Spaces, Lemma
\ref{decent-spaces-lemma-conditions-on-point-in-fibre-and-qf}.
Hence $x = x'$ and (5) holds. Conversely, if (5) holds, then
(5) holds for the morphism of schemes $X' \to Y'$ (see above)
and we can use
Morphisms, Lemma \ref{morphisms-lemma-finite-degree}
to see that (1) holds.

\medskip\noindent
Observe that (6) implies the equivalent conditions (1) -- (5)
without any further assumptions on $f$. To finish the proof
we have to show the equivalent conditions (1) -- (5) imply (6).
This follows from Decent Spaces, Lemma
\ref{decent-spaces-lemma-finite-over-dense-open}.
\end{proof}

\begin{definition}
\label{definition-degree}
Let $S$ be a scheme.
Let $X$ and $Y$ be integral algebraic spaces over $S$.
Let $f : X \to Y$ be locally of finite type and dominant.
Assume any of the equivalent conditions (1) -- (5) of
Lemma \ref{lemma-finite-degree}. Let $x \in |X|$ and $y \in |Y|$
be the generic points. Then the positive integer
$$
\deg(X/Y) = [\kappa(x) : \kappa(y)]
$$
is called the {\it degree of $X$ over $Y$}.
\end{definition}

\begin{lemma}
\label{lemma-degree-composition}
Let $S$ be a scheme.
Let $X$, $Y$, $Z$ be integral algebraic spaces over $S$.
Let $f : X \to Y$ and $g : Y \to Z$ be dominant morphisms locally
of finite type. Assume any of the equivalent conditions
(1) -- (5) of Lemma \ref{lemma-finite-degree} hold for $f$ and $g$. Then
$$
\deg(X/Z) = \deg(X/Y) \deg(Y/Z).
$$
\end{lemma}

\begin{proof}
This comes from the multiplicativity of degrees in towers
of finite extensions of fields, see
Fields, Lemma \ref{fields-lemma-multiplicativity-degrees}.
\end{proof}










\section{Weil divisors}
\label{section-Weil-divisors}

\noindent
This section is the analogue of Divisors, Section
\ref{divisors-section-Weil-divisors}.

\medskip\noindent
We will introduce Weil divisors and rational equivalence of Weil
divisors for locally Noetherian integral algebraic spaces.
Since we are not assuming our algebraic spaces are quasi-compact we have
to be a little careful when defining Weil divisors. We have to allow
infinite sums of prime divisors because a rational function may have
infinitely many poles for example. In the quasi-compact case our
Weil divisors are finite sums as usual. Here is a basic lemma we will
often use to prove collections of closed subspaces are locally finite.

\begin{lemma}
\label{lemma-components-locally-finite}
Let $S$ be a scheme and let $X$ be a locally Noetherian
algebraic space over $S$. If $T \subset |X|$ is a closed subset,
then the collection of irreducible components of $T$ is locally finite.
\end{lemma}

\begin{proof}
The topological space $|X|$ is locally Noetherian
(Properties of Spaces, Lemma \ref{spaces-properties-lemma-Noetherian-topology}).
A Noetherian topological space has a finite number of
irreducible components and a subspace of a Noetherian space is Noetherian
(Topology, Lemma \ref{topology-lemma-Noetherian}).
Thus the lemma follows from the definition of locally finite
(Topology, Definition \ref{topology-definition-locally-finite}).
\end{proof}

\noindent
Let $S$ be a scheme. Let $X$ be a decent algebraic space over $S$.
Let $Z$ be an integral closed subspace of $X$ and let
$\xi \in |Z|$ be the generic point. Then the codimension of
$|Z|$ in $|X|$ is equal to the dimension of the local ring
of $X$ at $\xi$ by
Decent Spaces, Lemma \ref{decent-spaces-lemma-codimension-local-ring}.
Recall that we also indicate this by saying that
{\it $\xi$ is a point of codimension $1$ on $X$}, see
Properties of Spaces, Definition
\ref{spaces-properties-definition-dimension-local-ring}.

\begin{definition}
\label{definition-Weil-divisor}
Let $S$ be a scheme.
Let $X$ be a locally Noetherian integral algebraic space over $S$.
\begin{enumerate}
\item A {\it prime divisor} is an integral closed subspace $Z \subset X$
of codimension $1$, i.e., the generic point of $|Z|$ is a point
of codimension $1$ on $X$.
\item A {\it Weil divisor} is a formal sum $D = \sum n_Z Z$ where
the sum is over prime divisors of $X$ and the collection
$\{|Z| : n_Z \not = 0\}$ is locally finite in $|X|$
(Topology, Definition \ref{topology-definition-locally-finite}).
\end{enumerate}
The group of all Weil divisors on $X$ is denoted $\text{Div}(X)$.
\end{definition}

\noindent
Our next task is to define the Weil divisor associated to a rational
function. In order to do this we need to define the order of vanishing of a
rational function on a locally Noetherian integral algebraic space $X$
along a prime divisor $Z$. Let $\xi \in |Z|$ be the generic point.
Here we run into the problem that the local ring $\mathcal{O}_{X, \xi}$
doesn't exist and the henselian local ring $\mathcal{O}_{X, \xi}^h$
may not be a domain, see Example \ref{example-not-a-domain}.
To get around this we use the following lemma.

\begin{lemma}
\label{lemma-order-vanishing}
Let $S$ be a scheme. Let $X$ be a locally Noetherian integral algebraic space
over $S$. Let $Z \subset X$ be a prime divisor and let $\xi \in |Z|$ be
the generic point. Then the henselian local ring $\mathcal{O}_{X, \xi}^h$
is a reduced $1$-dimensional Noetherian local ring and
there is a canonical injective map
$$
R(X) \longrightarrow Q(\mathcal{O}_{X, \xi}^h)
$$
from the function field $R(X)$ of $X$ into the total ring of fractions.
\end{lemma}

\begin{proof}
We will use the results of Decent Spaces, Section
\ref{decent-spaces-section-residue-fields-henselian-local-rings}.
Let $(U, u) \to (X, \xi)$ be an elementary \'etale neighbourhood.
Observe that $U$ is locally Noetherian and reduced.
Thus $\mathcal{O}_{U, u}$ is a $1$-dimensional
(by our definition of prime divisors)
reduced Noetherian ring.
After replacing $U$ by an affine open neighbourhood of $u$
we may assume $U$ is Noetherian and affine.
After replacing $U$ by a smaller open, we may assume every irreducible
component of $U$ passes through $u$.
Since $U \to X$ is open and $X$ irreducible, $U \to X$ is dominant.
Hence we obtain a ring map $R(X) \to R(U)$ by composing rational maps, see
Morphisms of Spaces, Section \ref{spaces-morphisms-section-rational-maps}.
Since $R(X)$ is a field, this map is injective.
By our choice of $U$ we see that $R(U)$ is the total quotient
ring $Q(\mathcal{O}_{U, u})$, see
Morphisms, Lemma \ref{morphisms-lemma-integral-scheme-rational-functions}
and
Algebra, Lemma \ref{algebra-lemma-total-ring-fractions-no-embedded-points}.

\medskip\noindent
At this point we have proved all the statements in the lemma with
$\mathcal{O}_{U, u}$ in stead of $\mathcal{O}_{X, \xi}^h$.
However, $\mathcal{O}_{X, \xi}^h$ is the henselization of
$\mathcal{O}_{U, u}$. Thus $\mathcal{O}_{X, \xi}^h$ is a $1$-dimensional
reduced Noetherian ring, see
More on Algebra, Lemmas
\ref{more-algebra-lemma-henselization-reduced},
\ref{more-algebra-lemma-henselization-dimension}, and
\ref{more-algebra-lemma-henselization-noetherian}.
Since $\mathcal{O}_{U, u} \to \mathcal{O}_{X, \xi}^h$ is
faithfully flat by More on Algebra, Lemma
\ref{more-algebra-lemma-dumb-properties-henselization}
it sends nonzerodivisors to nonzerodivisors.
Therefore we obtain a canonical map
$Q(\mathcal{O}_{U, u}) \to Q(\mathcal{O}_{X, \xi}^h)$
and we obtain our map.
We omit the verification that the map is independent of
the choice of $(U, u) \to (X, x)$; a slightly better
approach would be to first observe that
$\colim Q(\mathcal{O}_{U, u}) = Q(\mathcal{O}_{X, \xi}^h)$.
\end{proof}

\begin{definition}
\label{definition-order-vanishing}
Let $S$ be a scheme. Let $X$ be a locally Noetherian integral algebraic
space over $S$. Let $f \in R(X)^*$. For every prime divisor
$Z \subset X$ we define the {\it order of vanishing of $f$ along $Z$}
as the integer
$$
\text{ord}_Z(f) =
\text{length}_{\mathcal{O}_{X, \xi}^h}
(\mathcal{O}_{X, \xi}^h/a \mathcal{O}_{X, \xi}^h) -
\text{length}_{\mathcal{O}_{X, \xi}^h}
(\mathcal{O}_{X, \xi}^h/b \mathcal{O}_{X, \xi}^h)
$$
where $a, b \in \mathcal{O}_{X, \xi}^h$ are nonzerodivisors
such that the image of $f$ in $Q(\mathcal{O}_{X, \xi}^h)$
(Lemma \ref{lemma-order-vanishing}) is equal to $a/b$.
This is well defined by
Algebra, Lemma \ref{algebra-lemma-ord-additive}.
\end{definition}

\noindent
If $\mathcal{O}_{X, \xi}^h$ happens to be a domain, then we obtain
$$
\text{ord}_Z(f) = \text{ord}_{\mathcal{O}_{X, \xi}^h}(f)
$$
where the right hand side is the notion of
Algebra, Definition \ref{algebra-definition-ord}.
Note that for $f, g \in R(X)^*$ we have
$$
\text{ord}_Z(fg) = \text{ord}_Z(f) + \text{ord}_Z(g).
$$
Of course it can happen that $\text{ord}_Z(f) < 0$.
In this case we say that $f$ has a {\it pole} along $Z$
and that $-\text{ord}_Z(f) > 0$ is the
{\it order of pole of $f$ along $Z$}. It is important to note
that the condition $\text{ord}_Z(f) \geq 0$ is {\bf not} equivalent
to the condition $f \in \mathcal{O}_{X, \xi}^h$ unless the local
ring $\mathcal{O}_{X, \xi}$ is a discrete valuation ring.

\begin{lemma}
\label{lemma-order-vanishing-agrees}
Let $S$ be a scheme. Let $X$ be a locally Noetherian integral algebraic
space over $S$. Let $f \in R(X)^*$. If the prime divisor
$Z \subset X$ meets the schematic locus of $X$, then the order
of vanishing $\text{ord}_Z(f)$ of Definition \ref{definition-order-vanishing}
agrees with the order of vanishing of
Divisors, Definition \ref{divisors-definition-order-vanishing}.
\end{lemma}

\begin{proof}
After shrinking $X$ we may assume $X$ is an integral Noetherian scheme.
If $\xi \in Z$ denotes the generic point, then we find that
$\mathcal{O}_{X, \xi}^h$ is the henselization of $\mathcal{O}_{X, \xi}$
(Decent Spaces, Lemma \ref{decent-spaces-lemma-describe-henselian-local-ring}).
To prove the lemma it suffices and is necessary to show that
$$
\text{length}_{\mathcal{O}_{X, \xi}}
(\mathcal{O}_{X, \xi}/a \mathcal{O}_{X, \xi}) =
\text{length}_{\mathcal{O}_{X, \xi}^h}
(\mathcal{O}_{X, \xi}^h/a \mathcal{O}_{X, \xi}^h)
$$
This follows immediately from
Algebra, Lemma \ref{algebra-lemma-pullback-module}
(and the fact that $\mathcal{O}_{X, \xi} \to \mathcal{O}_{X, \xi}^h$
is a flat local ring homomorphism of local Noetherian rings).
\end{proof}

\begin{lemma}
\label{lemma-divisor-locally-finite}
Let $S$ be a scheme. Let $X$ be a locally Noetherian integral algebraic space
over $S$. Let $f \in R(X)^*$. Then the collections
$$
\{Z \subset X \mid Z\text{ a prime divisor with generic point }\xi
\text{ and }f\text{ not in }\mathcal{O}_{X, \xi}\}
$$
and
$$
\{Z \subset X \mid Z \text{ a prime divisor and }\text{ord}_Z(f) \not = 0\}
$$
are locally finite in $X$.
\end{lemma}

\begin{proof}
There exists a nonempty open subspace $U \subset X$
such that $f$ corresponds to a section of $\Gamma(U, \mathcal{O}_X^*)$.
Hence the prime divisors which can occur in the sets of the lemma all
correspond to irreducible components of $|X| \setminus |U|$.
Hence Lemma \ref{lemma-components-locally-finite} gives the desired result.
\end{proof}

\noindent
This lemma allows us to make the following definition.

\begin{definition}
\label{definition-principal-divisor}
Let $S$ be a scheme.
Let $X$ be a locally Noetherian integral algebraic space over $S$.
Let $f \in R(X)^*$.
The {\it principal Weil divisor associated to $f$} is the Weil divisor
$$
\text{div}(f) = \text{div}_X(f) = \sum \text{ord}_Z(f) [Z]
$$
where the sum is over prime divisors and $\text{ord}_Z(f)$ is as in
Definition \ref{definition-order-vanishing}. This makes sense
by Lemma \ref{lemma-divisor-locally-finite}.
\end{definition}

\begin{lemma}
\label{lemma-div-additive}
Let $S$ be a scheme.
Let $X$ be a locally Noetherian integral algebraic space over $S$.
Let $f, g \in R(X)^*$. Then
$$
\text{div}_X(fg) = \text{div}_X(f) + \text{div}_X(g)
$$
as Weil divisors on $X$.
\end{lemma}

\begin{proof}
This is clear from the additivity of the $\text{ord}$ functions.
\end{proof}

\noindent
We see from the lemma above that the collection of principal Weil divisors
form a subgroup of the group of all Weil divisors. This leads to the following
definition.

\begin{definition}
\label{definition-class-group}
Let $S$ be a scheme.
Let $X$ be a locally Noetherian integral algebraic space over $S$. The
{\it Weil divisor class group} of $X$ is the quotient of
the group of Weil divisors by the subgroup of principal Weil divisors.
Notation: $\text{Cl}(X)$.
\end{definition}

\noindent
By construction we obtain an exact complex
\begin{equation}
\label{equation-Weil-divisor-class}
R(X)^* \xrightarrow{\text{div}} \text{Div}(X) \to \text{Cl}(X) \to 0
\end{equation}
which we can think of as a presentation of $\text{Cl}(X)$. Our next task
is to relate the Weil divisor class group to the Picard group.

\begin{example}
\label{example-Z-mod-2}
This is a continuation of Morphisms of Spaces, Example
\ref{spaces-morphisms-example-universal-homeomorphism}.
Consider the algebraic space
$X = \mathbf{A}^1_k/\{t \sim -t \mid t \not = 0\}$.
This is a smooth algebraic space over the field $k$.
There is a universal homeomorphism
$$
X \longrightarrow \mathbf{A}^1_k = \Spec(k[t])
$$
which is an isomorphism over $\mathbf{A}^1_k \setminus \{0\}$.
We conclude that $X$ is Noetherian and integral.
Since $\dim(X) = 1$, we see that the prime divisors of $X$ are
the closed points of $X$.
Consider the unique closed point $x \in |X|$ lying over $0 \in \mathbf{A}^1_k$.
Since $X \setminus \{x\}$ maps isomorphically to $\mathbf{A}^1 \setminus \{0\}$
we see that the classes in $\text{Cl}(X)$ of closed points different
from $x$ are zero. However, the divisor of $t$ on $X$ is $2[x]$.
We conclude that $\text{Cl}(X) = \mathbf{Z}/2\mathbf{Z}$.
\end{example}

\begin{example}
\label{example-not-a-domain}
Let $k$ be a field. Let
$$
U = \Spec(k[x, y]/(xy))
$$
be the union of the coordinate axes in $\mathbf{A}^2_k$.
Denote $\Delta : U \to U \times_k U$ the diagonal and
$\Delta' : U \to U \times_k U$ the map $u \mapsto (u, \sigma(u))$
where $\sigma : U \to U$, $(x, y) \mapsto (y, x)$
is the automorphism flipping the coordinate axes. Set
$$
R = \Delta(U) \amalg \Delta'(U \setminus \{0_U\})
$$
where $0_U \in U$ is the origin. It is easy to see that
$R$ is an \'etale equivalence relation on $U$. The quotient $X = U/R$
is an algebraic space. The morphism $U \to \mathbf{A}^1_k$,
$(x, y) \mapsto x + y$ is $R$-invariant and hence defines
a morphism
$$
X \longrightarrow \mathbf{A}^1_k
$$
This morphism is a universal homeomorphism and an isomorphism over
$\mathbf{A}^1_k \setminus \{0\}$. It follows that $X$ is integral
and Noetherian. Exactly as in Example \ref{example-Z-mod-2}
the reader shows that $\text{Cl}(X) = \mathbf{Z}/2\mathbf{Z}$
with generator corresponding to the unique closed point $x \in |X|$
mapping to $0 \in \mathbf{A}^1_k$. However, in this case the henselian
local ring of $X$ at $x$ isn't a domain, as it is the henselization
of $\mathcal{O}_{U, 0_U}$.
\end{example}














\section{The Weil divisor class associated to an invertible module}
\label{section-c1}

\noindent
In this section we go through exactly the same progression as in
Section \ref{section-Weil-divisors} to define a canonical map
$\Pic(X) \to \text{Cl}(X)$
on a locally Noetherian integral algebraic space.

\medskip\noindent
Let $S$ be a scheme. Let $X$ be a locally Noetherian integral algebraic space
over $S$. Let $\mathcal{L}$ be an invertible $\mathcal{O}_X$-module.
By Divisors on Spaces, Lemma
\ref{spaces-divisors-lemma-regular-meromorphic-section-exists}
there exists a regular meromorphic section
$s \in \Gamma(X, \mathcal{K}_X(\mathcal{L}))$.
In fact, by Divisors on Spaces, Lemma
\ref{spaces-divisors-lemma-compute-meromorphic}
this is the same thing as a nonzero element in
$\mathcal{L}_\eta$ where $\eta \in |X|$ is the generic point.
The same lemma tells us that if $\mathcal{L} = \mathcal{O}_X$,
then $s$ is the same thing as a nonzero rational function on $X$
(so what we will do below matches the construction in
Section \ref{section-Weil-divisors}).

\medskip\noindent
Let $Z \subset X$ be a prime divisor and let $\xi \in |Z|$ be the generic point.
We are going to define the order of vanishing of $s$ along $Z$.
Consider the canonical morphism
$$
c_\xi : \Spec(\mathcal{O}_{X, \xi}^h) \longrightarrow X
$$
whose source is the spectrum of the henselian local ring of $X$ as $\xi$
(Decent Spaces, Definition \ref{decent-spaces-definition-henselian-local-ring}).
The pullback $\mathcal{L}_\xi = c_\xi^*\mathcal{L}$
is an invertible module and hence trivial; choose a generator $s_\xi$ of
$\mathcal{L}_\xi$. Since $c_\xi$ is flat, pullbacks of meromorphic functions
and (regular) sections are defined for $c_\xi$, see
Divisors on Spaces, Definition
\ref{spaces-divisors-definition-pullback-meromorphic-sections} and
Lemmas \ref{spaces-divisors-lemma-pullback-meromorphic-sections-defined} and
\ref{spaces-divisors-lemma-meromorphic-sections-pullback}.
Thus we get
$$
c_\xi^*(s) = f s_\xi
$$
for some nonzerodivisor $f \in Q(\mathcal{O}_{X, \xi}^h)$.
Here we are using Divisors, Lemma \ref{divisors-lemma-locally-Noetherian-K}
to identify the space of meromorphic sections of
$\mathcal{L}_\xi \cong \mathcal{O}_{\Spec(\mathcal{O}_{X, \xi}^h)}$
in terms of the total ring of fractions of $\mathcal{O}_{X, \xi}^h$.
Let us agree to denote this element
$$
s/s_\xi = f \in  Q(\mathcal{O}_{X, \xi}^h)
$$
Observe that $f = s/s_\xi$
is replaced by $uf$ where $u \in \mathcal{O}_{X, \xi}^h$
is a unit if we change our choice of $s_\xi$.

\begin{definition}
\label{definition-order-vanishing-meromorphic}
Let $S$ be a scheme. Let $X$ be a locally Noetherian integral
algebraic algebraic space over $S$. Let $\mathcal{L}$ be an
invertible $\mathcal{O}_X$-module.
Let $s \in \Gamma(X, \mathcal{K}_X(\mathcal{L}))$
be a regular meromorphic section of $\mathcal{L}$.
For every prime divisor $Z \subset X$ with generic point $\xi \in |Z|$
we define the
{\it order of vanishing of $s$ along $Z$}
as the integer
$$
\text{ord}_{Z, \mathcal{L}}(s) =
\text{length}_{\mathcal{O}_{X, \xi}^h}
(\mathcal{O}_{X, \xi}^h/a \mathcal{O}_{X, \xi}^h) -
\text{length}_{\mathcal{O}_{X, \xi}^h}
(\mathcal{O}_{X, \xi}^h/b \mathcal{O}_{X, \xi}^h)
$$
where $a, b \in \mathcal{O}_{X, \xi}^h$ are nonzerodivisors
such that the element $s/s_\xi$ of $Q(\mathcal{O}_{X, \xi}^h)$
constructed above is equal to $a/b$.
This is well defined by the above and
Algebra, Lemma \ref{algebra-lemma-ord-additive}.
\end{definition}

\noindent
As explained above, a regular meromorphic section $s$ of $\mathcal{O}_X$
can be written $s = f \cdot 1$ where $f$
is a nonzero rational function on $X$ and we have
$\text{ord}_Z(f) = \text{ord}_{Z, \mathcal{O}_X}(s)$.
As in the case of principal divisors we have the following lemma.

\begin{lemma}
\label{lemma-divisor-meromorphic-locally-finite}
Let $S$ be a scheme. Let $X$ be a locally Noetherian integral algebraic space
over $S$. Let $\mathcal{L}$ be an invertible $\mathcal{O}_X$-module.
Let $s \in \mathcal{K}_X(\mathcal{L})$ be a
regular (i.e., nonzero) meromorphic section of $\mathcal{L}$. Then the sets
$$
\{Z \subset X \mid Z \text{ a prime divisor with generic point }\xi
\text{ and }s\text{ not in }\mathcal{L}_\xi\}
$$
and
$$
\{Z \subset X \mid Z \text{ is a prime divisor and }
\text{ord}_{Z, \mathcal{L}}(s) \not = 0\}
$$
are locally finite in $X$.
\end{lemma}

\begin{proof}
There exists a nonempty open subspace $U \subset X$ such that $s$
corresponds to a section of $\Gamma(U, \mathcal{L})$ which generates
$\mathcal{L}$ over $U$. Hence the prime divisors which can occur
in the sets of the lemma all correspond to irreducible components of
$|X| \setminus |U|$. Hence Lemma \ref{lemma-components-locally-finite}.
gives the desired result.
\end{proof}

\begin{lemma}
\label{lemma-divisor-meromorphic-well-defined}
Let $S$ be a scheme. Let $X$ be a locally Noetherian integral algebraic space
over $S$ Let $\mathcal{L}$ be an invertible $\mathcal{O}_X$-module.
Let $s, s' \in \mathcal{K}_X(\mathcal{L})$ be nonzero
meromorphic sections of $\mathcal{L}$. Then $f = s/s'$
is an element of $R(X)^*$ and we have
$$
\sum \text{ord}_{Z, \mathcal{L}}(s)[Z]
=
\sum \text{ord}_{Z, \mathcal{L}}(s')[Z]
+
\text{div}(f)
$$
as Weil divisors.
\end{lemma}

\begin{proof}
This is clear from the definitions.
Note that Lemma \ref{lemma-divisor-meromorphic-locally-finite}
guarantees that the sums are indeed Weil divisors.
\end{proof}

\begin{definition}
\label{definition-divisor-invertible-sheaf}
Let $S$ be a scheme. Let $X$ be a locally Noetherian integral algebraic space
over $S$. Let $\mathcal{L}$ be an invertible $\mathcal{O}_X$-module.
\begin{enumerate}
\item For any nonzero meromorphic section $s$ of $\mathcal{L}$
we define the {\it Weil divisor associated to $s$} as
$$
\text{div}_\mathcal{L}(s) =
\sum \text{ord}_{Z, \mathcal{L}}(s) [Z] \in \text{Div}(X)
$$
where the sum is over prime divisors. This is well defined by
Lemma \ref{lemma-divisor-meromorphic-locally-finite}.
\item We define {\it Weil divisor class associated to $\mathcal{L}$}
as the image of $\text{div}_\mathcal{L}(s)$ in $\text{Cl}(X)$
where $s$ is any nonzero meromorphic section of $\mathcal{L}$ over $X$.
This is well defined by
Lemma \ref{lemma-divisor-meromorphic-well-defined}.
\end{enumerate}
\end{definition}

\noindent
As expected this construction is additive in the invertible module.

\begin{lemma}
\label{lemma-c1-additive}
Let $S$ be a scheme. Let $X$ be a locally Noetherian integral algebraic space
over $S$. Let $\mathcal{L}$, $\mathcal{N}$ be invertible
$\mathcal{O}_X$-modules. Let $s$, resp.\ $t$ be a nonzero meromorphic section
of $\mathcal{L}$, resp.\ $\mathcal{N}$. Then $st$ is a nonzero
meromorphic section of $\mathcal{L} \otimes_{\mathcal{O}_X} \mathcal{N}$ and
$$
\text{div}_{\mathcal{L} \otimes \mathcal{N}}(st)
=
\text{div}_\mathcal{L}(s) + \text{div}_\mathcal{N}(t)
$$
in $\text{Div}(X)$. In particular, the Weil divisor class of
$\mathcal{L} \otimes_{\mathcal{O}_X} \mathcal{N}$ is the sum
of the Weil divisor classes of $\mathcal{L}$ and $\mathcal{N}$.
\end{lemma}

\begin{proof}
Let $s$, resp.\ $t$ be a nonzero meromorphic section
of $\mathcal{L}$, resp.\ $\mathcal{N}$. Then $st$ is a nonzero
meromorphic section of $\mathcal{L} \otimes \mathcal{N}$.
Let $Z \subset X$ be a prime divisor. Let $\xi \in |Z|$ be its generic
point. Choose generators $s_\xi \in \mathcal{L}_\xi$, and
$t_\xi \in \mathcal{N}_\xi$ with notation as described earlier
in this section. Then $s_\xi \otimes t_\xi$ is a generator
for $(\mathcal{L} \otimes \mathcal{N})_\xi$.
So $st/(s_\xi t_\xi) = (s/s_\xi)(t/t_\xi)$ in
$Q(\mathcal{O}_{X, \xi}^h)$. Applying the additivity of
Algebra, Lemma \ref{algebra-lemma-ord-additive}
we conclude that
$$
\text{div}_{\mathcal{L} \otimes \mathcal{N}, Z}(st)
=
\text{div}_{\mathcal{L}, Z}(s) + \text{div}_{\mathcal{N}, Z}(t)
$$
Some details omitted.
\end{proof}

\noindent
Let $S$ be a scheme. Let $X$ be a locally Noetherian integral algebraic space
over $S$. By the constructions and lemmas above we obtain a homomorphism
of abelian groups
\begin{equation}
\label{equation-c1}
\Pic(X) \longrightarrow \text{Cl}(X)
\end{equation}
which assigns to an invertible module its Weil divisor class.

\begin{lemma}
\label{lemma-normal-c1-injective}
Let $S$ be a scheme. Let $X$ be a locally Noetherian integral algebraic space
over $S$. If $X$ is normal, then the map (\ref{equation-c1})
$\Pic(X) \to \text{Cl}(X)$ is injective.
\end{lemma}

\begin{proof}
Let $\mathcal{L}$ be an invertible $\mathcal{O}_X$-module whose
associated Weil divisor class is trivial. Let $s$ be a regular
meromorphic section of $\mathcal{L}$. The assumption means that
$\text{div}_\mathcal{L}(s) = \text{div}(f)$ for some
$f \in R(X)^*$. Then we see that $t = f^{-1}s$ is a regular
meromorphic section of $\mathcal{L}$ with
$\text{div}_\mathcal{L}(t) = 0$, see
Lemma \ref{lemma-divisor-meromorphic-well-defined}.
We claim that $t$ defines a trivialization of $\mathcal{L}$.
The claim finishes the proof of the lemma.
Our proof of the claim is a bit awkward as we
don't yet have a lot of theory at our dispposal; we suggest
the reader skip the proof.

\medskip\noindent
We may check our claim \'etale locally. Let $U \in X_\etale$ be affine
such that $\mathcal{L}|_U$ is trivial. Say $s_U \in \Gamma(U, \mathcal{L}|_U)$
is a trivialization. By
Properties, Lemma \ref{properties-lemma-normal-locally-finite-nr-irreducibles}
we may also assume $U$ is integral. Write $U = \Spec(A)$ as the spectrum of a
normal Noetherian domain $A$ with fraction field $K$.
We may write $t|_U = f s_U$ for some element $f$ of $K$, see
Divisors on Spaces, Lemma \ref{spaces-divisors-lemma-meromorphic-quasi-coherent}
for example. Let $\mathfrak p \subset A$ be a height one prime
corresponding to a codimension $1$ point $u \in U$ which maps
to a codimension $1$ point $\xi \in |X|$. Choose a trivialization
$s_\xi$ of $\mathcal{L}_\xi$ as in the beginning of this section.
Choose a geometric point $\overline{u}$ of $U$ lying over $u$.
Then
$$
(\mathcal{O}_{X, \xi}^h)^{sh} =
\mathcal{O}_{X, \overline{u}} =
\mathcal{O}_{U, u}^{sh} = (A_\mathfrak p)^{sh}
$$
see Decent Spaces, Lemmas
\ref{decent-spaces-lemma-henselian-local-ring-strict}
and
Properties of Spaces, Lemma
\ref{spaces-properties-lemma-describe-etale-local-ring}.
The normality of $X$ shows that all of these are
discrete valuation rings. The trivializations $s_U$ and $s_\xi$
differ by a unit as sections of $\mathcal{L}$ pulled back to
$\Spec(\mathcal{O}_{X, \overline{u}})$.
Write $t = f_\xi s_\xi$ with $f_\xi \in Q(\mathcal{O}_{X, \xi}^h)$.
We conclude that $f_\xi$ and $f$ differ by a unit
in $Q(\mathcal{O}_{X, \overline{u}})$.
If $Z \subset X$ denotes the prime divisor corresponding to $\xi$
(Lemma \ref{lemma-decent-irreducible-closed}), then
$0 = \text{ord}_{Z, \mathcal{L}}(t) =
\text{ord}_{\mathcal{O}_{X, \xi}^h}(f_\xi)$
and since $\mathcal{O}_{X, \xi}^h$ is a discrete valuation ring
we see that $f_\xi$ is a unit.
Thus $f$ is a unit in $\mathcal{O}_{X, \overline{u}}$
and hence in particular $f \in A_\mathfrak p$.
This implies $f \in A$ by Algebra, Lemma
\ref{algebra-lemma-normal-domain-intersection-localizations-height-1}.
We conclude that $t \in \Gamma(X, \mathcal{L})$.
Repeating the argument with $t^{-1}$ viewed as a meromorphic
section of $\mathcal{L}^{\otimes -1}$ finishes the proof.
\end{proof}

















\section{Modifications and alterations}
\label{section-modifications-alterations}

\noindent
Using our notion of an integral algebraic space we can define a modification
as follows.

\begin{definition}
\label{definition-modification}
Let $S$ be a scheme. Let $X$ be an integral algebraic space over $S$. A
{\it modification of $X$} is a birational proper morphism
$f : X' \to X$ of algebraic spaces over $S$ with $X'$ integral.
\end{definition}

\noindent
For birational morphisms of algebraic spaces, see
Decent Spaces, Definition \ref{decent-spaces-definition-birational}.

\begin{lemma}
\label{lemma-modification-iso-over-open}
Let $f : X' \to X$ be a modification as in
Definition \ref{definition-modification}.
There exists a nonempty open $U \subset X$ such that $f^{-1}(U) \to U$
is an isomorphism.
\end{lemma}

\begin{proof}
By
Lemma \ref{lemma-finite-degree} there exists a nonempty $U \subset X$ such
that $f^{-1}(U) \to U$ is finite. By generic flatness
(Morphisms of Spaces, Proposition
\ref{spaces-morphisms-proposition-generic-flatness-reduced})
we may assume $f^{-1}(U) \to U$ is flat and of finite presentation.
So $f^{-1}(U) \to U$ is finite locally free
(Morphisms of Spaces, Lemma \ref{spaces-morphisms-lemma-finite-flat}).
Since $f$ is birational, the degree of $X'$ over $X$ is $1$.
Hence $f^{-1}(U) \to U$ is finite locally free of degree $1$,
in other words it is an isomorphism.
\end{proof}

\begin{definition}
\label{definition-alteration}
Let $S$ be a scheme. Let $X$ be an integral algebraic space over $S$.
An {\it alteration of $X$} is a proper dominant morphism $f : Y \to X$
of algebraic spaces over $S$ with $Y$ integral such that $f^{-1}(U) \to U$
is finite for some nonempty open $U \subset X$.
\end{definition}

\noindent
If $f : Y \to X$ is a dominant and proper morphism between integral
algebraic spaces, then it is an alteration as soon as the induced
extension of residue fields in generic points is finite. Here is the
precise statement.

\begin{lemma}
\label{lemma-alteration-generically-finite}
Let $S$ be a scheme. Let $f : X \to Y$ be a proper dominant morphism of
integral algebraic spaces over $S$. Then $f$ is an alteration
if and only if any of the equivalent conditions (1) -- (6) of
Lemma \ref{lemma-finite-degree} hold.
\end{lemma}

\begin{proof}
Immediate consequence of the lemma referenced in the statement.
\end{proof}

\begin{lemma}
\label{lemma-alteration-contained-in}
Let $S$ be a scheme. Let $f : X \to Y$ be a proper surjective morphism of
algebraic spaces over $S$. Assume $Y$ is integral. Then
there exists an integral closed subspace $X' \subset X$ such that
$f' = f|_{X'} : X' \to Y$ is an alteration.
\end{lemma}

\begin{proof}
Let $V \subset Y$ be a nonempty open affine
(Decent Spaces, Theorem \ref{decent-spaces-theorem-decent-open-dense-scheme}).
Let $\eta \in V$ be the generic point. Then
$X_\eta$ is a nonempty proper algebraic space over $\eta$.
Choose a closed point $x \in |X_\eta|$
(exists because $|X_\eta|$ is a quasi-compact, sober
topological space, see Decent Spaces, Proposition
\ref{decent-spaces-proposition-reasonable-sober}
and Topology, Lemma \ref{topology-lemma-quasi-compact-closed-point}.)
Let $X'$ be the reduced induced closed subspace structure on
$\overline{\{x\}} \subset |X|$ (Properties of Spaces, Definition
\ref{spaces-properties-definition-reduced-induced-space}.
Then $f' : X' \to Y$ is surjective as the image contains $\eta$.
Also $f'$ is proper as a composition of a closed immersion
and a proper morphism. Finally, the fibre $X'_\eta$ has a
single point; to see this use
Decent Spaces, Lemma \ref{decent-spaces-lemma-topology-fibre}
for both $X \to Y$ and $X' \to Y$ and the point $\eta$.
Since $Y$ is decent and $X' \to Y$ is separated we see that $X'$ is decent
(Decent Spaces, Lemmas
\ref{decent-spaces-lemma-properties-trivial-implications} and
\ref{decent-spaces-lemma-property-over-property}).
Thus $f'$ is an alteration by
Lemma \ref{lemma-alteration-generically-finite}.
\end{proof}








\section{Schematic locus}
\label{section-schematic}

\noindent
We have already proven a number of results on the schematic locus
of an algebraic space. Here is a list of references:
\begin{enumerate}
\item Properties of Spaces, Sections
\ref{spaces-properties-section-schematic} and
\ref{spaces-properties-section-getting-a-scheme},
\item Decent Spaces, Section \ref{decent-spaces-section-schematic},
\item Properties of Spaces, Lemma
\ref{spaces-properties-lemma-point-like-spaces}
$\Leftarrow$
Decent Spaces, Lemma \ref{decent-spaces-lemma-decent-point-like-spaces}
$\Leftarrow$
Decent Spaces, Lemma \ref{decent-spaces-lemma-when-field},
\item Limits of Spaces, Section \ref{spaces-limits-section-affine}, and
\item Limits of Spaces, Section \ref{spaces-limits-section-representable}.
\end{enumerate}
There are some cases where certain types of morphisms of algebraic spaces
are automatically representable, for example
separated, locally quasi-finite morphisms (Morphisms of Spaces, Lemma
\ref{spaces-morphisms-lemma-locally-quasi-finite-separated-representable}),
and flat monomorphisms (More on Morphisms of Spaces, Lemma
\ref{spaces-more-morphisms-lemma-flat-case}).
In Section \ref{section-schematic-and-field-extension}
we will study what happens with the schematic
locus under extension of base field.

\begin{lemma}
\label{lemma-locally-finite-type-dim-zero}
Let $S$ be a scheme. Let $X$ be an algebraic space over $S$.
Assume $X$ satisfies at least one of the following conditions
\begin{enumerate}
\item $X$ is quasi-separated and $\dim(X) = 0$,
\item $X$ is locally of finite type over a field $k$ and $\dim(X) = 0$,
\item $X$ is Noetherian and $\dim(X) = 0$, or
\item add more here.
\end{enumerate}
Then $X$ is a separated scheme and any quasi-compact open of $X$ is affine.
\end{lemma}

\begin{proof}
If we prove that any quasi-compact open of $X$ is affine, then
$X$ is a separated scheme. Thus we may assume $X$ is quasi-compact
and we aim to show that $X$ is affine.
Cases (2) and (3) follow immediately from case (1) but we will give a separate
proofs of (2) and (3) as these proofs use significantly less theory.

\medskip\noindent
Proof of (3). Let $U$ be an affine scheme and let $U \to X$ be an
\'etale morphism. Set $R = U \times_X U$. The two projection
morphisms $s, t : R \to U$ are \'etale morphisms of schemes. By
Properties of Spaces, Definition \ref{spaces-properties-definition-dimension}
we see that $\dim(U) = 0$ and $\dim(R) = 0$.
Since $R$ is a locally Noetherian scheme of dimension $0$,
we see that $R$ is a disjoint union of spectra of
Artinian local rings
(Properties, Lemma \ref{properties-lemma-locally-Noetherian-dimension-0}).
Since we assumed that $X$ is Noetherian (so quasi-separated) we
conclude that $R$ is quasi-compact. Hence $R$ is an affine scheme
(use Schemes, Lemma \ref{schemes-lemma-disjoint-union-affines}).
The \'etale morphisms $s, t : R \to U$ induce finite residue field
extensions. Hence $s$ and $t$ are finite by
Algebra, Lemma
\ref{algebra-lemma-essentially-of-finite-type-into-artinian-local}
(small detail omitted). 
Thus
Groupoids, Proposition \ref{groupoids-proposition-finite-flat-equivalence}
shows that $X = U/R$ is an affine scheme.

\medskip\noindent
Proof of (2) -- almost identical to the proof of (3).
Let $U$ be an affine scheme and let $U \to X$ be a surjective \'etale morphism.
Set $R = U \times_X U$. The two projection morphisms
$s, t : R \to U$ are \'etale morphisms of schemes. By
Properties of Spaces, Definition \ref{spaces-properties-definition-dimension}
we see that $\dim(U) = 0$ and similarly $\dim(R) = 0$.
On the other hand, the morphism $U \to \Spec(k)$ is locally of finite
type as the composition of the \'etale morphism $U \to X$ and
$X \to \Spec(k)$, see
Morphisms of Spaces,
Lemmas \ref{spaces-morphisms-lemma-composition-finite-type} and
\ref{spaces-morphisms-lemma-etale-locally-finite-type}.
Similarly, $R \to \Spec(k)$ is locally of finite type.
Hence by
Varieties, Lemma \ref{varieties-lemma-algebraic-scheme-dim-0}
we see that $U$ and $R$ are disjoint unions of spectra of
local Artinian $k$-algebras finite over $k$. The same thing
is therefore true of $U \times_{\Spec(k)} U$. As
$$
R = U \times_X U \longrightarrow U \times_{\Spec(k)} U
$$
is a monomorphism, we see that $R$ is a finite(!) union of spectra of
finite $k$-algebras. It follows that $R$ is affine, see
Schemes, Lemma \ref{schemes-lemma-disjoint-union-affines}.
Applying
Varieties, Lemma \ref{varieties-lemma-algebraic-scheme-dim-0}
once more we see that $R$ is finite over $k$. Hence $s, t$
are finite, see
Morphisms, Lemma \ref{morphisms-lemma-finite-permanence}.
Thus
Groupoids, Proposition \ref{groupoids-proposition-finite-flat-equivalence}
shows that $X = U/R$ is an affine scheme.

\medskip\noindent
Cohomological proof of (1). By Cohomology of Spaces, Lemma
\ref{spaces-cohomology-lemma-vanishing-above-dimension}
we have vanishing of higher cohomology groups for all
quasi-coherent sheaves $\mathcal{F}$ on $X$. Hence $X$
is affine (in particular a scheme) by
Cohomology of Spaces, Proposition
\ref{spaces-cohomology-proposition-vanishing-affine}.

\medskip\noindent
Geometric proof of (1). Choose a stratification
$$
\emptyset = U_{n + 1} \subset
U_n \subset U_{n - 1} \subset \ldots \subset U_1 = X
$$
and \'etale morphisms $f_p : V_p \to U_p$ as in
Decent Spaces, Lemma
\ref{decent-spaces-lemma-filter-quasi-compact-quasi-separated}
(we will use all their properties below).
Then $\dim(V_p) = 0$ by our definition of dimension of algebraic
spaces. Thus Properties, Lemma \ref{properties-lemma-dimension-zero}
applies to each $V_p$. Then $f_p^{-1}(U_{p + 1}) \subset V_p$
is quasi-compact open and hence is affine as well as closed.
It follows that $|T_p| \subset |U_p|$ (see locus citatus)
is open as well as closed. Hence $X$ is a disjoint union
of open and closed subspaces whose reduced structures are schemes.
It follows that $X$ is a scheme
(Limits of Spaces, Lemma \ref{spaces-limits-lemma-reduction-scheme}).
Then the proof is finished by the case of schemes that we
already referenced above.
\end{proof}

\noindent
The following lemma tells us that a quasi-separated algebraic space
is a scheme away from codimension $1$.

\begin{lemma}
\label{lemma-generic-point-in-schematic-locus}
Let $S$ be a scheme. Let $X$ be a quasi-separated algebraic space over $S$.
Let $x \in |X|$. The following are equivalent
\begin{enumerate}
\item $x$ is a point of codimension $0$ on $X$,
\item the local ring of $X$ at $x$ has dimension $0$, and
\item $x$ is a generic point of an irreducible component of $|X|$.
\end{enumerate}
If true, then there exists an open subspace of $X$
containing $x$ which is a scheme.
\end{lemma}

\begin{proof}
The equivalence of (1), (2), and (3) follows from
Decent Spaces, Lemma \ref{decent-spaces-lemma-decent-generic-points}
and the fact that a quasi-separated algebraic space is decent
(Decent Spaces, Section \ref{decent-spaces-section-reasonable-decent}).
However in the next paragraph we will give a more elementary proof of the
equivalence.

\medskip\noindent
Note that (1) and (2) are equivalent by definition
(Properties of Spaces, Definition
\ref{spaces-properties-definition-dimension-local-ring}).
To prove the equivalence of (1) and (3) we may assume $X$ is quasi-compact.
Choose
$$
\emptyset = U_{n + 1} \subset
U_n \subset U_{n - 1} \subset \ldots \subset U_1 = X
$$
and $f_i : V_i \to U_i$ as in Decent Spaces, Lemma
\ref{decent-spaces-lemma-filter-quasi-compact-quasi-separated}.
Say $x \in U_i$, $x \not \in U_{i + 1}$. Then $x = f_i(y)$ for
a unique $y \in V_i$. If (1) holds, then $y$ is a generic point of
an irreducible component of $V_i$ (Properties of Spaces, Lemma
\ref{spaces-properties-lemma-codimension-0-points}).
Since $f_i^{-1}(U_{i + 1})$ is a quasi-compact open of $V_i$
not containing $y$, there is an open neighbourhood $W \subset V_i$
of $y$ disjoint from $f_i^{-1}(V_i)$
(see
Properties, Lemma \ref{properties-lemma-generic-point-in-constructible}
or more simply Algebra, Lemma
\ref{algebra-lemma-standard-open-containing-maximal-point}).
Then $f_i|_W : W \to X$ is an isomorphism onto its image and hence
$x = f_i(y)$ is a generic point of $|X|$. Conversely, assume (3) holds.
Then $f_i$ maps $\overline{\{y\}}$ onto the irreducible component
$\overline{\{x\}}$ of $|U_i|$. Since $|f_i|$ is bijective over
$\overline{\{x\}}$, it follows that $\overline{\{y\}}$
is an irreducible component of $U_i$. Thus $x$ is a point of
codimension $0$.

\medskip\noindent
The final statement of the lemma is
Properties of Spaces, Proposition
\ref{spaces-properties-proposition-locally-quasi-separated-open-dense-scheme}.
\end{proof}

\noindent
The following lemma says that a separated locally Noetherian algebraic
space is a scheme in codimension $1$, i.e., away from codimension $2$.

\begin{lemma}
\label{lemma-codim-1-point-in-schematic-locus}
\begin{slogan}
Separated algebraic spaces are schemes in codimension 1.
\end{slogan}
Let $S$ be a scheme. Let $X$ be an algebraic space over $S$.
Let $x \in |X|$. If $X$ is separated, locally Noetherian, and
the dimension of the local ring of $X$ at $x$ is $\leq 1$
(Properties of Spaces, Definition
\ref{spaces-properties-definition-dimension-local-ring}),
then there exists an open subspace of $X$ containing $x$ which is a scheme.
\end{lemma}

\begin{proof}
(Please see the remark below for a different approach avoiding the material on
finite groupoids.) We can replace $X$ by an quasi-compact neighbourhood of
$x$, hence we may assume $X$ is quasi-compact, separated, and Noetherian.
There exists a scheme $U$ and a finite surjective morphism $U \to X$,
see Limits of Spaces, Proposition
\ref{spaces-limits-proposition-there-is-a-scheme-finite-over}.
Let $R = U \times_X U$. Then $j : R \to U \times_S U$ is an equivalence
relation and we obtain a groupoid scheme $(U, R, s, t, c)$ over $S$
with $s, t$ finite and $U$ Noetherian and separated.
Let $\{u_1, \ldots, u_n\} \subset U$ be the set of points mapping to $x$. 
Then $\dim(\mathcal{O}_{U, u_i}) \leq 1$ by
Decent Spaces, Lemma
\ref{decent-spaces-lemma-dimension-local-ring-quasi-finite}.

\medskip\noindent
By More on Groupoids, Lemma
\ref{more-groupoids-lemma-find-affine-codimension-1}
there exists an $R$-invariant affine open $W \subset U$ containing
the orbit $\{u_1, \ldots, u_n\}$. Since $U \to X$ is finite surjective
the continuous map $|U| \to |X|$ is closed surjective, hence
submersive by Topology, Lemma
\ref{topology-lemma-closed-morphism-quotient-topology}.
Thus $f(W)$ is open and there is an open subspace $X' \subset X$
with $f : W \to X'$ a surjective finite morphism.
Then $X'$ is an affine scheme by
Cohomology of Spaces, Lemma
\ref{spaces-cohomology-lemma-image-affine-finite-morphism-affine-Noetherian}
and the proof is finished.
\end{proof}

\begin{remark}
\label{remark-alternate-proof-scheme-codim-1}
Here is a sketch of a proof of
Lemma \ref{lemma-codim-1-point-in-schematic-locus}
which avoids using
More on Groupoids, Lemma
\ref{more-groupoids-lemma-find-affine-codimension-1}.

\medskip\noindent
Step 1. We may assume $X$ is a reduced Noetherian separated algebraic space
(for example by Cohomology of Spaces, Lemma
\ref{spaces-cohomology-lemma-image-affine-finite-morphism-affine-Noetherian}
or by
Limits of Spaces, Lemma \ref{spaces-limits-lemma-reduction-scheme})
and we may choose a finite surjective morphism
$Y \to X$ where $Y$ is a Noetherian scheme (by
Limits of Spaces, Proposition
\ref{spaces-limits-proposition-there-is-a-scheme-finite-over}).

\medskip\noindent
Step 2. After replacing $X$ by an open neighbourhood of $x$, there
exists a birational finite morphism $X' \to X$ and a closed subscheme
$Y' \subset X' \times_X Y$ such that $Y' \to X'$ is surjective
finite locally free. Namely, because $X$ is reduced there is a dense
open subspace $U \subset X$ over which $Y$ is flat (Morphisms of Spaces,
Proposition \ref{spaces-morphisms-proposition-generic-flatness-reduced}).
Then we can choose a $U$-admissible blowup $b : \tilde X \to X$ such
that the strict transform $\tilde Y$ of $Y$ is flat over $\tilde X$, see
More on Morphisms of Spaces, Lemma
\ref{spaces-more-morphisms-lemma-flat-after-blowing-up}.
(An alternative is to use Hilbert schemes if one wants to avoid using
the result on blowups).
Then we let $X' \subset \tilde X$ be the scheme theoretic
closure of $b^{-1}(U)$ and $Y' = X' \times_{\tilde X} \tilde Y$.
Since $x$ is a codimension $1$ point, we see that $X' \to X$ is finite over a
neighbourhood of $x$ (Lemma \ref{lemma-finite-in-codim-1}).

\medskip\noindent
Step 3. After shrinking $X$ to a smaller neighbourhood of $x$ we get that
$X'$ is a scheme. This holds because $Y'$ is a scheme and $Y' \to X'$
being finite locally free and because every finite set of codimension $1$
points of $Y'$ is contained in an affine open. Use
Properties of Spaces, Proposition
\ref{spaces-properties-proposition-finite-flat-equivalence-global}
and
Varieties, Proposition
\ref{varieties-proposition-finite-set-of-points-of-codim-1-in-affine}.

\medskip\noindent
Step 4. There exists an affine open $W' \subset X'$ containing all points
lying over $x$ which is the inverse image of an open subspace of $X$.
To prove this let $Z \subset X$ be the closure of the set of points
where $X' \to X$ is not an isomorphism. We may assume $x \in Z$ otherwise
we are already done. Then $x$ is a generic point of an irreducible
component of $Z$ and after shrinking $X$ we may assume $Z$ is an affine scheme
(Lemma \ref{lemma-generic-point-in-schematic-locus}).
Then the inverse image $Z' \subset X'$ is an affine scheme as well.
Say $x_1, \ldots, x_n \in Z'$ are the points mapping to $x$.
Then we can find an affine open $W'$ in $X'$ whose intersection with
$Z'$ is the inverse image of a principal open of $Z$ containing $x$.
Namely, we first pick an affine open $W' \subset X'$ containing
$x_1, \ldots, x_n$ using Varieties, Proposition
\ref{varieties-proposition-finite-set-of-points-of-codim-1-in-affine}.
Then we pick a principal open $D(f) \subset Z$ containing $x$
whose inverse image $D(f|_{Z'})$ is contained in $W' \cap Z'$.
Then we pick $f' \in \Gamma(W', \mathcal{O}_{W'})$ restricting
to $f|_{Z'}$ and we replace $W'$ by $D(f') \subset W'$.
Since $X' \to X$ is an isomorphism away from $Z' \to Z$ the choice
of $W'$ guarantees that the image $W \subset X$ of $W'$ is open
with inverse image $W'$ in $X'$.

\medskip\noindent
Step 5. Then $W' \to W$ is a finite surjective morphism and $W$ is a scheme by
Cohomology of Spaces, Lemma
\ref{spaces-cohomology-lemma-image-affine-finite-morphism-affine-Noetherian}
and the proof is complete.
\end{remark}





\section{Schematic locus and field extension}
\label{section-schematic-and-field-extension}

\noindent
It can happen that a nonrepresentable algebraic space over a field $k$
becomes representable (i.e., a scheme) after base change to an extension
of $k$. See Spaces, Example \ref{spaces-example-non-representable-descent}.
In this section we address this issue.

\begin{lemma}
\label{lemma-scheme-after-purely-inseparable-base-change}
Let $k$ be a field. Let $X$ be an algebraic space over $k$.
If there exists a purely inseparable field extension $k'/k$
such that $X_{k'}$ is a scheme, then $X$ is a scheme.
\end{lemma}

\begin{proof}
The morphism $X_{k'} \to X$ is integral, surjective, and
universally injective. Hence this lemma follows from
Limits of Spaces, Lemma
\ref{spaces-limits-lemma-integral-universally-bijective-scheme}.
\end{proof}

\begin{lemma}
\label{lemma-when-scheme-after-base-change}
Let $k$ be a field with algebraic closure $\overline{k}$.
Let $X$ be a quasi-separated algebraic space over $k$.
\begin{enumerate}
\item If there exists a field extension $K/k$ such that
$X_K$ is a scheme, then $X_{\overline{k}}$ is a scheme.
\item If $X$ is quasi-compact and there exists a field extension
$K/k$ such that $X_K$ is a scheme, then $X_{k'}$
is a scheme for some finite separable extension $k'$ of $k$.
\end{enumerate}
\end{lemma}

\begin{proof}
Since every algebraic space is the union of its quasi-compact open
subspaces, we see that the first part of the lemma follows from
the second part (some details omitted). Thus we assume $X$ is quasi-compact
and we assume given an extension $K/k$ with $X_K$ representable.
Write $K = \bigcup A$ as the colimit of finitely generated $k$-subalgebras
$A$. By Limits of Spaces, Lemma \ref{spaces-limits-lemma-limit-is-scheme}
we see that $X_A$ is a scheme for some $A$. Choose a maximal ideal
$\mathfrak m \subset A$. By the Hilbert Nullstellensatz
(Algebra, Theorem \ref{algebra-theorem-nullstellensatz})
the residue field $k' = A/\mathfrak m$ is a finite extension of $k$.
Thus we see that $X_{k'}$ is a scheme. If $k' \supset k$ is not
separable, let $k'/k''/k$ be the subextension
found in Fields, Lemma \ref{fields-lemma-separable-first}.
Since $k'/k''$ is purely inseparable, by
Lemma \ref{lemma-scheme-after-purely-inseparable-base-change}
the algebraic space $X_{k''}$ is a scheme. Since $k''|k$ is separable
the proof is complete.
\end{proof}

\begin{lemma}
\label{lemma-base-change-by-Galois}
Let $k'/k$ be a finite Galois extension with Galois group $G$.
Let $X$ be an algebraic space over $k$. Then $G$ acts freely on the
algebraic space $X_{k'}$ and $X = X_{k'}/G$ in the sense of
Properties of Spaces, Lemma \ref{spaces-properties-lemma-quotient}.
\end{lemma}

\begin{proof}
Omitted. Hints: First show that $\Spec(k) = \Spec(k')/G$.
Then use compatibility of taking quotients with base change.
\end{proof}

\begin{lemma}
\label{lemma-when-quotient-scheme-at-point}
Let $S$ be a scheme. Let $X$ be an algebraic space over $S$ and
let $G$ be a finite group acting freely on $X$. Set $Y = X/G$ as
in Properties of Spaces, Lemma \ref{spaces-properties-lemma-quotient}.
For $y \in |Y|$ the following are equivalent
\begin{enumerate}
\item $y$ is in the schematic locus of $Y$, and
\item there exists an affine open $U \subset X$
containing the preimage of $y$.
\end{enumerate}
\end{lemma}

\begin{proof}
It follows from the construction of $Y = X/G$ in
Properties of Spaces, Lemma \ref{spaces-properties-lemma-quotient}
that the morphism $X \to Y$ is surjective and \'etale.
Of course we have $X \times_Y X = X \times G$ hence the morphism
$X \to Y$ is even finite \'etale. It is also surjective.
Thus the lemma follows from
Decent Spaces, Lemma \ref{decent-spaces-lemma-when-quotient-scheme-at-point}.
\end{proof}

\begin{lemma}
\label{lemma-scheme-after-purely-transcendental-base-change}
Let $k$ be a field. Let $X$ be a quasi-separated
algebraic space over $k$. If there exists a purely transcendental
field extension $K/k$ such that $X_K$ is a scheme, then
$X$ is a scheme.
\end{lemma}

\begin{proof}
Since every algebraic space is the union of its quasi-compact open
subspaces, we may assume $X$ is quasi-compact (some details omitted).
Recall (Fields, Definition \ref{fields-definition-transcendence})
that the assumption on the extension $K/k$ signifies that
$K$ is the fraction field of a polynomial ring (in possibly infinitely
many variables) over $k$. Thus $K = \bigcup A$ is the union of subalgebras
each of which is a localization of a finite polynomial algebra over $k$.
By Limits of Spaces, Lemma \ref{spaces-limits-lemma-limit-is-scheme}
we see that $X_A$ is a scheme for some $A$. Write
$$
A = k[x_1, \ldots, x_n][1/f]
$$
for some nonzero $f \in k[x_1, \ldots, x_n]$.

\medskip\noindent
If $k$ is infinite then we can finish the proof as follows: choose
$a_1, \ldots, a_n \in k$ with $f(a_1, \ldots, a_n) \not = 0$.
Then $(a_1, \ldots, a_n)$ define an $k$-algebra map $A \to k$
mapping $x_i$ to $a_i$ and $1/f$ to $1/f(a_1, \ldots, a_n)$.
Thus the base change $X_A \times_{\Spec(A)} \Spec(k) \cong X$ is a
scheme as desired.

\medskip\noindent
In this paragraph we finish the proof in case $k$ is finite. In this
case we write $X = \lim X_i$ with $X_i$ of finite presentation over $k$
and with affine transition morphisms
(Limits of Spaces, Lemma \ref{spaces-limits-lemma-relative-approximation}).
Using Limits of Spaces, Lemma \ref{spaces-limits-lemma-limit-is-scheme}
we see that $X_{i, A}$ is a scheme for some $i$. Thus we may assume
$X \to \Spec(k)$ is of finite presentation. Let $x \in |X|$ be a closed
point. We may represent $x$ by a closed immersion
$\Spec(\kappa) \to X$
(Decent Spaces, Lemma \ref{decent-spaces-lemma-decent-space-closed-point}).
Then $\Spec(\kappa) \to \Spec(k)$ is of finite type, hence $\kappa$
is a finite extension of $k$ (by the Hilbert Nullstellensatz, see
Algebra, Theorem \ref{algebra-theorem-nullstellensatz};
some details omitted). Say $[\kappa : k] = d$. Choose an integer
$n \gg 0$ prime to $d$ and let $k'/k$ be the extension
of degree $n$. Then $k'/k$ is Galois with $G = \text{Aut}(k'/k)$
cyclic of order $n$. If $n$ is large enough there will be $k$-algebra
homomorphism $A \to k'$ by the same reason as above.
Then $X_{k'}$ is a scheme and $X = X_{k'}/G$
(Lemma \ref{lemma-base-change-by-Galois}).
On the other hand, since $n$ and $d$ are relatively prime we see that
$$
\Spec(\kappa) \times_{X} X_{k'} =
\Spec(\kappa) \times_{\Spec(k)} \Spec(k') =
\Spec(\kappa \otimes_k k')
$$
is the spectrum of a field. In other words, the fibre of $X_{k'} \to X$
over $x$ consists of a single point. Thus by
Lemma \ref{lemma-when-quotient-scheme-at-point}
we see that $x$ is in the schematic locus of $X$ as desired.
\end{proof}

\begin{remark}
\label{remark-when-does-the-argument-work}
Let $k$ be a finite field. Let $K/k$ be a geometrically
irreducible field extension. Then $K$ is the limit of geometrically
irreducible finite type $k$-algebras $A$. Given $A$ the estimates
of Lang and Weil \cite{LW}, show that for $n \gg 0$ there exists
an $k$-algebra homomorphism $A \to k'$ with $k'/k$ of degree $n$.
Analyzing the argument given in the proof of
Lemma \ref{lemma-scheme-after-purely-transcendental-base-change}
we see that if $X$ is a quasi-separated algebraic space over $k$
and $X_K$ is a scheme, then $X$ is a scheme. If we ever need this
result we will precisely formulate it and prove it here.
\end{remark}

\begin{lemma}
\label{lemma-scheme-over-algebraic-closure-enough-affines}
Let $k$ be a field with algebraic closure $\overline{k}$. Let $X$
be an algebraic space over $k$ such that
\begin{enumerate}
\item $X$ is decent and locally of finite type over $k$,
\item $X_{\overline{k}}$ is a scheme, and
\item any finite set of $\overline{k}$-rational points of $X_{\overline{k}}$
is contained in an affine.
\end{enumerate}
Then $X$ is a scheme.
\end{lemma}

\begin{proof}
If $K/k$ is an extension, then the base change $X_K$ is
decent (Decent Spaces, Lemma
\ref{decent-spaces-lemma-representable-named-properties})
and locally of finite type
over $K$ (Morphisms of Spaces, Lemma
\ref{spaces-morphisms-lemma-base-change-finite-type}).
By Lemma \ref{lemma-scheme-after-purely-inseparable-base-change}
it suffices to prove that $X$ becomes a scheme after base change to
the perfection of $k$, hence we may assume $k$ is a perfect field
(this step isn't strictly necessary, but makes the other arguments
easier to think about).
By covering $X$ by quasi-compact opens we see that it suffices to prove
the lemma in case $X$ is quasi-compact (small detail omitted).
In this case $|X|$ is a sober topological space
(Decent Spaces, Proposition
\ref{decent-spaces-proposition-reasonable-sober}).
Hence it suffices to show that every closed point in $|X|$
is contained in the schematic locus of $X$
(use Properties of Spaces, Lemma \ref{spaces-properties-lemma-subscheme} and
Topology, Lemma \ref{topology-lemma-quasi-compact-closed-point}).

\medskip\noindent
Let $x \in |X|$ be a closed point. By Decent Spaces, Lemma
\ref{decent-spaces-lemma-decent-space-closed-point}
we can find a closed immersion $\Spec(l) \to X$ representing $x$.
Then $\Spec(l) \to \Spec(k)$ is of finite type (Morphisms of Spaces,
Lemma \ref{spaces-morphisms-lemma-composition-finite-type}) and we
conclude that $l$ is a finite extension of $k$
by the Hilbert Nullstellensatz (Algebra, Theorem
\ref{algebra-theorem-nullstellensatz}). It is separable because
$k$ is perfect. Thus the scheme
$$
\Spec(l) \times_X X_{\overline{k}} =
\Spec(l) \times_{\Spec(k)} \Spec(\overline{k}) =
\Spec(l \otimes_k \overline{k})
$$
is the disjoint union of a finite number of $\overline{k}$-rational points.
By assumption (3) we can find an affine open $W \subset X_{\overline{k}}$
containing these points.

\medskip\noindent
By Lemma \ref{lemma-when-scheme-after-base-change} we see that $X_{k'}$
is a scheme for some finite extension $k'/k$. After enlarging
$k'$ we may assume that there exists an affine open $U' \subset X_{k'}$
whose base change to $\overline{k}$ recovers $W$
(use that $X_{\overline{k}}$ is the limit of the schemes $X_{k''}$
for $k' \subset k'' \subset \overline{k}$ finite and use
Limits, Lemmas \ref{limits-lemma-descend-opens} and
\ref{limits-lemma-limit-affine}). We may assume
that $k'/k$ is a Galois extension (take the normal closure
Fields, Lemma \ref{fields-lemma-normal-closure} and use
that $k$ is perfect). Set $G = \text{Gal}(k'/k)$.
By construction the $G$-invariant closed subscheme
$\Spec(l) \times_X X_{k'}$ is contained in $U'$.
Thus $x$ is in the schematic locus by
Lemmas \ref{lemma-base-change-by-Galois} and
\ref{lemma-when-quotient-scheme-at-point}.
\end{proof}

\noindent
The following two lemmas should go somewhere else.
Please compare the next lemma to
Decent Spaces, Lemma \ref{decent-spaces-lemma-conditions-on-space-over-field}.

\begin{lemma}
\label{lemma-locally-quasi-finite-over-field}
Let $k$ be a field. Let $X$ be an algebraic space over $k$.
The following are equivalent
\begin{enumerate}
\item $X$ is locally quasi-finite over $k$,
\item $X$ is locally of finite type over $k$ and has dimension $0$,
\item $X$ is a scheme and is locally quasi-finite over $k$,
\item $X$ is a scheme and is locally of finite type over $k$ and has
dimension $0$, and
\item $X$ is a disjoint union of spectra of Artinian local $k$-algebras
$A$ over $k$ with $\dim_k(A) < \infty$.
\end{enumerate}
\end{lemma}

\begin{proof}
Because we are over a field relative dimension of $X/k$ is the same as
the dimension of $X$. Hence by
Morphisms of Spaces,
Lemma \ref{spaces-morphisms-lemma-locally-quasi-finite-rel-dimension-0}
we see that (1) and (2) are equivalent. Hence it follows from
Lemma \ref{lemma-locally-finite-type-dim-zero}
(and trivial implications) that (1) -- (4) are equivalent.
Finally,
Varieties, Lemma \ref{varieties-lemma-algebraic-scheme-dim-0}
shows that (1) -- (4) are equivalent with (5).
\end{proof}

\begin{lemma}
\label{lemma-mono-towards-locally-quasi-finite-over-field}
Let $k$ be a field. Let $f : X \to Y$ be a monomorphism of algebraic spaces
over $k$. If $Y$ is locally quasi-finite over $k$ so is $X$.
\end{lemma}

\begin{proof}
Assume $Y$ is locally quasi-finite over $k$. By
Lemma \ref{lemma-locally-quasi-finite-over-field}
we see that $Y = \coprod \Spec(A_i)$ where each $A_i$ is an
Artinian local ring finite over $k$. By
Decent Spaces, Lemma
\ref{decent-spaces-lemma-monomorphism-toward-disjoint-union-dim-0-rings}
we see that $X$ is a scheme. Consider $X_i = f^{-1}(\Spec(A_i))$.
Then $X_i$ has either one or zero points. If $X_i$ has zero points there
is nothing to prove. If $X_i$ has one point, then
$X_i = \Spec(B_i)$ with $B_i$ a zero dimensional local ring
and $A_i \to B_i$ is an epimorphism of rings. In particular
$A_i/\mathfrak m_{A_i} = B_i/\mathfrak m_{A_i}B_i$ and we see that
$A_i \to B_i$ is surjective by Nakayama's lemma,
Algebra, Lemma \ref{algebra-lemma-NAK}
(because $\mathfrak m_{A_i}$ is a nilpotent ideal!).
Thus $B_i$ is a finite local $k$-algebra, and we conclude by
Lemma \ref{lemma-locally-quasi-finite-over-field}
that $X \to \Spec(k)$ is locally quasi-finite.
\end{proof}











\section{Geometrically reduced algebraic spaces}
\label{section-geometrically-reduced}

\noindent
If $X$ is a reduced algebraic space over a field, then it can happen that $X$
becomes nonreduced after extending the ground field. This does not happen
for geometrically reduced algebraic spaces.

\begin{definition}
\label{definition-geometrically-reduced}
Let $k$ be a field.
Let $X$ be an algebraic space over $k$.
\begin{enumerate}
\item Let $x \in |X|$ be a point. We say $X$ is
{\it geometrically reduced at $x$} if $\mathcal{O}_{X, \overline{x}}$
is geometrically reduced over $k$.
\item We say $X$ is {\it geometrically reduced} over $k$
if $X$ is geometrically reduced at every point of $X$.
\end{enumerate}
\end{definition}

\noindent
Observe that if $X$ is geometrically reduced at $x$, then the local
ring of $X$ at $x$ is reduced (Properties of Spaces, Lemma
\ref{spaces-properties-lemma-reduced-local-ring}).
Similarly, if $X$ is geometrically reduced over $k$, then $X$ is reduced
(by Properties of Spaces, Lemma \ref{spaces-properties-lemma-reduced-space}).
The following lemma in particular implies this definition does not clash
with the corresponding property for schemes over a field.

\begin{lemma}
\label{lemma-geometrically-reduced-at-point}
Let $k$ be a field. Let $X$ be an algebraic space over $k$.
Let $x \in |X|$. The following are equivalent
\begin{enumerate}
\item $X$ is geometrically reduced at $x$,
\item for some \'etale neighbourhood $(U, u) \to (X, x)$
where $U$ is a scheme, $U$ is geometrically reduced at $u$,
\item for any \'etale neighbourhood $(U, u) \to (X, x)$
where $U$ is a scheme, $U$ is geometrically reduced at $u$.
\end{enumerate}
\end{lemma}

\begin{proof}
Recall that the local ring $\mathcal{O}_{X, \overline{x}}$
is the strict henselization of $\mathcal{O}_{U, u}$, see
Properties of Spaces, Lemma
\ref{spaces-properties-lemma-describe-etale-local-ring}.
By Varieties, Lemma \ref{varieties-lemma-geometrically-reduced-at-point}
we find that $U$ is geometrically reduced at $u$ if and only
if $\mathcal{O}_{U, u}$ is geometrically reduced over $k$.
Thus we have to show: if $A$ is a local $k$-algebra, then
$A$ is geometrically reduced over $k$ if and only if
$A^{sh}$ is geometrically reduced over $k$.
We check this using the definition of geometrically reduced
algebras (Algebra, Definition \ref{algebra-definition-geometrically-reduced}).
Let $K/k$ be a field extension.
Since $A \to A^{sh}$ is faithfully flat
(More on Algebra, Lemma \ref{more-algebra-lemma-dumb-properties-henselization})
we see that
$A \otimes_k K \to A^{sh} \otimes_k K$ is faithfully flat
(Algebra, Lemma \ref{algebra-lemma-flat-base-change}).
Hence if $A^{sh} \otimes_k K$ is reduced, so is
$A \otimes_k K$ by Algebra, Lemma \ref{algebra-lemma-descent-reduced}.
Conversely, recall that $A^{sh}$ is a colimit of
\'etale $A$-algebra, see
Algebra, Lemma \ref{algebra-lemma-strict-henselization}.
Thus $A^{sh} \otimes_k K$ is a filtered
colimit of \'etale $A \otimes_k K$-algebras.
We conclude by Algebra, Lemma \ref{algebra-lemma-reduced-goes-up}.
\end{proof}

\begin{lemma}
\label{lemma-geometrically-reduced}
Let $k$ be a field. Let $X$ be an algebraic space over $k$.
The following are equivalent
\begin{enumerate}
\item $X$ is geometrically reduced,
\item for some surjective \'etale morphism $U \to X$ where $U$
is a scheme, $U$ is geometrically reduced,
\item for any \'etale morphism $U \to X$
where $U$ is a scheme, $U$ is geometrically reduced.
\end{enumerate}
\end{lemma}

\begin{proof}
Immediate from the definitions and
Lemma \ref{lemma-geometrically-reduced-at-point}.
\end{proof}

\noindent
The notion isn't interesting in characteristic zero.

\begin{lemma}
\label{lemma-perfect-reduced}
Let $X$ be an algebraic space over a perfect field $k$ (for example
$k$ has characteristic zero).
\begin{enumerate}
\item For $x \in |X|$, if $\mathcal{O}_{X, \overline{x}}$ is
reduced, then $X$ is geometrically reduced at $x$.
\item If $X$ is reduced, then $X$ is geometrically reduced over $k$.
\end{enumerate}
\end{lemma}

\begin{proof}
The first statement follows from
Algebra, Lemma \ref{algebra-lemma-separable-extension-preserves-reducedness}
and the definition of a perfect field
(Algebra, Definition \ref{algebra-definition-perfect}).
The second statement follows from the first.
\end{proof}

\begin{lemma}
\label{lemma-geometrically-reduced-positive-characteristic}
Let $k$ be a field of characteristic $p > 0$. Let $X$ be an algebraic space
over $k$. The following are equivalent
\begin{enumerate}
\item $X$ is geometrically reduced over $k$,
\item $X_{k'}$ is reduced for every field extension $k'/k$,
\item $X_{k'}$ is reduced for every
finite purely inseparable field extension $k'/k$,
\item $X_{k^{1/p}}$ is reduced,
\item $X_{k^{perf}}$ is reduced, and
\item $X_{\bar k}$ is reduced.
\end{enumerate}
\end{lemma}

\begin{proof}
Choose a surjective \'etale morphism $U \to X$ where $U$ is a scheme.
Via Lemma \ref{lemma-geometrically-reduced} the lemma follows from
the result for $U$ over $k$.
See Varieties, Lemma \ref{varieties-lemma-geometrically-reduced}.
\end{proof}

\begin{lemma}
\label{lemma-geometrically-reduced-upstairs}
Let $k$ be a field. Let $X$ be an algebraic space over $k$.
Let $k'/k$ be a field extension. Let $x \in |X|$ be a point and let
$x' \in |X_{k'}|$ be a point lying over $x$.
The following are equivalent
\begin{enumerate}
\item $X$ is geometrically reduced at $x$,
\item $X_{k'}$ is geometrically reduced at $x'$.
\end{enumerate}
In particular, $X$ is geometrically reduced over $k$ if and only if
$X_{k'}$ is geometrically reduced over $k'$.
\end{lemma}

\begin{proof}
Choose an \'etale morphism $U \to X$ where $U$ is a scheme
and a point $u \in U$ mapping to $x \in |X|$.
By Properties of Spaces, Lemma \ref{spaces-properties-lemma-points-cartesian}
we may choose a point $u' \in U_{k'} = U \times_X X_{k'}$ mapping
to both $u$ and $x'$. By Lemma \ref{lemma-geometrically-reduced-at-point}
the lemma follows from the lemma for $U, u, u'$ which is
Varieties, Lemma \ref{varieties-lemma-geometrically-reduced-upstairs}.
\end{proof}

\begin{lemma}
\label{lemma-geometrically-reduced-etale-local}
Let $k$ be a field. Let $f : X \to Y$ be a morphism of
algebraic spaces over $k$. Let $x \in |X|$ be a point
with image $y \in |Y|$.
\begin{enumerate}
\item if $f$ is \'etale at $x$, then
$X$ is geometrically reduced at $x$ $\Leftrightarrow$
$Y$ is geometrically reduced at $y$,
\item if $f$ is surjective \'etale, then
$X$ is geometrically reduced $\Leftrightarrow$
$Y$ is geometrically reduced.
\end{enumerate}
\end{lemma}

\begin{proof}
Part (1) is clear because
$\mathcal{O}_{X, \overline{x}} = \mathcal{O}_{Y, \overline{y}}$
if $f$ is \'etale at $x$.
Part (2) follows immediately from part (1).
\end{proof}






\section{Geometrically connected algebraic spaces}
\label{section-geometrically-connected}

\noindent
If $X$ is a connected algebraic space over a field, then it can happen that
$X$ becomes disconnected after extending the ground field. This does not
happen for geometrically connected algebraic spaces.

\begin{definition}
\label{definition-geometrically-connected}
Let $X$ be an algebraic space over the field $k$. We say $X$ is
{\it geometrically connected} over $k$ if the base change $X_{k'}$
is connected for every field extension $k'$ of $k$.
\end{definition}

\noindent
By convention a connected topological space is nonempty; hence a fortiori
geometrically connected algebraic spaces are nonempty.

\begin{lemma}
\label{lemma-geometrically-connected-check-after-extension}
Let $X$ be an algebraic space over the field $k$.
Let $k'/k$ be a field extension.
Then $X$ is geometrically connected over $k$ if and only if
$X_{k'}$ is geometrically connected over $k'$.
\end{lemma}

\begin{proof}
If $X$ is geometrically connected over $k$, then it is clear that
$X_{k'}$ is geometrically connected over $k'$. For the converse, note
that for any field extension $k''/k$ there exists a common
field extension $k'''/k'$ and $k'''/k'$. As the
morphism $X_{k'''} \to X_{k''}$ is surjective (as a base change of
a surjective morphism between spectra of fields) we see that the
connectedness of $X_{k'''}$ implies the connectedness of $X_{k''}$.
Thus if $X_{k'}$ is geometrically connected over $k'$ then
$X$ is geometrically connected over $k$.
\end{proof}

\begin{lemma}
\label{lemma-bijection-connected-components}
Let $k$ be a field. Let $X$, $Y$ be algebraic spaces over $k$.
Assume $X$ is geometrically connected over $k$.
Then the projection morphism
$$
p : X \times_k Y \longrightarrow Y
$$
induces a bijection between connected components.
\end{lemma}

\begin{proof}
Let $y \in |Y|$ be represented by a morphism $\Spec(K) \to Y$
where $K$ is a field. The fibre of $|X \times_k Y| \to |Y|$ over $y$
is the image of $|X_K| \to |X \times_k Y|$ by
Properties of Spaces, Lemma \ref{spaces-properties-lemma-points-cartesian}.
Thus these fibres are connected by our assumption that $X$ is
geometrically connected. By
Morphisms of Spaces, Lemma
\ref{spaces-morphisms-lemma-space-over-field-universally-open}
the map $|p|$ is open.
Thus we may apply Topology,
Lemma \ref{topology-lemma-connected-fibres-connected-components}
to conclude.
\end{proof}

\begin{lemma}
\label{lemma-separably-closed-field-connected-components}
Let $k'/k$ be an extension of fields. Let $X$ be an algebraic space
over $k$. Assume $k$ separably algebraically closed. Then the morphism
$X_{k'} \to X$ induces a bijection of connected components. In particular,
$X$ is geometrically connected over $k$ if and only if $X$ is connected.
\end{lemma}

\begin{proof}
Since $k$ is separably algebraically closed we see that
$k'$ is geometrically connected over $k$, see
Algebra,
Lemma \ref{algebra-lemma-separably-closed-connected-implies-geometric}.
Hence $Z = \Spec(k')$ is geometrically connected over $k$ by
Varieties, Lemma \ref{varieties-lemma-affine-geometrically-connected}.
Since $X_{k'} = Z \times_k X$ the result is a special case of
Lemma \ref{lemma-bijection-connected-components}.
\end{proof}

\begin{lemma}
\label{lemma-characterize-geometrically-connected}
Let $k$ be a field. Let $X$ be an algebraic space over $k$.
Let $\overline{k}$ be a separable algebraic closure of $k$.
Then $X$ is geometrically connected if and only if the base change
$X_{\overline{k}}$ is connected.
\end{lemma}

\begin{proof}
Assume $X_{\overline{k}}$ is connected. Let $k'/k$ be a field
extension. There exists a field extension $\overline{k}'/\overline{k}$
such that $k'$ embeds into $\overline{k}'$ as an extension of $k$.
By Lemma \ref{lemma-separably-closed-field-connected-components}
we see that $X_{\overline{k}'}$ is connected.
Since $X_{\overline{k}'} \to X_{k'}$ is surjective we conclude
that $X_{k'}$ is connected as desired.
\end{proof}

\noindent
Let $k$ be a field. Let $\overline{k}/k$ be a (possibly infinite)
Galois extension. For example $\overline{k}$ could be the
separable algebraic closure of $k$.
For any $\sigma \in \text{Gal}(\overline{k}/k)$ we get a corresponding
automorphism
$
\Spec(\sigma) :
\Spec(\overline{k})
\longrightarrow
\Spec(\overline{k})
$.
Note that
$\Spec(\sigma) \circ \Spec(\tau) = \Spec(\tau \circ \sigma)$.
Hence we get an action
$$
\text{Gal}(\overline{k}/k)^{opp} \times \Spec(\overline{k})
\longrightarrow
\Spec(\overline{k})
$$
of the opposite group on the scheme $\Spec(\overline{k})$.
Let $X$ be an algebraic space over $k$. Since
$X_{\overline{k}} =
\Spec(\overline{k}) \times_{\Spec(k)} X$
by definition we see that the action above induces a canonical action
\begin{equation}
\label{equation-galois-action-base-change-kbar}
\text{Gal}(\overline{k}/k)^{opp} \times X_{\overline{k}}
\longrightarrow
X_{\overline{k}}.
\end{equation}

\begin{lemma}
\label{lemma-Galois-action-quasi-compact-open}
Let $k$ be a field. Let $X$ be an algebraic space over $k$.
Let $\overline{k}$ be a (possibly infinite) Galois extension of $k$.
Let $V \subset X_{\overline{k}}$ be a quasi-compact open.
Then
\begin{enumerate}
\item there exists a finite subextension $\overline{k}/k'/k$
and a quasi-compact open $V' \subset X_{k'}$ such that
$V = (V')_{\overline{k}}$,
\item there exists an open subgroup $H \subset \text{Gal}(\overline{k}/k)$
such that $\sigma(V) = V$ for all $\sigma \in H$.
\end{enumerate}
\end{lemma}

\begin{proof}
Choose a scheme $U$ and a surjective \'etale morphism $U \to X$.
Choose a quasi-compact open $W \subset U_{\overline{k}}$ whose
image in $X_{\overline{k}}$ is $V$. This is possible because
$|U_{\overline{k}}| \to |X_{\overline{k}}|$ is continuous and because
$|U_{\overline{k}}|$ has a basis of quasi-compact opens. We can apply
Varieties, Lemma
\ref{varieties-lemma-Galois-action-quasi-compact-open}
to $W \subset U_{\overline{k}}$ to obtain the lemma.
\end{proof}

\begin{lemma}
\label{lemma-closed-fixed-by-Galois}
Let $k$ be a field. Let $\overline{k}/k$ be a (possibly infinite)
Galois extension. Let $X$ be an algebraic space over $k$. Let
$\overline{T} \subset |X_{\overline{k}}|$ have the following properties
\begin{enumerate}
\item $\overline{T}$ is a closed subset of $|X_{\overline{k}}|$,
\item for every $\sigma \in \text{Gal}(\overline{k}/k)$
we have $\sigma(\overline{T}) = \overline{T}$.
\end{enumerate}
Then there exists a closed subset $T \subset |X|$ whose inverse image
in $|X_{k'}|$ is $\overline{T}$.
\end{lemma}

\begin{proof}
Let $T \subset |X|$ be the image of $\overline{T}$.
Since $|X_{\overline{k}}| \to |X|$ is surjective, the statement means
that $T$ is closed and that its inverse image is $\overline{T}$.
Choose a scheme $U$ and a surjective \'etale morphism $U \to X$.
By the case of schemes
(see Varieties, Lemma \ref{varieties-lemma-closed-fixed-by-Galois})
there exists a closed subset $T' \subset |U|$ whose inverse image
in $|U_{\overline{k}}|$ is the inverse image of $\overline{T}$.
Since $|U_{\overline{k}}| \to |X_{\overline{k}}|$ is surjective,
we see that $T'$ is the inverse image of $T$ via $|U| \to |X|$.
By our construction of the topology on $|X|$ this means that $T$ is
closed. In the same manner one sees that $\overline{T}$ is the inverse
image of $T$.
\end{proof}

\begin{lemma}
\label{lemma-characterize-geometrically-disconnected}
Let $k$ be a field. Let $X$ be an algebraic space over $k$.
The following are equivalent
\begin{enumerate}
\item $X$ is geometrically connected,
\item for every finite separable field extension $k'/k$
the algebraic space $X_{k'}$ is connected.
\end{enumerate}
\end{lemma}

\begin{proof}
This proof is identical to the proof of
Varieties, Lemma \ref{varieties-lemma-characterize-geometrically-disconnected}
except that
we replace
Varieties, Lemma \ref{varieties-lemma-characterize-geometrically-connected}
by Lemma \ref{lemma-characterize-geometrically-connected},
we replace
Varieties, Lemma \ref{varieties-lemma-Galois-action-quasi-compact-open}
by Lemma \ref{lemma-Galois-action-quasi-compact-open}, and
we replace
Varieties, Lemma \ref{varieties-lemma-closed-fixed-by-Galois}
by Lemma \ref{lemma-closed-fixed-by-Galois}.
We urge the reader to read that proof in stead of this one.

\medskip\noindent
It follows immediately from the definition that (1) implies (2).
Assume that $X$ is not geometrically connected.
Let $k \subset \overline{k}$ be a separable algebraic
closure of $k$. By
Lemma \ref{lemma-characterize-geometrically-connected}
it follows that $X_{\overline{k}}$ is disconnected.
Say $X_{\overline{k}} = \overline{U} \amalg \overline{V}$
with $\overline{U}$ and $\overline{V}$ open, closed, and nonempty
algebraic subspaces of $X_{\overline{k}}$.

\medskip\noindent
Suppose that $W \subset X$ is any quasi-compact open subspace.
Then $W_{\overline{k}} \cap \overline{U}$ and
$W_{\overline{k}} \cap \overline{V}$ are open and closed subspaces of
$W_{\overline{k}}$. In particular $W_{\overline{k}} \cap \overline{U}$ and
$W_{\overline{k}} \cap \overline{V}$ are quasi-compact, and by
Lemma \ref{lemma-Galois-action-quasi-compact-open}
both $W_{\overline{k}} \cap \overline{U}$ and
$W_{\overline{k}} \cap \overline{V}$
are defined over a finite subextension and invariant under an
open subgroup of $\text{Gal}(\overline{k}/k)$.
We will use this without further mention in the following.

\medskip\noindent
Pick $W_0 \subset X$ quasi-compact open subspace such that both
$W_{0, \overline{k}} \cap \overline{U}$ and
$W_{0, \overline{k}} \cap \overline{V}$ are nonempty.
Choose a finite subextension $\overline{k}/k'/k$
and a decomposition $W_{0, k'} = U_0' \amalg V_0'$ into open and closed
subsets such that
$W_{0, \overline{k}} \cap \overline{U} = (U'_0)_{\overline{k}}$ and
$W_{0, \overline{k}} \cap \overline{V} = (V'_0)_{\overline{k}}$.
Let $H = \text{Gal}(\overline{k}/k') \subset \text{Gal}(\overline{k}/k)$.
In particular
$\sigma(W_{0, \overline{k}} \cap \overline{U}) =
W_{0, \overline{k}} \cap \overline{U}$ and similarly for
$\overline{V}$.

\medskip\noindent
Having chosen $W_0$, $k'$ as above, for every quasi-compact open subspace
$W \subset X$ we set
$$
U_W =
\bigcap\nolimits_{\sigma \in H} \sigma(W_{\overline{k}} \cap \overline{U}),
\quad
V_W =
\bigcup\nolimits_{\sigma \in H} \sigma(W_{\overline{k}} \cap \overline{V}).
$$
Now, since $W_{\overline{k}} \cap \overline{U}$ and
$W_{\overline{k}} \cap \overline{V}$ are fixed by an open subgroup of
$\text{Gal}(\overline{k}/k)$ we see that the union and intersection
above are finite. Hence $U_W$ and $V_W$ are both open and closed subspaces.
Also, by construction $W_{\bar k} = U_W \amalg V_W$.

\medskip\noindent
We claim that if $W \subset W' \subset X$ are quasi-compact
open subspaces, then $W_{\overline{k}} \cap U_{W'} = U_W$ and
$W_{\overline{k}} \cap V_{W'} = V_W$. Verification omitted.
Hence we see that upon defining $U = \bigcup_{W \subset X} U_W$
and $V = \bigcup_{W \subset X} V_W$ we obtain
$X_{\overline{k}} = U \amalg V$ is a disjoint union of open
and closed subsets.
It is clear that $V$ is nonempty as it is constructed by taking
unions (locally). On the other hand, $U$ is nonempty since it contains
$W_0 \cap \overline{U}$ by construction. Finally, $U, V \subset X_{\bar k}$
are closed and $H$-invariant by construction. Hence by
Lemma \ref{lemma-closed-fixed-by-Galois}
we have $U = (U')_{\bar k}$, and $V = (V')_{\bar k}$ for some
closed $U', V' \subset X_{k'}$. Clearly $X_{k'} = U' \amalg V'$
and we see that $X_{k'}$ is disconnected as desired.
\end{proof}






\section{Geometrically irreducible algebraic spaces}
\label{section-geometrically-irreducible}

\noindent
Spaces, Example \ref{spaces-example-infinite-product} shows that it
is best not to think about irreducible algebraic spaces in complete
generality\footnote{To be sure, if we say ``the algebraic space $X$
is irreducible'', we probably mean to say ``the topological space $|X|$
is irreducible''.}. For decent (for example quasi-separated) algebraic spaces
this kind of disaster doesn't happen. Thus we make the following
definition only under the assumption that our algebraic space is decent.

\begin{definition}
\label{definition-geometrically-irreducible}
Let $k$ be a field. 
Let $X$ be a decent algebraic space over $k$.
We say $X$ is {\it geometrically irreducible} if the
topological space $|X_{k'}|$ is
irreducible\footnote{An irreducible space is nonempty.}
for any field extension $k'$ of $k$.
\end{definition}

\noindent
Observe that $X_{k'}$ is a decent algebraic space
(Decent Spaces, Lemma
\ref{decent-spaces-lemma-representable-named-properties}).
Hence the topological space $|X_{k'}|$ is sober.
Decent Spaces, Proposition \ref{decent-spaces-proposition-reasonable-sober}.





\section{Geometrically integral algebraic spaces}
\label{section-geometrically-integral}

\noindent
Recall that integral algebraic spaces are by definition decent, see
Section \ref{section-integral-spaces}.

\begin{definition}
\label{definition-geometrically-integral}
Let $X$ be an algebraic space over the field $k$. We say $X$ is
{\it geometrically integral} over $k$ if the algebraic space
$X_{k'}$ is integral (Definition \ref{definition-integral-algebraic-space})
for every field extension $k'$ of $k$.
\end{definition}

\noindent
In particular $X$ is a decent algebraic space.
We can relate this to being geometrically reduced and
geometrically irreducible as follows.

\begin{lemma}
\label{lemma-geometrically-integral}
Let $k$ be a field. Let $X$ be a decent algebraic space over $k$.
Then $X$ is geometrically integral over $k$ if and only if
$X$ is both geometrically reduced and geometrically irreducible
over $k$.
\end{lemma}

\begin{proof}
This is an immediate consequence of the definitions because our
notion of integral (in the presence of decency) is equivalent to
reduced and irreducible.
\end{proof}

\begin{lemma}
\label{lemma-proper-geometrically-reduced-global-sections}
Let $k$ be a field. Let $X$ be a proper algebraic space over $k$.
\begin{enumerate}
\item $A = H^0(X, \mathcal{O}_X)$ is a finite dimensional $k$-algebra,
\item $A = \prod_{i = 1, \ldots, n} A_i$ is a product of Artinian
local $k$-algebras, one factor for each connected component of $|X|$,
\item if $X$ is reduced, then $A = \prod_{i = 1, \ldots, n} k_i$
is a product of fields, each a finite extension of $k$,
\item if $X$ is geometrically reduced, then $k_i$ is finite separable
over $k$,
\item if $X$ is geometrically connected, then $A$ is geometrically
irreducible over $k$,
\item if $X$ is geometrically irreducible, then $A$ is geometrically
irreducible over $k$,
\item if $X$ is geometrically reduced and connected, then $A = k$, and
\item if $X$ is geometrically integral, then $A = k$.
\end{enumerate}
\end{lemma}

\begin{proof}
By Cohomology of Spaces, Lemma
\ref{spaces-cohomology-lemma-proper-over-affine-cohomology-finite}
we see that $A = H^0(X, \mathcal{O}_X)$ is a finite dimensional
$k$-algebra. This proves (1).

\medskip\noindent
Then $A$ is a product of local rings by
Algebra, Lemma \ref{algebra-lemma-finite-dimensional-algebra} and
Algebra, Proposition \ref{algebra-proposition-dimension-zero-ring}.
If $X = Y \amalg Z$ with $Y$ and $Z$ open subspaces of $X$, then we obtain
an idempotent $e \in A$ by taking the section of $\mathcal{O}_X$
which is $1$ on $Y$ and $0$ on $Z$. Conversely, if $e \in A$
is an idempotent, then we get a corresponding decomposition of $|X|$.
Finally, as $|X|$ is a Noetherian topological space
(by Morphisms of Spaces, Lemma
\ref{spaces-morphisms-lemma-finite-presentation-noetherian} and
Properties of Spaces, Lemma
\ref{spaces-properties-lemma-Noetherian-topology})
its connected components are open. Hence the connected components
of $|X|$ correspond $1$-to-$1$ with primitive idempotents of $A$.
This proves (2).

\medskip\noindent
If $X$ is reduced, then $A$ is reduced
(Properties of Spaces, Lemma \ref{spaces-properties-lemma-reduced-space}).
Hence the local rings $A_i = k_i$ are reduced and therefore fields
(for example by Algebra, Lemma \ref{algebra-lemma-minimal-prime-reduced-ring}).
This proves (3).

\medskip\noindent
If $X$ is geometrically reduced, then same thing is true for
$A \otimes_k \overline{k} =
H^0(X_{\overline{k}}, \mathcal{O}_{X_{\overline{k}}})$
(see Cohomology of Spaces, Lemma
\ref{spaces-cohomology-lemma-flat-base-change-cohomology} for equality).
This implies that $k_i \otimes_k \overline{k}$ is a product
of fields and hence $k_i/k$ is separable for example by
Algebra,
Lemmas \ref{algebra-lemma-characterize-separable-field-extensions} and
\ref{algebra-lemma-geometrically-reduced-finite-purely-inseparable-extension}.
This proves (4).

\medskip\noindent
If $X$ is geometrically connected, then $A \otimes_k \overline{k} =
H^0(X_{\overline{k}}, \mathcal{O}_{X_{\overline{k}}})$
is a zero dimensional local ring by part (2) and hence its
spectrum has one point, in particular it is irreducible.
Thus $A$ is geometrically irreducible. This proves (5).
Of course (5) implies (6).

\medskip\noindent
If $X$ is geometrically reduced and connected, then
$A = k_1$ is a field and the extension $k_1/k$ is finite separable and
geometrically irreducible. However, then $k_1 \otimes_k \overline{k}$
is a product of $[k_1 : k]$ copies of $\overline{k}$ and we conclude
that $k_1 = k$. This proves (7). Of course (7) implies (8).
\end{proof}

\begin{lemma}
\label{lemma-characterize-trivial-pic-integral}
Let $k$ be a field. Let $X$ be a proper integral algebraic space over $k$.
Let $\mathcal{L}$ be an invertible $\mathcal{O}_X$-module.
If $H^0(X, \mathcal{L})$ and $H^0(X, \mathcal{L}^{\otimes - 1})$
are both nonzero, then $\mathcal{L} \cong \mathcal{O}_X$.
\end{lemma}

\begin{proof}
Let $s \in H^0(X, \mathcal{L})$ and $t \in H^0(X, \mathcal{L}^{\otimes - 1})$
be nonzero sections. Let $x \in |X|$ be a point in the support of $s$.
Choose an affine \'etale neighbourhood $(U, u) \to (X, x)$ such that
$\mathcal{L}|_U \cong \mathcal{O}_U$. Then $s|_U$ corresponds to a nonzero
regular function on the reduced (because $X$ is reduced) scheme $U$ and hence
is nonvanishing in a generic point of an irreducible component of $U$. By
Decent Spaces, Lemma \ref{decent-spaces-lemma-decent-generic-points}
we conclude that the generic point $\eta$ of $|X|$ is in the support of $s$.
The same is true for $t$. Then of course $st$ must be nonzero because
the local ring of $X$ at $\eta$ is a field (by aforementioned lemma
the local ring has dimension zero, as $X$ is reduced the local ring is
reduced, and Algebra, Lemma \ref{algebra-lemma-minimal-prime-reduced-ring}).
However, we have seen that $K = H^0(X, \mathcal{O}_X)$
is a field in Lemma \ref{lemma-proper-geometrically-reduced-global-sections}.
Thus $st$ is everywhere nonzero and we see that
$s : \mathcal{O}_X \to \mathcal{L}$ is an isomorphism.
\end{proof}






\section{Dimension}
\label{section-dimension}

\noindent
In this section we continue the discussion about dimension.
Here is a list of previous material:
\begin{enumerate}
\item dimension is defined in
Properties of Spaces, Section
\ref{spaces-properties-section-dimension},
\item dimension of local ring is defined in
Properties of Spaces, Section
\ref{spaces-properties-section-dimension-local-ring},
\item a couple of results in Properties of Spaces, Lemmas
\ref{spaces-properties-lemma-dimension-local-ring} and
\ref{spaces-properties-lemma-dimension-decent-invariant-under-etale},
\item relative dimension is defined in
Morphisms of Spaces, Section \ref{spaces-morphisms-section-relative-dimension},
\item results on dimension of fibres in
Morphisms of Spaces, Section \ref{spaces-morphisms-section-dimension-fibres},
\item a weak form of the dimension formula
Morphisms of Spaces, Section \ref{spaces-morphisms-section-dimension-formula},
\item a result on smoothness and dimension Morphisms of Spaces, Lemma
\ref{spaces-morphisms-lemma-smoothness-dimension-spaces},
\item dimension is $\dim(|X|)$ for decent spaces
Decent Spaces, Lemma \ref{decent-spaces-lemma-dimension-decent-space},
\item quasi-finite maps and dimension
Decent Spaces, Lemmas
\ref{decent-spaces-lemma-dimension-local-ring-quasi-finite} and
\ref{decent-spaces-lemma-dimension-quasi-finite}.
\end{enumerate}
In More on Morphisms of Spaces, Section
\ref{spaces-more-morphisms-section-dimension}
we will discuss jumping of
dimension in fibres of a finite type morphism.

\begin{lemma}
\label{lemma-integral-dimension}
Let $S$ be a scheme. Let $f : X \to Y$ be an integral morphism
of algebraic spaces. Then $\dim(X) \leq \dim(Y)$.
If $f$ is surjective then $\dim(X) = \dim(Y)$.
\end{lemma}

\begin{proof}
Choose $V \to Y$ surjective \'etale with $V$ a scheme.
Then $U = X \times_Y V$ is a scheme and $U \to V$ is integral
(and surjective if $f$ is surjective).
By Properties of Spaces, Lemma
\ref{spaces-properties-lemma-dimension-decent-invariant-under-etale}
we have $\dim(X) = \dim(U)$ and $\dim(Y) = \dim(V)$.
Thus the result follows from the case of schemes
which is Morphisms, Lemma \ref{morphisms-lemma-integral-dimension}.
\end{proof}

\begin{lemma}
\label{lemma-alteration-dimension}
Let $S$ be a scheme. Let $f : X \to Y$ be a morphism of algebraic spaces
over $S$. Assume that
\begin{enumerate}
\item $Y$ is locally Noetherian,
\item $X$ and $Y$ are integral algebraic spaces,
\item $f$ is dominant, and
\item $f$ is locally of finite type.
\end{enumerate}
If $x \in |X|$ and $y \in |Y|$ are the generic points, then
$$
\dim(X) \leq \dim(Y) + \text{transcendence degree of }x/y.
$$
If $f$ is proper, then equality holds.
\end{lemma}

\begin{proof}
Recall that $|X|$ and $|Y|$ are irreducible sober topological spaces, see
discussion following Definition \ref{definition-integral-algebraic-space}.
Thus the fact that $f$ is dominant means that $|f|$ maps $x$ to $y$.
Moreover, $x \in |X|$ is the unique point at which the
local ring of $X$ has dimension $0$, see
Decent Spaces, Lemma \ref{decent-spaces-lemma-decent-generic-points}.
By Morphisms of Spaces, Lemma
\ref{spaces-morphisms-lemma-dimension-formula-general}
we see that the dimension of the local ring of $X$ at
any point $x' \in |X|$ is at most the dimension of the local
ring of $Y$ at $y' = f(x')$ plus the transcendence degree of $x/y$.
Since the dimension of $X$, resp.\ dimension of $Y$ is the
supremum of the dimensions of the local rings at $x'$, resp.\ $y'$
(Properties of Spaces, Lemma \ref{spaces-properties-lemma-dimension})
we conclude the inequality holds.

\medskip\noindent
Assume $f$ is proper.
Let $V \subset Y$ be a nonempty quasi-compact open subspace.
If we can prove the equality for the morphism $f^{-1}(V) \to V$,
then we get the equality for $X \to Y$. Thus we may assume that
$X$ and $Y$ are quasi-compact.
Observe that $X$ is quasi-separated as
a locally Noetherian decent algebraic space, see
Decent Spaces, Lemma
\ref{decent-spaces-lemma-locally-Noetherian-decent-quasi-separated}.
Thus we may choose $Y' \to Y$ finite surjective where $Y'$
is a scheme, see Limits of Spaces, Proposition
\ref{spaces-limits-proposition-there-is-a-scheme-finite-over}.
After replacing $Y'$ by a suitable closed subscheme, we
may assume $Y'$ is integral, see for example the more general
Lemma \ref{lemma-alteration-contained-in}.
By the same lemma, we may choose a closed subspace
$X' \subset X \times_Y Y'$ such that $X'$ is integral
and $X' \to X$ is finite surjective.
Now $X'$ is also locally Noetherian
(Morphisms of Spaces, Lemma
\ref{spaces-morphisms-lemma-locally-finite-type-locally-noetherian})
and we can use Limits of Spaces, Proposition
\ref{spaces-limits-proposition-there-is-a-scheme-finite-over}
once more to choose a finite surjective morphism $X'' \to X'$
with $X''$ a scheme. As before we may assume that $X''$
is integral. Picture
$$
\xymatrix{
X'' \ar[d] \ar[r] & X \ar[d]^f \\
Y' \ar[r] & Y
}
$$
By Lemma \ref{lemma-integral-dimension} we have $\dim(X'') = \dim(X)$
and $\dim(Y') = \dim(Y)$. Since $X$ and $Y$ have open neighbourhoods
of $x$, resp.\ $y$ which are schemes, we readily see that the generic
points $x'' \in X''$, resp.\ $y' \in Y'$ are the unique points mapping
to $x$, resp.\ $y$ and that the residue field extensions
$\kappa(x'')/\kappa(x)$ and $\kappa(y')/\kappa(y)$ are finite.
This implies that the transcendence degree of $x''/y'$ is the
same as the transcendence degree of $x/y$. Thus the equality follows
from the case of schemes whicn is
Morphisms, Lemma \ref{morphisms-lemma-alteration-dimension}.
\end{proof}





\section{Spaces smooth over fields}
\label{section-smooth}

\noindent
This section is the analogue of
Varieties, Section \ref{varieties-section-smooth}.

\begin{lemma}
\label{lemma-smooth-regular}
Let $k$ be a field.
Let $X$ be an algebraic space smooth over $k$.
Then $X$ is a regular algebraic space.
\end{lemma}

\begin{proof}
Choose a scheme $U$ and a surjective \'etale morphism $U \to X$.
The morphism $U \to \Spec(k)$ is smooth as a composition of
an \'etale (hence smooth) morphism and a smooth morphism (see
Morphisms of Spaces, Lemmas \ref{spaces-morphisms-lemma-etale-smooth}
and \ref{spaces-morphisms-lemma-composition-smooth}).
Hence $U$ is regular by
Varieties, Lemma \ref{varieties-lemma-smooth-regular}.
By
Properties of Spaces, Definition
\ref{spaces-properties-definition-type-property}
this means that $X$ is regular.
\end{proof}

\begin{lemma}
\label{lemma-smooth-separable-closed-points-dense}
Let $k$ be a field. Let $X$ be an algebraic space smooth over $\Spec(k)$.
The set of $x \in |X|$ which are image of morphisms $\Spec(k') \to X$
with $k' \supset k$ finite separable is dense in $|X|$.
\end{lemma}

\begin{proof}
Choose a scheme $U$ and a surjective \'etale morphism $U \to X$.
The morphism $U \to \Spec(k)$ is smooth as a composition of
an \'etale (hence smooth) morphism and a smooth morphism (see
Morphisms of Spaces, Lemmas \ref{spaces-morphisms-lemma-etale-smooth}
and \ref{spaces-morphisms-lemma-composition-smooth}).
Hence we can apply Varieties, Lemma
\ref{varieties-lemma-smooth-separable-closed-points-dense} to see that
the closed points of $U$ whose residue fields are finite separable over
$k$ are dense. This implies the lemma by our definition of the
topology on $|X|$.
\end{proof}







\section{Euler characteristics}
\label{section-euler}

\noindent
In this section we prove some elementary properties of Euler characteristics
of coherent sheaves on algebraic spaces proper over fields.

\begin{definition}
\label{definition-euler-characteristic}
Let $k$ be a field. Let $X$ be a proper algebraic over $k$. Let $\mathcal{F}$
be a coherent $\mathcal{O}_X$-module. In this situation the
{\it Euler characteristic of $\mathcal{F}$} is the integer
$$
\chi(X, \mathcal{F}) = \sum\nolimits_i (-1)^i \dim_k H^i(X, \mathcal{F}).
$$
For justification of the formula see below.
\end{definition}

\noindent
In the situation of the definition only a finite number of the vector spaces
$H^i(X, \mathcal{F})$ are nonzero (Cohomology of Spaces, Lemma
\ref{spaces-cohomology-lemma-vanishing-quasi-separated})
and each of these spaces is finite dimensional
(Cohomology of Spaces, Lemma
\ref{spaces-cohomology-lemma-proper-over-affine-cohomology-finite}). Thus
$\chi(X, \mathcal{F}) \in \mathbf{Z}$ is well defined. Observe that
this definition depends on the field $k$ and not just on the pair
$(X, \mathcal{F})$.

\begin{lemma}
\label{lemma-euler-characteristic-additive}
Let $k$ be a field. Let $X$ be a proper algebraic space over $k$.
Let $0 \to \mathcal{F}_1 \to \mathcal{F}_2 \to \mathcal{F}_3 \to 0$
be a short exact sequence of coherent modules on $X$. Then
$$
\chi(X, \mathcal{F}_2) = \chi(X, \mathcal{F}_1) + \chi(X, \mathcal{F}_3)
$$
\end{lemma}

\begin{proof}
Consider the long exact sequence of cohomology
$$
0 \to H^0(X, \mathcal{F}_1) \to H^0(X, \mathcal{F}_2) \to
H^0(X, \mathcal{F}_3) \to H^1(X, \mathcal{F}_1) \to \ldots
$$
associated to the short exact sequence of the lemma. The rank-nullity theorem
in linear algebra shows that
$$
0 = \dim H^0(X, \mathcal{F}_1) - \dim H^0(X, \mathcal{F}_2)
+ \dim H^0(X, \mathcal{F}_3) - \dim H^1(X, \mathcal{F}_1) + \ldots
$$
This immediately implies the lemma.
\end{proof}

\begin{lemma}
\label{lemma-euler-characteristic-morphism}
Let $k$ be a field. Let $f : Y \to X$ be a morphism of
algebraic spaces proper over $k$. Let $\mathcal{G}$ be a
coherent $\mathcal{O}_Y$-module. Then
$$
\chi(Y, \mathcal{G}) = \sum (-1)^i \chi(X, R^if_*\mathcal{G})
$$
\end{lemma}

\begin{proof}
The formula makes sense: the sheaves $R^if_*\mathcal{G}$ are coherent
and only a finite number of them are nonzero, see
Cohomology of Spaces, Lemmas
\ref{spaces-cohomology-lemma-proper-pushforward-coherent} and
\ref{spaces-cohomology-lemma-vanishing-higher-direct-images}.
By Cohomology on Sites, Lemma \ref{sites-cohomology-lemma-Leray}
there is a spectral sequence with
$$
E_2^{p, q} = H^p(X, R^qf_*\mathcal{G})
$$
converging to $H^{p + q}(Y, \mathcal{G})$. By finiteness of cohomology
on $X$ we see that only a finite number of $E_2^{p, q}$ are nonzero
and each $E_2^{p, q}$ is a finite dimensional vector space. It follows
that the same is true for $E_r^{p, q}$ for $r \geq 2$ and that
$$
\sum (-1)^{p + q} \dim_k E_r^{p, q}
$$
is independent of $r$. Since for $r$ large enough we have
$E_r^{p, q} = E_\infty^{p, q}$ and since convergence means there
is a filtration on $H^n(Y, \mathcal{G})$ whose graded pieces are
$E_\infty^{p, q}$ with $p + 1 = n$ (this is the meaning of convergence
of the spectral sequence), we conclude.
\end{proof}








\section{Numerical intersections}
\label{section-num}

\noindent
In this section we play around with the Euler characteristic of
coherent sheaves on proper algebraic spaces to obtain numerical intersection
numbers for invertible modules. Our main tool will be the following
lemma.

\begin{lemma}
\label{lemma-numerical-polynomial-from-euler}
Let $k$ be a field. Let $X$ be a proper algebraic space over $k$.
Let $\mathcal{F}$ be a coherent $\mathcal{O}_X$-module. Let
$\mathcal{L}_1, \ldots, \mathcal{L}_r$ be invertible $\mathcal{O}_X$-modules.
The map
$$
(n_1, \ldots, n_r) \longmapsto
\chi(X, \mathcal{F} \otimes
\mathcal{L}_1^{\otimes n_1} \otimes \ldots \otimes
\mathcal{L}_r^{\otimes n_r})
$$
is a numerical polynomial in $n_1, \ldots, n_r$ of total degree at
most the dimension of the scheme theoretic support of $\mathcal{F}$.
\end{lemma}

\begin{proof}
Let $Z \subset X$ be the scheme theoretic support of $\mathcal{F}$.
Then $\mathcal{F} = i_*\mathcal{G}$ for some coherent
$\mathcal{O}_Z$-module $\mathcal{G}$
(Cohomology of Spaces, Lemma
\ref{spaces-cohomology-lemma-coherent-support-closed})
and we have
$$
\chi(X, \mathcal{F} \otimes
\mathcal{L}_1^{\otimes n_1} \otimes \ldots \otimes
\mathcal{L}_r^{\otimes n_r}) =
\chi(Z, \mathcal{G} \otimes
i^*\mathcal{L}_1^{\otimes n_1} \otimes \ldots \otimes
i^*\mathcal{L}_r^{\otimes n_r})
$$
by the projection formula
(Cohomology on Sites, Lemma \ref{sites-cohomology-lemma-projection-formula})
and Cohomology of Spaces, Lemma
\ref{spaces-cohomology-lemma-relative-affine-cohomology}.
Since $|Z| = \text{Supp}(\mathcal{F})$ we see that it suffices
to show
$$
P_\mathcal{F}(n_1, \ldots, n_r) :
(n_1, \ldots, n_r)
\longmapsto
\chi(X, \mathcal{F} \otimes
\mathcal{L}_1^{\otimes n_1} \otimes \ldots \otimes
\mathcal{L}_r^{\otimes n_r})
$$
is a numerical polynomial in $n_1, \ldots, n_r$ of total degree at
most $\dim(X)$. Let us say property $\mathcal{P}$ holds for the
coherent $\mathcal{O}_X$-module $\mathcal{F}$ if the above is true.

\medskip\noindent
We will prove this statement by devissage, more precisely we will
check conditions (1), (2), and (3) of
Cohomology of Spaces, Lemma
\ref{spaces-cohomology-lemma-property-higher-rank-cohomological-variant}
are satisfied.

\medskip\noindent
Verification of condition (1). Let
$$
0 \to \mathcal{F}_1 \to \mathcal{F}_2 \to \mathcal{F}_3 \to 0
$$
be a short exact sequence of coherent sheaves on $X$.
By Lemma \ref{lemma-euler-characteristic-additive} we have
$$
P_{\mathcal{F}_2}(n_1, \ldots, n_r) =
P_{\mathcal{F}_1}(n_1, \ldots, n_r) +
P_{\mathcal{F}_3}(n_1, \ldots, n_r)
$$
Then it is clear that if 2-out-of-3 of the sheaves $\mathcal{F}_i$
have property $\mathcal{P}$, then so does the third.

\medskip\noindent
Condition (2) follows because
$P_{\mathcal{F}^{\oplus m}}(n_1, \ldots, n_r) =
mP_\mathcal{F}(n_1, \ldots, n_r)$.

\medskip\noindent
Proof of (3). Let $i : Z \to X$ be a reduced closed subspace with
$|Z|$ irreducible. We have to find a coherent module $\mathcal{G}$
on $X$ whose support is $Z$ such that $\mathcal{P}$ holds for $\mathcal{G}$.
We will give two constructions: one using Chow's lemma and one
using a finite cover by a scheme.

\medskip\noindent
Proof existence $\mathcal{G}$ using a finite cover by a scheme.
Choose $\pi : Z' \to Z$ finite surjective where $Z'$ is a scheme, see
Limits of Spaces, Proposition
\ref{spaces-limits-proposition-there-is-a-scheme-finite-over}.
Set $\mathcal{G} = i_*\pi_*\mathcal{O}_{Z'} = (i \circ \pi)_*\mathcal{O}_{Z'}$.
Note that $Z'$ is proper over $k$ and that the support of $\mathcal{G}$ is $Y$
(details omitted). We have
$$
R(\pi \circ i)_*(\mathcal{O}_{Z'}) = \mathcal{G}
\quad\text{and}\quad
R(\pi \circ i)_*(\pi^*i^*(\mathcal{L}_1^{\otimes n_1} \otimes \ldots \otimes
\mathcal{L}_r^{\otimes n_r})
) = \mathcal{G} \otimes \mathcal{L}_1^{\otimes n_1} \otimes \ldots \otimes
\mathcal{L}_r^{\otimes n_r}
$$
The first equality holds because $i \circ \pi$ is affine
(Cohomology of Spaces, Lemma
\ref{spaces-cohomology-lemma-affine-vanishing-higher-direct-images})
and the second equality follows from the first and the projection formula
(Cohomology on Sites, Lemma \ref{sites-cohomology-lemma-projection-formula}).
Using Leray
(Cohomology on Sites, Lemma \ref{sites-cohomology-lemma-apply-Leray})
we obtain
$$
P_\mathcal{G}(n_1, \ldots, n_r) =
\chi(Z', \pi^*i^*(\mathcal{L}_1^{\otimes n_1} \otimes \ldots \otimes
\mathcal{L}_r^{\otimes n_r}))
$$
By the case of schemes
(Varieties, Lemma \ref{varieties-lemma-numerical-polynomial-from-euler})
this is a numerical polynomial in
$n_1, \ldots, n_r$ of degree at most $\dim(Z')$.
We conclude because $\dim(Z') \leq \dim(Z) \leq \dim(X)$.
The first inequality follows from
Decent Spaces, Lemma \ref{decent-spaces-lemma-dimension-quasi-finite}.

\medskip\noindent
Proof existence $\mathcal{G}$ using Chow's lemma. We apply
Cohomology of Spaces, Lemma \ref{spaces-cohomology-lemma-weak-chow}
to the morphism $Z \to \Spec(k)$. Thus we get a
surjective proper morphism $f : Y \to Z$ over $\Spec(k)$
where $Y$ is a closed subscheme of $\mathbf{P}^m_k$ for some $m$.
After replacing $Y$ by a closed subscheme we may assume that $Y$
is integral and $f : Y \to Z$ is an alteration, see
Lemma \ref{lemma-alteration-contained-in}.
Denote $\mathcal{O}_Y(n)$ the pullback of $\mathcal{O}_{\mathbf{P}^m_k}(n)$.
Pick $n > 0$ such that $R^pf_*\mathcal{O}_Y(n) = 0$
for $p > 0$, see Cohomology of Spaces, Lemma
\ref{spaces-cohomology-lemma-kill-by-twisting}.
We claim that $\mathcal{G} = i_*f_*\mathcal{O}_Y(n)$ satisfies $\mathcal{P}$.
Namely, by the case of schemes
(Varieties, Lemma \ref{varieties-lemma-numerical-polynomial-from-euler})
we know that
$$
(n_1, \ldots, n_r)
\longmapsto
\chi(Y, \mathcal{O}_Y(n) \otimes
f^*i^*(\mathcal{L}_1^{\otimes n_1} \otimes \ldots \otimes
\mathcal{L}_r^{\otimes n_r}))
$$
is a numerical polynomial in $n_1, \ldots, n_r$ of total degree at
most $\dim(Y)$. On the other hand, by the projection formula
(Cohomology on Sites, Lemma \ref{sites-cohomology-lemma-projection-formula})
\begin{align*}
i_*Rf_*\left(
\mathcal{O}_Y(n) \otimes
f^*i^*(\mathcal{L}_1^{\otimes n_1} \otimes \ldots \otimes
\mathcal{L}_r^{\otimes n_r})\right)
& =
i_*Rf_*\mathcal{O}_Y(n) \otimes
\mathcal{L}_1^{\otimes n_1} \otimes \ldots \otimes
\mathcal{L}_r^{\otimes n_r} \\
& =
\mathcal{G} \otimes \mathcal{L}_1^{\otimes n_1} \otimes \ldots \otimes
\mathcal{L}_r^{\otimes n_r}
\end{align*}
the last equality by our choice of $n$. By
Leray (Cohomology on Sites, Lemma \ref{sites-cohomology-lemma-apply-Leray})
we get
$$
\chi(Y, \mathcal{O}_Y(n) \otimes
f^*i^*(\mathcal{L}_1^{\otimes n_1} \otimes \ldots \otimes
\mathcal{L}_r^{\otimes n_r})) =
P_\mathcal{G}(n_1, \ldots, n_r)
$$
and we conclude because $\dim(Y) \leq \dim(Z) \leq \dim(X)$.
The first inequality holds by
Morphisms of Spaces, Lemma
\ref{spaces-morphisms-lemma-alteration-dimension-general}
and the fact that $Y \to Z$ is an alteration (and hence the
induced extension of residue fields in generic points is finite).
\end{proof}

\noindent
The following lemma roughly shows that the leading coefficient only depends
on the length of the coherent module in the generic points of its
support.

\begin{lemma}
\label{lemma-numerical-polynomial-leading-term}
Let $k$ be a field. Let $X$ be a proper algebraic space over $k$. Let
$\mathcal{F}$ be a coherent $\mathcal{O}_X$-module. Let
$\mathcal{L}_1, \ldots, \mathcal{L}_r$ be invertible $\mathcal{O}_X$-modules.
Let $d = \dim(\text{Supp}(\mathcal{F}))$.
Let $Z_i \subset X$ be the irreducible components
of $\text{Supp}(\mathcal{F})$ of dimension $d$. Let $\overline{x}_i$
be a geometric generic point of $Z_i$ and set
$m_i = \text{length}_{\mathcal{O}_{X, \overline{x}_i}}
(\mathcal{F}_{\overline{x}_i})$.
Then
$$
\chi(X, \mathcal{F} \otimes \mathcal{L}_1^{\otimes n_1} \otimes \ldots \otimes
\mathcal{L}_r^{\otimes n_r}) -
\sum\nolimits_i
m_i\ \chi(Z_i, \mathcal{L}_1^{\otimes n_1} \otimes \ldots \otimes
\mathcal{L}_r^{\otimes n_r}|_{Z_i})
$$
is a numerical polynomial in $n_1, \ldots, n_r$ of total degree $< d$.
\end{lemma}

\begin{proof}
We first prove a slightly weaker statement. Namely, say
$\dim(X) = N$ and let $X_i \subset X$ be the irreducible
components of dimension $N$. Let $\overline{x}_i$ be a geometric
generic point of $X_i$. The \'etale local ring
$\mathcal{O}_{X, \overline{x}_i}$ is Noetherian of
dimension $0$, hence for every coherent $\mathcal{O}_X$-module
$\mathcal{F}$ the length
$$
m_i(\mathcal{F}) = \text{length}_{\mathcal{O}_{X, \overline{x}_i}}
(\mathcal{F}_{\overline{x}_i})
$$
is an integer $\geq 0$. We claim that
$$
E(\mathcal{F}) =
\chi(X, \mathcal{F} \otimes \mathcal{L}_1^{\otimes n_1} \otimes \ldots \otimes
\mathcal{L}_r^{\otimes n_r}) -
\sum\nolimits_i
m_i(\mathcal{F})\ \chi(Z_i, \mathcal{L}_1^{\otimes n_1} \otimes \ldots \otimes
\mathcal{L}_r^{\otimes n_r}|_{Z_i})
$$
is a numerical polynomial in $n_1, \ldots, n_r$ of total degree $< N$.
We will prove this using Cohomology of Spaces, Lemma
\ref{spaces-cohomology-lemma-property-higher-rank-cohomological-variant}.
For any short exact sequence $0 \to \mathcal{F}' \to \mathcal{F} \to
\mathcal{F}'' \to 0$ we have
$E(\mathcal{F}) = E(\mathcal{F}') + E(\mathcal{F}'')$.
This follows from additivity of Euler characteristics
(Lemma \ref{lemma-euler-characteristic-additive})
and additivity of lengths
(Algebra, Lemma \ref{algebra-lemma-length-additive}).
This immediately implies properties (1) and (2) of Cohomology of Spaces, Lemma
\ref{spaces-cohomology-lemma-property-higher-rank-cohomological-variant}.
Finally, property (3) holds because for $\mathcal{G} = \mathcal{O}_Z$
for any $Z \subset X$ irreducible reduced closed subspace.
Namely, if $Z = Z_{i_0}$ for some $i_0$, then
$m_i(\mathcal{G}) = \delta_{i_0i}$ and we conclude $E(\mathcal{G}) = 0$.
If $Z \not = Z_i$ for any $i$, then $m_i(\mathcal{G}) = 0$ for all $i$,
$\dim(Z) < N$ and we get the result from
Lemma \ref{lemma-numerical-polynomial-from-euler}.

\medskip\noindent
Proof of the statement as in the lemma.
Let $Z \subset X$ be the scheme theoretic support of $\mathcal{F}$.
Then $\mathcal{F} = i_*\mathcal{G}$ for some coherent
$\mathcal{O}_Z$-module $\mathcal{G}$
(Cohomology of Spaces, Lemma
\ref{spaces-cohomology-lemma-coherent-support-closed})
and we have
$$
\chi(X, \mathcal{F} \otimes
\mathcal{L}_1^{\otimes n_1} \otimes \ldots \otimes
\mathcal{L}_r^{\otimes n_r}) =
\chi(Z, \mathcal{G} \otimes
i^*\mathcal{L}_1^{\otimes n_1} \otimes \ldots \otimes
i^*\mathcal{L}_r^{\otimes n_r})
$$
by the projection formula
(Cohomology on Sites, Lemma \ref{sites-cohomology-lemma-projection-formula})
and Cohomology of Spaces, Lemma
\ref{spaces-cohomology-lemma-relative-affine-cohomology}.
Since $|Z| = \text{Supp}(\mathcal{F})$ we see that $Z_i \subset Z$
for all $i$ and we see that these are the irreducible components
of $Z$ of dimension $d$. We may and do think of $\overline{x}_i$
as a geometric point of $Z$. The map
$i^\sharp : \mathcal{O}_X \to i_*\mathcal{O}_Z$
determines a surjection
$$
\mathcal{O}_{X, \overline{x}_i} \to \mathcal{O}_{Z, \overline{x}_i}
$$
Via this map we have an isomorphism of modules
$\mathcal{G}_{\overline{x}_i} = \mathcal{F}_{\overline{x}_i}$
as $\mathcal{F} = i_*\mathcal{G}$. This implies that
$$
m_i = \text{length}_{\mathcal{O}_{X, \overline{x}_i}}
(\mathcal{F}_{\overline{x}_i}) =
\text{length}_{\mathcal{O}_{Z, \overline{x}_i}}
(\mathcal{G}_{\overline{x}_i})
$$
Thus we see that the expression in the lemma is equal to
$$
\chi(Z, \mathcal{G} \otimes \mathcal{L}_1^{\otimes n_1} \otimes \ldots \otimes
\mathcal{L}_r^{\otimes n_r}) -
\sum\nolimits_i
m_i\ \chi(Z_i, \mathcal{L}_1^{\otimes n_1} \otimes \ldots \otimes
\mathcal{L}_r^{\otimes n_r}|_{Z_i})
$$
and the result follows from the discussion in the first paragraph
(applied with $Z$ in stead of $X$).
\end{proof}

\begin{definition}
\label{definition-intersection-number}
Let $k$ be a field. Let $X$ be a proper algebraic space over $k$. Let
$i : Z \to X$ be a closed subspace of dimension $d$. Let
$\mathcal{L}_1, \ldots, \mathcal{L}_d$ be invertible
$\mathcal{O}_X$-modules. We define the {\it intersection number}
$(\mathcal{L}_1 \cdots \mathcal{L}_d \cdot Z)$
as the coefficient of $n_1 \ldots n_d$ in the numerical polynomial
$$
\chi(X, i_*\mathcal{O}_Z \otimes \mathcal{L}_1^{\otimes n_1} \otimes
\ldots \otimes \mathcal{L}_d^{\otimes n_d}) =
\chi(Z, \mathcal{L}_1^{\otimes n_1} \otimes
\ldots \otimes \mathcal{L}_d^{\otimes n_d}|_Z)
$$
In the special
case that $\mathcal{L}_1 = \ldots = \mathcal{L}_d = \mathcal{L}$
we write $(\mathcal{L}^d \cdot Z)$.
\end{definition}

\noindent
The displayed equality in the definition follows from
the projection formula
(Cohomology, Section \ref{cohomology-section-projection-formula}) and
Cohomology of Schemes, Lemma
\ref{coherent-lemma-relative-affine-cohomology}.
We prove a few lemmas for these intersection numbers.

\begin{lemma}
\label{lemma-intersection-number-integer}
In the situation of Definition \ref{definition-intersection-number}
the intersection number
$(\mathcal{L}_1 \cdots \mathcal{L}_d \cdot Z)$
is an integer.
\end{lemma}

\begin{proof}
Any numerical polynomial of degree $e$ in $n_1, \ldots, n_d$ can be
written uniquely as a $\mathbf{Z}$-linear combination of the functions
${n_1 \choose k_1}{n_2 \choose k_2} \ldots {n_d \choose k_d}$ with
$k_1 + \ldots + k_d \leq e$. Apply this with $e = d$.
Left as an exercise.
\end{proof}

\begin{lemma}
\label{lemma-intersection-number-additive}
In the situation of Definition \ref{definition-intersection-number}
the intersection number
$(\mathcal{L}_1 \cdots \mathcal{L}_d \cdot Z)$
is additive: if $\mathcal{L}_i = \mathcal{L}_i' \otimes \mathcal{L}_i''$,
then we have
$$
(\mathcal{L}_1 \cdots \mathcal{L}_i \cdots \mathcal{L}_d \cdot Z) =
(\mathcal{L}_1 \cdots \mathcal{L}_i' \cdots \mathcal{L}_d \cdot Z) +
(\mathcal{L}_1 \cdots \mathcal{L}_i'' \cdots \mathcal{L}_d \cdot Z)
$$
\end{lemma}

\begin{proof}
This is true because by Lemma \ref{lemma-numerical-polynomial-from-euler}
the function
$$
(n_1, \ldots, n_{i - 1}, n_i', n_i'', n_{i + 1}, \ldots, n_d)
\mapsto
\chi(Z, \mathcal{L}_1^{\otimes n_1} \otimes
\ldots \otimes (\mathcal{L}_i')^{\otimes n_i'} \otimes
(\mathcal{L}_i'')^{\otimes n_i''} \otimes \ldots \otimes
\mathcal{L}_d^{\otimes n_d}|_Z)
$$
is a numerical polynomial of total degree at most $d$ in $d + 1$ variables.
\end{proof}

\begin{lemma}
\label{lemma-intersection-number-in-terms-of-components}
In the situation of Definition \ref{definition-intersection-number}
let $Z_i \subset Z$ be the irreducible components of dimension $d$. Let
$m_i = \text{length}_{\mathcal{O}_{X, \overline{x}_i}}
(\mathcal{O}_{Z, \overline{x}_i})$
where $\overline{x}_i$ is a geometric generic point of $Z_i$. Then
$$
(\mathcal{L}_1 \cdots \mathcal{L}_d \cdot Z) =
\sum m_i(\mathcal{L}_1 \cdots \mathcal{L}_d \cdot Z_i)
$$
\end{lemma}

\begin{proof}
Immediate from Lemma \ref{lemma-numerical-polynomial-leading-term}
and the definitions.
\end{proof}

\begin{lemma}
\label{lemma-intersection-number-and-pullback}
Let $k$ be a field. Let $f : Y \to X$ be a morphism of
algebraic spaces proper over $k$.
Let $Z \subset Y$ be an integral closed subspace of dimension $d$ and let
$\mathcal{L}_1, \ldots, \mathcal{L}_d$ be invertible $\mathcal{O}_X$-modules.
Then
$$
(f^*\mathcal{L}_1 \cdots f^*\mathcal{L}_d \cdot Z) =
\deg(f|_Z : Z \to f(Z)) (\mathcal{L}_1 \cdots \mathcal{L}_d \cdot f(Z))
$$
where $\deg(Z \to f(Z))$ is as in Definition \ref{definition-degree}
or $0$ if $\dim(f(Z)) < d$.
\end{lemma}

\begin{proof}
In the statement $f(Z) \subset X$ is the scheme theoretic image of $f$
and it is also the reduced induced algebraic space structure on the
closed subset $f(|Z|) \subset X$, see Morphisms of Spaces, Lemma
\ref{spaces-morphisms-lemma-scheme-theoretic-image-reduced}.
Then $Z$ and $f(Z)$ are reduced, proper (hence decent) algebraic spaces
over $k$, whence integral
(Definition \ref{definition-integral-algebraic-space}).
The left hand side is computed using the coefficient of $n_1 \ldots n_d$
in the function
$$
\chi(Y, \mathcal{O}_Z \otimes f^*\mathcal{L}_1^{\otimes n_1} \otimes
\ldots \otimes f^*\mathcal{L}_d^{\otimes n_d}) =
\sum (-1)^i
\chi(X, R^if_*\mathcal{O}_Z \otimes
\mathcal{L}_1^{\otimes n_1} \otimes \ldots \otimes
\mathcal{L}_d^{\otimes n_d})
$$
The equality follows from Lemma \ref{lemma-euler-characteristic-morphism}
and the projection formula
(Cohomology, Lemma \ref{cohomology-lemma-projection-formula}).
If $f(Z)$ has dimension $< d$, then the right hand side
is a polynomial of total degree $< d$ by
Lemma \ref{lemma-numerical-polynomial-from-euler}
and the result is true. Assume $\dim(f(Z)) = d$. Then
by dimension theory (Lemma \ref{lemma-alteration-dimension})
we find that the equivalent conditions (1) -- (5) of
Lemma \ref{lemma-finite-degree} hold. Thus
$\deg(Z \to f(Z))$ is well defined.
By the already used Lemma \ref{lemma-finite-degree}
we find $f : Z \to f(Z)$ is finite over a nonempty open
$V$ of $f(Z)$; after possibly shrinking $V$ we may assume
$V$ is a scheme. Let $\xi \in V$ be the generic point.
Thus $\deg(f : Z \to f(Z))$ the length of the stalk of
$f_*\mathcal{O}_Z$ at $\xi$ over $\mathcal{O}_{X, \xi}$
and the stalk of $R^if_*\mathcal{O}_X$ at $\xi$ is zero for $i > 0$
(for example by Cohomology of Spaces, Lemma
\ref{spaces-cohomology-lemma-finite-higher-direct-image-zero}).
Thus the terms $\chi(X, R^if_*\mathcal{O}_Z \otimes
\mathcal{L}_1^{\otimes n_1} \otimes \ldots \otimes
\mathcal{L}_d^{\otimes n_d})$ with $i > 0$ have total
degree $< d$ and
$$
\chi(X, f_*\mathcal{O}_Z \otimes
\mathcal{L}_1^{\otimes n_1} \otimes \ldots \otimes
\mathcal{L}_d^{\otimes n_d})
=
\deg(f : Z \to f(Z)) \chi(f(Z),
\mathcal{L}_1^{\otimes n_1} \otimes \ldots \otimes
\mathcal{L}_d^{\otimes n_d}|_{f(Z)})
$$
modulo a polynomial of total degree $< d$ by
Lemma \ref{lemma-numerical-polynomial-leading-term}.
The desired result follows.
\end{proof}

\begin{lemma}
\label{lemma-numerical-intersection-effective-Cartier-divisor}
Let $k$ be a field. Let $X$ be a proper algebraic space over $k$.
Let $Z \subset X$ be a closed subspace of dimension $d$.
Let $\mathcal{L}_1, \ldots, \mathcal{L}_d$
be invertible $\mathcal{O}_X$-modules. Assume there exists an
effective Cartier divisor $D \subset Z$ such that
$\mathcal{L}_1|_Z \cong \mathcal{O}_Z(D)$. Then
$$
(\mathcal{L}_1 \cdots \mathcal{L}_d \cdot Z) =
(\mathcal{L}_2 \cdots \mathcal{L}_d \cdot D)
$$
\end{lemma}

\begin{proof}
We may replace $X$ by $Z$ and $\mathcal{L}_i$ by $\mathcal{L}_i|_Z$.
Thus we may assume $X = Z$ and $\mathcal{L}_1 = \mathcal{O}_X(D)$.
Then $\mathcal{L}_1^{-1}$ is the ideal sheaf of $D$ and we can
consider the short exact sequence
$$
0 \to \mathcal{L}_1^{\otimes -1} \to \mathcal{O}_X \to \mathcal{O}_D \to 0
$$
Set
$P(n_1, \ldots, n_d) =
\chi(X, \mathcal{L}_1^{\otimes n_1} \otimes \ldots \otimes
\mathcal{L}_d^{\otimes n_d})$
and
$Q(n_1, \ldots, n_d) =
\chi(D, \mathcal{L}_1^{\otimes n_1} \otimes \ldots \otimes
\mathcal{L}_d^{\otimes n_d}|_D)$.
We conclude from additivity
(Lemma \ref{lemma-euler-characteristic-additive})
that
$$
P(n_1, \ldots, n_d) - P(n_1 - 1, n_2, \ldots, n_d) =
Q(n_1, \ldots, n_d)
$$
Because the total degree of $P$ is at most $d$, we see that
the coefficient of $n_1 \ldots n_d$ in $P$ is equal to the coefficient
of $n_2 \ldots n_d$ in $Q$.
\end{proof}








\begin{multicols}{2}[\section{Other chapters}]
\noindent
Preliminaries
\begin{enumerate}
\item \hyperref[introduction-section-phantom]{Introduction}
\item \hyperref[conventions-section-phantom]{Conventions}
\item \hyperref[sets-section-phantom]{Set Theory}
\item \hyperref[categories-section-phantom]{Categories}
\item \hyperref[topology-section-phantom]{Topology}
\item \hyperref[sheaves-section-phantom]{Sheaves on Spaces}
\item \hyperref[sites-section-phantom]{Sites and Sheaves}
\item \hyperref[stacks-section-phantom]{Stacks}
\item \hyperref[fields-section-phantom]{Fields}
\item \hyperref[algebra-section-phantom]{Commutative Algebra}
\item \hyperref[brauer-section-phantom]{Brauer Groups}
\item \hyperref[homology-section-phantom]{Homological Algebra}
\item \hyperref[derived-section-phantom]{Derived Categories}
\item \hyperref[simplicial-section-phantom]{Simplicial Methods}
\item \hyperref[more-algebra-section-phantom]{More on Algebra}
\item \hyperref[smoothing-section-phantom]{Smoothing Ring Maps}
\item \hyperref[modules-section-phantom]{Sheaves of Modules}
\item \hyperref[sites-modules-section-phantom]{Modules on Sites}
\item \hyperref[injectives-section-phantom]{Injectives}
\item \hyperref[cohomology-section-phantom]{Cohomology of Sheaves}
\item \hyperref[sites-cohomology-section-phantom]{Cohomology on Sites}
\item \hyperref[dga-section-phantom]{Differential Graded Algebra}
\item \hyperref[dpa-section-phantom]{Divided Power Algebra}
\item \hyperref[sdga-section-phantom]{Differential Graded Sheaves}
\item \hyperref[hypercovering-section-phantom]{Hypercoverings}
\end{enumerate}
Schemes
\begin{enumerate}
\setcounter{enumi}{25}
\item \hyperref[schemes-section-phantom]{Schemes}
\item \hyperref[constructions-section-phantom]{Constructions of Schemes}
\item \hyperref[properties-section-phantom]{Properties of Schemes}
\item \hyperref[morphisms-section-phantom]{Morphisms of Schemes}
\item \hyperref[coherent-section-phantom]{Cohomology of Schemes}
\item \hyperref[divisors-section-phantom]{Divisors}
\item \hyperref[limits-section-phantom]{Limits of Schemes}
\item \hyperref[varieties-section-phantom]{Varieties}
\item \hyperref[topologies-section-phantom]{Topologies on Schemes}
\item \hyperref[descent-section-phantom]{Descent}
\item \hyperref[perfect-section-phantom]{Derived Categories of Schemes}
\item \hyperref[more-morphisms-section-phantom]{More on Morphisms}
\item \hyperref[flat-section-phantom]{More on Flatness}
\item \hyperref[groupoids-section-phantom]{Groupoid Schemes}
\item \hyperref[more-groupoids-section-phantom]{More on Groupoid Schemes}
\item \hyperref[etale-section-phantom]{\'Etale Morphisms of Schemes}
\end{enumerate}
Topics in Scheme Theory
\begin{enumerate}
\setcounter{enumi}{41}
\item \hyperref[chow-section-phantom]{Chow Homology}
\item \hyperref[intersection-section-phantom]{Intersection Theory}
\item \hyperref[pic-section-phantom]{Picard Schemes of Curves}
\item \hyperref[weil-section-phantom]{Weil Cohomology Theories}
\item \hyperref[adequate-section-phantom]{Adequate Modules}
\item \hyperref[dualizing-section-phantom]{Dualizing Complexes}
\item \hyperref[duality-section-phantom]{Duality for Schemes}
\item \hyperref[discriminant-section-phantom]{Discriminants and Differents}
\item \hyperref[derham-section-phantom]{de Rham Cohomology}
\item \hyperref[local-cohomology-section-phantom]{Local Cohomology}
\item \hyperref[algebraization-section-phantom]{Algebraic and Formal Geometry}
\item \hyperref[curves-section-phantom]{Algebraic Curves}
\item \hyperref[resolve-section-phantom]{Resolution of Surfaces}
\item \hyperref[models-section-phantom]{Semistable Reduction}
\item \hyperref[functors-section-phantom]{Functors and Morphisms}
\item \hyperref[equiv-section-phantom]{Derived Categories of Varieties}
\item \hyperref[pione-section-phantom]{Fundamental Groups of Schemes}
\item \hyperref[etale-cohomology-section-phantom]{\'Etale Cohomology}
\item \hyperref[crystalline-section-phantom]{Crystalline Cohomology}
\item \hyperref[proetale-section-phantom]{Pro-\'etale Cohomology}
\item \hyperref[relative-cycles-section-phantom]{Relative Cycles}
\item \hyperref[more-etale-section-phantom]{More \'Etale Cohomology}
\item \hyperref[trace-section-phantom]{The Trace Formula}
\end{enumerate}
Algebraic Spaces
\begin{enumerate}
\setcounter{enumi}{64}
\item \hyperref[spaces-section-phantom]{Algebraic Spaces}
\item \hyperref[spaces-properties-section-phantom]{Properties of Algebraic Spaces}
\item \hyperref[spaces-morphisms-section-phantom]{Morphisms of Algebraic Spaces}
\item \hyperref[decent-spaces-section-phantom]{Decent Algebraic Spaces}
\item \hyperref[spaces-cohomology-section-phantom]{Cohomology of Algebraic Spaces}
\item \hyperref[spaces-limits-section-phantom]{Limits of Algebraic Spaces}
\item \hyperref[spaces-divisors-section-phantom]{Divisors on Algebraic Spaces}
\item \hyperref[spaces-over-fields-section-phantom]{Algebraic Spaces over Fields}
\item \hyperref[spaces-topologies-section-phantom]{Topologies on Algebraic Spaces}
\item \hyperref[spaces-descent-section-phantom]{Descent and Algebraic Spaces}
\item \hyperref[spaces-perfect-section-phantom]{Derived Categories of Spaces}
\item \hyperref[spaces-more-morphisms-section-phantom]{More on Morphisms of Spaces}
\item \hyperref[spaces-flat-section-phantom]{Flatness on Algebraic Spaces}
\item \hyperref[spaces-groupoids-section-phantom]{Groupoids in Algebraic Spaces}
\item \hyperref[spaces-more-groupoids-section-phantom]{More on Groupoids in Spaces}
\item \hyperref[bootstrap-section-phantom]{Bootstrap}
\item \hyperref[spaces-pushouts-section-phantom]{Pushouts of Algebraic Spaces}
\end{enumerate}
Topics in Geometry
\begin{enumerate}
\setcounter{enumi}{81}
\item \hyperref[spaces-chow-section-phantom]{Chow Groups of Spaces}
\item \hyperref[groupoids-quotients-section-phantom]{Quotients of Groupoids}
\item \hyperref[spaces-more-cohomology-section-phantom]{More on Cohomology of Spaces}
\item \hyperref[spaces-simplicial-section-phantom]{Simplicial Spaces}
\item \hyperref[spaces-duality-section-phantom]{Duality for Spaces}
\item \hyperref[formal-spaces-section-phantom]{Formal Algebraic Spaces}
\item \hyperref[restricted-section-phantom]{Algebraization of Formal Spaces}
\item \hyperref[spaces-resolve-section-phantom]{Resolution of Surfaces Revisited}
\end{enumerate}
Deformation Theory
\begin{enumerate}
\setcounter{enumi}{89}
\item \hyperref[formal-defos-section-phantom]{Formal Deformation Theory}
\item \hyperref[defos-section-phantom]{Deformation Theory}
\item \hyperref[cotangent-section-phantom]{The Cotangent Complex}
\item \hyperref[examples-defos-section-phantom]{Deformation Problems}
\end{enumerate}
Algebraic Stacks
\begin{enumerate}
\setcounter{enumi}{93}
\item \hyperref[algebraic-section-phantom]{Algebraic Stacks}
\item \hyperref[examples-stacks-section-phantom]{Examples of Stacks}
\item \hyperref[stacks-sheaves-section-phantom]{Sheaves on Algebraic Stacks}
\item \hyperref[criteria-section-phantom]{Criteria for Representability}
\item \hyperref[artin-section-phantom]{Artin's Axioms}
\item \hyperref[quot-section-phantom]{Quot and Hilbert Spaces}
\item \hyperref[stacks-properties-section-phantom]{Properties of Algebraic Stacks}
\item \hyperref[stacks-morphisms-section-phantom]{Morphisms of Algebraic Stacks}
\item \hyperref[stacks-limits-section-phantom]{Limits of Algebraic Stacks}
\item \hyperref[stacks-cohomology-section-phantom]{Cohomology of Algebraic Stacks}
\item \hyperref[stacks-perfect-section-phantom]{Derived Categories of Stacks}
\item \hyperref[stacks-introduction-section-phantom]{Introducing Algebraic Stacks}
\item \hyperref[stacks-more-morphisms-section-phantom]{More on Morphisms of Stacks}
\item \hyperref[stacks-geometry-section-phantom]{The Geometry of Stacks}
\end{enumerate}
Topics in Moduli Theory
\begin{enumerate}
\setcounter{enumi}{107}
\item \hyperref[moduli-section-phantom]{Moduli Stacks}
\item \hyperref[moduli-curves-section-phantom]{Moduli of Curves}
\end{enumerate}
Miscellany
\begin{enumerate}
\setcounter{enumi}{109}
\item \hyperref[examples-section-phantom]{Examples}
\item \hyperref[exercises-section-phantom]{Exercises}
\item \hyperref[guide-section-phantom]{Guide to Literature}
\item \hyperref[desirables-section-phantom]{Desirables}
\item \hyperref[coding-section-phantom]{Coding Style}
\item \hyperref[obsolete-section-phantom]{Obsolete}
\item \hyperref[fdl-section-phantom]{GNU Free Documentation License}
\item \hyperref[index-section-phantom]{Auto Generated Index}
\end{enumerate}
\end{multicols}


\bibliography{my}
\bibliographystyle{amsalpha}

\end{document}
