\IfFileExists{stacks-project.cls}{%
\documentclass{stacks-project}
}{%
\documentclass{amsart}
}

% For dealing with references we use the comment environment
\usepackage{verbatim}
\newenvironment{reference}{\comment}{\endcomment}
%\newenvironment{reference}{}{}
\newenvironment{slogan}{\comment}{\endcomment}
\newenvironment{history}{\comment}{\endcomment}

% For commutative diagrams we use Xy-pic
\usepackage[all]{xy}

% We use 2cell for 2-commutative diagrams.
\xyoption{2cell}
\UseAllTwocells

% We use multicol for the list of chapters between chapters
\usepackage{multicol}

% This is generally recommended for better output
\usepackage{lmodern}
\usepackage[T1]{fontenc}

% For cross-file-references
\usepackage{xr-hyper}

% Package for hypertext links:
\usepackage{hyperref}

% For any local file, say "hello.tex" you want to link to please
% use \externaldocument[hello-]{hello}
\externaldocument[introduction-]{introduction}
\externaldocument[conventions-]{conventions}
\externaldocument[sets-]{sets}
\externaldocument[categories-]{categories}
\externaldocument[topology-]{topology}
\externaldocument[sheaves-]{sheaves}
\externaldocument[sites-]{sites}
\externaldocument[stacks-]{stacks}
\externaldocument[fields-]{fields}
\externaldocument[algebra-]{algebra}
\externaldocument[brauer-]{brauer}
\externaldocument[homology-]{homology}
\externaldocument[derived-]{derived}
\externaldocument[simplicial-]{simplicial}
\externaldocument[more-algebra-]{more-algebra}
\externaldocument[smoothing-]{smoothing}
\externaldocument[modules-]{modules}
\externaldocument[sites-modules-]{sites-modules}
\externaldocument[injectives-]{injectives}
\externaldocument[cohomology-]{cohomology}
\externaldocument[sites-cohomology-]{sites-cohomology}
\externaldocument[dga-]{dga}
\externaldocument[dpa-]{dpa}
\externaldocument[sdga-]{sdga}
\externaldocument[hypercovering-]{hypercovering}
\externaldocument[schemes-]{schemes}
\externaldocument[constructions-]{constructions}
\externaldocument[properties-]{properties}
\externaldocument[morphisms-]{morphisms}
\externaldocument[coherent-]{coherent}
\externaldocument[divisors-]{divisors}
\externaldocument[limits-]{limits}
\externaldocument[varieties-]{varieties}
\externaldocument[topologies-]{topologies}
\externaldocument[descent-]{descent}
\externaldocument[perfect-]{perfect}
\externaldocument[more-morphisms-]{more-morphisms}
\externaldocument[flat-]{flat}
\externaldocument[groupoids-]{groupoids}
\externaldocument[more-groupoids-]{more-groupoids}
\externaldocument[etale-]{etale}
\externaldocument[chow-]{chow}
\externaldocument[intersection-]{intersection}
\externaldocument[pic-]{pic}
\externaldocument[weil-]{weil}
\externaldocument[adequate-]{adequate}
\externaldocument[dualizing-]{dualizing}
\externaldocument[duality-]{duality}
\externaldocument[discriminant-]{discriminant}
\externaldocument[derham-]{derham}
\externaldocument[local-cohomology-]{local-cohomology}
\externaldocument[algebraization-]{algebraization}
\externaldocument[curves-]{curves}
\externaldocument[resolve-]{resolve}
\externaldocument[models-]{models}
\externaldocument[functors-]{functors}
\externaldocument[equiv-]{equiv}
\externaldocument[pione-]{pione}
\externaldocument[etale-cohomology-]{etale-cohomology}
\externaldocument[proetale-]{proetale}
\externaldocument[relative-cycles-]{relative-cycles}
\externaldocument[more-etale-]{more-etale}
\externaldocument[trace-]{trace}
\externaldocument[crystalline-]{crystalline}
\externaldocument[spaces-]{spaces}
\externaldocument[spaces-properties-]{spaces-properties}
\externaldocument[spaces-morphisms-]{spaces-morphisms}
\externaldocument[decent-spaces-]{decent-spaces}
\externaldocument[spaces-cohomology-]{spaces-cohomology}
\externaldocument[spaces-limits-]{spaces-limits}
\externaldocument[spaces-divisors-]{spaces-divisors}
\externaldocument[spaces-over-fields-]{spaces-over-fields}
\externaldocument[spaces-topologies-]{spaces-topologies}
\externaldocument[spaces-descent-]{spaces-descent}
\externaldocument[spaces-perfect-]{spaces-perfect}
\externaldocument[spaces-more-morphisms-]{spaces-more-morphisms}
\externaldocument[spaces-flat-]{spaces-flat}
\externaldocument[spaces-groupoids-]{spaces-groupoids}
\externaldocument[spaces-more-groupoids-]{spaces-more-groupoids}
\externaldocument[bootstrap-]{bootstrap}
\externaldocument[spaces-pushouts-]{spaces-pushouts}
\externaldocument[spaces-chow-]{spaces-chow}
\externaldocument[groupoids-quotients-]{groupoids-quotients}
\externaldocument[spaces-more-cohomology-]{spaces-more-cohomology}
\externaldocument[spaces-simplicial-]{spaces-simplicial}
\externaldocument[spaces-duality-]{spaces-duality}
\externaldocument[formal-spaces-]{formal-spaces}
\externaldocument[restricted-]{restricted}
\externaldocument[spaces-resolve-]{spaces-resolve}
\externaldocument[formal-defos-]{formal-defos}
\externaldocument[defos-]{defos}
\externaldocument[cotangent-]{cotangent}
\externaldocument[examples-defos-]{examples-defos}
\externaldocument[algebraic-]{algebraic}
\externaldocument[examples-stacks-]{examples-stacks}
\externaldocument[stacks-sheaves-]{stacks-sheaves}
\externaldocument[criteria-]{criteria}
\externaldocument[artin-]{artin}
\externaldocument[quot-]{quot}
\externaldocument[stacks-properties-]{stacks-properties}
\externaldocument[stacks-morphisms-]{stacks-morphisms}
\externaldocument[stacks-limits-]{stacks-limits}
\externaldocument[stacks-cohomology-]{stacks-cohomology}
\externaldocument[stacks-perfect-]{stacks-perfect}
\externaldocument[stacks-introduction-]{stacks-introduction}
\externaldocument[stacks-more-morphisms-]{stacks-more-morphisms}
\externaldocument[stacks-geometry-]{stacks-geometry}
\externaldocument[moduli-]{moduli}
\externaldocument[moduli-curves-]{moduli-curves}
\externaldocument[examples-]{examples}
\externaldocument[exercises-]{exercises}
\externaldocument[guide-]{guide}
\externaldocument[desirables-]{desirables}
\externaldocument[coding-]{coding}
\externaldocument[obsolete-]{obsolete}
\externaldocument[fdl-]{fdl}
\externaldocument[index-]{index}

% Theorem environments.
%
\theoremstyle{plain}
\newtheorem{theorem}[subsection]{Theorem}
\newtheorem{proposition}[subsection]{Proposition}
\newtheorem{lemma}[subsection]{Lemma}

\theoremstyle{definition}
\newtheorem{definition}[subsection]{Definition}
\newtheorem{example}[subsection]{Example}
\newtheorem{exercise}[subsection]{Exercise}
\newtheorem{situation}[subsection]{Situation}

\theoremstyle{remark}
\newtheorem{remark}[subsection]{Remark}
\newtheorem{remarks}[subsection]{Remarks}

\numberwithin{equation}{subsection}

% Macros
%
\def\lim{\mathop{\mathrm{lim}}\nolimits}
\def\colim{\mathop{\mathrm{colim}}\nolimits}
\def\Spec{\mathop{\mathrm{Spec}}}
\def\Hom{\mathop{\mathrm{Hom}}\nolimits}
\def\Ext{\mathop{\mathrm{Ext}}\nolimits}
\def\SheafHom{\mathop{\mathcal{H}\!\mathit{om}}\nolimits}
\def\SheafExt{\mathop{\mathcal{E}\!\mathit{xt}}\nolimits}
\def\Sch{\mathit{Sch}}
\def\Mor{\mathop{\mathrm{Mor}}\nolimits}
\def\Ob{\mathop{\mathrm{Ob}}\nolimits}
\def\Sh{\mathop{\mathit{Sh}}\nolimits}
\def\NL{\mathop{N\!L}\nolimits}
\def\CH{\mathop{\mathrm{CH}}\nolimits}
\def\proetale{{pro\text{-}\acute{e}tale}}
\def\etale{{\acute{e}tale}}
\def\QCoh{\mathit{QCoh}}
\def\Ker{\mathop{\mathrm{Ker}}}
\def\Im{\mathop{\mathrm{Im}}}
\def\Coker{\mathop{\mathrm{Coker}}}
\def\Coim{\mathop{\mathrm{Coim}}}

% Boxtimes
%
\DeclareMathSymbol{\boxtimes}{\mathbin}{AMSa}{"02}

%
% Macros for moduli stacks/spaces
%
\def\QCohstack{\mathcal{QC}\!\mathit{oh}}
\def\Cohstack{\mathcal{C}\!\mathit{oh}}
\def\Spacesstack{\mathcal{S}\!\mathit{paces}}
\def\Quotfunctor{\mathrm{Quot}}
\def\Hilbfunctor{\mathrm{Hilb}}
\def\Curvesstack{\mathcal{C}\!\mathit{urves}}
\def\Polarizedstack{\mathcal{P}\!\mathit{olarized}}
\def\Complexesstack{\mathcal{C}\!\mathit{omplexes}}
% \Pic is the operator that assigns to X its picard group, usage \Pic(X)
% \Picardstack_{X/B} denotes the Picard stack of X over B
% \Picardfunctor_{X/B} denotes the Picard functor of X over B
\def\Pic{\mathop{\mathrm{Pic}}\nolimits}
\def\Picardstack{\mathcal{P}\!\mathit{ic}}
\def\Picardfunctor{\mathrm{Pic}}
\def\Deformationcategory{\mathcal{D}\!\mathit{ef}}


% OK, start here.
%
\begin{document}

\title{Moduli Stacks}

\maketitle

\phantomsection
\label{section-phantom}

\tableofcontents




\section{Introduction}
\label{section-introduction}

\noindent
In this chapter we verify basic properties of moduli spaces
and moduli stacks such as
$\mathit{Hom}$, $\mathit{Isom}$, $\Cohstack_{X/B}$,
$\Quotfunctor_{\mathcal{F}/X/B}$, $\Hilbfunctor_{X/B}$,
$\Picardstack_{X/B}$, $\Picardfunctor_{X/B}$, $\mathit{Mor}_B(Z, X)$,
$\Spacesstack'_{fp, flat, proper}$, $\Polarizedstack$, and
$\Complexesstack_{X/B}$.
We have already shown these algebraic spaces or algebraic stacks
under suitable hypotheses, see Quot, Sections
\ref{quot-section-hom},
\ref{quot-section-isom},
\ref{quot-section-stack-coherent-sheaves},
\ref{quot-section-not-flat},
\ref{quot-section-quot},
\ref{quot-section-hilb},
\ref{quot-section-picard-stack},
\ref{quot-section-picard-functor},
\ref{quot-section-relative-morphisms},
\ref{quot-section-stack-of-spaces},
\ref{quot-section-polarized}, and
\ref{quot-section-moduli-complexes}.
The stack of curves, denoted $\textit{Curves}$ and introduced
in Quot, Section \ref{quot-section-curves}, is discussed in the
chapter on moduli of curves, see
Moduli of Curves, Section \ref{moduli-curves-section-stack-curves}.

\medskip\noindent
In some sense this chapter is following the footsteps of
Grothendieck's lectures \cite{Gr-I},
\cite{Gr-II},
\cite{Gr-III},
\cite{Gr-IV},
\cite{Gr-V}, and
\cite{Gr-VI}.







\section{Conventions and abuse of language}
\label{section-conventions}

\noindent
We continue to use the conventions and the abuse of language
introduced in
Properties of Stacks, Section \ref{stacks-properties-section-conventions}.
Unless otherwise mentioned our base scheme will be $\Spec(\mathbf{Z})$.






\section{Properties of Hom and Isom}
\label{section-hom-isom}

\noindent
Let $f : X \to B$ be a morphism of algebraic spaces which is
of finite presentation. Assume $\mathcal{F}$ and $\mathcal{G}$
are quasi-coherent $\mathcal{O}_X$-modules.
If $\mathcal{G}$ is of finite presentation, flat over $B$
with support proper over $B$, then the functor
$\mathit{Hom}(\mathcal{F}, \mathcal{G})$ defined by
$$
T/B \longmapsto \Hom_{\mathcal{O}_{X_T}}(\mathcal{F}_T, \mathcal{G}_T)
$$
is an algebraic space affine over $B$. If $\mathcal{F}$ is of
finite presentation, then
$\mathit{Hom}(\mathcal{F}, \mathcal{G}) \to B$
is of finite presentation. See
Quot, Proposition \ref{quot-proposition-hom}.

\medskip\noindent
If both $\mathcal{F}$ and $\mathcal{G}$ are of finite presentation,
flat over $B$ with support proper over $B$, then the subfunctor
$$
\mathit{Isom}(\mathcal{F}, \mathcal{G}) \subset
\mathit{Hom}(\mathcal{F}, \mathcal{G})
$$
is an algebraic space affine of finite presentation over $B$.
See Quot, Proposition \ref{quot-proposition-isom}.




\section{Properties of the stack of coherent sheaves}
\label{section-stack-coherent-sheaves}

\noindent
Let $f : X \to B$ be a morphism of algebraic spaces which is
separated and of finite presentation. Then the stack
$\Cohstack_{X/B}$ parametrizing flat families of coherent
modules with proper support is algebraic. See
Quot, Theorem \ref{quot-theorem-coherent-algebraic-general}.

\begin{lemma}
\label{lemma-coherent-diagonal-affine-fp}
The diagonal of $\Cohstack_{X/B}$ over $B$ is affine
and of finite presentation.
\end{lemma}

\begin{proof}
The representability of the diagonal by algebraic spaces
was shown in Quot, Lemma \ref{quot-lemma-coherent-diagonal}.
From the proof we find that we have to show
$\mathit{Isom}(\mathcal{F}, \mathcal{G}) \to T$
is affine and of finite presentation for a pair of
finitely presented $\mathcal{O}_{X_T}$-modules
$\mathcal{F}$, $\mathcal{G}$ flat over $T$ with support
proper over $T$. This was discussed in Section \ref{section-hom-isom}.
\end{proof}

\begin{lemma}
\label{lemma-coherent-qs-lfp}
The morphism $\Cohstack_{X/B} \to B$ is quasi-separated and
locally of finite presentation.
\end{lemma}

\begin{proof}
To check $\Cohstack_{X/B} \to B$ is quasi-separated we have to
show that its diagonal is quasi-compact and quasi-separated.
This is immediate from Lemma \ref{lemma-coherent-diagonal-affine-fp}.
To prove that $\Cohstack_{X/B} \to B$ is locally of finite
presentation, we have to show that $\Cohstack_{X/B} \to B$
is limit preserving, see
Limits of Stacks, Proposition
\ref{stacks-limits-proposition-characterize-locally-finite-presentation}.
This follows from Quot, Lemma \ref{quot-lemma-coherent-limits}
(small detail omitted).
\end{proof}

\begin{lemma}
\label{lemma-coherent-existence-part}
Assume $X \to B$ is proper as well as of finite presentation.
Then $\Cohstack_{X/B} \to B$ satisfies the existence part
of the valuative criterion (Morphisms of Stacks, Definition
\ref{stacks-morphisms-definition-existence}).
\end{lemma}

\begin{proof}
Taking base change, this immediately reduces to the following
problem: given a valuation ring $R$ with fraction field $K$ and
an algebraic space $X$ proper over $R$ and a coherent
$\mathcal{O}_{X_K}$-module $\mathcal{F}_K$, show there exists
a finitely presented $\mathcal{O}_X$-module $\mathcal{F}$
flat over $R$ whose generic fibre is $\mathcal{F}_K$.
Observe that by Flatness on Spaces, Theorem
\ref{spaces-flat-theorem-finite-type-flat}
any finite type quasi-coherent $\mathcal{O}_X$-module
$\mathcal{F}$ flat over $R$ is of finite presentation.
Denote $j : X_K \to X$ the embedding of the generic fibre.
As a base change of the affine morphism $\Spec(K) \to \Spec(R)$
the morphism $j$ is affine. Thus $j_*\mathcal{F}_K$ is
quasi-coherent. Write
$$
j_*\mathcal{F}_K = \colim \mathcal{F}_i
$$
as a filtered colimit of its finite type quasi-coherent
$\mathcal{O}_X$-submodules, see
Limits of Spaces, Lemma \ref{spaces-limits-lemma-directed-colimit-finite-type}.
Since $j_*\mathcal{F}_K$ is a sheaf of $K$-vector spaces over $X$,
it is flat over $\Spec(R)$. Thus each $\mathcal{F}_i$ is flat
over $R$ as flatness over a valuation ring is the same as being
torsion free
(More on Algebra, Lemma
\ref{more-algebra-lemma-valuation-ring-torsion-free-flat})
and torsion freeness is inherited by submodules.
Finally, we have to show that the map
$j^*\mathcal{F}_i \to \mathcal{F}_K$
is an isomorphism for some $i$.
Since $j^*j_*\mathcal{F}_K = \mathcal{F}_K$ (small detail omitted)
and since $j^*$ is exact, we see that $j^*\mathcal{F}_i \to \mathcal{F}_K$
is injective for all $i$.
Since $j^*$ commutes with colimits, we have
$\mathcal{F}_K = j^*j_*\mathcal{F}_K = \colim j^*\mathcal{F}_i$.
Since $\mathcal{F}_K$ is coherent (i.e., finitely presented),
there is an $i$ such that $j^*\mathcal{F}_i$ contains all the
(finitely many) generators over an affine \'etale cover of $X$.
Thus we get surjectivity of $j^*\mathcal{F}_i \to \mathcal{F}_K$
for $i$ large enough.
\end{proof}

\begin{lemma}
\label{lemma-coherent-functorial}
Let $B$ be an algebraic space. Let $\pi : X \to Y$ be a quasi-finite
morphism of algebraic spaces which are separated and of finite presentation
over $B$. Then $\pi_*$ induces a morphism
$\Cohstack_{X/B} \to \Cohstack_{Y/B}$.
\end{lemma}

\begin{proof}
Let $(T \to B, \mathcal{F})$ be an object of $\Cohstack_{X/B}$.
We claim
\begin{enumerate}
\item[(a)] $(T \to B, \pi_{T, *}\mathcal{F})$ is an object
of $\Cohstack_{Y/B}$ and
\item[(b)] for $T' \to T$ we have
$\pi_{T', *}(X_{T'} \to X_T)^*\mathcal{F} =
(Y_{T'} \to Y_T)^*\pi_{T, *}\mathcal{F}$.
\end{enumerate}
Part (b) guarantees that this construction defines a functor
$\Cohstack_{X/B} \to \Cohstack_{Y/B}$ as desired.

\medskip\noindent
Let $i : Z \to X_T$ be the closed subspace cut out by the zeroth
fitting ideal of $\mathcal{F}$
(Divisors on Spaces, Section
\ref{spaces-divisors-section-fitting-ideals}).
Then $Z \to B$ is proper by assumption (see
Derived Categories of Spaces, Section
\ref{spaces-perfect-section-proper-over-base}).
On the other hand $i$ is of finite presentation
(Divisors on Spaces, Lemma
\ref{spaces-divisors-lemma-fitting-ideal-of-finitely-presented} and
Morphisms of Spaces, Lemma
\ref{spaces-morphisms-lemma-closed-immersion-finite-presentation}).
There exists a quasi-coherent $\mathcal{O}_Z$-module
$\mathcal{G}$ of finite type with $i_*\mathcal{G} = \mathcal{F}$
(Divisors on Spaces, Lemma
\ref{spaces-divisors-lemma-on-subscheme-cut-out-by-Fit-0}).
In fact $\mathcal{G}$ is of finite presentation as an $\mathcal{O}_Z$-module
by Descent on Spaces, Lemma
\ref{spaces-descent-lemma-finite-finitely-presented-module}.
Observe that $\mathcal{G}$ is flat over $B$, for example
because the stalks of $\mathcal{G}$ and $\mathcal{F}$ agree
(Morphisms of Spaces, Lemma \ref{spaces-morphisms-lemma-stalk-push-closed}).
Observe that $\pi_T \circ i : Z \to Y_T$ is quasi-finite as a composition
of quasi-finite morphisms and that
$\pi_{T, *}\mathcal{F} = (\pi_T \circ i)_*\mathcal{G})$.
Since $i$ is affine, formation of $i_*$ commutes with base change
(Cohomology of Spaces, Lemma \ref{spaces-cohomology-lemma-affine-base-change}).
Therefore we may replace $B$ by $T$, $X$ by $Z$,
$\mathcal{F}$ by $\mathcal{G}$, and $Y$ by $Y_T$
to reduce to the case discussed in the next paragraph.

\medskip\noindent
Assume that $X \to B$ is proper. Then $\pi$ is proper
by Morphisms of Spaces, Lemma
\ref{spaces-morphisms-lemma-universally-closed-permanence}
and hence finite by
More on Morphisms of Spaces,
Lemma \ref{spaces-more-morphisms-lemma-characterize-finite}.
Since a finite morphism is affine we see that (b) holds by
Cohomology of Spaces, Lemma \ref{spaces-cohomology-lemma-affine-base-change}.
On the other hand, $\pi$ is of finite presentation by
Morphisms of Spaces, Lemma
\ref{spaces-morphisms-lemma-finite-presentation-permanence}.
Thus $\pi_{T, *}\mathcal{F}$ is of finite presentation by
Descent on Spaces, Lemma
\ref{spaces-descent-lemma-finite-finitely-presented-module}.
Finally, $\pi_{T, *}\mathcal{F} $ is flat over $B$ for example
by looking at stalks using
Cohomology of Spaces, Lemma \ref{spaces-cohomology-lemma-stalk-push-finite}.
\end{proof}

\begin{lemma}
\label{lemma-coherent-open}
Let $B$ be an algebraic space. Let $\pi : X \to Y$ be an open immersion
of algebraic spaces which are separated and of finite presentation over $B$.
Then the morphism $\Cohstack_{X/B} \to \Cohstack_{Y/B}$ of
Lemma \ref{lemma-coherent-functorial} is an open immersion.
\end{lemma}

\begin{proof}
Omitted. Hint: If $\mathcal{F}$ is an object of $\Cohstack_{Y/B}$ over $T$
and for $t \in T$ we have $\text{Supp}(\mathcal{F}_t) \subset |X_t|$,
then the same is true for $t' \in T$ in a neighbourhood of $t$.
\end{proof}

\begin{lemma}
\label{lemma-coherent-closed}
Let $B$ be an algebraic space. Let $\pi : X \to Y$ be a closed immersion
of algebraic spaces which are separated and of finite presentation over $B$.
Then the morphism $\Cohstack_{X/B} \to \Cohstack_{Y/B}$ of
Lemma \ref{lemma-coherent-functorial} is a closed immersion.
\end{lemma}

\begin{proof}
Let $\mathcal{I} \subset \mathcal{O}_Y$ be the sheaf of ideals cutting
out $X$ as a closed subspace of $Y$. Recall that $\pi_*$ induces
an equivalence between the category of quasi-coherent $\mathcal{O}_X$-modules
and the category of quasi-coherent $\mathcal{O}_Y$-modules annihilated
by $\mathcal{I}$, see Morphisms of Spaces, Lemma
\ref{spaces-morphisms-lemma-i-star-equivalence}.
The same, mutatis mutandis, is true after base by $T \to B$ with
$\mathcal{I}$ replaced by the ideal sheaf
$\mathcal{I}_T = \Im((Y_T \to Y)^*\mathcal{I} \to \mathcal{O}_{Y_T})$.
Analyzing the proof of Lemma \ref{lemma-coherent-functorial}
we find that the essential image of
$\Cohstack_{X/B} \to \Cohstack_{Y/B}$
is exactly the objects $\xi = (T \to B, \mathcal{F})$
where $\mathcal{F}$ is annihilated by $\mathcal{I}_T$.
In other words, $\xi$ is in the essential image if and only if
the multiplication map
$$
\mathcal{F} \otimes_{\mathcal{O}_{Y_T}} (Y_T \to Y)^*\mathcal{I}
\longrightarrow
\mathcal{F}
$$
is zero and similarly after any further base change $T' \to T$.
Note that
$$
(Y_{T'} \to Y_T)^*(
\mathcal{F} \otimes_{\mathcal{O}_{Y_T}} (Y_T \to Y)^*\mathcal{I}) =
(Y_{T'} \to Y_T)^*\mathcal{F} \otimes_{\mathcal{O}_{Y_{T'}}}
(Y_{T'} \to Y)^*\mathcal{I})
$$
Hence the vanishing of the multiplication map on $T'$
is representable by a closed subspace of $T$ by
Flatness on Spaces, Lemma \ref{spaces-flat-lemma-F-zero-closed-proper}.
\end{proof}

\begin{situation}[Numerical invariants]
\label{situation-numerical}
Let $f : X \to B$ be as in the introduction to this section. Let $I$
be a set and for $i \in I$ let $E_i \in D(\mathcal{O}_X)$ be perfect.
Given an object $(T \to B, \mathcal{F})$ of $\Cohstack_{X/B}$
denote $E_{i, T}$ the derived pullback of $E_i$ to $X_T$.
The object
$$
K_i = Rf_{T, *}(E_{i, T} \otimes_{\mathcal{O}_{X_T}}^\mathbf{L} \mathcal{F})
$$
of $D(\mathcal{O}_T)$ is perfect and its formation commutes with base change,
see Derived Categories of Spaces, Lemma
\ref{spaces-perfect-lemma-base-change-tensor-perfect}.
Thus the function
$$
\chi_i : |T| \longrightarrow \mathbf{Z},\quad
\chi_i(t) =
\chi(X_t, E_{i, t} \otimes_{\mathcal{O}_{X_t}}^\mathbf{L} \mathcal{F}_t) =
\chi(K_i \otimes_{\mathcal{O}_T}^\mathbf{L} \kappa(t))
$$
is locally constant by Derived Categories of Spaces, Lemma
\ref{spaces-perfect-lemma-chi-locally-constant}.
Let $P : I \to \mathbf{Z}$ be a map. Consider the substack
$$
\Cohstack^P_{X/B} \subset \Cohstack_{X/B}
$$
consisting of flat families of coherent sheaves with proper support
whose numerical invariants agree with $P$. More precisely, an object
$(T \to B, \mathcal{F})$ of $\Cohstack_{X/B}$ is in
$\Cohstack^P_{X/B}$ if and only if $\chi_i(t) = P(i)$ for all $i \in I$
and $t \in T$.
\end{situation}

\begin{lemma}
\label{lemma-open-P}
In Situation \ref{situation-numerical} the stack
$\Cohstack^P_{X/B}$ is algebraic and
$$
\Cohstack^P_{X/B} \longrightarrow \Cohstack_{X/B}
$$
is a flat closed immersion. If $I$ is finite or $B$ is locally
Noetherian, then $\Cohstack^P_{X/B}$ is an open and closed substack of
$\Cohstack_{X/B}$.
\end{lemma}

\begin{proof}
This is immediately clear if $I$ is finite, because the functions
$t \mapsto \chi_i(t)$ are locally constant. If $I$ is infinite, then
we write
$$
I = \bigcup\nolimits_{I' \subset I\text{ finite}} I'
$$
and we denote $P' = P|_{I'}$. Then we have
$$
\Cohstack^P_{X/B} = \bigcap\nolimits_{I' \subset I\text{ finite}}
\Cohstack^{P'}_{X/B}
$$
Therefore, $\Cohstack^P_{X/B}$ is always an algebraic stack and the morphism
$\Cohstack^P_{X/B} \subset \Cohstack_{X/B}$ is always a flat closed immersion,
but it may no longer be an open substack. (We leave it to the reader to
make examples). However, if $B$ is locally Noetherian, then so
is $\Cohstack_{X/B}$ by Lemma \ref{lemma-coherent-qs-lfp} and
Morphisms of Stacks, Lemma
\ref{stacks-morphisms-lemma-locally-finite-type-locally-noetherian}.
Hence if $U \to \Cohstack_{X/B}$ is a smooth surjective morphism
where $U$ is a locally Noetherian scheme, then the inverse images of
the open and closed substacks $\Cohstack^{P'}_{X/B}$
have an open intersection in $U$ (because connected components of
locally Noetherian topological spaces are open).
Thus the result in this case.
\end{proof}

\begin{lemma}
\label{lemma-finite-list-perfect-objects}
Let $f : X \to B$ be as in the introduction to this section.
Let $E_1, \ldots, E_r \in D(\mathcal{O}_X)$ be perfect.
Let $I = \mathbf{Z}^{\oplus r}$ and consider the map
$$
I \longrightarrow D(\mathcal{O}_X),\quad
(n_1, \ldots, n_r) \longmapsto
E_1^{\otimes n_1}
\otimes \ldots \otimes
E_r^{\otimes n_r}
$$
Let $P : I \to \mathbf{Z}$ be a map. Then
$\Cohstack^P_{X/B} \subset \Cohstack_{X/B}$
as defined in Situation \ref{situation-numerical}
is an open and closed substack.
\end{lemma}

\begin{proof}
We may work \'etale locally on $B$, hence we may assume that $B$ is affine.
In this case we may perform absolute Noetherian reduction; we suggest
the reader skip the proof. Namely, say $B = \Spec(\Lambda)$.
Write $\Lambda = \colim \Lambda_i$ as a filtered colimit with each $\Lambda_i$
of finite type over $\mathbf{Z}$. For some $i$ we can find
a morphism of algebraic spaces $X_i \to \Spec(\Lambda_i)$
which is separated and of finite presentation and whose base change
to $\Lambda$ is $X$. See Limits of Spaces, Lemmas
\ref{spaces-limits-lemma-descend-finite-presentation} and
\ref{spaces-limits-lemma-descend-separated-morphism}.
Then after increasing $i$ we may assume there exist
perfect objects $E_{1, i}, \ldots, E_{r, i}$
in $D(\mathcal{O}_{X_i})$ whose derived pullback to $X$
are isomorphic to $E_1, \ldots, E_r$, see
Derived Categories of Spaces, Lemma
\ref{spaces-perfect-lemma-perfect-on-limit}.
Clearly we have a cartesian square
$$
\xymatrix{
\Cohstack^P_{X/B} \ar[r] \ar[d] &
\Cohstack_{X/B} \ar[d] \\
\Cohstack^P_{X_i/\Spec(\Lambda_i)} \ar[r] &
\Cohstack_{X_i/\Spec(\Lambda_i)}
}
$$
and hence we may appeal to Lemma \ref{lemma-open-P}
to finish the proof.
\end{proof}

\begin{example}[Coherent sheaves with fixed Hilbert polynomial]
\label{example-hilbert-polynomial}
Let $f : X \to B$ be as in the introduction to this section.
Let $\mathcal{L}$ be an invertible $\mathcal{O}_X$-module.
Let $P : \mathbf{Z} \to \mathbf{Z}$ be a numerical polynomial.
Then we can consider the open and closed algebraic substack
$$
\Cohstack^P_{X/B} =
\Cohstack^{P, \mathcal{L}}_{X/B}
\subset \Cohstack_{X/B}
$$
consisting of flat families of coherent sheaves with proper support
whose numerical invariants agree with $P$: an object
$(T \to B, \mathcal{F})$ of $\Cohstack_{X/B}$ lies in
$\Cohstack^P_{X/B}$ if and only if
$$
P(n) =
\chi(X_t, \mathcal{F}_t \otimes_{\mathcal{O}_{X_t}} \mathcal{L}_t^{\otimes n})
$$
for all $n \in \mathbf{Z}$ and $t \in T$. Of course this is a
special case of Situation \ref{situation-numerical}
where $I = \mathbf{Z} \to D(\mathcal{O}_X)$ is given by
$n \mapsto \mathcal{L}^{\otimes n}$. It follows from
Lemma \ref{lemma-finite-list-perfect-objects}
that this is an open and closed substack. Since the functions
$n \mapsto
\chi(X_t, \mathcal{F}_t \otimes_{\mathcal{O}_{X_t}} \mathcal{L}_t^{\otimes n})$
are always numerical polynomials (Spaces over Fields, Lemma
\ref{spaces-over-fields-lemma-numerical-polynomial-from-euler})
we conclude that
$$
\Cohstack_{X/B} = \coprod\nolimits_{P\text{ numerical polynomial}}
\Cohstack^P_{X/B}
$$
is a disjoint union decomposition.
\end{example}







\section{Properties of Quot}
\label{section-quot}

\noindent
Let $f : X \to B$ be a morphism of algebraic spaces which is
separated and of finite presentation. Let $\mathcal{F}$ be a
quasi-coherent $\mathcal{O}_X$-module. Then
$\Quotfunctor_{\mathcal{F}/X/B}$ is an algebraic space.
If $\mathcal{F}$ is of finite presentation, then
$\Quotfunctor_{\mathcal{F}/X/B} \to B$ is locally of finite
presentation. See Quot, Proposition \ref{quot-proposition-quot}.

\begin{lemma}
\label{lemma-quot-diagonal-closed}
The diagonal of $\Quotfunctor_{\mathcal{F}/X/B} \to B$ is a closed immersion.
If $\mathcal{F}$ is of finite type, then the diagonal is a closed
immersion of finite presentation.
\end{lemma}

\begin{proof}
Suppose we have a scheme $T/B$ and two quotients
$\mathcal{F}_T \to \mathcal{Q}_i$, $i = 1, 2$ corresponding
to $T$-valued points of $\Quotfunctor_{\mathcal{F}/X/B}$ over $B$.
Denote $\mathcal{K}_1$ the kernel of the first one and set
$u : \mathcal{K}_1 \to \mathcal{Q}_2$ the composition.
By Flatness on Spaces, Lemma \ref{spaces-flat-lemma-F-zero-closed-proper}
there is a closed subspace of $T$ such that $T' \to T$
factors through it if and only if the pullback $u_{T'}$ is zero.
This proves the diagonal is a closed immersion.
Moreover, if $\mathcal{F}$ is of finite type, then
$\mathcal{K}_1$ is of finite type
(Modules on Sites, Lemma
\ref{sites-modules-lemma-kernel-surjection-finite-onto-finite-presentation})
and we see that the diagonal is of finite presentation by
the same lemma.
\end{proof}

\begin{lemma}
\label{lemma-quot-s-lfp}
The morphism $\Quotfunctor_{\mathcal{F}/X/B} \to B$ is separated.
If $\mathcal{F}$ is of finite presentation, then it is also
locally of finite presentation.
\end{lemma}

\begin{proof}
To check $\Quotfunctor_{\mathcal{F}/X/B} \to B$ is separated we have to
show that its diagonal is a closed immersion. This
is true by Lemma \ref{lemma-quot-diagonal-closed}.
The second statement is part of
Quot, Proposition \ref{quot-proposition-quot}.
\end{proof}

\begin{lemma}
\label{lemma-quot-existence-part}
Assume $X \to B$ is proper as well as of finite presentation
and $\mathcal{F}$ quasi-coherent of finite type.
Then $\Quotfunctor_{\mathcal{F}/X/B} \to B$ satisfies the existence part
of the valuative criterion (Morphisms of Spaces, Definition
\ref{spaces-morphisms-definition-valuative-criterion}).
\end{lemma}

\begin{proof}
Taking base change, this immediately reduces to the following
problem: given a valuation ring $R$ with fraction field $K$,
an algebraic space $X$ proper over $R$, a finite type quasi-coherent
$\mathcal{O}_X$-module $\mathcal{F}$, and a coherent
quotient $\mathcal{F}_K \to \mathcal{Q}_K$, show there exists
a quotient $\mathcal{F} \to \mathcal{Q}$ where $\mathcal{Q}$ is a
finitely presented $\mathcal{O}_X$-module
flat over $R$ whose generic fibre is $\mathcal{Q}_K$.
Observe that by Flatness on Spaces, Theorem
\ref{spaces-flat-theorem-finite-type-flat}
any finite type quasi-coherent $\mathcal{O}_X$-module
$\mathcal{F}$ flat over $R$ is of finite presentation.
We first solve the existence of $\mathcal{Q}$ affine locally.

\medskip\noindent
Affine locally we arrive at the following problem:
let $R \to A$ be a finitely presented ring map,
let $M$ be a finite $A$-module, let $\varphi : M_K \to N_K$ be
an $A_K$-quotient module. Then we may consider
$$
L = \{x \in M \mid \varphi(x \otimes 1) = 0 \}
$$
The $M \to M/L$ is an $A$-module quotient which is
torsion free as an $R$-module. Hence it is flat as an
$R$-module (More on Algebra, Lemma
\ref{more-algebra-lemma-valuation-ring-torsion-free-flat}).
Since $M$ is finite as an $A$-module so is $L$ and we
conclude that $L$ is of finite presentation as an $A$-module
(by the reference above). Clearly $M/L$ is the unique such
quotient with $(M/L)_K = N_K$.

\medskip\noindent
The uniqueness in the construction of the previous paragraph
guarantees these quotients glue and give the desired $\mathcal{Q}$.
Here is a bit more detail. Choose a surjective \'etale morphism
$U \to X$ where $U$ is an affine scheme. Use the above construction
to construct a quotient $\mathcal{F}|_U \to \mathcal{Q}_U$
which is quasi-coherent, is flat over $R$, and recovers $\mathcal{Q}_K|U$
on the generic fibre. Since $X$ is separated, we see that
$U \times_X U$ is an affine scheme \'etale over $X$ as well.
Then $\mathcal{F}|_{U \times_X U} \to \text{pr}_1^*\mathcal{Q}_U$ and
$\mathcal{F}|_{U \times_X U} \to \text{pr}_2^*\mathcal{Q}_U$
agree as quotients by the uniquess in the construction. Hence we may descend
$\mathcal{F}|_U \to \mathcal{Q}_U$ to a surjection
$\mathcal{F} \to \mathcal{Q}$ as desired (Properties of Spaces,
Proposition \ref{spaces-properties-proposition-quasi-coherent}).
\end{proof}

\begin{lemma}
\label{lemma-quot-functorial}
Let $B$ be an algebraic space. Let $\pi : X \to Y$ be an affine quasi-finite
morphism of algebraic spaces which are separated and of finite presentation
over $B$. Let $\mathcal{F}$ be a quasi-coherent $\mathcal{O}_X$-module.
Then $\pi_*$ induces a morphism
$\Quotfunctor_{\mathcal{F}/X/B} \to \Quotfunctor_{\pi_*\mathcal{F}/Y/B}$.
\end{lemma}

\begin{proof}
Set $\mathcal{G} = \pi_*\mathcal{F}$. Since $\pi$ is affine we see that for
any scheme $T$ over $B$ we have $\mathcal{G}_T = \pi_{T, *}\mathcal{F}_T$ by
Cohomology of Spaces, Lemma \ref{spaces-cohomology-lemma-affine-base-change}.
Moreover $\pi_T$ is affine, hence $\pi_{T, *}$ is exact and transforms
quotients into quotients. Observe that a quasi-coherent quotient
$\mathcal{F}_T \to \mathcal{Q}$ defines a point of $\Quotfunctor_{X/B}$
if and only if $\mathcal{Q}$ defines an object of $\Cohstack_{X/B}$
over $T$ (similarly for $\mathcal{G}$ and $Y$). Since we've seen in
Lemma \ref{lemma-coherent-functorial}
that $\pi_*$ induces a morphism $\Cohstack_{X/B} \to \Cohstack_{Y/B}$
we see that if $\mathcal{F}_T \to \mathcal{Q}$ is in
$\Quotfunctor_{\mathcal{F}/X/B}(T)$, then
$\mathcal{G}_T \to \pi_{T, *}\mathcal{Q}$ is
in $\Quotfunctor_{\mathcal{G}/Y/B}(T)$.
\end{proof}

\begin{lemma}
\label{lemma-quot-open}
Let $B$ be an algebraic space. Let $\pi : X \to Y$ be an affine open immersion
of algebraic spaces which are separated and of finite presentation over $B$.
Let $\mathcal{F}$ be a quasi-coherent $\mathcal{O}_X$-module. Then the morphism
$\Quotfunctor_{\mathcal{F}/X/B} \to \Quotfunctor_{\pi_*\mathcal{F}/Y/B}$ of
Lemma \ref{lemma-quot-functorial} is an open immersion.
\end{lemma}

\begin{proof}
Omitted. Hint: If $(\pi_*\mathcal{F})_T \to \mathcal{Q}$ is an element of
$\Quotfunctor_{\pi_*\mathcal{F}/Y/B}(T)$
and for $t \in T$ we have $\text{Supp}(\mathcal{Q}_t) \subset |X_t|$,
then the same is true for $t' \in T$ in a neighbourhood of $t$.
\end{proof}

\begin{lemma}
\label{lemma-quot-better-open}
Let $B$ be an algebraic space. Let $j : X \to Y$ be an open immersion
of algebraic spaces which are separated and of finite presentation over $B$.
Let $\mathcal{G}$ be a quasi-coherent $\mathcal{O}_Y$-module and set
$\mathcal{F} = j^*\mathcal{G}$. Then there is an open immersion
$$
\Quotfunctor_{\mathcal{F}/X/B}
\longrightarrow
\Quotfunctor_{\mathcal{G}/Y/B}
$$
of algebraic spaces over $B$.
\end{lemma}

\begin{proof}
If $\mathcal{F}_T \to \mathcal{Q}$ is an element of
$\Quotfunctor_{\mathcal{F}/X/B}(T)$ then we can consider
$\mathcal{G}_T \to j_{T, *}\mathcal{F}_T \to j_{T, *}\mathcal{Q}$.
Looking at stalks one finds that this is surjective.
By Lemma \ref{lemma-coherent-functorial}
we see that $j_{T, *}\mathcal{Q}$ is finitely presented, flat over $B$
with support proper over $B$. Thus we obtain a $T$-valued
point of $\Quotfunctor_{\mathcal{G}/Y/B}$.
This defines the morphism of the lemma.
We omit the proof that this is an open immersion. Hint:
If $\mathcal{G}_T \to \mathcal{Q}$ is an element of
$\Quotfunctor_{\mathcal{G}/Y/B}(T)$
and for $t \in T$ we have $\text{Supp}(\mathcal{Q}_t) \subset |X_t|$,
then the same is true for $t' \in T$ in a neighbourhood of $t$.
\end{proof}

\begin{lemma}
\label{lemma-quot-closed}
Let $B$ be an algebraic space. Let $\pi : X \to Y$ be a closed immersion
of algebraic spaces which are separated and of finite presentation over $B$.
Let $\mathcal{F}$ be a quasi-coherent $\mathcal{O}_X$-module.
Then the morphism
$\Quotfunctor_{\mathcal{F}/X/B} \to \Quotfunctor_{\pi_*\mathcal{F}/Y/B}$ of
Lemma \ref{lemma-quot-functorial} is an isomorphism.
\end{lemma}

\begin{proof}
For every scheme $T$ over $B$ the morphism $\pi_T : X_T \to Y_T$
is a closed immersion. Then $\pi_{T, *}$ is an equivalence of
categories between $\QCoh(\mathcal{O}_{X_T})$ and the full subcategory
of $\QCoh(\mathcal{O}_{Y_T})$ whose objects are those quasi-coherent
modules annihilated by the ideal sheaf of $X_T$, see
Morphisms of Spaces, Lemma \ref{spaces-morphisms-lemma-i-star-equivalence}.
Since a qotient of
$(\pi_*\mathcal{F})_T$ is annihilated by this ideal we obtain the
bijectivity of the map
$\Quotfunctor_{\mathcal{F}/X/B}(T) \to \Quotfunctor_{\pi_*\mathcal{F}/Y/B}(T)$
for all $T$ as desired.
\end{proof}

\begin{lemma}
\label{lemma-quot-quotient}
Let $X \to B$ be as in the introduction to this section. Let
$\mathcal{F} \to \mathcal{G}$ be a surjection of quasi-coherent
$\mathcal{O}_X$-modules. Then there is a canonical closed immersion
$\Quotfunctor_{\mathcal{G}/X/B} \to \Quotfunctor_{\mathcal{F}/X/B}$.
\end{lemma}

\begin{proof}
Let $\mathcal{K} = \Ker(\mathcal{F} \to \mathcal{G})$. By right
exactness of pullbacks we find that
$\mathcal{K}_T \to \mathcal{F}_T \to \mathcal{G}_T \to 0$
is an exact sequecnce for all schemes $T$ over $B$.
In particular, a quotient of $\mathcal{G}_T$
determines a quotient of $\mathcal{F}_T$ and we obtain our transformation
of functors
$\Quotfunctor_{\mathcal{G}/X/B} \to \Quotfunctor_{\mathcal{F}/X/B}$.
This transformation is a closed immersion by
Flatness on Spaces, Lemma \ref{spaces-flat-lemma-F-zero-closed-proper}.
Namely, given an element $\mathcal{F}_T \to \mathcal{Q}$ of
$\Quotfunctor_{\mathcal{F}/X/B}(T)$, then we see that the pull
back to $T'/T$ is in the image of the transformation if and
only if $\mathcal{K}_{T'} \to \mathcal{Q}_{T'}$ is zero.
\end{proof}

\begin{remark}[Numerical invariants]
\label{remark-quot-numerical}
Let $f : X \to B$ and $\mathcal{F}$ be as in the introduction to this section.
Let $I$ be a set and for $i \in I$ let $E_i \in D(\mathcal{O}_X)$ be perfect.
Let $P : I \to \mathbf{Z}$ be a function. Recall that we have a morphism
$$
\Quotfunctor_{\mathcal{F}/X/B} \longrightarrow \Cohstack_{X/B}
$$
which sends the element $\mathcal{F}_T \to \mathcal{Q}$
of $\Quotfunctor_{\mathcal{F}/X/B}(T)$ to the object $\mathcal{Q}$
of $\Cohstack_{X/B}$ over $T$, see proof of
Quot, Proposition \ref{quot-proposition-quot}. Hence we can form
the fibre product diagram
$$
\xymatrix{
\Quotfunctor^P_{\mathcal{F}/X/B} \ar[r] \ar[d] &
\Cohstack^P_{X/B} \ar[d] \\
\Quotfunctor_{\mathcal{F}/X/B} \ar[r] &
\Cohstack_{X/B}
}
$$
This is the defining diagram for the algebraic space in the
upper left corner. The left vertical arrow is a
flat closed immersion which is an open and closed immersion
for example if $I$ is finite, or $B$ is locally Noetherian, or
$I = \mathbf{Z}$ and $E_i = \mathcal{L}^{\otimes i}$ for some
invertible $\mathcal{O}_X$-module $\mathcal{L}$ (in the last
case we sometimes use the notation
$\Quotfunctor^{P, \mathcal{L}}_{\mathcal{F}/X/B}$).
See Situation \ref{situation-numerical} and
Lemmas \ref{lemma-open-P} and \ref{lemma-finite-list-perfect-objects} and
Example \ref{example-hilbert-polynomial}.
\end{remark}

\begin{lemma}
\label{lemma-quot-tensor-invertible}
Let $f : X \to B$ and $\mathcal{F}$ be as in the introduction to this section.
Let $\mathcal{L}$ be an invertible $\mathcal{O}_X$-module.
Then tensoring with $\mathcal{L}$ defines an isomorphism
$$
\Quotfunctor_{\mathcal{F}/X/B}
\longrightarrow
\Quotfunctor_{\mathcal{F} \otimes_{\mathcal{O}_X} \mathcal{L}/X/B}
$$
Given a numerical polynomial $P(t)$, then setting $P'(t) = P(t + 1)$
this map induces an isomorphism
$\Quotfunctor^P_{\mathcal{F}/X/B}
\longrightarrow
\Quotfunctor^{P'}_{\mathcal{F} \otimes_{\mathcal{O}_X} \mathcal{L}/X/B}$
of open and closed substacks.
\end{lemma}

\begin{proof}
Set $\mathcal{G} = \mathcal{F} \otimes_{\mathcal{O}_X} \mathcal{L}$.
Observe that
$\mathcal{G}_T = \mathcal{F}_T \otimes_{\mathcal{O}_{X_T}} \mathcal{L}_T$.
If $\mathcal{F}_T \to \mathcal{Q}$ is an element of
$\Quotfunctor_{\mathcal{F}/X/B}(T)$, then we send it
to the element
$\mathcal{G}_T \to \mathcal{Q} \otimes_{\mathcal{O}_{X_T}} \mathcal{L}_T$
of
$\Quotfunctor_{\mathcal{F} \otimes_{\mathcal{O}_X} \mathcal{L}/X/B}(T)$.
This is compatible with pullbacks and hence
defines a transformation of functors as desired.
Since there is an obvious inverse transformation,
it is an isomorphism. We omit the proof of the final statement.
\end{proof}

\begin{lemma}
\label{lemma-quot-power-invertible}
Let $f : X \to B$ and $\mathcal{F}$ be as in the introduction to this section.
Let $\mathcal{L}$ be an invertible $\mathcal{O}_X$-module.
Then
$$
\Quotfunctor^{P, \mathcal{L}}_{\mathcal{F}/X/B} =
\Quotfunctor^{P', \mathcal{L}^{\otimes n}}_{\mathcal{F}/X/B}
$$
where $P'(t) = P(nt)$.
\end{lemma}

\begin{proof}
Follows immediately after unwinding all the definitions.
\end{proof}






\section{Boundedness for Quot}
\label{section-quot-bounded}

\noindent
Contrary to what happens classically, we already know the Quot
functor is an algebraic space, but we don't know that it is
ever represented by a finite type algebraic space.

\begin{lemma}
\label{lemma-quot-Pn}
Let $n \geq 0$, $r \geq 1$, $P \in \mathbf{Q}[t]$.
The algebraic space
$$
X = \Quotfunctor^P_{\mathcal{O}^{\oplus r}_{\mathbf{P}^n_\mathbf{Z}}/
\mathbf{P}^n_\mathbf{Z}/\mathbf{Z}}
$$
parametrizing quotients of $\mathcal{O}_{\mathbf{P}^n_\mathbf{Z}}^{\oplus r}$
with Hilbert polynomial $P$ is proper over $\Spec(\mathbf{Z})$.
\end{lemma}

\begin{proof}
We already know that $X \to \Spec(\mathbf{Z})$ is separated and
locally of finite presentation (Lemma \ref{lemma-quot-s-lfp}).
We also know that $X \to \Spec(\mathbf{Z})$ satisfies the
existence part of the valuative criterion, see
Lemma \ref{lemma-quot-existence-part}.
By the valuative criterion for properness, it suffices to
prove our Quot space is quasi-compact, see
Morphisms of Spaces,
Lemma \ref{spaces-morphisms-lemma-characterize-proper}.
Thus it suffices to find a quasi-compact scheme $T$ and a surjective
morphism $T \to X$. Let $m$ be the integer found in
Varieties, Lemma \ref{varieties-lemma-bound-quotients-free}.
Let
$$
N = r{m + n \choose n} - P(m)
$$
We will write $\mathbf{P}^n$ for
$\mathbf{P}^n_\mathbf{Z} = \text{Proj}(\mathbf{Z}[T_0, \ldots, T_n])$
and unadorned products will mean products over $\Spec(\mathbf{Z})$.
The idea of the proof is to construct a ``universal'' map
$$
\Psi :
\mathcal{O}_{T \times \mathbf{P}^n}(-m)^{\oplus N}
\longrightarrow
\mathcal{O}_{T \times \mathbf{P}^n}^{\oplus r}
$$
over an affine scheme $T$ and show that every point of $X$
corresponds to a cokernel of this in some point of $T$.

\medskip\noindent
Definition of $T$ and $\Psi$. We take $T = \Spec(A)$ where
$$
A = \mathbf{Z}[a_{i, j, E}]
$$
where $i \in \{1, \ldots, r\}$, $j \in \{1, \ldots, N\}$
and $E = (e_0, \ldots, e_n)$ runs through the multi-indices
of total degree $|E| = \sum_{k = 0, \ldots n} e_k = m$.
Then we define $\Psi$ to be the map whose $(i, j)$ matrix
entry is the map
$$
\sum\nolimits_{E = (e_0, \ldots, e_n)}
a_{i, j, E} T_0^{e_0} \ldots T_n^{e_n} :
\mathcal{O}_{T \times \mathbf{P}^n}(-m)
\longrightarrow
\mathcal{O}_{T \times \mathbf{P}^n}
$$
where the sum is over $E$ as above (but $i$ and $j$ are fixed of course).

\medskip\noindent
Consider the quotient $\mathcal{Q} = \Coker(\Psi)$ on $T \times \mathbf{P}^n$.
By More on Morphisms, Lemma
\ref{more-morphisms-lemma-generic-flatness-stratification}
there exists a $t \geq 0$ and closed subschemes
$$
T = T_0 \supset T_1 \supset \ldots \supset T_t = \emptyset
$$
such that the pullback $\mathcal{Q}_p$ of $\mathcal{Q}$ to
$(T_p \setminus T_{p + 1}) \times \mathbf{P}^n$ is flat over
$T_p \setminus T_{p + 1}$. Observe that we
have an exact sequence
$$
\mathcal{O}_{(T_p \setminus T_{p + 1}) \times \mathbf{P}^n}(-m)^{\oplus N}
\to
\mathcal{O}_{(T_p \setminus T_{p + 1}) \times \mathbf{P}^n}^{\oplus r}
\to
\mathcal{Q}_p
\to
0
$$
by pulling back the exact sequence defining $\mathcal{Q} = \Coker(\Psi)$.
Therefore we obtain a morphism
$$
\coprod (T_p \setminus T_{p + 1})
\longrightarrow
\Quotfunctor_{\mathcal{O}^{\oplus r}/\mathbf{P}/\mathbf{Z}}
\supset
\Quotfunctor^P_{\mathcal{O}^{\oplus r}/\mathbf{P}/\mathbf{Z}} = X
$$
Since the left hand side is a Noetherian scheme and the inclusion
on the right hand side is open, it
suffices to show that any point of $X$ is in the image of this morphism.

\medskip\noindent
Let $k$ be a field and let $x \in X(k)$. Then $x$ corresponds to
a surjection $\mathcal{O}_{\mathbf{P}^n_k}^{\oplus r} \to \mathcal{F}$
of coherent $\mathcal{O}_{\mathbf{P}^n_k}$-modules
such that the Hilbert polynomial of $\mathcal{F}$ is $P$.
Consider the short exact sequence
$$
0 \to \mathcal{K} \to
\mathcal{O}_{\mathbf{P}^n_k}^{\oplus r} \to
\mathcal{F} \to 0
$$
By Varieties, Lemma \ref{varieties-lemma-bound-quotients-free}
and our choice of $m$ we see that $\mathcal{K}$ is $m$-regular.
By Varieties, Lemma \ref{varieties-lemma-m-regular-globally-generated}
we see that $\mathcal{K}(m)$ is globally generated.
By Varieties, Lemma \ref{varieties-lemma-m-regular-up}
and the definition of $m$-regularity we see that
$H^i(\mathbf{P}^n_k, \mathcal{K}(m)) = 0$ for $i > 0$.
Hence we see that
$$
\dim_k H^0(\mathbf{P}^n_k, \mathcal{K}(m)) =
\chi(\mathcal{K}(m)) =
\chi(\mathcal{O}_{\mathbf{P}^n_k}(m)^{\oplus r}) -
\chi(\mathcal{F}(m)) = N
$$
by our choice of $N$. This gives a surjection
$$
\mathcal{O}_{\mathbf{P}^n_k}^{\oplus N}
\longrightarrow
\mathcal{K}(m)
$$
Twisting back down and using the short exact sequence above
we see that $\mathcal{F}$ is the cokernel of a map
$$
\Psi_x :
\mathcal{O}_{\mathbf{P}^n_k}(-m)^{\oplus N}
\to
\mathcal{O}_{\mathbf{P}^n_k}^{\oplus r}
$$
There is a unique ring map $\tau : A \to k$ such that the base change
of $\Psi$ by the corresponding morphism $t = \Spec(\tau) : \Spec(k) \to T$
is $\Psi_x$. This is true because the entries of the $N \times r$
matrix defining $\Psi_x$ are homogeneous polynomials
$\sum \lambda_{i, j, E} T_0^{e_0} \ldots T_n^{e_n}$
of degree $m$ in $T_0, \ldots, T_n$ with coefficients
$\lambda_{i, j, E} \in k$ and we can set
$\tau(a_{i, j, E}) = \lambda_{i, j, E}$.
Then $t \in T_p \setminus T_{p + 1}$ for some $p$ and
the image of $t$ under the morphism above is $x$ as desired.
\end{proof}

\begin{lemma}
\label{lemma-quot-Pn-over-base}
Let $B$ be an algebraic space. Let $X = B \times \mathbf{P}^n_\mathbf{Z}$.
Let $\mathcal{L}$ be the pullback of $\mathcal{O}_{\mathbf{P}^n}(1)$ to $X$.
Let $\mathcal{F}$ be an $\mathcal{O}_X$-module of finite
presentation. The algebraic space $\Quotfunctor^P_{\mathcal{F}/X/B}$
parametrizing quotients of $\mathcal{F}$
having Hilbert polynomial $P$ with respect to $\mathcal{L}$
is proper over $B$.
\end{lemma}

\begin{proof}
The question is \'etale local over $B$, see
Morphisms of Spaces, Lemma \ref{spaces-morphisms-lemma-proper-local}.
Thus we may assume $B$ is an affine scheme.
In this case $\mathcal{L}$ is an ample invertible module on $X$
(by Constructions, Lemma \ref{constructions-lemma-ample-on-proj}
and the definition of ample invertible modules in
Properties, Definition \ref{properties-definition-ample}).
Thus we can find $r' \geq 0$ and $r \geq 0$ and a surjection
$$
\mathcal{O}_X^{\oplus r} \longrightarrow
\mathcal{F} \otimes_{\mathcal{O}_X} \mathcal{L}^{\otimes r'}
$$
by Properties, Proposition \ref{properties-proposition-characterize-ample}.
By Lemma \ref{lemma-quot-tensor-invertible}
we may replace $\mathcal{F}$ by
$\mathcal{F} \otimes_{\mathcal{O}_X} \mathcal{L}^{\otimes r'}$
and $P(t)$ by $P(t + r')$.
By Lemma \ref{lemma-quot-quotient}
we obtain a closed immersion
$$
\Quotfunctor^P_{\mathcal{F}/X/B}
\longrightarrow
\Quotfunctor^P_{\mathcal{O}_X^{\oplus r}/X/B}
$$
Since we've shown that $\Quotfunctor^P_{\mathcal{O}_X^{\oplus r}/X/B} \to B$
is proper in Lemma \ref{lemma-quot-Pn} we conclude.
\end{proof}

\begin{lemma}
\label{lemma-quot-proper-over-base}
Let $f : X \to B$ be a proper morphism of finite presentation
of algebraic spaces. Let $\mathcal{F}$ be a finitely presented
$\mathcal{O}_X$-module. Let $\mathcal{L}$ be an invertible
$\mathcal{O}_X$-module ample on $X/B$, see
Divisors on Spaces, Definition
\ref{spaces-divisors-definition-relatively-ample}.
The algebraic space $\Quotfunctor^P_{\mathcal{F}/X/B}$
parametrizing quotients of $\mathcal{F}$
having Hilbert polynomial $P$ with respect to $\mathcal{L}$
is proper over $B$.
\end{lemma}

\begin{proof}
The question is \'etale local over $B$, see
Morphisms of Spaces, Lemma \ref{spaces-morphisms-lemma-proper-local}.
Thus we may assume $B$ is an affine scheme.
Then we can find a closed immersion $i : X \to \mathbf{P}^n_B$
such that $i^*\mathcal{O}_{\mathbf{P}^n_B}(1) \cong \mathcal{L}^{\otimes d}$
for some $d \geq 1$. See
Morphisms, Lemma \ref{morphisms-lemma-quasi-projective-finite-type-over-S}.
Changing $\mathcal{L}$ into $\mathcal{L}^{\otimes d}$ and
the numerical polynomial $P(t)$ into $P(dt)$ leaves
$\Quotfunctor^P_{\mathcal{F}/X/B}$ unaffected; some details omitted.
Hence we may assume $\mathcal{L} = i^*\mathcal{O}_{\mathbf{P}^n_B}(1)$.
Then the isomorphism
$\Quotfunctor_{\mathcal{F}/X/B} \to
\Quotfunctor_{i_*\mathcal{F}/\mathbf{P}^n_B/B}$ of
Lemma \ref{lemma-quot-closed} induces an isomorphism
$\Quotfunctor^P_{\mathcal{F}/X/B} \cong
\Quotfunctor^P_{i_*\mathcal{F}/\mathbf{P}^n_B/B}$.
Since $\Quotfunctor^P_{i_*\mathcal{F}/\mathbf{P}^n_B/B}$
is proper over $B$ by Lemma \ref{lemma-quot-Pn-over-base}
we conclude.
\end{proof}

\begin{lemma}
\label{lemma-quot-qc-over-base}
Let $f : X \to B$ be a separated morphism of finite presentation
of algebraic spaces. Let $\mathcal{F}$ be a finitely presented
$\mathcal{O}_X$-module. Let $\mathcal{L}$ be an invertible
$\mathcal{O}_X$-module ample on $X/B$, see
Divisors on Spaces, Definition
\ref{spaces-divisors-definition-relatively-ample}.
The algebraic space $\Quotfunctor^P_{\mathcal{F}/X/B}$
parametrizing quotients of $\mathcal{F}$
having Hilbert polynomial $P$ with respect to $\mathcal{L}$
is separated of finite presentation over $B$.
\end{lemma}

\begin{proof}
We have already seen that $\Quotfunctor_{\mathcal{F}/X/B} \to B$
is separated and locally of finite presentation, see
Lemma \ref{lemma-quot-s-lfp}. Thus it suffices to show that
the open subspace $\Quotfunctor^P_{\mathcal{F}/X/B}$
of Remark \ref{remark-quot-numerical} is quasi-compact over $B$.

\medskip\noindent
The question is \'etale local on $B$
(Morphisms of Spaces, Lemma \ref{spaces-morphisms-lemma-quasi-compact-local}).
Thus we may assume $B$ is affine.

\medskip\noindent
Assume $B = \Spec(\Lambda)$. Write $\Lambda = \colim \Lambda_i$ as the
colimit of its finite type $\mathbf{Z}$-subalgebras. Then
we can find an $i$ and a system $X_i, \mathcal{F}_i, \mathcal{L}_i$
as in the lemma over $B_i = \Spec(\Lambda_i)$ whose base change to
$B$ gives $X, \mathcal{F}, \mathcal{L}$.
This follows from
Limits of Spaces, Lemmas
\ref{spaces-limits-lemma-descend-finite-presentation} (to find $X_i$),
\ref{spaces-limits-lemma-descend-modules-finite-presentation} (to find
$\mathcal{F}_i$), \ref{spaces-limits-lemma-descend-invertible-modules}
(to find $\mathcal{L}_i$), and \ref{spaces-limits-lemma-descend-separated}
(to make $X_i$ separated). Because
$$
\Quotfunctor_{\mathcal{F}/X/B} = B \times_{B_i}
\Quotfunctor_{\mathcal{F}_i/X_i/B_i}
$$
and similarly for $\Quotfunctor^P_{\mathcal{F}/X/B}$ we reduce
to the case discussed in the next paragraph.

\medskip\noindent
Assume $B$ is affine and Noetherian. We may replace $\mathcal{L}$
by a positive power, see Lemma \ref{lemma-quot-power-invertible}.
Thus we may assume there exists an immersion $i : X \to \mathbf{P}^n_B$
such that $i^*\mathcal{O}_{\mathbf{P}^n}(1) = \mathcal{L}$. By
Morphisms, Lemma \ref{morphisms-lemma-quasi-compact-immersion}
there exists a closed subscheme $X' \subset \mathbf{P}^n_B$
such that $i$ factors through an open immersion $j : X \to X'$.
By Properties, Lemma \ref{properties-lemma-lift-finite-presentation}
there exists a finitely presented $\mathcal{O}_{X'}$-module
$\mathcal{G}$ such that $j^*\mathcal{G} = \mathcal{F}$.
Thus we obtain an open immersion
$$
\Quotfunctor_{\mathcal{F}/X/B}
\longrightarrow
\Quotfunctor_{\mathcal{G}/X'/B}
$$
by Lemma \ref{lemma-quot-better-open}. Clearly this open immersion
sends $\Quotfunctor^P_{\mathcal{F}/X/B}$ into
$\Quotfunctor^P_{\mathcal{G}/X'/B}$. Now
$\Quotfunctor^P_{\mathcal{G}/X'/B}$ is proper over $B$ by
Lemma \ref{lemma-quot-proper-over-base}.
Therefore it is Noetherian and since any open of a Noetherian
algebraic space is quasi-compact we win.
\end{proof}




\section{Properties of the Hilbert functor}
\label{section-hilb}

\noindent
Let $f : X \to B$ be a morphism of algebraic spaces which is
separated and of finite presentation. Then
$\Hilbfunctor_{X/B}$ is an algebraic space locally of finite
presentation over $B$. See Quot, Proposition \ref{quot-proposition-hilb}.

\begin{lemma}
\label{lemma-hilb-diagonal-closed}
The diagonal of $\Hilbfunctor_{X/B} \to B$ is a closed immersion
of finite presentation.
\end{lemma}

\begin{proof}
In Quot, Lemma \ref{quot-lemma-hilb-is-quot} we have seen that
$\Hilbfunctor_{X/B} = \Quotfunctor_{\mathcal{O}_X/X/B}$.
Hence this follows from Lemma \ref{lemma-quot-diagonal-closed}.
\end{proof}

\begin{lemma}
\label{lemma-hilb-s-lfp}
The morphism $\Hilbfunctor_{X/B} \to B$ is separated
and locally of finite presentation.
\end{lemma}

\begin{proof}
To check $\Hilbfunctor_{X/B} \to B$ is separated we have to
show that its diagonal is a closed immersion. This
is true by Lemma \ref{lemma-hilb-diagonal-closed}.
The second statement is part of
Quot, Proposition \ref{quot-proposition-hilb}.
\end{proof}

\begin{lemma}
\label{lemma-hilb-existence-part}
Assume $X \to B$ is proper as well as of finite presentation.
Then $\Hilbfunctor_{X/B} \to B$ satisfies the existence part
of the valuative criterion (Morphisms of Spaces, Definition
\ref{spaces-morphisms-definition-valuative-criterion}).
\end{lemma}

\begin{proof}
In Quot, Lemma \ref{quot-lemma-hilb-is-quot} we have seen that
$\Hilbfunctor_{X/B} = \Quotfunctor_{\mathcal{O}_X/X/B}$.
Hence this follows from Lemma \ref{lemma-quot-existence-part}.
\end{proof}

\begin{lemma}
\label{lemma-hilb-open}
Let $B$ be an algebraic space. Let $\pi : X \to Y$ be an open immersion
of algebraic spaces which are separated and of finite presentation over $B$.
Then $\pi$ induces an open immersion
$\Hilbfunctor_{X/B} \to \Hilbfunctor_{Y/B}$.
\end{lemma}

\begin{proof}
Omitted. Hint: If $Z \subset X_T$ is a closed subscheme which is
proper over $T$, then $Z$ is also closed in $Y_T$. Thus we obtain
the transformation $\Hilbfunctor_{X/B} \to \Hilbfunctor_{Y/B}$.
If $Z \subset Y_T$ is an element of $\Hilbfunctor_{Y/B}(T)$
and for $t \in T$ we have $|Z_t| \subset |X_t|$,
then the same is true for $t' \in T$ in a neighbourhood of $t$.
\end{proof}

\begin{lemma}
\label{lemma-hilb-closed}
Let $B$ be an algebraic space. Let $\pi : X \to Y$ be a closed immersion
of algebraic spaces which are separated and of finite presentation
over $B$. Then $\pi$ induces a closed immersion
$\Hilbfunctor_{X/B} \to \Hilbfunctor_{Y/B}$.
\end{lemma}

\begin{proof}
Since $\pi$ is a closed immersion, it is immediate that given a
closed subscheme $Z \subset X_T$,
we can view $Z$ as a closed subscheme of $X_T$. Thus we obtain
the transformation $\Hilbfunctor_{X/B} \to \Hilbfunctor_{Y/B}$.
This transformation is immediately seen to be a monomorphism.
To prove that it is a closed immersion, you can use
Lemma \ref{lemma-quot-quotient} for the map
$\mathcal{O}_Y \to \mathcal{O}_X$ and the identifications
$\Hilbfunctor_{X/B} = \Quotfunctor_{\mathcal{O}_X/X/B}$,
$\Hilbfunctor_{Y/B} = \Quotfunctor_{\mathcal{O}_Y/Y/B}$
of Quot, Lemma \ref{quot-lemma-hilb-is-quot}.
\end{proof}

\begin{remark}[Numerical invariants]
\label{remark-hilb-numerical}
Let $f : X \to B$ be as in the introduction to this section.
Let $I$ be a set and for $i \in I$ let $E_i \in D(\mathcal{O}_X)$ be perfect.
Let $P : I \to \mathbf{Z}$ be a function. Recall that
$\Hilbfunctor_{X/B} = \Quotfunctor_{\mathcal{O}_X/X/B}$, see
Quot, Lemma \ref{quot-lemma-hilb-is-quot}.
Thus we can define
$$
\Hilbfunctor^P_{X/B} = \Quotfunctor^P_{\mathcal{O}_X/X/B}
$$
where $\Quotfunctor^P_{\mathcal{O}_X/X/B}$ is as in
Remark \ref{remark-quot-numerical}. The morphism
$$
\Hilbfunctor^P_{X/B} \longrightarrow \Hilbfunctor_{X/B}
$$
is a flat closed immersion which is an open and closed immersion
for example if $I$ is finite, or $B$ is locally Noetherian, or
$I = \mathbf{Z}$ and $E_i = \mathcal{L}^{\otimes i}$ for some
invertible $\mathcal{O}_X$-module $\mathcal{L}$. In the last case
we sometimes use the notation $\Hilbfunctor^{P, \mathcal{L}}_{X/B}$.
\end{remark}

\begin{lemma}
\label{lemma-hilb-proper-over-base}
Let $f : X \to B$ be a proper morphism of finite presentation
of algebraic spaces. Let $\mathcal{L}$ be an invertible
$\mathcal{O}_X$-module ample on $X/B$, see
Divisors on Spaces, Definition
\ref{spaces-divisors-definition-relatively-ample}.
The algebraic space $\Hilbfunctor^P_{X/B}$
parametrizing closed subschemes
having Hilbert polynomial $P$ with respect to $\mathcal{L}$
is proper over $B$.
\end{lemma}

\begin{proof}
Recall that $\Hilbfunctor_{X/B} = \Quotfunctor_{\mathcal{O}_X/X/B}$, see
Quot, Lemma \ref{quot-lemma-hilb-is-quot}.
Thus this lemma is an immediate consequence of
Lemma \ref{lemma-quot-proper-over-base}.
\end{proof}

\begin{lemma}
\label{lemma-hilb-qc-over-base}
Let $f : X \to B$ be a separated morphism of finite presentation
of algebraic spaces. Let $\mathcal{L}$ be an invertible
$\mathcal{O}_X$-module ample on $X/B$, see
Divisors on Spaces, Definition
\ref{spaces-divisors-definition-relatively-ample}.
The algebraic space $\Hilbfunctor^P_{X/B}$
parametrizing closed subschemes
having Hilbert polynomial $P$ with respect to $\mathcal{L}$
is separated of finite presentation over $B$.
\end{lemma}

\begin{proof}
Recall that $\Hilbfunctor_{X/B} = \Quotfunctor_{\mathcal{O}_X/X/B}$, see
Quot, Lemma \ref{quot-lemma-hilb-is-quot}.
Thus this lemma is an immediate consequence of
Lemma \ref{lemma-quot-qc-over-base}.
\end{proof}









\section{Properties of the Picard stack}
\label{section-picard-stack}

\noindent
Let $f : X \to B$ be a morphism of algebraic spaces which is flat,
proper, and of finite presentation. Then the stack
$\Picardstack_{X/B}$ parametrizing invertible sheaves on $X/B$
is algebraic, see Quot, Proposition \ref{quot-proposition-pic}.

\begin{lemma}
\label{lemma-pic-diagonal-affine-fp}
The diagonal of $\Picardstack_{X/B}$ over $B$ is affine
and of finite presentation.
\end{lemma}

\begin{proof}
In Quot, Lemma \ref{quot-lemma-picard-stack-open-in-coh} we have seen that
$\Picardstack_{X/B}$ is an open substack of
$\Cohstack_{X/B}$. Hence this follows from
Lemma \ref{lemma-coherent-diagonal-affine-fp}.
\end{proof}

\begin{lemma}
\label{lemma-pic-qs-lfp}
The morphism $\Picardstack_{X/B} \to B$ is quasi-separated and
locally of finite presentation.
\end{lemma}

\begin{proof}
In Quot, Lemma \ref{quot-lemma-picard-stack-open-in-coh} we have seen that
$\Picardstack_{X/B}$ is an open substack of
$\Cohstack_{X/B}$. Hence this follows from
Lemma \ref{lemma-coherent-qs-lfp}.
\end{proof}

\begin{lemma}
\label{lemma-pic-existence-part}
Assume $X \to B$ is smooth in addition to being proper.
Then $\Picardstack_{X/B} \to B$ satisfies the existence part
of the valuative criterion (Morphisms of Stacks, Definition
\ref{stacks-morphisms-definition-existence}).
\end{lemma}

\begin{proof}
Taking base change, this immediately reduces to the following
problem: given a valuation ring $R$ with fraction field $K$ and
an algebraic space $X$ proper and smooth over $R$ and an invertible
$\mathcal{O}_{X_K}$-module $\mathcal{L}_K$, show there exists
an invertible $\mathcal{O}_X$-module $\mathcal{L}$
whose generic fibre is $\mathcal{L}_K$.
Observe that $X_K$ is Noetherian, separated, and regular
(use Morphisms of Spaces, Lemma
\ref{spaces-morphisms-lemma-finite-presentation-noetherian}
and
Spaces over Fields, Lemma \ref{spaces-over-fields-lemma-smooth-regular}).
Thus we can write
$\mathcal{L}_K$ as the difference in the Picard group of
$\mathcal{O}_{X_K}(D_K)$ and $\mathcal{O}_{X_K}(D'_K)$
for two effective Cartier divisors $D_K, D'_K$ in $X_K$, see
Divisors on Spaces, Lemma
\ref{spaces-divisors-lemma-Noetherian-regular-separated-pic-effective-Cartier}.
Finally, we know that $D_K$ and $D'_K$ are restrictions of
effective Cartier divisors $D, D' \subset X$, see
Divisors on Spaces, Lemma
\ref{spaces-divisors-lemma-smooth-over-valuation-ring-effective-Cartier}.
\end{proof}

\begin{lemma}
\label{lemma-pic-inertia}
Assume $f_{T, *}\mathcal{O}_{X_T} \cong \mathcal{O}_T$ for all
schemes $T$ over $B$. Then the inertia stack of $\Picardstack_{X/B}$
is equal to $\mathbf{G}_m \times \Picardstack_{X/B}$.
\end{lemma}

\begin{proof}
This is explained in Examples of Stacks, Example
\ref{examples-stacks-example-inertia-stack-of-picard}.
\end{proof}

\begin{lemma}
\label{lemma-pic-curves-smooth}
Assume $f : X \to B$ has relative dimension $\leq 1$ in addition to
the other assumptions in this section. Then $\Picardstack_{X/B} \to B$
is smooth.
\end{lemma}

\begin{proof}
We already know that $\Picardstack_{X/B} \to B$ is
locally of finite presentation, see Lemma \ref{lemma-pic-qs-lfp}.
Thus it suffices to show that $\Picardstack_{X/B} \to B$ is
formally smooth, see More on Morphisms of Stacks, Lemma
\ref{stacks-more-morphisms-lemma-smooth-formally-smooth}.
Taking base change, this immediately reduces to the following
problem: given a first order thickening $T \subset T'$
of affine schemes, given $X' \to T'$ proper, flat, of finite
presentation and of relative dimension $\leq 1$, and
for $X = T \times_{T'} X'$ given an invertible $\mathcal{O}_X$-module
$\mathcal{L}$, prove that there exists an invertible
$\mathcal{O}_{X'}$-module $\mathcal{L}'$ whose
restriction to $X$ is $\mathcal{L}$.
Since $T \subset T'$ is a first order thickening, the
same is true for $X \subset X'$, see
More on Morphisms of Spaces, Lemma
\ref{spaces-more-morphisms-lemma-base-change-thickening}.
By More on Morphisms of Spaces, Lemma
\ref{spaces-more-morphisms-lemma-picard-group-first-order-thickening}
we see that it suffices to show $H^2(X, \mathcal{I}) = 0$
where $\mathcal{I}$ is the quasi-coherent ideal cutting out $X$ in $X'$.
Denote $f : X \to T$ the structure morphism.
By Cohomology of Spaces, Lemma
\ref{spaces-cohomology-lemma-higher-direct-images-zero-above-dimension-fibre}
we see that $R^pf_*\mathcal{I} = 0$ for $p > 1$.
Hence we get the desired vanishing by
Cohomology of Spaces, Lemma
\ref{spaces-cohomology-lemma-quasi-coherence-higher-direct-images-application}
(here we finally use that $T$ is affine).
\end{proof}




\section{Properties of the Picard functor}
\label{section-picard-functor}

\noindent
Let $f : X \to B$ be a morphism of algebraic spaces which is flat,
proper, and of finite presentation such that moreover for every $T/B$
the canonical map
$$
\mathcal{O}_T \longrightarrow f_{T, *}\mathcal{O}_{X_T}
$$
is an isomorphism. Then the Picard functor $\Picardfunctor_{X/B}$ is an
algebraic space, see Quot, Proposition \ref{quot-proposition-pic-functor}.
There is a closed relationship with the Picard stack.

\begin{lemma}
\label{lemma-pic-gerbe-over-pic-functor}
The morphism $\Picardstack_{X/B} \to \Picardfunctor_{X/B}$
turns the Picard stack into a gerbe over the Picard functor.
\end{lemma}

\begin{proof}
The definition of $\Picardstack_{X/B} \to \Picardfunctor_{X/B}$ being
a gerbe is given in Morphisms of Stacks, Definition
\ref{stacks-morphisms-definition-gerbe}, which in turn refers to
Stacks, Definition \ref{stacks-definition-gerbe-over-stack-in-groupoids}.
To prove it, we will check conditions (2)(a) and (2)(b) of
Stacks, Lemma \ref{stacks-lemma-when-gerbe}. This follows immediately from
Quot, Lemma \ref{quot-lemma-pic-over-pic}; here is a detailed explanation.

\medskip\noindent
Condition (2)(a).
Suppose that $\xi \in \Picardfunctor_{X/B}(U)$ for some scheme $U$ over $B$.
Since $\Picardfunctor_{X/B}$ is the fppf sheafification of the rule
$T \mapsto \Pic(X_T)$ on schemes over $B$
(Quot, Situation \ref{quot-situation-pic}), we see that there exists an
fppf covering $\{U_i \to U\}$ such that $\xi|_{U_i}$ corresponds
to some invertible module $\mathcal{L}_i$ on $X_{U_i}$.
Then $(U_i \to B, \mathcal{L}_i)$ is an object of
$\Picardstack_{X/B}$ over $U_i$ mapping to $\xi|_{U_i}$.

\medskip\noindent
Condition (2)(b). Suppose that $U$ is a scheme over $B$ and
$\mathcal{L}, \mathcal{N}$ are invertible modules on $X_U$
which map to the same element of $\Picardfunctor_{X/B}(U)$.
Then there exists an fppf covering $\{U_i \to U\}$
such that $\mathcal{L}|_{X_{U_i}}$ is isomorphic to $\mathcal{N}|_{X_{U_i}}$.
Thus we find isomorphisms between
$(U \to B, \mathcal{L})|_{U_i} \to (U \to B, \mathcal{N})|_{U_i}$
as desired.
\end{proof}

\begin{lemma}
\label{lemma-pic-functor-diagonal-qc-immersion}
The diagonal of $\Picardfunctor_{X/B}$ over $B$ is a quasi-compact immersion.
\end{lemma}

\begin{proof}
The diagonal is an immersion by Quot, Lemma \ref{quot-lemma-diagonal-pic}.
To finish we show that the diagonal is quasi-compact.
The diagonal of $\Picardstack_{X/B}$ is quasi-compact
by Lemma \ref{lemma-pic-diagonal-affine-fp} and
$\Picardstack_{X/B}$ is a gerbe over $\Picardfunctor_{X/B}$ by
Lemma \ref{lemma-pic-gerbe-over-pic-functor}.
We conclude by Morphisms of Stacks, Lemma
\ref{stacks-morphisms-lemma-gerbe-diagonal-quasi-compact}.
\end{proof}

\begin{lemma}
\label{lemma-pic-functor-qs-lfp}
The morphism $\Picardfunctor_{X/B} \to B$ is quasi-separated and
locally of finite presentation.
\end{lemma}

\begin{proof}
To check $\Picardfunctor_{X/B} \to B$ is quasi-separated we have to
show that its diagonal is quasi-compact. This is immediate from
Lemma \ref{lemma-pic-functor-diagonal-qc-immersion}.
Since the morphism $\Picardstack_{X/B} \to \Picardfunctor_{X/B}$
is surjective, flat, and locally of finite presentation
(by Lemma \ref{lemma-pic-gerbe-over-pic-functor} and
Morphisms of Stacks, Lemma \ref{stacks-morphisms-lemma-gerbe-fppf})
it suffices to prove that $\Picardstack_{X/B} \to B$
is locally of finite presentation, see
Morphisms of Stacks, Lemma
\ref{stacks-morphisms-lemma-flat-finite-presentation-permanence}.
This follows
from Lemma \ref{lemma-pic-qs-lfp}.
\end{proof}

\begin{lemma}
\label{lemma-pic-functor-uniqueness-part}
Assume the geometric fibres of $X \to B$ are integral
in addition to the other assumptions in this section.
Then $\Picardfunctor_{X/B} \to B$ is separated.
\end{lemma}

\begin{proof}
Since $\Picardfunctor_{X/B} \to B$ is quasi-separated, it suffices
to check the uniqueness part of the valuative criterion, see
Morphisms of Spaces, Lemma
\ref{spaces-morphisms-lemma-valuative-criterion-separatedness}.
This immediately reduces to the following problem: given
\begin{enumerate}
\item a valuation ring $R$ with fraction field $K$,
\item an algebraic space $X$ proper and flat over $R$
with integral geometric fibre,
\item an element $a \in \Picardfunctor_{X/R}(R)$ with
$a|_{\Spec(K)} = 0$,
\end{enumerate}
then we have to prove $a = 0$. Applying
Morphisms of Stacks, Lemma
\ref{stacks-morphisms-lemma-lift-valuation-ring-through-flat-morphism}
to the surjective flat morphism
$\Picardstack_{X/R} \to \Picardfunctor_{X/R}$
(surjective and flat by Lemma \ref{lemma-pic-gerbe-over-pic-functor} and
Morphisms of Stacks, Lemma \ref{stacks-morphisms-lemma-gerbe-fppf})
after replacing $R$ by an extension we may assume
$a$ is given by an invertible $\mathcal{O}_X$-module
$\mathcal{L}$. Since $a|_{\Spec(K)} = 0$ we find
$\mathcal{L}_K \cong \mathcal{O}_{X_K}$ by
Quot, Lemma \ref{quot-lemma-flat-geometrically-connected-fibres}.

\medskip\noindent
Denote $f : X \to \Spec(R)$ the structure morphism.
Let $\eta, 0 \in \Spec(R)$ be the generic and closed point.
Consider the perfect complexes
$K = Rf_*\mathcal{L}$ and $M = Rf_*(\mathcal{L}^{\otimes -1})$
on $\Spec(R)$, see Derived Categories of Spaces, Lemma
\ref{spaces-perfect-lemma-flat-proper-perfect-direct-image-general}.
Consider the functions
$\beta_{K, i}, \beta_{M, i} : \Spec(R) \to \mathbf{Z}$
of Derived Categories of Spaces, Lemma
\ref{spaces-perfect-lemma-jump-loci} associated to $K$ and $M$.
Since the formation of $K$ amd $M$ commutes with
base change (see lemma cited above) we find
$\beta_{K, 0}(\eta) = \beta_{M, 0}(\beta) = 1$ by
Spaces over Fields, Lemma
\ref{spaces-over-fields-lemma-proper-geometrically-reduced-global-sections}
and our assumption on the fibres of $f$.
By upper semi-continuity we find
$\beta_{K, 0}(0) \geq 1$ and $\beta_{M, 0} \geq 1$.
By 
Spaces over Fields, Lemma
\ref{spaces-over-fields-lemma-characterize-trivial-pic-integral}
we conclude that the restriction of $\mathcal{L}$
to the special fibre $X_0$ is trivial. In turn this gives
$\beta_{K, 0}(0) = \beta_{M, 0} = 1$ as above.
Then by More on Algebra, Lemma
\ref{more-algebra-lemma-lift-pseudo-coherent-from-residue-field}
we can represent $K$ by a complex of the form
$$
\ldots \to 0 \to R \to R^{\oplus \beta_{K, 1}(0)} \to
R^{\oplus \beta_{K, 2}(0)} \to \ldots
$$
Now $R \to R^{\oplus \beta_{K, 1}(0)}$ is zero
because $\beta_{K, 0}(\eta) = 1$. In other words
$K = R \oplus \tau_{\geq 1}(K)$ in $D(R)$ where $\tau_{\geq 1}(K)$
has tor amplitude in $[1, b]$ for some $b \in \mathbf{Z}$.
Hence there is a global section $s \in H^0(X, \mathcal{L})$
whose restriction $s_0$
to $X_0$ is nonvanishing (again because formation of $K$
commutes with base change). Then $s : \mathcal{O}_X \to \mathcal{L}$
is a map of invertible sheaves whose restriction to $X_0$
is an isomorphism and hence is an isomorphism as desired.
\end{proof}

\begin{lemma}
\label{lemma-pic-functor-curves-smooth}
Assume $f : X \to B$ has relative dimension $\leq 1$ in addition to
the other assumptions in this section. Then $\Picardfunctor_{X/B} \to B$
is smooth.
\end{lemma}

\begin{proof}
By Lemma \ref{lemma-pic-curves-smooth} we know that
$\Picardstack_{X/B} \to B$ is smooth. The morphism
$\Picardstack_{X/B} \to \Picardfunctor_{X/B}$ is surjective
and smooth by combining Lemma \ref{lemma-pic-gerbe-over-pic-functor} with
Morphisms of Stacks, Lemma \ref{stacks-morphisms-lemma-gerbe-smooth}.
Thus if $U$ is a scheme and $U \to \Picardstack_{X/B}$ is surjective
and smooth, then $U \to \Picardfunctor_{X/B}$ is surjective and smooth
and $U \to B$ is surjective and smooth (because these properties
are preserved by composition). Thus $\Picardfunctor_{X/B} \to B$
is smooth for example by
Descent on Spaces, Lemma
\ref{spaces-descent-lemma-syntomic-smooth-etale-permanence}.
\end{proof}






\section{Properties of relative morphisms}
\label{section-relative-morphisms}

\noindent
Let $B$ be an algebraic space. Let $X$ and $Y$ be algebraic spaces
over $B$ such that $Y \to B$ is flat, proper, and of finite presentation
and $X \to B$ is separated and of finite presentation.
Then the functor $\mathit{Mor}_B(Y, X)$ of relative morphisms
is an algebraic space locally of finite presentation over $B$.
See Quot, Proposition \ref{quot-proposition-Mor}.

\begin{lemma}
\label{lemma-Mor-diagonal-closed}
The diagonal of $\mathit{Mor}_B(Y, X) \to B$ is a closed immersion
of finite presentation.
\end{lemma}

\begin{proof}
There is an open immersion
$\mathit{Mor}_B(Y, X) \to \Hilbfunctor_{Y \times_B X/B}$, see
Quot, Lemma \ref{quot-lemma-Mor-into-Hilb-open}.
Thus the lemma follows from
Lemma \ref{lemma-hilb-diagonal-closed}.
\end{proof}

\begin{lemma}
\label{lemma-Mor-s-lfp}
The morphism $\mathit{Mor}_B(Y, X) \to B$ is separated
and locally of finite presentation.
\end{lemma}

\begin{proof}
To check $\mathit{Mor}_B(Y, X) \to B$ is separated we have to
show that its diagonal is a closed immersion. This
is true by Lemma \ref{lemma-Mor-diagonal-closed}.
The second statement is part of
Quot, Proposition \ref{quot-proposition-Mor}.
\end{proof}

\begin{lemma}
\label{lemma-Isom-in-Mor}
With $B, X, Y$ as in the introduction of this section, in addition
assume $X \to B$ is proper. Then the
subfunctor $\mathit{Isom}_B(Y, X) \subset \mathit{Mor}_B(Y, X)$
of isomorphisms is an open subspace.
\end{lemma}

\begin{proof}
Follows immediately from More on Morphisms of Spaces, Lemma
\ref{spaces-more-morphisms-lemma-where-isomorphism}.
\end{proof}

\begin{remark}[Numerical invariants]
\label{remark-Mor-numerical}
Let $B, X, Y$ be as in the introduction to this section.
Let $I$ be a set and for $i \in I$ let
$E_i \in D(\mathcal{O}_{Y \times_B X})$ be perfect.
Let $P : I \to \mathbf{Z}$ be a function. Recall that
$$
\mathit{Mor}_B(Y, X) \subset
\Hilbfunctor_{Y \times_B X/B}
$$
is an open subspace, see Quot, Lemma \ref{quot-lemma-Mor-into-Hilb-open}.
Thus we can define
$$
\mathit{Mor}^P_B(Y, X) =
\mathit{Mor}_B(Y, X) \cap \Hilbfunctor^P_{Y \times_B X/B}
$$
where $\Hilbfunctor^P_{Y \times_B X/B}$ is as in
Remark \ref{remark-hilb-numerical}. The morphism
$$
\mathit{Mor}^P_B(Y, X) \longrightarrow \mathit{Mor}_B(Y, X)
$$
is a flat closed immersion which is an open and closed immersion
for example if $I$ is finite, or $B$ is locally Noetherian, or
$I = \mathbf{Z}$, $E_i = \mathcal{L}^{\otimes i}$
for some invertible $\mathcal{O}_{Y \times_B X}$-module $\mathcal{L}$.
In the last case we sometimes use the notation
$\mathit{Mor}^{P, \mathcal{L}}_B(Y, X)$.
\end{remark}

\begin{lemma}
\label{lemma-Mor-qc-over-base}
With $B, X, Y$ as in the introduction of this section, let
$\mathcal{L}$ be ample on $X/B$ and let $\mathcal{N}$ be ample on $Y/B$.
See Divisors on Spaces, Definition
\ref{spaces-divisors-definition-relatively-ample}.
Let $P$ be a numerical polynomial. Then
$$
\mathit{Mor}^{P, \mathcal{M}}_B(Y, X) \longrightarrow B
$$
is separated and of finite presentation where
$\mathcal{M} = \text{pr}_1^*\mathcal{N}
\otimes_{\mathcal{O}_{Y \times_B X}} \text{pr}_2^*\mathcal{L}$.
\end{lemma}

\begin{proof}
By Lemma \ref{lemma-Mor-s-lfp} the morphism $\mathit{Mor}_B(Y, X) \to B$
is separated and locally of finite presentation. Thus it suffices to
show that the open and closed subspace $\mathit{Mor}^{P, \mathcal{M}}_B(Y, X)$
of Remark \ref{remark-Mor-numerical} is quasi-compact over $B$.

\medskip\noindent
The question is \'etale local on $B$
(Morphisms of Spaces, Lemma \ref{spaces-morphisms-lemma-quasi-compact-local}).
Thus we may assume $B$ is affine.

\medskip\noindent
Assume $B = \Spec(\Lambda)$. Note that $X$ and
$Y$ are schemes and that $\mathcal{L}$ and $\mathcal{N}$ are ample
invertible sheaves on $X$ and $Y$ (this follows immediately from the
definitions). Write $\Lambda = \colim \Lambda_i$ as the
colimit of its finite type $\mathbf{Z}$-subalgebras. Then
we can find an $i$ and a system $X_i, Y_i, \mathcal{L}_i, \mathcal{N}_i$
as in the lemma over $B_i = \Spec(\Lambda_i)$ whose base change to
$B$ gives $X, Y, \mathcal{L}, \mathcal{N}$. This follows from
Limits, Lemmas
\ref{limits-lemma-descend-finite-presentation} (to find $X_i$, $Y_i$),
\ref{limits-lemma-descend-invertible-modules} (to find $\mathcal{L}_i$,
$\mathcal{N}_i$), \ref{limits-lemma-descend-separated-finite-presentation}
(to make $X_i \to B_i$ separated), \ref{limits-lemma-eventually-proper}
(to make $Y_i \to B_i$ proper), and \ref{limits-lemma-limit-ample}
(to make $\mathcal{L}_i$, $\mathcal{N}_i$ ample).
Because
$$
\mathit{Mor}_B(Y, X) = B \times_{B_i} \mathit{Mor}_{B_i}(Y_i, X_i)
$$
and similarly for $\mathit{Mor}^P_B(Y, X)$ we reduce
to the case discussed in the next paragraph.

\medskip\noindent
Assume $B$ is a Noetherian affine scheme. By
Properties, Lemma \ref{properties-lemma-ample-on-product}
we see that $\mathcal{M}$ is ample. By Lemma \ref{lemma-hilb-qc-over-base}
we see that $\Hilbfunctor^{P, \mathcal{M}}_{Y \times_B X/B}$ is of
finite presentation over $B$ and hence Noetherian.
By construction
$$
\mathit{Mor}^{P, \mathcal{M}}_B(Y, X) =
\mathit{Mor}_B(Y, X) \cap
\Hilbfunctor^{P, \mathcal{M}}_{Y \times_B X/B}
$$
is an open subspace of $\Hilbfunctor^{P, \mathcal{M}}_{Y \times_B X/B}$ and
hence quasi-compact (as an open of a Noetherian algebraic space
is quasi-compact).
\end{proof}






\section{Properties of the stack of polarized proper schemes}
\label{section-polarized}

\noindent
In this section we discuss properties of the moduli stack
$$
\Polarizedstack \longrightarrow \Spec(\mathbf{Z})
$$
whose category of sections over a scheme $S$ is the category of
proper, flat, finitely presented scheme over $S$ endowed
with a relatively ample invertible sheaf. This is an algebraic
stack by Quot, Theorem \ref{quot-theorem-polarized-algebraic}.

\begin{lemma}
\label{lemma-polarized-diagonal-separated-fp}
The diagonal of $\Polarizedstack$ is separated
and of finite presentation.
\end{lemma}

\begin{proof}
Recall that $\Polarizedstack$ is a limit preserving algebraic stack, see
Quot, Lemma \ref{quot-lemma-polarized-limits}.
By Limits of Stacks, Lemma \ref{stacks-limits-lemma-limit-preserving-diagonal}
this implies that
$\Delta : \Polarizedstack \to \Polarizedstack \times \Polarizedstack$
is limit preserving. Hence $\Delta$ is locally of finite presentation
by Limits of Stacks, Proposition
\ref{stacks-limits-proposition-characterize-locally-finite-presentation}.

\medskip\noindent
Let us prove that $\Delta$ is separated. To see this, it suffices to show
that given an affine scheme $U$ and two objects
$\upsilon = (Y, \mathcal{N})$ and $\chi = (X, \mathcal{L})$
of $\Polarizedstack$ over $U$, the algebraic
space
$$
\mathit{Isom}_{\Polarizedstack}(\upsilon, \chi)
$$
is separated. The rule which to an isomorphism $\upsilon_T \to \chi_T$
assigns the underlying isomorphism $Y_T \to X_T$ defines a morphism
$$
\mathit{Isom}_{\Polarizedstack}(\upsilon, \chi)
\longrightarrow
\mathit{Isom}_U(Y, X)
$$
Since we have seen in Lemmas \ref{lemma-Mor-s-lfp} and
\ref{lemma-Isom-in-Mor} that the target is
a separated algebraic space, it suffices to prove that this morphism
is separated. Given an isomorphism $f : Y_T \to X_T$
over some scheme $T/U$, then clearly
$$
\mathit{Isom}_{\Polarizedstack}(\upsilon, \chi)
\times_{\mathit{Isom}_U(Y, X), [f]} T
=
\mathit{Isom}(\mathcal{N}_T, f^*\mathcal{L}_T)
$$
Here $[f] : T \to \mathit{Isom}_U(Y, X)$ indicates the $T$-valued
point corresponding to $f$ and
$\mathit{Isom}(\mathcal{N}_T, f^*\mathcal{L}_T)$
is the algebraic space discussed in Section \ref{section-hom-isom}.
Since this algebraic space is affine over $U$, the claim implies
$\Delta$ is separated.

\medskip\noindent
To finish the proof we show that $\Delta$ is quasi-compact. Since
$\Delta$ is representable by algebraic spaces, it suffice to check
the base change of $\Delta$ by a surjective smooth morphism
$U \to \Polarizedstack \times \Polarizedstack$ is quasi-compact
(see for example Properties of Stacks, Lemma
\ref{stacks-properties-lemma-check-property-covering}).
We can assume $U = \coprod U_i$ is a disjoint union of affine opens.
Since $\Polarizedstack$ is limit preserving (see above), we
see that $\Polarizedstack \to \Spec(\mathbf{Z})$ is locally of
finite presentation, hence $U_i \to \Spec(\mathbf{Z})$ is
locally of finite presentation
(Limits of Stacks, Proposition
\ref{stacks-limits-proposition-characterize-locally-finite-presentation}
and Morphisms of Stacks, Lemmas
\ref{stacks-morphisms-lemma-composition-finite-presentation} and
\ref{stacks-morphisms-lemma-smooth-locally-finite-presentation}).
In particular, $U_i$ is Noetherian affine. This reduces us to the
case discussed in the next paragraph.

\medskip\noindent
In this paragraph, given a Noetherian affine scheme $U$ and two objects
$\upsilon = (Y, \mathcal{N})$ and $\chi = (X, \mathcal{L})$
of $\Polarizedstack$ over $U$, we show the algebraic space
$$
\mathit{Isom}_{\Polarizedstack}(\upsilon, \chi)
$$
is quasi-compact. Since the connected components of $U$ are open and closed
we may replace $U$ by these. Thus we may and do assume $U$ is connected.
Let $u \in U$ be a point. Let $P$ be the Hilbert polynomial
$n \mapsto \chi(Y_u, \mathcal{N}_u^{\otimes n})$, see
Varieties, Lemma \ref{varieties-lemma-numerical-polynomial-from-euler}.
Since $U$ is connected and since
the functions $u \mapsto \chi(Y_u, \mathcal{N}_u^{\otimes n})$
are locally constant (see 
Derived Categories of Schemes, Lemma
\ref{perfect-lemma-chi-locally-constant-geometric})
we see that we get the same Hilbert polynomial in every point of $U$.
Set
$\mathcal{M} = \text{pr}_1^*\mathcal{N}
\otimes_{\mathcal{O}_{Y \times_U X}} \text{pr}_2^*\mathcal{L}$
on $Y \times_U X$. Given
$(f, \varphi) \in \mathit{Isom}_{\Polarizedstack}(\upsilon, \chi)(T)$
for some scheme $T$ over $U$ then for every $t \in T$ we have
$$
\chi(Y_t, (\text{id} \times f)^*\mathcal{M}^{\otimes n}) =
\chi(Y_t,
\mathcal{N}_t^{\otimes n} \otimes_{\mathcal{O}_{Y_t}}
f_t^*\mathcal{L}_t^{\otimes n}) =
\chi(Y_t, \mathcal{N}_t^{\otimes 2n}) = P(2n)
$$
where in the middle equality we use the isomorphism
$\varphi : f^*\mathcal{L}_T \to \mathcal{N}_T$.
Setting $P'(t) = P(2t)$ we find that the morphism
$$
\mathit{Isom}_{\Polarizedstack}(\upsilon, \chi)
\longrightarrow
\mathit{Isom}_U(Y, X)
$$
(see earlier) has image contained in the intersection
$$
\mathit{Isom}_U(Y, X) \cap \mathit{Mor}^{P', \mathcal{M}}_U(Y, X)
$$
The intersection is an intersection of open subspaces of
$\mathit{Mor}_U(Y, X)$ (see Lemma \ref{lemma-Isom-in-Mor} and
Remark \ref{remark-Mor-numerical}).
Now $\mathit{Mor}^{P', \mathcal{M}}_U(Y, X)$
is a Noetherian algebraic space as it is of finite
presentation over $U$ by Lemma \ref{lemma-Mor-qc-over-base}.
Thus the intersection
is a Noetherian algebraic space too. Since the morphism
$$
\mathit{Isom}_{\Polarizedstack}(\upsilon, \chi)
\longrightarrow
\mathit{Isom}_U(Y, X) \cap \mathit{Mor}^{P', \mathcal{M}}_U(Y, X)
$$
is affine (see above) we conclude.
\end{proof}

\begin{lemma}
\label{lemma-polarized-qs-lfp}
The morphism $\Polarizedstack \to \Spec(\mathbf{Z})$ is quasi-separated and
locally of finite presentation.
\end{lemma}

\begin{proof}
To check $\Polarizedstack \to \Spec(\mathbf{Z})$ is quasi-separated we have to
show that its diagonal is quasi-compact and quasi-separated.
This is immediate from Lemma \ref{lemma-polarized-diagonal-separated-fp}.
To prove that $\Polarizedstack \to \Spec(\mathbf{Z})$ is locally of finite
presentation, it suffices to show that $\Polarizedstack$
is limit preserving, see Limits of Stacks, Proposition
\ref{stacks-limits-proposition-characterize-locally-finite-presentation}.
This is Quot, Lemma \ref{quot-lemma-polarized-limits}.
\end{proof}

\begin{lemma}
\label{lemma-bounded-polarized}
Let $n \geq 1$ be an integer and let $P$ be a numerical polynomial.
Let
$$
T \subset |\Polarizedstack|
$$
be a subset with the following property: for every $\xi \in T$
there exists a field $k$ and an object $(X, \mathcal{L})$
of $\Polarizedstack$ over $k$ representing $\xi$ such that
\begin{enumerate}
\item the Hilbert polynomial of $\mathcal{L}$ on $X$ is $P$, and
\item there exists a closed immersion $i : X \to \mathbf{P}^n_k$
such that $i^*\mathcal{O}_{\mathbf{P}^n}(1) \cong \mathcal{L}$.
\end{enumerate}
Then $T$ is a Noetherian topological space, in particular quasi-compact.
\end{lemma}

\begin{proof}
Observe that $|\Polarizedstack|$ is a locally Noetherian topological
space, see Morphisms of Stacks, Lemma
\ref{stacks-morphisms-lemma-Noetherian-topology}
(this also uses that $\Spec(\mathbf{Z})$ is Noetherian and
hence $\Polarizedstack$ is a locally Noetherian algebraic stack
by Lemma \ref{lemma-polarized-qs-lfp} and
Morphisms of Stacks, Lemma
\ref{stacks-morphisms-lemma-locally-finite-type-locally-noetherian}).
Thus any quasi-compact subset of $|\Polarizedstack|$ is
a Noetherian topological space and any subset of such is
also Noetherian, see
Topology, Lemmas \ref{topology-lemma-finite-union-Noetherian} and
\ref{topology-lemma-Noetherian}.
Thus all we have to do is a find a quasi-compact subset
containing $T$.

\medskip\noindent
By Lemma \ref{lemma-hilb-proper-over-base} the algebraic space
$$
H =
\Hilbfunctor^{P, \mathcal{O}(1)}_{\mathbf{P}^n_\mathbf{Z}/\Spec(\mathbf{Z})}
$$
is proper over $\Spec(\mathbf{Z})$. By
Quot, Lemma \ref{quot-lemma-extend-hilb-to-spaces}\footnote{We will see
later (insert future reference here) that $H$ is a scheme and hence the
use of this lemma and Quot, Lemma \ref{quot-lemma-extend-polarized-to-spaces}
isn't necessary.} the identity morphism of $H$ corresponds
to a closed subspace
$$
Z \subset \mathbf{P}^n_H
$$
which is proper, flat, and of finite presentation over $H$ and
such that the restriction $\mathcal{N} = \mathcal{O}(1)|_Z$
is relatively ample on $Z/H$ and has Hilbert polynomial $P$
on the fibres of $Z \to H$. In particular, the pair $(Z \to H, \mathcal{N})$
defines a morphism
$$
H \longrightarrow \Polarizedstack
$$
which sends a morphism of schemes $U \to H$ to the classifying morphism
of the family $(Z_U \to U, \mathcal{N}_U)$, see
Quot, Lemma \ref{quot-lemma-extend-polarized-to-spaces}.
Since $H$ is a Noetherian algebraic space
(as it is proper over $\mathbf{Z})$)
we see that $|H|$ is Noetherian and hence quasi-compact. The map
$$
|H| \longrightarrow |\Polarizedstack|
$$
is continuous, hence the image is quasi-compact.
Thus it suffices to prove $T$ is contained
in the image of $|H| \to |\Polarizedstack|$.
However, assumptions (1) and (2) exactly express the fact
that this is the case: any choice of a closed immersion
$i : X \to \mathbf{P}^n_k$ with
$i^*\mathcal{O}_{\mathbf{P}^n}(1) \cong \mathcal{L}$ we get a
$k$-valued point of $H$ by the moduli interpretation of $H$.
This finishes the proof of the lemma.
\end{proof}






\section{Properties of moduli of complexes on a proper morphism}
\label{section-complexes}

\noindent
Let $f : X \to B$ be a morphism of algebraic spaces which is proper,
flat, and of finite presentation. Then the stack
$\Complexesstack_{X/B}$ parametrizing relatively perfect complexes
with vanishing negative self-exts is algebraic. See
Quot, Theorem \ref{quot-theorem-complexes-algebraic}.

\begin{lemma}
\label{lemma-complexes-diagonal-affine-fp}
The diagonal of $\Complexesstack_{X/B}$ over $B$ is affine
and of finite presentation.
\end{lemma}

\begin{proof}
The representability of the diagonal by algebraic spaces
was shown in Quot, Lemma \ref{quot-lemma-complexes-diagonal}.
From the proof we find that we have to show:
given a scheme $T$ over $B$ and objects
$E, E' \in D(\mathcal{O}_{X_T})$ such that
$(T, E)$ and $(T, E')$ are objects of the fibre category
of $\Complexesstack_{X/B}$ over $T$, then
$\mathit{Isom}(E, E') \to T$
is affine and of finite presentation.
Here $\mathit{Isom}(E, E')$ is the functor
$$
(\Sch/T)^{opp} \to \textit{Sets},\quad
T' \mapsto \{\varphi : E_{T'} \to E'_{T'}
\text{ isomorphism in }D(\mathcal{O}_{X_{T'}})\}
$$
where $E_{T'}$ and $E'_{T'}$ are the derived pullbacks of $E$ and $E'$
to $X_{T'}$. Consider the functor $H = \SheafHom(E, E')$ defined
by the rule
$$
(\Sch/T)^{opp} \to \textit{Sets},\quad
T' \mapsto \Hom_{\mathcal{O}_{X_{T'}}}(E_T, E'_T)
$$
By Quot, Lemma \ref{quot-lemma-complexes-open-neg-exts-vanishing}
this is an algebraic space affine and of finite presentation over $T$.
The same is true for $H' = \SheafHom(E', E)$, $I = \SheafHom(E, E)$, and
$I' = \SheafHom(E', E')$. Therefore we see that
$$
\mathit{Isom}(E, E') = (H' \times_T H) \times_{c, I \times_T I', \sigma} T
$$
where $c(\varphi', \varphi) = (\varphi \circ \varphi', \varphi' \circ \varphi)$
and $\sigma = (\text{id}, \text{id})$ (compare with the proof of
Quot, Proposition \ref{quot-proposition-isom}). Thus
$\mathit{Isom}(E, E')$ is affine over $T$ as a fibre product of
schemes affine over $T$. Similarly, $\mathit{Isom}(E, E')$ is
of finite presentation over $T$.
\end{proof}

\begin{lemma}
\label{lemma-complexes-qs-lfp}
The morphism $\Complexesstack_{X/B} \to B$ is quasi-separated and
locally of finite presentation.
\end{lemma}

\begin{proof}
To check $\Complexesstack_{X/B} \to B$ is quasi-separated we have to
show that its diagonal is quasi-compact and quasi-separated.
This is immediate from Lemma \ref{lemma-complexes-diagonal-affine-fp}.
To prove that $\Complexesstack_{X/B} \to B$ is locally of finite
presentation, we have to show that $\Complexesstack_{X/B} \to B$
is limit preserving, see
Limits of Stacks, Proposition
\ref{stacks-limits-proposition-characterize-locally-finite-presentation}.
This follows from Quot, Lemma \ref{quot-lemma-complexes-limits}
(small detail omitted).
\end{proof}





\begin{multicols}{2}[\section{Other chapters}]
\noindent
Preliminaries
\begin{enumerate}
\item \hyperref[introduction-section-phantom]{Introduction}
\item \hyperref[conventions-section-phantom]{Conventions}
\item \hyperref[sets-section-phantom]{Set Theory}
\item \hyperref[categories-section-phantom]{Categories}
\item \hyperref[topology-section-phantom]{Topology}
\item \hyperref[sheaves-section-phantom]{Sheaves on Spaces}
\item \hyperref[sites-section-phantom]{Sites and Sheaves}
\item \hyperref[stacks-section-phantom]{Stacks}
\item \hyperref[fields-section-phantom]{Fields}
\item \hyperref[algebra-section-phantom]{Commutative Algebra}
\item \hyperref[brauer-section-phantom]{Brauer Groups}
\item \hyperref[homology-section-phantom]{Homological Algebra}
\item \hyperref[derived-section-phantom]{Derived Categories}
\item \hyperref[simplicial-section-phantom]{Simplicial Methods}
\item \hyperref[more-algebra-section-phantom]{More on Algebra}
\item \hyperref[smoothing-section-phantom]{Smoothing Ring Maps}
\item \hyperref[modules-section-phantom]{Sheaves of Modules}
\item \hyperref[sites-modules-section-phantom]{Modules on Sites}
\item \hyperref[injectives-section-phantom]{Injectives}
\item \hyperref[cohomology-section-phantom]{Cohomology of Sheaves}
\item \hyperref[sites-cohomology-section-phantom]{Cohomology on Sites}
\item \hyperref[dga-section-phantom]{Differential Graded Algebra}
\item \hyperref[dpa-section-phantom]{Divided Power Algebra}
\item \hyperref[sdga-section-phantom]{Differential Graded Sheaves}
\item \hyperref[hypercovering-section-phantom]{Hypercoverings}
\end{enumerate}
Schemes
\begin{enumerate}
\setcounter{enumi}{25}
\item \hyperref[schemes-section-phantom]{Schemes}
\item \hyperref[constructions-section-phantom]{Constructions of Schemes}
\item \hyperref[properties-section-phantom]{Properties of Schemes}
\item \hyperref[morphisms-section-phantom]{Morphisms of Schemes}
\item \hyperref[coherent-section-phantom]{Cohomology of Schemes}
\item \hyperref[divisors-section-phantom]{Divisors}
\item \hyperref[limits-section-phantom]{Limits of Schemes}
\item \hyperref[varieties-section-phantom]{Varieties}
\item \hyperref[topologies-section-phantom]{Topologies on Schemes}
\item \hyperref[descent-section-phantom]{Descent}
\item \hyperref[perfect-section-phantom]{Derived Categories of Schemes}
\item \hyperref[more-morphisms-section-phantom]{More on Morphisms}
\item \hyperref[flat-section-phantom]{More on Flatness}
\item \hyperref[groupoids-section-phantom]{Groupoid Schemes}
\item \hyperref[more-groupoids-section-phantom]{More on Groupoid Schemes}
\item \hyperref[etale-section-phantom]{\'Etale Morphisms of Schemes}
\end{enumerate}
Topics in Scheme Theory
\begin{enumerate}
\setcounter{enumi}{41}
\item \hyperref[chow-section-phantom]{Chow Homology}
\item \hyperref[intersection-section-phantom]{Intersection Theory}
\item \hyperref[pic-section-phantom]{Picard Schemes of Curves}
\item \hyperref[weil-section-phantom]{Weil Cohomology Theories}
\item \hyperref[adequate-section-phantom]{Adequate Modules}
\item \hyperref[dualizing-section-phantom]{Dualizing Complexes}
\item \hyperref[duality-section-phantom]{Duality for Schemes}
\item \hyperref[discriminant-section-phantom]{Discriminants and Differents}
\item \hyperref[derham-section-phantom]{de Rham Cohomology}
\item \hyperref[local-cohomology-section-phantom]{Local Cohomology}
\item \hyperref[algebraization-section-phantom]{Algebraic and Formal Geometry}
\item \hyperref[curves-section-phantom]{Algebraic Curves}
\item \hyperref[resolve-section-phantom]{Resolution of Surfaces}
\item \hyperref[models-section-phantom]{Semistable Reduction}
\item \hyperref[functors-section-phantom]{Functors and Morphisms}
\item \hyperref[equiv-section-phantom]{Derived Categories of Varieties}
\item \hyperref[pione-section-phantom]{Fundamental Groups of Schemes}
\item \hyperref[etale-cohomology-section-phantom]{\'Etale Cohomology}
\item \hyperref[crystalline-section-phantom]{Crystalline Cohomology}
\item \hyperref[proetale-section-phantom]{Pro-\'etale Cohomology}
\item \hyperref[relative-cycles-section-phantom]{Relative Cycles}
\item \hyperref[more-etale-section-phantom]{More \'Etale Cohomology}
\item \hyperref[trace-section-phantom]{The Trace Formula}
\end{enumerate}
Algebraic Spaces
\begin{enumerate}
\setcounter{enumi}{64}
\item \hyperref[spaces-section-phantom]{Algebraic Spaces}
\item \hyperref[spaces-properties-section-phantom]{Properties of Algebraic Spaces}
\item \hyperref[spaces-morphisms-section-phantom]{Morphisms of Algebraic Spaces}
\item \hyperref[decent-spaces-section-phantom]{Decent Algebraic Spaces}
\item \hyperref[spaces-cohomology-section-phantom]{Cohomology of Algebraic Spaces}
\item \hyperref[spaces-limits-section-phantom]{Limits of Algebraic Spaces}
\item \hyperref[spaces-divisors-section-phantom]{Divisors on Algebraic Spaces}
\item \hyperref[spaces-over-fields-section-phantom]{Algebraic Spaces over Fields}
\item \hyperref[spaces-topologies-section-phantom]{Topologies on Algebraic Spaces}
\item \hyperref[spaces-descent-section-phantom]{Descent and Algebraic Spaces}
\item \hyperref[spaces-perfect-section-phantom]{Derived Categories of Spaces}
\item \hyperref[spaces-more-morphisms-section-phantom]{More on Morphisms of Spaces}
\item \hyperref[spaces-flat-section-phantom]{Flatness on Algebraic Spaces}
\item \hyperref[spaces-groupoids-section-phantom]{Groupoids in Algebraic Spaces}
\item \hyperref[spaces-more-groupoids-section-phantom]{More on Groupoids in Spaces}
\item \hyperref[bootstrap-section-phantom]{Bootstrap}
\item \hyperref[spaces-pushouts-section-phantom]{Pushouts of Algebraic Spaces}
\end{enumerate}
Topics in Geometry
\begin{enumerate}
\setcounter{enumi}{81}
\item \hyperref[spaces-chow-section-phantom]{Chow Groups of Spaces}
\item \hyperref[groupoids-quotients-section-phantom]{Quotients of Groupoids}
\item \hyperref[spaces-more-cohomology-section-phantom]{More on Cohomology of Spaces}
\item \hyperref[spaces-simplicial-section-phantom]{Simplicial Spaces}
\item \hyperref[spaces-duality-section-phantom]{Duality for Spaces}
\item \hyperref[formal-spaces-section-phantom]{Formal Algebraic Spaces}
\item \hyperref[restricted-section-phantom]{Algebraization of Formal Spaces}
\item \hyperref[spaces-resolve-section-phantom]{Resolution of Surfaces Revisited}
\end{enumerate}
Deformation Theory
\begin{enumerate}
\setcounter{enumi}{89}
\item \hyperref[formal-defos-section-phantom]{Formal Deformation Theory}
\item \hyperref[defos-section-phantom]{Deformation Theory}
\item \hyperref[cotangent-section-phantom]{The Cotangent Complex}
\item \hyperref[examples-defos-section-phantom]{Deformation Problems}
\end{enumerate}
Algebraic Stacks
\begin{enumerate}
\setcounter{enumi}{93}
\item \hyperref[algebraic-section-phantom]{Algebraic Stacks}
\item \hyperref[examples-stacks-section-phantom]{Examples of Stacks}
\item \hyperref[stacks-sheaves-section-phantom]{Sheaves on Algebraic Stacks}
\item \hyperref[criteria-section-phantom]{Criteria for Representability}
\item \hyperref[artin-section-phantom]{Artin's Axioms}
\item \hyperref[quot-section-phantom]{Quot and Hilbert Spaces}
\item \hyperref[stacks-properties-section-phantom]{Properties of Algebraic Stacks}
\item \hyperref[stacks-morphisms-section-phantom]{Morphisms of Algebraic Stacks}
\item \hyperref[stacks-limits-section-phantom]{Limits of Algebraic Stacks}
\item \hyperref[stacks-cohomology-section-phantom]{Cohomology of Algebraic Stacks}
\item \hyperref[stacks-perfect-section-phantom]{Derived Categories of Stacks}
\item \hyperref[stacks-introduction-section-phantom]{Introducing Algebraic Stacks}
\item \hyperref[stacks-more-morphisms-section-phantom]{More on Morphisms of Stacks}
\item \hyperref[stacks-geometry-section-phantom]{The Geometry of Stacks}
\end{enumerate}
Topics in Moduli Theory
\begin{enumerate}
\setcounter{enumi}{107}
\item \hyperref[moduli-section-phantom]{Moduli Stacks}
\item \hyperref[moduli-curves-section-phantom]{Moduli of Curves}
\end{enumerate}
Miscellany
\begin{enumerate}
\setcounter{enumi}{109}
\item \hyperref[examples-section-phantom]{Examples}
\item \hyperref[exercises-section-phantom]{Exercises}
\item \hyperref[guide-section-phantom]{Guide to Literature}
\item \hyperref[desirables-section-phantom]{Desirables}
\item \hyperref[coding-section-phantom]{Coding Style}
\item \hyperref[obsolete-section-phantom]{Obsolete}
\item \hyperref[fdl-section-phantom]{GNU Free Documentation License}
\item \hyperref[index-section-phantom]{Auto Generated Index}
\end{enumerate}
\end{multicols}


\bibliography{my}
\bibliographystyle{amsalpha}

\end{document}
